
\documentclass[sigchi]{acmart}
\settopmatter{printacmref=false, printccs=true, printfolios=true}
%%
%% \BibTeX command to typeset BibTeX logo in the docs
\AtBeginDocument{%
  \providecommand\BibTeX{{%
    \normalfont B\kern-0.5em{\scshape i\kern-0.25em b}\kern-0.8em\TeX}}}

%% Rights management information.  This information is sent to you
%% when you complete the rights form.  These commands have SAMPLE
%% values in them; it is your responsibility as an author to replace
%% the commands and values with those provided to you when you
%% complete the rights form.
\setcopyright{none}
%\copyrightyear{none}
\acmYear{2021}
\acmDOI{}

%% These commands are for a PROCEEDINGS abstract or paper.
\acmConference[Technical Report]{May}{{\today}}{Cologne, Germany}
\acmBooktitle{}
\acmPrice{}
\acmISBN{}

\usepackage{subfig}

\begin{document}

\title{The impact of the avatar representation on team trust and effectiveness in a shared virtual environment.}

\author{Hannes Hinrichs}
\email{hhinrich@smail.th-koeln.de}
\affiliation{%
  \institution{TH-Köln}
  \city{Cologne}
  \country{Germany}
}

\begin{abstract}
The goal of this work is to find out if different avatar representations have an impact on the formed
trust and effectiveness of a team. The first question here is whether an inverse-kinematic humanlike
representation or an abstract non-human-like representation is more effective in generating
trust in a newly formed virtual team. The second question addresses whether the trust formed
by the different representations influences the effectiveness of the virtual team. To answer these
questions, a quantitative study was conducted in which different participants in a three-person
team performed a collaborative task in a shared virtual environment. No significant differences in
effectiveness were found between the teams. The results of the study also shows that in the threeperson
teams in the shared virtual environment, significantly more trust was built with non-humanlike
avatars, as well as that there was a significant relationship between the cognitive trust formed
and team effectiveness. This means that the simplicity of a non-human-like avatar in a newly formed
team in a shared virtual environment can be effective in creating a trusting work atmosphere.
\end{abstract}

\keywords{Virtual-Reality, Trust, Team formation, virtual Team, Avatar}

%% A "teaser" image appears between the author and affiliation
%% information and the body of the document, and typically spans the
%% page.
\begin{teaserfigure}
  \includegraphics[width=\textwidth]{Abbildungen/RoundSuccsessful2}
  \caption{This figure represents the developed Shared-Virual-Environment infront of there Podests. A green sphere appears clearly visible for all participants when a round is successfully completed.}
  \Description{This figure represents the developed Shared-Virual-Environment. The green sphere appears clearly visible for all participants when a round is successfully completed.}
  \label{fig:teaser}
\end{teaserfigure}

%%
%% This command processes the author and affiliation and title
%% information and builds the first part of the formatted document.
\maketitle

\section{INTRODUCTION}
Mit voranschreitender technologischer Entwicklung rückt die digitale Kommunikation immer mehr in den Mittelpunkt. Unternehmen weltweit setzen schon seit langem darauf, räumliche und zeitliche Grenzen zu überwinden.
Neue Generationen von sozialen Netzwerksystemen werden unter der Prämisse erstellt, die Kommunikation zu entfernten Personen zu verbessern.
Einsatzfelder sind beispielsweise die \textit{Mobil- und Internettelefonie}, die\textit{ FOIP/VOIP- Telefonkonferenzen} oder die \textit{sozialen virtuellen 3D- Umgebungen}.
All diese Technologien teilen dasselbe Ziel: 
\begin{quote}
"Die Verbesserung der \textit{sozialen Präsenz}, sodass der Nutzer das Gefühl hat zu einem gewissen Grad Einblicke in die kognitiven und affektiven Zustände des anderen zu haben" \citep{biocca2002defining} \citep[S. 407–447]{biocca2001plugging}.
\end{quote}
Mitarbeiter befinden sich sehr häufig nicht am selben Ort, wobei sich viele Unternehmen trotzdem eine effektive Gestaltung ihrer Teams wünschen \citep[S. 791-792]{jarvenpaa1999communication}. \textit{Virtuelle Teams} können hierbei Abhilfe schaffen. 
Vor der Corona-Pandemie, im 2. Quartal 2020, haben 4\% aller Angestellten in Deutschland im Homeoffice gearbeitet. Dieser Anteil ist im Laufe des Jahres - Stand 01.01.2021 - auf 24\% gestiegen und es könnten theoretisch 80\% der Belegschaften von zu Hause arbeiten \citep{statistaCorona2020}. Durch diese Entwicklung mussten sich Unternehmen zwangsläufig mit der Funktionsweise von virtuellen Teams beschäftigen.
Trifft sich ein virtuelles Team in einer virtuellen Realität, können Avatare zur Repräsentation des eigenen Individuums eingesetzt werden. Durch diese wird mit anderen Teilnehmern des Shared-Virtual-Environment interagiert und kommuniziert. 
In einem räumlich getrennten Team zu arbeiten, das sich gegenseitig nicht vertraut oder nicht richtig zusammenarbeitet, hemmt dessen Performance \citep[S. 98-107]{huang1998supporting} \citep[S. 399-417]{turoff1993distributed}.
Diese Arbeit zielt darauf ab, das Konstrukt \textit{Vertrauen} in der virtuellen Welt besser zu verstehen und mit diesem umzugehen.
Es gilt herauszufinden, welche Art von Repräsentation in einem Shared-Virtual-Environment, während der Neugründung eines virtuellen Teams, mehr zwischenmenschliches Vertrauen aufbaut. Es wird ein Fokus wird auf die beiden Avatarkonditionen IK sowie NIK gelegt, um zu analysieren, ob es einen Zusammenhang zwischen dem gebildeten \textit{kognitiven Vertrauen} im Team und der \textit{Teameffektivität} bei unterschiedlichen Repräsentationsarten des Avatars während der erstmaligen Zusammenarbeit gibt.
Zudem wird sich auf das den \textit{generellen Hang zum Vertrauen} der einzelnen Personen konzentriert, um zu untersuchen, ob der \textit{generelle Hang zum Vertrauen} einen Einfluss auf die \textit{Teameffektivität} und das \textit{kognitive Vertrauen} im Hinblick auf die unterschiedlichen Konditionen besitzt. Dazu bestreitet ein Drei-Personen-Team, dessen Mitglieder sich nicht kennen, in einem Shared-Virtual-Environment eine kooperative Aufgabe.

\section{RELEATED WORD}
\subsection{TRUST}
Die meist verbreitetste Definition stammt von Meyer et al. \citep[S. 712]{mayer1995integrative}. So definieren sie Vertrauen als:
\begin{quote} "die Bereitschaft einer Person, für die Handlungen einer anderen Person anfällig zu sein, basierend auf der Erwartung, dass der Vertrauensnehmer eine bestimmte, für den Vertrauensgeber wichtige Handlung ausführen wird, unabhängig von der Möglichkeit, diese Person zu überwachen oder zu kontrollieren." \end{quote}

Jede zwischenmenschliche Beziehung beginnt mit einer frühen Phase der Vertrauensbildung. Diese frühe Phase kann von Unsicherheiten und Zweifel geprägt sein. Das gegenseitige Vertrauen, das man sich anschließend schenkt, muss anfänglich erst einmal ausgelotet werden \citep[S. 166-168]{meyerson1996swift}.
Während der frühen Phase der Vertrauensbildung entscheidet sich, ob eine Beziehung aufrechterhalten wird oder nicht. Unterbewusst bildet sich ein Gefühl von Zuversicht und Sicherheit oder ein Gefühl von Spannung, Zweifel und Skepsis dem Interaktionspartner gegenüber. 
Dabei ist es egal, ob sich dafür entschieden wird, jemandem zu vertrauen, oder nicht. Auf jeden Fall beeinflusst die Stärke des positiven oder negativen Vertrauensgefühls die Effektivität der Zusammenarbeit. Vertrauen kann es einfach oder schwierig machen, mit einer anderen Person zu arbeiten und Ziele in einer Gruppe oder einem Team zu erreichen \citep[S. 405-406]{bigley1998straining}.

Die initiale Phase der Vertrauensbildung wirkt sich auf das \textit{kognitive} und das \textit{affektive Vertrauen aus}, welche einen starken Einfluss auf das sich entwickelnde Vertrauensmodell zu einer Person haben. In dieser Phase sind diese beiden Vertrauensarten anfällig für Veränderungen \citep[S. 461-462]{baldwin1992relational}.
Meinungen und Annahmen, die sich frühzeitig bilden, prägen somit auch stark die zukünftige Meinung über die vertrauensnehmenden Personen.

Vertrauen wird nicht statisch und einseitig betrachtet. Eine Person kann nicht nur \textit{vertrauen} oder \textit{nicht vertrauen}. Vertrauen ist ein dynamisches Konstrukt, welches sich mit der Zeit verändert. Es kann in eine Bildungs-, Stabilisierungs- und Abnehmphase unterteilt werden \citep[S. 396]{rousseau1998not}.

Viele Psychologen, die sich mit dem Thema Vertrauen beschäftigen, gehen heute davon aus, dass \textit{zwischenmenschliches} Vertrauen aus einem zweidimensionalen Konstrukt besteht \citep{johnson2005cognitive}; \citep{cook1980new}. So sind Mooradian et al. der Ansicht, dass Vertrauen als \textit{Eigenschaft} oder als \textit{Zustand} gesehen wird \citep[S. 524-525]{mooradian2006trusts}.

\subsubsection{TRUST AS A TRAIT }
Wird Vertrauen als Eigenschaft betrachtet, spiegelt dies die Einstellung zum Vertrauen einer Person wider. Diese Einstellung zum Vertrauen ist langlebig und wird nicht allzu schnell auf oder abgebaut. Unabhängig von einer Situation, in der sich diese Person befindet, wird davon ausgegangen, dass diese Eigenschaft aus dem Temperament oder der Lebenserfahrung einer Person entsteht. Dieses Vertrauen ist das Grundlevel an Vertrauen, das eine Person in eine neue zwischenmenschliche Beziehung von Anfang an mitbringt \citep[S. 11]{couch1996assessment}. Es ist jedoch nicht nachgewiesen, wie \textit{generelles Vertrauen} genau gebildet wird \citep[S. 409]{stolle2002trusting}.

\textit{Generelles Vertrauen} impliziert, dass den meisten Personen vertraut werden kann, oder dass im Fall von generellem Misstrauen, Personen nicht vertraut werden kann \citep[S. 409]{stolle2002trusting}.

Der \textit{generelle Hang zum Vertrauen} ist nicht situationsabhängig, sondern stellt eine längerfristige Konstante auf Basis des Grundvertrauens einer Person dar. Grundvertrauen setzt sich dabei aus der individuellen Eigenschaft des Hangs zum Vertrauen einer einzelnen Person sowie der Grundstimmung gegenüber Personen im Allgemeinen, zusammen \citep[S. 11]{couch1996assessment}.

\subsubsection{TRUST AS A STATE}

Wird \textit{Vertrauen als Zustand} betrachtet, so kann sich dieses Vertrauen im Laufe der Zeit, z.B. durch Interaktion mit einer anderen Person, ändern \citep[S. 712]{mayer1995integrative}.
Das Konzept des \textit{Vertrauens als Zustand} lässt sich laut Lewis et al. \citep[S. 970-971]{lewis1985trust} in eine \textit{kognitive Komponente} und eine \textit{affektive Komponente} unterteilen.

\paragraph{COGNITIVE BASED TRUST}

Laut den Studie von Lewis et al. basiert unter anderem 
\begin{quote}
"auf einem kognitiven Prozess, der zwischen vertrauenswürdigen, misstrauischen und unbekannten Personen und Institutionen unterscheidet. In diesem Sinne wählen wir kognitiv aus, wem wir in welcher Hinsicht und unter welchen Umständen vertrauen, und wir stützen die Wahl auf das, was wir als "gute Gründe" ansehen, die einen Beweis für die Vertrauenswürdigkeit darstellen" \citep[S. 970]{lewis1985trust}.
\end{quote}

Somit basiert das \textbf{kognitiv aufgebaute Vertrauen} auf einer von uns definierten Logik statt auf einer emotionalen Komponente. Diese "guten Gründe" können auch leicht gebrochen werden, indem der Vertrauensvorschuss, den wir durch das kognitive Vertrauen unserem Interaktionspartner geben, gebrochen wird.
Das kognitive Vertrauen kann kurzfristig aufgebaut werden und ist leicht anfällig gegen äußerliche Einflüsse \citep[S. 970]{lewis1985trust}. 

\paragraph{AFFECTIVE BASED TRUST}

Weiterhin besitzt Vertrauen als \textit{Zustand} eine affektive Komponente:
\begin{quote}
"Diese \textbf{affektive Komponente des Vertrauens} besteht in einer emotionalen Bindung zwischen allen, die an der Beziehung beteiligt sind. Wie die affektiven Bindungen der Freundschaft und der Liebe schafft Vertrauen eine soziale Situation, in der intensive emotionale Investitionen getätigt werden können, und deshalb weckt der Verrat eines persönlichen Vertrauens ein Gefühl der emotionalen Empörung bei dem Betrogenen. Der Vertrauensbruch trifft die Grundlage der Beziehung selbst, nicht nur den spezifischen Inhalt des Verrats. Diese emotionale Komponente ist bei allen Arten von Vertrauen vorhanden, aber normalerweise ist sie bei engem zwischenmenschlichem Vertrauen am intensivsten" \citep[S. 971]{lewis1985trust}.
\end{quote}
Ein Beispiel für affektives Vertrauen ist die Liebesbeziehung zwischen zwei Personen. Das affektive Vertrauen baut sich mit der Zeit langsam auf und kann durch verschiedene Ereignisse erschüttert oder gestärkt werden. Es kann eher durch Emotionen statt durch Logik charakterisiert werden.
\textbf{Affektives Vertrauen} ergibt sich aus \textit{zwischenmenschlichen emotionalen Verbindungen und gegenseitiger Fürsorge}, während individuelles \textbf{kognitives Vertrauen} auf der \textit{Überzeugung in die Fähigkeiten oder in die Zuverlässigkeit eines anderen basiert} \citep[S. 30]{mcallister1995affect}.

\subsection{VIRTUAL TEAMS}
Ein Team wird als eine "kleine Gruppe von Menschen mit gleichartigen Fähigkeiten, welche sich in gleicher Weise für das gleiche Ziel und gleiche Arbeitsweisen einsetzen und dies verfolgen" \citep[S. 2]{zenun2007effects}, definiert.

\textit{Virtuelle Teams} teilen viele Eigenschaften herkömmlicher \textit{Teams}. Es muss jedoch unterschieden werden, wie die virtuelle Komponente des Teams definiert wird und weshalb diese Komponente ein herkömmliches Team zu einem virtuellen Team macht.

\textit{Virtuelle Teams} besitzen laut Schweizer et al. \citep[S. 270]{schweitzer2010conceptualizing} noch vier weitere Kennzeichen, um als \textit{virtuelles Team} zu gelten. Laut ihnen sind diese:
\label{AnforderungenLautSchweitzer}
\begin{itemize}
\item Zustande gekommen mithilfe von Kommunikationstechnologie. Durch technische Hilfsmittel wird kommuniziert, es werden Entscheidungen getroffen oder Informationen ausgetauscht.
\item Räumlich getrennt. \textit{Virtuelle Teams} arbeiten \textit{nicht} am selben Arbeitsplatz.
\item Grenzübergreifend. Die Teammitglieder stammen aus verschiedenen Organisationen oder Organisationseinheiten.
\item Asynchron. \textit{Virtuelle Teams} arbeiten zu unterschiedlichen Zeiten, in verschiedenen Zeitzonen oder in derselben Zeitzone in unterschiedlichen Schichten.
\end{itemize}

\textit{Virtuelle Teams} werden häufig aufgrund eines anstehenden Projektes gebildet und wieder aufgelöst, wenn das Projekt beendet ist. Somit sind \textit{virtuelle Teams}  oftmals nur von kurzer Lebensdauer. Dies impliziert auch, dass die Aufgabenverteilung in virtuellen Teams je nach Projekt oder Aufgabe immer unterschiedlich ist und keine einheitliche Hierarchie innerhalb des \textit{virtuellen Teams} zustande kommt \citep{wong2000virtual}.
Das \textit{virtuelle Team} zu gründen, stellt laut Dyer nicht die eigentliche Herausforderung dar. Die Herausforderung ergibt sich aus den unterschiedlichen Kulturen, Entfernungen und Zeitzonen, die ein \textit{virtuelles Team} mitbringt. Wenn die einzelnen Teammitglieder eines \textit{virtuellen Teams} sich gegenseitig vertrauen, kann der eigentliche Nachteil der verschiedenen Kulturen, Entfernungen und der Zeitzonen auch zum Vorteil werden. Es wird kulturelle Diversität gefordert und neue Verhaltensmuster erworben, wodurch neue, kreative Sichtweisen gefördert werden. Durch diese Faktoren ist es möglich, innovativer zu arbeiten und zu denken \citep{dyer1995team} \citep[S. 405-416]{milliken1996searching}.

Es wird davon ausgegangen, dass Virtualität als Kontinuum gesehen werden kann, bei dem jedes Team ein gewisses Maß an Virtualität besitzt. Dieses Kontinuum reicht von Face-to-Face bis zur vollständigen, nur über Kommunikationstechnologie stattfindenden Kommunikation \citep{martins2004virtual} (siehe \textit{Abbildung \ref{virtualTeamsVirtuality}}).

\begin{figure}[H]
		\begin{footnotesize}
		\centering
			\includegraphics[width=\linewidth]{Abbildungen/GlobalVirtualTeam.PNG}	
			\caption[Virtualität eines virtuellen Teams]{Grad an Virtualität, das ein Team laut Javenpaa et al. \citep{jarvenpaa1999communication} besitzen muss, um als \textit{virtuelles Team} zu gelten.}
			\label{virtualTeamsVirtuality}
		\end{footnotesize}
	\end{figure}

Die Mitglieder eines \textit{virtuellen Teams} haben im Gegensatz zu traditionell geformten Teams weniger Möglichkeiten, sich zu sehen, zu interagieren oder Konflikte zu lösen. 
Respekt und gegenseitiges Verständnis sind die Grundbausteine, um Kreativität und Innovation innerhalb eines Teams zu fördern. Die Effektivität eines Teams ist eine direkte Konsequenz daraus \citep[S. 378]{ren2007applying}.

\paragraph{VIRTUAL TEAMS AND TEAMEFFECTIVENESS}

Wird ein Vergleich zwischen \textit{traditionell} geformten Teams und \textit{virtuellen Teams} gezogen, gehen Schweitzer et al. \citep{schweitzer2010conceptualizing} davon aus, dass traditionell geformte Teams effektiver als \textit{virtuelle Teams} sind und die \textit{Teameffektivität} abnimmt je höher der Grad an Virtualität (siehe \textit{Abbildung \ref{virtualTeamsVirtuality}}) ist.
Diese Meinung teilt auch Becker et al., denn laut ihnen leidet der Austausch des Informationsgehaltes sowie der Vertrauensbildung aufgrund steigender Virtualität und kann nicht dieselbe Effektivität wie ein Face-to-Face Team erreichen \citep{becker2002fuhrung}.
Eine andere Meinung nehmen dabei Dufner et al. ein \citep{dufner2002asynchronous}. Ein zeitlich asynchroner Informationsaustausch von \textit{virtuellen Teams} hat laut ihrer Untersuchung einen positiven Einfluss auf die \textit{Teameffektivität}, da die Teammitglieder mehr Zeit haben, um über Probleme nachzudenken, bevor Informationen ausgetauscht werden.

Bisherige Studien haben positive Zusammenhänge \citep{davis2000trusted}, keine Zusammenhänge \citep{hertel2004managing} sowie negative Zusammenhänge \citep{dirks1999effects} zwischen Vertrauen und \textit{Teameffektivität} in \textit{virtuellen Teams} festgestellt.
Trotz der sich widersprechenden Studienergebnisse wird im Allgemeinen die Meinung vertreten, dass Vertrauen einen positiven Einfluss auf die \textit{Teameffektivität} besitzt \citep{de2016trust}. 
Vertrauen in sein Team hilft dabei, eigene Unsicherheiten auszublenden, um sicherer und effektiver arbeiten zu können \citep{de2010does}. Weiterhin entsteht durch vorhandenes Vertrauen in sein Team ein größeres Interesse an den Teammitgliedern, was Synergieeffekte freischaltet und eine direktere und effektivere Interaktion ermöglicht \citep{dirks1999effects}. 

\subsection{AVATARS AND TRUST}

Es stellt sich die Frage, ob ein Avatar in einem Shared-Virtual-Environment menschenähnlich aussehen sollte. Dieser Frage gingen George et al. \citep{george2018trusting} in ihrer Forschung nach und verglichen, ob sich mehr Vertrauen zwischen einem menschenähnlichen oder einem roboterartigen Avatar aufbauen lässt.
%Dazu schufen sie ein Szenario, in dem Personen mittels eines Head-Mounted-Display ein Social-Dilemma-Scenario\footnote{Situationen, in denen - die rationale Verfolgung von Eigeninteressen zu einer kollektiven Katastrophe führen kann.} erlebten.
Sie fanden keinen signifikanten Unterschied in der Vertrauenswürdigkeit zwischen menschenähnlichen und roboterartigen Avataren. Jedoch wurde ein größeres Gefühl von Gemeinsamkeit festgestellt, wenn mit einem menschenähnlichen Avatar interagiert wurde.
% \citep{kerr1983motivation}.
George et al. erwähnten weiterhin in ihrer Studie, dass gute Grafik und realistisches Verhalten durch beispielsweise Mikrogestikulationen und soziale Interaktionen den Aufbau von \textit{Co-Präsenz} unterstützen \citep{george2018trusting}.

Um den Einfluss des Grades an Realismus unter Avataren zu erforschen, führten Riedl et al. \citep{riedl2014trusting} eine Studie zum Vertrauensaufbau unter Menschen im Vergleich zu Avataren mit menschenähnlichen Gesichtern durch. Sie fanden heraus, dass es Personen leichter fällt, einer realen Person, zu vertrauen als einem Avatar mit menschenähnlichem Gesicht. Es wurde der Frontalkortex - die Gehirnregion, die dafür verantwortlich ist, die Gedanken und Gefühle des Gegenübers zu erahnen - bei Interaktionen mit Menschen mehr angeregt als bei Interaktionen mit Avataren mit menschenähnlichen Gesichtern.
Vertrauen zwischen Menschen wird jedoch in der gleichen Geschwindigkeit aufgebaut wie zwischen Menschen und Avataren.

Somit lässt sich feststellen, dass ein höherer Grad an Realismus den Vertrauensaufbau fördert, jedoch kein signifikanter Unterschied in der Geschwindigkeit  des Vertrauensaufbaus zwischen einem menschenähnlichen sowie roboterähnlichen Avatar besteht. Diese Vermutung bestätigten auch Bente et al. \citep[S. 54-59]{bente2004social}, indem sie eine Studie zur \textit{sozialen Präsenz} von Avataren in einem \textit{Shared-Virtual-Environment} durchführten. Der Aufbau des \textit{Shared-Virtual-Environment} ähnelte einer Videokonferenz. Es waren keine \textit{Head-Mounted-Display}s vorhanden und die Teilnehmer haben sich während des Experiments nicht gesehen. In der Studie wurden die Kommunikationsarten, Face-to-Face, Chat und auf Avataren basierende Kommunikationsmedien untereinander verglichen, um Unterschiede in der \textit{Sozialen-Präsenz} sowie dem zwischenmenschlichen Vertrauen zu untersuchen.
Es wurde festgestellt, dass wenig \textit{kognitives Vertrauen} während der Nutzung des \textit{Shared-Virtual-Environment} zu Avataren aufgebaut werden konnte, während Face-to-Face, Telefon- und Chatkommunikationen besser abschnitten. Weiterhin wurde weniger \textit{affektives Vertrauen} im \textit{Shared-Virtual-Environment} als bei der Nutzung eines Telefons oder während der Face-to-Face Kommunikation aufgebaut.
Bente et al. \citep[S. 54-59]{bente2004social} gehen davon aus, dass dies mit der Neuheit der Technologie zusammenhängt.

\section{METHODS}

Um die beiden unabhängigen Variablen IK und NIK innerhalb der Versuchsumgebung zu untersuchen, wird das \textit{A/B-Testing} in Kombination mit einem induktiven quantitativen Forschungsdesign gewählt.
Gruppe A bekommt dabei die Kondition IK zugeteilt, während Gruppe B die Kondition NIK zugeteilt wird. Diese Gruppeneinteilung der Probanden erfolgt nach dem Zufallsprinzip. 

Die Analysen dieser Studie werden auf unterschiedlichen Ebenen durchgeführt.
Da die Teilnehmer als Team arbeiten und unterschiedliche Teams unterschiedliche Konditionen aufweisen, sind einige Zusammenhänge auf \textit{Individualebene}, einige auf \textit{Konditionsebene} und einige auf \textit{Teamebene} zu betrachten.

Die \textbf{Individualebene} sagt etwas über eine einzelne Person aus. Dadurch können alle Teilnehmer individuell betrachtet werden. Die Betrachtung ist dabei unabhängig vom Team oder den verschiedenen Avatar-Konditionen. 

Die \textbf{Konditionsebene} unterscheidet zwischen den Konditionen IK und NIK. Die Konditionsebene ordnet den einzelnen Teilnehmern die Kondition zu, die diese im Shared-Virtual-Environment als \textit{andere} Avatare wahrnehmen. 

Die Konditionsebene kann in einzelne Teams von jeweils 3 Personen aufgeteilt werden. Diese Aufteilung wird als \textbf{Teamebene} bezeichnet und betrachtet das gesamte Team als Einheit. Jedes Teammitglied besitzt dieselbe Kondition. Wird das Team auf Teamebene betrachtet, ist es möglich Aussagen über das Team zu treffen. 
Die \textit{Abbildung \ref{DifferentLevels}} zeigt die Hierarchie der verschiedenen Ebenen.

\begin{figure}[H]
		\begin{footnotesize}
		\centering
			\includegraphics[width=0.5\linewidth]{Abbildungen/DifferentLevels.JPG}	
			\caption[Die Hierarchieebenen]{Die Hierarchie der Individualebene, Konditionsebene und Teamebene.}
			\label{DifferentLevels}
		\end{footnotesize}
	\end{figure}

\paragraph{Generelles Vertrauen}
Es wird der Zusammenhang zwischen dem \textit{generellen Vertrauen} und dem \textit{kognitiven Vertrauen} auf \textit{Konditionsebene} analysiert.
Es wird auf \textit{Teamebene} analysiert, ob ein Zusammenhang zwischen dem \textit{generellen Vertrauen} und der \textit{Teameffektivität} besteht.

\paragraph{Kognitives Vertrauen}
Es wird der Zusammenhang zwischen dem \textit{gebildeten kognitiven Vertrauen} und der \textit{Teameffektivität} auf \textit{Teamebene} analysiert.

\paragraph{Avatar-Darstellung}
Es wird der \textit{Unterschied} zwischen dem \textit{gebildeten kognitiven Vertrauen} bei unterschiedlichen Avatar-Konditionen auf \textit{Konditionsebene} analysiert.
Es wird der \textit{Unterschied} \textit{der Teameffektivität} bei unterschiedlichen Avatar-Konditionen auf \textit{Teamebene} analysiert.

Anhand des im Folgenden grafisch dargestellten Frameworks (siehe \textit{Abbildung \ref{Versuchshypothesen}}) wurden Hypothesen entwickelt, die im nächsten Kapitel genauer definiert und erklärt werden.

\begin{figure}[H]
		\begin{footnotesize}
			\includegraphics[width=\linewidth]{Abbildungen/Versuchshypothesen_02.JPG}		
			\caption[Das Framework der Versuchshypothesen]{Dieses Framework verdeutlicht, wie die Hypothesen zusammenhängen.}
			\label{Versuchshypothesen}
		\end{footnotesize}
	\end{figure}	

Es wurden folgende Hypothesen aufgestellt :

\textbf{H1$_{1}$}: Die Mittelwerte der erzielten \textit{kognitiven Vertrauenswerte} unterscheiden sich bei den Konditionen IK und NIK signifikant voneinander.

\textbf{H2$_{1}$}: Je höher der erzielte \textit{generelle Vertrauenswert} einer Person ist, desto höher ist der erzielte \textit{kognitive Vertrauenswert} einer Person.

\textbf{H3$_{1}$}: Der Zusammenhang zwischen dem \textit{kognitiven Vertrauenswert von Teams} und der \textit{Teameffektivität von Teams} mit der Kondition IK ist stärker als der von Teams mit der Kondition NIK.

\textbf{H4$_{1}$}: Die Mittelwerte der \textit{Teameffektivität} unterscheiden sich bei den Konditionen IK und NIK signifikant voneinander.

\textbf{H5$_{1}$}: Der Zusammenhang zwischen dem \textit{generellen Vertrauenswert eines Teams} und der \textit{Teameffektivität eines Teams} mit der Kondition IK ist stärker als der von Teams mit der Kondition NIK.

\begin{itemize}
\item \textbf{Generelles Vertrauen} Das generelle Vertrauen bezieht sich in dieser Studie darauf, wie sehr die Teilnehmer dazu neigen, anderen Personen einen Vertrauensvorschuss zu gewähren. \citep[S. 30]{mcallister1995affect}.
\item \textbf{Kognitives Vertrauen} Das \textit{kognitive Vertrauen} bezieht sich auf die \textit{Überzeugung in die Fähigkeiten oder in die Zuverlässigkeit eines anderen} \citep[S. 30]{mcallister1995affect}.
\item \textbf{Teameffektivität} Die \textit{Teameffektivität} wird anhand der Anzahl der abgeschlossenen Runden im Team gemessen.
\end{itemize}

\subsection{MEASURING METHODS}
Der General-Trust-Scale ($\alpha =,91$) \citep{couch1996assessment} wurde eingesetzt, um den generellen Vertrauenswert der einzelnen Teilnehmer zu messen. 

Der Cognitive-Trust-Scale ($\alpha =,91$) Fragebogen ist ein Auszug des von McAllister et al. \citep[S. 37]{mcallister1995affect} entwickelten Fragebogens. Er überprüft, wie viel \textit{kognitives Vertrauen} die Teilnehmer während des Versuchs aufbauen.

Gonzales-Rom et al. \citep[S. 1049]{gonzalez2014climate} entwickelten 2004 einen Fragebogen, um die \textit{Qualität von Teamkommunikation} ($\alpha =,76$) zu messen.

Gibson et al. \citep[S. 469]{gibson2003team} entwickelten 2003 einen Fragebogen, der die \textit{wahrgenommene Teameffektivität} ($\alpha =,62-,88$) misst. In dieser Studie wurde ein Auszug des Fragebogens verwendet. Er misst das \textit{subjektive Ausmaß der wahrgenommenen Teameffektivität}.

Der NASA-TLX ($\alpha =,84$) erfragt die allgemeine Belastung der Probanden, die sie während des Experiments empfunden haben. 

Der IPQ ($\alpha =,85$) dient zur \textit{Messung des Präsenz-Gefühls} in einer virtuellen Umgebung. Er misst, inwieweit sich der Nutzer in der virtuellen Umgebung anwesend fühlt, inwieweit der Nutzer seine Aufmerksamkeit der virtuellen Umgebung schenkt und wie real die virtuelle Umgebung dem Nutzer erschien. 

Mithilfe des \textit{Co-Präsenz-Fragebogens} können die \textit{selbst gemeldete Co-Präsenz} ($\alpha =,78$), die \textit{wahrgenommene Präsenz des anderen} ($\alpha =,90$), die \textit{Telepräsenz} ($\alpha =,88$) sowie die \textit{soziale Präsenz} ($\alpha =,82$) ermittelt werden.
\subsection{PARTICIPANTS}

Die Teilnehmer werden über zwei Wege akquiriert. Zum einen werden im Bekanntenkreis Personen angesprochen, denen die notwendige Hardware zur Verfügung gestellt wird. Zum anderen werden in verschiedenen Foren (z.B. VRForum.de, Computerbase.de, Hardwareluxx.de, etc.) in Form eines extra dafür angelegten Threads Teilnehmer gesucht, die an der Studie teilnehmen wollen. Weiterhin werden Teilnehmer mithilfe verschiedener sozialer Netzwerke mit einem Bezug zu VR sowie zufälliger WhatsApp-Chatgruppen mit 50 oder mehr Mitgliedern akquiriert. Da der gesamte Versuch, die Fragebögen sowie das Erklärvideo auf deutscher Sprache erstellt wurde, findet die Teilnehmerfindung nur im deutschsprachigen Raum statt.

Um an dem Experiment teilnehmen zu können, benötigen die Teilnehmer ein in vollem Umfang funktionierendes SteamVR, Windows-Mixed-Reality oder ein Oculus Rift/Rift-S Head-Mounted-Display mit kompatiblen Controllern sowie einen leistungsstarken VR-fähigen PC. Der Spectator, der das Experiment von außerhalb steuert und verwaltet, nutzt einen PC, auf dem die Anwendung ohne Head-Mounted-Display ausführbar ist.

\subsection{PROCEDURE}
Zunächst wird jedem Team entweder die Kondition IK oder NIK zugeordnet. Es gab fünf Teams mit der Kondition IK sowie fünf Teams mit der Kondition NIK.
Es werden jeweils drei Personen in einem Zeitslot untergebracht, um ein Team zu bilden. Insgesamt nehmen somit drei Personen an einem Versuch zur selben Zeit mit derselben Kondition teil. Die Teilnehmer werden sich untereinander \textit{nicht} Face-to-Face vorgestellt und sehen sich während des gesamten Experiments nur als Repräsentation eines Avatars im Shared-Virtual-Environment . 
Ein Zeitslot wird auf 25 Minuten festgelegt und teilt sich auf in
		\begin{itemize}
			\item 5 Minuten Pre-Questionnaire,
			\item 5 Minuten Videoerklärung,
			\item 10 Minuten Versuchsdurchführung,
			\item 15 Minuten Post-Questionnaire.
		\end{itemize}
Jeder Teilnehmer bekommt zu Beginn seines Zeitslots einen Pre-Questionnaire ausgehändigt, den er selbstständig ausfüllt. Alle Teilnehmer schauen sich anschließend ein Erklärvideo über das Experiment an, in dem alle relevanten Mechaniken und Funktionsweisen sowie der grobe Spielablauf erklärt werden. Durch das Erklärvideo wird sichergestellt, dass alle teilnehmende Person denselben Informationsgehalt über die Art und Weise des Ablaufs des Experiments besitzen. Alle Mitglieder eines Teams starten dadurch mit einem einheitlichen Wissensstand. Nachdem alle Personen die Videoerklärung angeschaut haben, beginnt das Experiment. Dazu starten die jeweiligen Teilnehmer die Anwendung. Es wird sich automatisch mit dem Online-Server des Shared-Virtual-Environment verbunden. Die Teilnehmer haben nun 10 Minuten Zeit, möglichst viele Runden im Team zu absolvieren. Am Ende der Versuchsdurchführung wird ein Post-Questionnaire ausgeteilt, den die Teilnehmer ausfüllen müssen. Die maximale Versuchsdauer nach Start der Anwendung beträgt exakt 10 Minuten (600 Sekunden) und es können maximal 15 Runden absolviert werden. Die Runden werden dabei inkrementell schwieriger, da in jeder dritten Runde jeweils ein Symbol in den Pool der zu erratenden Symbole hinzukommt.
 \textit{Abbildung \ref{RoundDifficulty}} zeigt die steigenden Schwierigkeitswerte, anhand derer in diesem Experiment die \textit{Teameffektivität} gemessen wurde.

\begin{figure}[H]
		\begin{footnotesize}
		\centering
			\includegraphics[width=\linewidth]{Abbildungen/RoundDifficulty.JPG}	
			\caption[Der Schwierigkeitsgrad der Runden]{Die steigende Schwierigkeit der zu erratenden Symbole der einzelnen Runden. In Runde 1-3 muss ein Symbol erraten werden, in Runde 3-6 zwei Symbole usw.}
			\label{RoundDifficulty}
		\end{footnotesize}
	\end{figure}

Zu Beginn jeder neuen Runde können die Spieler die Zuteilung der Farben sehen und dadurch den Startspieler identifizieren. Dieser ist schwarz markiert und hat die Aufgabe, seinen Mitspielern die für ihn farblich gekennzeichneten Symbole zu erklären. Seine Mitspieler müssen die ihnen zugeteilten Symbole identifizieren und an ihrem Podest einloggen. Das Ziel ist, so viele Symbole wie möglich individuell korrekt zu erkennen, um dadurch gemeinsam in höhere Runden aufzusteigen.

Die Symbole auf dem Podest des schwarz markierten Spielers sind entweder durch die Farbe Grün, Rot oder Grün-Rot gekennzeichnet. Auf den Podesten der Mitspieler befinden sich ebenfalls Symbole, welche jedoch zufällig angeordnet sind und keine farblichen Markierungen haben. Der schwarz markierte Spieler versucht nun, mittels Hand- und Armbewegung, den rot und grün markierten Mitspielern die Symbole, die in der jeweiligen Spielerfarbe vor ihm markiert sind, zu erklären. Meint der gerade angesprochene Mitspieler ein Symbol erkannt zu haben, loggt dieser das Symbol durch das Herunterdrücken des passenden Knopfes an seinem Podest ein. Hat sich ein Spieler während des Einloggens der Symbole verklickt oder möchte seine Angabe ändern, muss das Symbol durch erneutes Herunterdrücken ausgeloggt werden. Anschließend kann es erneut eingeloggt werden. 

Werden alle gekennzeichneten Symbole vom roten und grünen Spieler erkannt und eingeloggt, erscheint eine leuchtend grüne Kugel, die das Ende einer Runde anzeigt. Erscheint diese grüne Kugel nicht, ist noch ein Symbol falsch eingeloggt und der schwarz markierte Spieler muss noch einmal versuchen, die korrekten Symbole den jeweiligen Mitspielern aufzuzeigen. 
In der nächsten Runde wird ein anderer Spieler eindeutig mit Schwarz, Rot oder Grün markiert.
In der folgenden Runde erhält jeder Spieler wieder eine andere der drei Farben. Mit steigender Anzahl an erfolgreich bestandenen Runden müssen immer mehr Symbole richtig erkannt werden.
\textit{Abbildung \ref{AvatareImEinsatz}} zeigt beide Avatar-Konditionen IK (a) und NIK (b) während der Versuchsdurchführung im Spectatorview.
	
\begin{figure}[h]
  \centering
  \subfloat[][]{\includegraphics[width=0.45\linewidth]{Abbildungen/Podeste_IK_Avatars.jpg}}
  \qquad
  \subfloat[][]{\includegraphics[width=0.45\linewidth]{Abbildungen/Podeste_Non_IK_Avatars.jpg}}
  \caption[Die Avatare und der Spectatorview]{Avatar-Konditionen IK (a) und NIK (b) während der Versuchsdurchführung im Spectatorview.}
  \label{AvatareImEinsatz}
\end{figure}

\subsection{IMPLEMENTATION}

Um den Versuch durchzuführen, wurde ein Shared-Virtual-Environment entwickelt, in dem sich die drei Teammitglieder gegenseitig als Avatare sehen und miteinander interagieren können. Das Shared-Virtual-Environment ist mit Unity 2019.4.3f1 und der HD-Render-Pipeline entwickelt worden. Um die Echtzeitkommunikation zwischen den einzelnen Clients zu gewährleisten, wurde das Multiplayer-Framework \textit{Normcore v2.0}\footnote{www.Normcore.io} genutzt.

\subsubsection{AVATAR}
	\paragraph{IK-Avatar}
Der Avatar hat keine Augen, Mund, Haare oder Beine. Lediglich der Oberkörper, der Kopf sowie die Arme sind zu erkennen. Dieser Avatar besitzt somit keine Beine und schwebt mit dem Torso über dem Boden.
Die Handbewegungen, die Unterarmbewegungen, die Oberarmbewegungen sowie die Kopf- und Torsorotation des Avatar IK werden invers-kinematisch dargestellt. Die Positions- und Rotationsdaten für den Kopf und den Oberkörper werden über das Head-Mounted-Display gewonnen. Die Positions- und Rotationsdaten der Arme werden über die beiden Controller gewonnen.

		\paragraph{Non-IK-Avatar}
Der NIK-Avatar besteht aus einer Kugel mit Mund sowie einer Repräsentation der linken und der rechten Hand. Durch den fehlenden Oberkörper und die fehlenden Beine, schwebt der Avatar über dem Boden. Der Kopf ist frei beweglich und unabhängig von den Händen. Der Mund des Avatars bewegt sich nicht, sondern dient am Kopf als Orientierungspunkt. Dadurch kann ausgemacht werden, in welche Richtung der Kopf des Avatars gedreht ist. Die Positions- und Rotationsdaten für den Kopf werden über das Head-Mounted-Display gewonnen. Die Positions- und Rotationsdaten für die Hände werden über die beiden Controller gewonnen.\\
\textit{Abbildung \ref{AvatareImEinsatz}} zeigt beide Avatar-Konditionen IK (a) und NIK (b) während der Versuchsdurchführung im Spectatorview.

	\begin{figure}[H]
		\begin{footnotesize}
		\centering
			\includegraphics[width=\linewidth]{Abbildungen/Avatars.JPG}	
			\caption[Die verwendeten Avatare]{Die verwendeten Avatare : Links: IK-Avatar und rechts: Non-IK-Avatar.}
			\label{AvatareAussehen}
		\end{footnotesize}
	\end{figure}

\subsubsection{THE ENVIRONMENT}

\paragraph{USER TEST}

\paragraph{THE TEST ENVIRONMENT}

\section{STATISTICAL RESULTS}

Insgesamt nahmen 30 Personen an der Studie teil, sodass es insgesamt zehn Teams gab. 19 (63,3\%) Personen waren männlich und 11 (36,7\%) weiblich. Im Schnitt waren die Teilnehmer 30 Jahre alt $(\bar{x} = 30,13)$, wobei das 2. Quartil bei $28$ Jahren liegt.

\subsection{ANALYSIS HYPOTHESIS 1}
Auf \textit{Konditionsebene} wird analysiert, ob die Teilnehmer aufgrund der unterschiedlichen Avatar-Konditionen unterschiedliches \textit{kognitives Vertrauen} bilden. Dazu wird der Mann-Whitney-U-Test durchgeführt.
Die Verteilungsformen der beiden Konditionen unterscheiden sich laut Kolmogorov-Smirnov ($p =,925 > \alpha =,05$) \textbf{nicht signifikant} voneinander. Es gab laut Mann- Whitney-U-Test \textbf{einen signifikanten Unterschied} zwischen den Mittelwerten der \textit{kognitiven Vertrauenswerte} der Avatar-Konditionen IK $(\bar{x} = 4,188)$ und NIK $(\bar{x} = 4,622)$ mit $U = 64,000; Z = -2,029; p =,042 < \alpha =,05; r =-,370$.

\subsection{ANALYSIS HYPOTHESIS 2}
Es wird analysiert, ob ein Zusammenhang zwischen dem \textit{generellen Vertrauenswert} und dem \textit{kognitiven Vertrauenswert} auf Konditionsebene besteht.

\paragraph{Spearman-Korrelationsanalyse}
Es ist eine positive starke Korrelation (vgl. \citep{cohen2013statistical}) mit dem Spearman-Korrelationskoeffizient $r =,406$ zwischen dem \textit{generellen Vertrauenswert} der Kondition IK und dem \textit{kognitiven Vertrauenswert} der Kondition IK vorhanden. Die Spearman-Korrelation ist \textbf{nicht signifikant} ($p =,134 > \alpha =,05$).

Weiterhin liegt eine positive Korrelation schwacher Stärke (vgl. \citep{cohen2013statistical}) mit dem Spearman-Korrelationskoeffizient $r = ,045$ zwischen dem \textit{generellen Vertrauenswert} der Kondition \newline NIK und dem \textit{kognitiven Vertrauenswert} der Kondition NIK vor. Die Spearman-Korrelation ist \textbf{nicht signifikant} ($p = ,872 > \alpha = ,05$) (siehe \textit{Abbildung \ref{H2_Korrelation_Auswertung}}).

\paragraph{Fisher-Z-Transformation für unabhängige Stichproben}
Um herauszufinden, ob sich beide Korrelationen signifikant voneinander unterscheiden, wird eine Fisher-Z-Transformation für unabhängige Stichproben durchgeführt.
Die beiden Korrelationen unterscheiden sich laut Fisher-Z-Wert für unabhängige Stichproben ($Z =,954$) \textbf{signifikant} voneinander ($p =,017 < \alpha =,05$).

\subsection{ANALYSIS HYPOTHESIS 3}
Es wird der Zusammenhang zwischen dem \textit{kognitiven Vertrauenswert von Teams} und der \textit{Teameffektivität von Teams} auf Teamebene analysiert.
In einem Versuchsdurchlauf besitzt jedes der 3 Teammitglieder die gleiche Avatar-Kondition. Die Anzahl der abgeschlossenen Runden jedes einzelnen Teams bildet die \textit{Teameffektivität des Teams}. 
Für jedes Team wird zudem ein gemeinsamer kognitiver Vertrauenswert berechnet. Dieser sagt aus, wie viel \textit{kognitives Vertrauen} das gesamte Team gebildet hat und ergibt sich aus der Summe der \textit{kognitiven Vertrauenswerte} der einzelnen Personen eines Teams. Dieser wurde für jedes einzelne Team derselben Kondition zusammengefasst (siehe \textit{Tabelle \ref{TeamCogTabelle}})

\paragraph{Spearman-Korrelation}

Der Spearman-Korrelationskoeffizient zwischen den \textit{kognitiven Vertrauenswerten der Teams} mit der Kondition IK und der \textit{Teameffektivität der Teams} mit der Kondition IK beträgt $r =,205$ Diese Korrelation moderater Stärke ist positiv (vgl. \citep{cohen2013statistical}) und ist \textbf{nicht signifikant} ($p =,741 > \alpha = ,05$). \\
Weiterhin ist eine starke positive Korrelation (vgl. \citep{cohen2013statistical}) mit dem Spearman- Korrelationskoeffizient $r =,975$ zwischen den \textit{kognitiven Vertrauenswerten der Teams} mit der Kondition \newline NIK und der \textit{Teameffektivität der Teams} mit der Kondition NIK vorhanden. Diese Spearman-Korrelation ist \textbf{signifikant} ($p =,005 < \alpha = ,05$) (siehe \textit{Abbildung \ref{H3_Korrelation_Auswertung}}).

\paragraph{Fisher-Z-Transformation für unabhängige Stichproben}
Obwohl die Stichprobe mit $N=5$ sehr klein ist, wird eine Fisher-Z-Transformation für unabhängige Stichproben durchgeführt, um zu überprüfen, ob die beiden Spearman-Korrelationen sich signifikant voneinander unterscheiden.
Es zeigt sich, dass sie sich laut Fisher-Z-Wert für unabhängige Stichproben ($Z=-1.977$), \textbf{signifikant} voneinander unterscheiden ($p =,024 < \alpha = ,05$).

\subsection{ANALYSIS HYPOTHESIS 4}
Auf \textit{Konditionsebene} wird analysiert, ob die Teilnehmer aufgrund unterschiedlicher Avatar-Konditionen eine unterschiedliche \textit{Teameffektivität} besitzen.
\paragraph{Mann-Whitney-U-Test}
Die Verteilungen der \textit{Teameffektivitätswerte} der Kondition IK und der \textit{Teameffektivitätswerte} der Kondition NIK unterscheiden sich laut Kolmogorov - Smirnov ($p =,181 > \alpha = ,05$) \textbf{nicht signifikant} voneinander. Es gibt \textbf{keinen signifikanten Unterschied} zwischen den Mittelwerten der \textit{Teameffektivität} der Kondition IK $(\bar{x} = 9,000$) und der Kondition NIK $(\bar{x} = 9,000)$ mit $U = 103,500; Z = -,377; p =,706 > \alpha = ,05; r = -,060$. \\

\subsection{ANALYSIS HYPOTHESIS 5}
Es wird der Zusammenhang zwischen dem \textit{generellen Vertrauenswert von Teams} und der \textit{Teameffektivität von Teams} auf Teamebene analysiert.

\paragraph{Spearman-Korrelation}
Es ist eine starke negative Korrelation (vgl. \citep{cohen2013statistical}) mit dem Spearman-Korrelationskoeffizient $r = -,800$ zwischen den \textit{generellen Vertrauenswerten der Teams} mit der Kondition IK und den \textit{Teameffektivitätswerten der Teams} mit der Kondition \newline IK vorhanden. Die Spearman-Korrelation ist \textbf{nicht signifikant} ($p =,104 > \alpha = ,05$).\\
Weiterhin ist eine starke negative Korrelation (vgl. \citep{cohen2013statistical}) mit dem Spearman- Korrelationskoeffizient $r = -,667$ zwischen den \textit{generellen Vertrauenswerten der Teams} mit der Kondition NIK und den \textit{Teameffektivitätswerten der Teams} mit der Kondition \newline NIK vorhanden. Die Spearman-Korrelation ist \textbf{nicht signifikant} ($p =,219 > \alpha = ,05$) (siehe \textit{Abbildung \ref{H5_Korrelation_Auswertung}}). 

\paragraph{Fisher-Z-Transformation für unabhängige Stichproben}
Es wird eine Fisher-Z-Transformation für unabhängige Stichproben durchgeführt.
Die beiden Spearman-Korrelationen unterscheiden sich laut Fisher-Z-Wert für unabhängige Stichproben ($Z=-,293$) \textbf{nicht signifikant} voneinander ($p =,385 > \alpha = ,05$).

\subsection{ANALYSIS OF THE SUBJECTIVE DATA}
Anhand eines Mann-Whitney-U-Test konnte bei der Kategorie \textit{Teamkommunikation} ein \textit{signifikanter Unterschied} zwischen den Mittelwerten beider Avatar-Konditionen festgestellt werden. Der Mittelwert der \textit{Teamkommunikation} der Kondition IK beträgt $\bar{x} = 4.013$, während der Mittelwert der Teamkommunikation der Kondition NIK $\bar{x} = 4,48$ beträgt. Somit ist die subjektiv empfundene Teamkommunikation bei der Kondition NIK höher als bei der Kondition IK und bei beiden Konditionen ist die Tendenz einer hohen Teamkommunikation ersichtlich ($\bar{x} = 4,013$; $\bar{x} = 4,48$ $ > 3$).

Das von allen Teilnehmern durchschnittlich angegebene \textit{Gefühl der Präsenz} kann mit dem Wert $\bar{x} = 4,446$ als stark interpretiert werden ($\bar{x} > 3,5$). Die \textit{Telepräsenz} ($\bar{x} = 5,286$) sowie die \textit{soziale Präsenz} ($\bar{x} > 6,409$) werden etwas stärker als durchschnittlich empfunden und deuten dadurch auf ein höheres Gefühl der Anwesenheit sowie auf ein stärkeres Gefühl der Nähe zwischen den Teilnehmern hin. Auch die \textit{selbst wahrgenommene Co-Präsenz} sowie die \textit{wahrgenommene Co-Präsenz des anderen} liegen mit den Werten $\bar{x} = 3,827$ und $\bar{x} = 3,877$ über dem Mittelwert der Antwortmöglichkeiten ($\bar{x} > 2,5$) und weisen somit eine Tendenz zur starken \textit{Co-Präsenz} auf.

Daneben liegt die durchschnittlich \textit{wahrgenommene Teameffektivität} mit dem Wert $\bar{x} > 4,886$ über dem Wert 3,5 und lässt sich somit im Bereich der als eher stark empfundenen Teameffektivität ($3,5 - 7$) verorten.
Die Werte des NASA-TLX (Allg. Anstrengung) ($\bar{x} < 7$) zeigen, dass das VR Experiment als mittelmäßig bis wenig anstrengend wahrgenommen wurde ($\bar{x} < 11$ ).

Auf Konditionsebene ist eine starke positive Korrelation (vgl. \citep{cohen2013statistical}) mit dem Spearman- Korrelationskoeffizient $r =,869$ zwischen den \textit{kognitiven Vertrauenswerten} und \textit{der wahrgenommenen Teameffektivität} der Kondition NIK zu erkennen. Die Spearman-Korrelation ist \textbf{signifikant} ($p =,000 < \alpha = ,05$) und ist in \textit{Abbildung \ref{SubSig1}} dargestellt.

Weiterhin ist eine positive Korrelation starken Effektes (vgl. \citep{cohen2013statistical}) mit dem Spearman- Korrelationskoeffizient $r =,676$ zwischen den \textit{kognitiven Vertrauenswerten} und der \textit{Team-Kommunikation} der Kondition NIK auf Konditionsebene zu erkennen. Die Spearman-Korrelation ist \textbf{signifikant} ($p =,006 < \alpha = ,05$) (siehe \textit{Abbildung \ref{SubSig2}})

\section{RESULTS}
Anhand der statistischen Analyse lässt sich feststellen, dass unterschiedliche Avatar-Konditionen einen \textit{signifikanten} Einfluss auf das gebildete \textit{kognitive Vertrauen} besitzen. Dabei haben Teilnehmer mit der Kondition NIK mehr \textit{kognitives Vertrauen gebildet} als Teilnehmer mit der Kondition NIK. Es wurde zudem ein \textit{signifikanter} Zusammenhang zwischen dem gebildeten \textit{kognitiven Vertrauen im Team} und der \textit{Teameffektivität} bei der Kondition NIK festgestellt, jedoch unterscheidet sich die \textit{Teameffektivität} \textit{nicht signifikant} aufgrund unterschiedlicher Avatar-Konditionen voneinander.
Weiterhin gibt es \textit{keine Anzeichen} dafür, dass es einen \textit{signifikanten} Zusammenhang zwischen dem \textit{generellen Vertrauen} oder dem \textit{generellen Vertrauen im Team} und der \textit{Teameffektivität} gibt.
Es wurde ein \textit{signifikanter} Unterschied der Mittelwerte der Teamkommunikation auf Konditionsebene festgestellt. Zudem kann festgehalten werden, dass sich ein erhöhtes Präsenz-Gefühl (Präsenz, Telepräsenz, selbst wahrgenommene Co-Präsenz, wahrgenommene Co-Präsenz des anderen und Soziale-Präsenz) auf Konditionsebene während der Versuchsdurchführung gebildet hat. Die allgemeine Belastung lag unter dem Durchschnitt der möglichen anzugebenden Werte und die wahrgenommene Teameffektivität war überdurchschnittlich hoch.
Weiterhin gibt es einen \textit{signifikant} positiven Zusammenhang zwischen dem \textit{kognitiven Vertrauen} und der \textit{wahrgenommenen Teameffektivität} und der \textit{Teamkommunikation} auf Konditionsebene bei der Kondition NIK.

\section{DISCUSSION}
\subsection{Building trust through the avatar conditions}
Die Ergebnisse des Experiments widersprechen der Untersuchung von Bente et al. (vgl. Kapitel \ref{AvatarTrust}), die vermuten ließ, dass bei der Kondition IK ein größerer \textit{kognitiver Vertrauensaufbau} stattfindet als bei der Kondition NIK (Hypothese 1). Es ist laut der statistischen Auswertung genau das Gegenteil der Fall, denn die Teilnehmer mit der Kondition NIK erzielen im Durchschnitt einen signifikant höheren kognitiven Vertrauenswert.
Ein Grund für dieses Ergebnis könnte sein, dass der verwendete IK-Avatar vom \textit{Uncanny Valley Effekt} betroffen ist.
Weiterhin könnte die in diesem Experiment verwendete Inverse-Kinematik, die eventuell zu unnatürlichen Bewegungsabläufen führte, als fremdartig empfunden worden sein. Auch dadurch falsch interpretierte Gestikulation kann zu einem geringeren Vertrauensaufbau in das jeweilige Gegenüber geführt haben.

\subsection{Building trust through the general trust}
Dass kein signifikanter Zusammenhang zwischen dem \textit{kognitiven Vertrauen} und dem \textit{generellen Vertrauen} bei unterschiedlichen Avatar- Konditionen besteht (Hypothese 2), kann auch als Vorteil gesehen werden.
So lässt sich vermuten, dass es während einer kurzfristigen Zusammenarbeit in einem Shared-Virtual-Environment nicht von Relevanz ist, wie hoch oder niedrig das \textit{generelle Vertrauen} einer Person ist. 
Da nur die unterschiedlichen Avatar-Konditionen einen signifikanten Einfluss auf die Bildung des kognitiven Vertrauens besitzen, kann davon ausgegangen werden, dass das \textit{generelle Vertrauen} während einer Kennenlernphase eines virtuellen Teams keine größere Rolle spielt und isoliert betrachtet werden kann. Natürlich sollte in längerfristig bestehenden virtuellen Teams darauf geachtet werden, dass sich das generelle Vertrauen durch den Einfluss der kognitiven und der affektiven Komponente (vgl. Kapitel \ref{Vertrauen als Zustand-Label}) weiterhin zum Positiven entwickelt, um die Teameffektivität zu steigern (vgl. Kapitel \ref{Vertrauensforschung}).

\subsection{Trust in the team and team effectiveness}
Die Hypothese 3, in der vermutet wurde, dass der Avatar mit der Kondition IK mehr \textit{kognitives Vertrauen} aufbaut, muss ebenfalls abgelehnt werden. Jedoch wurde während der Analyse dieser Hypothese festgestellt, dass es einen \textit{signifikanten Zusammenhang} zwischen dem gebildeten \textit{kognitiven Vertrauen} und der \textit{Teameffektivität} der Kondition NIK gibt. Da die Hypothese 4 nicht angenommen werden kann, jedoch ein signifikanter Zusammenhang mit der Kondition NIK bei der Analyse der Hypothese 3 festgestellt wurde, muss vermutet werden, dass die Ergebnisse der Hypothese 3 und Hypothese 4 entweder zufälliger Natur sind oder die Messung der Teameffektivität verbessert werden muss.
Die kleine Stichprobengröße der Studie könnte zudem eine Ursache dafür sein, dass die Ergebnisse von Hypothese 3 und Hypothese 4 keine eindeutigen Ergebnisse liefern. Bei einer deutlich größeren Stichprobengröße könnte bei einer größeren Varianz der \textit{Teameffektivitätswerte} gegebenenfalls ein signifikanter Unterschied oder ein eindeutiger signifikanter Zusammenhang mit dem \textit{kognitiven Vertrauen} und der \textit{Teameffektivitätswerte}, festgestellt werden.

Die Teameffektivität könnte auch vom Bildungsstand der Teilnehmer beeinflusst worden sein. Da der Großteil der Teilnehmer (80\%) ein abgeschlossenen Studium besaßen, kann dies die \textit{Teameffektivitätswerte} beider Konditionen angenähert haben, weshalb beide Teams die Selbe durchschnittliche \textit{Teameffektivität} besitzen. Weiterhin zeigt sich aufgrund des hohen Bildungsstandes der Teams, dass die Gruppe bei diesem Attribut keine Homogenität besitzt. Jedoch bilden die Teams dieser Studie reale Teams besser als wenn die Teams unterschiedliche Bildungsniveaus besäßen. Teams in der Realität bestehen häufig aus Mitgliedern mit Selben Bildungsniveau.

\subsection{LIMITATIONS}
Als technische Limitation dieser Arbeit kann die eingesetzte Technik der Gestikulation aufgeführt werden. Die unterstützten Head-Mounted-Displays boten keine Möglichkeit, die Finger oder die gesamten Hände in der Virtual-Reality abzubilden. Die Verständigung innerhalb des Shared-Virtual-Environment kann durch den Einsatz von Finger- und Handtracking intensiviert werden, da verschiedene Gesten genauer wiedergegeben werden können. Das Finger- und Handtracking ist beispielsweise durch eine Oculus Quest oder Oculus Quest 2 möglich. Ein weiterer Aspekt, um die Menschenähnlichkeit und den Realismus bei der Avatar-Kondition IK zu erhöhen, wäre der Einsatz von Mimik.
Eine weitere Limitierung dieser Untersuchung war, dass die Teilnehmer bevor das Experiment startete wussten, dass diese in einem Shared-Virtual-Environment mit realen Personen agieren würden. 

Darüber hinaus ist der Avatar mit der Kondition IK in dieser Studie nicht auf Menschenähnlichkeit, Realismus oder Ähnliches überprüft worden. 
Da sich diese Untersuchung hauptsächlich mit dem gebildeten \textit{kognitiven Vertrauen} und der daraus resultierenden \textit{Teameffektivität} beschäftigt, können keine Aussagen über den Einfluss verschiedener Avatar-Konditionen auf das \textit{affektive Vertrauen} einer Person getroffen werden. Um diesen Einfluss zu messen, müssten längerfristige Forschungen durchgeführt werden.
\section{CONCLUSION AND FUTURE WORK}
Shared-Virtual-Environments entwickeln sich aktuell sehr rasant. Die Coronapandemie hat gezeigt, dass virtuelle Kollaborationsmaßnahmen einen großen Einfluss auf Unternehmen weltweit haben. Es ist mehr Forschung darüber nötig, wie Teams effektiv in einem Shared-Virtual-Environment zusammenarbeiten können.

Es könnte nicht nur untersucht werden, welche Art eines Avatars in einem Shared-Virtual-Environment mehr Vertrauen schafft oder mehr \textit{Teameffektivität} erzeugt, sondern auch, wie die eingesetzte Sprache, die Mimik, die Gestik, die Größe, das Geschlecht oder der vorherige Bekanntheitsgrad der Personen sich auf das Vertrauen und die \textit{Teameffektivität} auswirkt.
Weiterhin könnte untersucht werden, wie die Art und Dauer der Nutzung des Head-Mounted-Displays, während ein Team zusammenarbeitet, sich auf das Vertrauen ins Team und die Teameffektivität auswirkt.

Eine Weiterführung dieser Studie könnte untersuchen, in welchem Maß sich der \textit{kognitive Vertrauensaufbau im Team} ändert, je nachdem, ob die Teilnehmer wissen, dass sie mit Menschen zusammenarbeiten oder nicht. Darüber hinaus wäre es interessant zu untersuchen, wie sehr sich der Unterschied zwischen einer verbalen und einer nonverbalen Kommunikation auf das gebildete Vertrauen im Team in einem Shared-Virtual-Environment auswirkt.  
Weiterhin könnten ähnliche Studien durchgeführt werden, bei denen die einzelnen Runden der Kollaborationsaufgabe schneller hintereinander ausgeführt werden oder die Avatare ein anderes Aussehen besitzen.

Es konnte ein signifikanter Unterschied beim gebildeten \textit{kognitiven Vertrauen} zwischen den Avatar-Konditionen festgestellt werden, wobei sich mehr \textit{kognitives Vertrauen} bei den nicht- menschenähnlichen Avataren bildete. Es konnte jedoch kein statistisch signifikanter Unterschied zwischen den Avatar-Konditionen und der \textit{Teameffektivität} festgestellt werden. Weiterhin zeigen die Ergebnisse keinen signifikanten Zusammenhang zwischen dem \textit{generellen Vertrauen} und dem gebildeten \textit{kognitiven Vertrauen} einer Person. Zwischen dem \textit{kognitiven Vertrauen} und der \textit{Effektivität eines Teams} konnte ein signifikanter Zusammenhang bei der Kondition IK festgestellt werden.

In einem virtuellen Team besitzt die Avatar-Kondition laut dieser Studie somit keinen eindeutigen Einfluss auf die \textit{Teameffektivität}. Es kann jedoch sinnvoll sein, den Avatar nicht zu menschenähnlich zu gestalten, um mehr \textit{kognitives Vertrauen} zu bilden.
%Es empfiehlt sich, diese Studie mit einer großen Stichprobe zu replizieren und die Einflüsse anders gestalteter Avatare zu untersuchen. 
Die Arbeit in einem virtuellen Team muss folglich nicht mit aufwendig gestalteten Avataren unterstützt werden. So können beispielsweise Unternehmen, die mit virtuellen Teams in einem Shared-Virtual-Environment arbeiten wollen, auf simple Avatar-Modelle zurückgreifen, um den Vertrauensaufbau im Team zu unterstützen.

%%
%% The next two lines define the bibliography style to be used, and
%% the bibliography file.
%\citestyle{acmauthoryear}
\bibliographystyle{ACM-Reference-Format}

\bibliography{sample-base}

%%
%% If your work has an appendix, this is the place to put it.
\appendix


\end{document}
\endinput
%%
%% End of file `sample-sigconf.tex'.
