%TODO : Mein eigenes Verwendetes System beschreiben, genauer, welche verschiedenen Interaktionen hatte ich, etc. Warum habe ich was gemacht, eventuell sogar mit UML Diagrammen arbeiten
%
%Auch weitere Hypoethesen in Sachen präsenz als zusätzliches Kapitel "Signifikanzanalyse" oder so mit hineinbringen. Nur die, die Signifikant sind.
%
%Analyse mit Signifikanzniveau von 10% anschauen, damit Ergebnisse schnelller Signifikant sind.
%
%Umbenennung von "VR" und "AVATARE" etc.
%
%Nicht alles so kleingliedrig
%
%Auf Verwendung von nicht parametrische Tests achten.
%
%Leseprobe senden!
\documentclass[a4paper,11pt]{article}%Schriftgröße

\usepackage{booktabs}
\usepackage{multirow}
\usepackage{rotating}
\newcommand\tabrotate[1]{\begin{turn}{90}\rlap{#1}\end{turn}}
\usepackage{varwidth}
\newcommand\tabvarwidth[2][3cm]{\begin{varwidth}[b]{#1}\centering #2\end{varwidth}}
\usepackage{tabularx}

\usepackage{amsmath}

\usepackage[T1]{fontenc} 
\usepackage[utf8]{inputenc}
\usepackage[ngerman]{babel}%Veröffentlichungssprache
\usepackage{graphicx}
\usepackage{ragged2e}
\usepackage[format=plain,justification=RaggedRight,singlelinecheck=false,font={small},labelsep=space]{caption}
\usepackage{float}
\usepackage{xcolor}	
\usepackage[a4paper]{geometry}
\usepackage{fancyhdr}
\usepackage[onehalfspacing]{setspace}%Zeilenabstand
\usepackage[onehalfspacing]{setspace}%Zeilenabstand
\usepackage{fancyhdr}
\usepackage[authoryear]{natbib}
\usepackage[colorlinks,pdfpagelabels,pdfstartview = FitH,bookmarksopen = true,bookmarksnumbered = true,linkcolor = black,urlcolor = black,plainpages = false,hypertexnames = false,citecolor = black] {hyperref}
\usepackage[printonlyused, withpage, smaller]{acronym}
\renewcommand{\\}{\vspace*{0.5\baselineskip} \newline}
\renewcommand*\MakeUppercase[1]{#1}	
\renewcommand{\familydefault}{\sfdefault}
\geometry{left=3.5cm,right=2.5cm,top=2.4cm,bottom=2cm}%Seitenränder
\pagestyle{fancy}
\renewcommand{\headrulewidth}{0pt}
\renewcommand{\footrulewidth}{0pt}
\fancyhead[R]{\footnotesize{\thepage}}
\fancyhead[L]{\footnotesize{\leftmark}}
\fancyfoot{}

\begin{document}
	\begin{titlepage}
		\begin{flushleft}
			\vspace*{-1cm}
			\includegraphics[scale=1]{Abbildungen/TH.png}\\
			\vspace*{1cm}
		\end{flushleft}
		
		\begin{huge}
			\noindent
			%Vergleich des Teambuildingerfolgs mittels eines nicht-vollkörpergetracktem- und vollkörpergetracktem Avatars in einer Immersive Virtual Reality anhand einer Escape-Room-Artigen Umgebung. \\
			
%			Vertrauensaufbau einer Teambuildingmaßnahme zwischen eines Hand- und Kopf getracktem Avatar und  eines Hand-, Kopf und Inverskinematisch-simuliertem Torsos getracktem Avatar in einem Shared Virtual Environment anhand einer Escape-Room-Artigen Umgebung. \\
			
%			Auswirkung des Vertrauens auf die Effizienz in einer Teambuildingmaßnahme anhand eines Hand- und Kopfgetracktem und einem Hand-, Kopf und Inverskinematisch-simuliertem Torsos getracktem Avatar ein einem Shared Virtual Environment. //
			
%			Auswirkung von Vertrauen zu Avataren auf die \mbox{Effizienz} in einer Teambuildingmaßnahme in ein einem Shared Virtual Environment.

%Auswirkung des Vertrauens einer Teambuildingmaßnahme anhand eines \flqq IK-Avatar\flqq und einem \flqq Non-IK-Avatar\frqq in einem Shared-Virtual-Environment. \\
%
%Auswirkung des Vertrauens auf die Effizienz einer Teambuildingmaßnahme anhand eines \flqq IK-Avatar\flqq und einem \flqq Non-IK-Avatar\frqq in einem Shared-Virtual-Environment. \\
%
%Welche Avatarkondition schafft in meinem Team mehr Vertrauen? Hat dieses geschaffene Vertrauen eine Auswirkung auf die Effizienz meiner Teambuildingmaßnahme? Welche Avatarkondition sollte ich also nutzen, um Vertrauen zu schaffen, sodass in einem Team ein effizienteres Arbeiten stattfinden kann?

%Auswirkung von Vertrauen auf die Anfangsphase einer virtuellen Teamgrün










Einfluss der Vertrauensbildung zwischen \ac{ik} und \ac{nik} Avataren auf die Team-Effektivität in einem kurzzeitig zusammenarbeitendem virtuellem Team.

















		\end{huge}
		
		Masterarbeit zur Erlangung des Master-Grades \newline
		\textit{Master of Science} im Studiengang Medientechnologie \newline
		an der Fakultät für Informations-, Medien- und Elektrotechnik \newline
		der Technischen Hochschule Köln \\
		~\\
		~\\
		~\\
		\noindent\begin{tabular}{ll}
			vorgelegt von: & Hannes Hinrichs \\
			Matrikel-Nr.: &	11121733 \\
			Adresse: & Zülpicher Straße 19 \\
			~ &	50674 Köln \\
			~ &	hannes.hinrichs@web.de \\
			~ & ~ \\
			eingereicht bei: & Prof. Dr. Arnulph Fuhrmann \\
			Zweitgutachter/in: & Prof. Dr. Stefan Grünvogel
		\end{tabular}	
		~\\
		~\\
		Ort, TT.MM.JJJJ
	\end{titlepage}
	\pagenumbering{Roman}
	\pagestyle{fancy}
	\newpage
\section*{Erklärung}\markboth{Erklärung}{Erklärung}\addcontentsline{toc}{section}{Erklärung}
	Ich versichere, die von mir vorgelegte Arbeit selbstständig verfasst zu haben. Alle Stellen, die wörtlich oder sinngemäß aus veröffentlichten oder nicht veröffentlichten Arbeiten anderer oder der Verfasserin/des Verfassers selbst entnommen sind, habe ich als entnommen kenntlich gemacht. Sämtliche Quellen und Hilfsmittel, die ich für die Arbeit benutzt habe, sind angegeben. Die Arbeit hat mit gleichem Inhalt bzw. in wesentlichen Teilen noch keiner anderen Prüfungsbehörde vorgelegen.\\
	Anmerkung: In einigen Studiengängen steht die Erklärung am Ende des Textes.\\
	~\\
	~\\
	\rule{0.35\textwidth}{0.4pt} \hspace*{3cm} \rule{0.45\textwidth}{0.4pt} \newline
	Ort, Datum	\hspace*{6.3cm}	Rechtsverbindliche Unterschrift
	\newpage
\section*{Kurzfassung/Abstract}\markboth{Kurzfassung/Abstract}{Kurzfassung/Abstract}\addcontentsline{toc}{section}{Kurzfassung/Abstract}
	Virtual Reality \ac{vr} hat in den letzten Jahren aufgrund von verbesserter Technologie und sinkenden Kosten an Bedeutung gewonnen. Verschiedene Felder, wie Medizin, Wirtschaft, Training oder die Industrie greifen auf diese Technologie zurück. Voranschreitende Forschung in diesem Feld, ist aufgrund von wachsendem Interesse von großer Bedeutung. Vertrauensbildung und das gesamte Konstrukt von Vertrauen ist in vielen Feldern eine wichtige Grundlage, wurde in der Forschung im Zusammenhang mit \ac{vr} jedoch in den letzten Jahren wenig betrieben. Diese Arbeit zielt darauf ab, das Konstrukt des Vertrauens in der Virtuellen Welt besser zu verstehen und mit diesem umzugehen. Dafür wurde eine Studie der Technischen Hochschule zu Köln durchgeführt, um einen Zusammenhang der Vertrauensbildung in einer Teambuildingmaßnahme zwischen einem Menschenähnlichen Avatar und einem nicht-Menschenähnlichen Avatar, darzustellen.
	
			\paragraph{Virtual-Reality, Vertrauen, Teambuilding, Avatar}
	\newpage
	\tableofcontents
	\newpage
	
\section*{Abkürzungsverzeichnis}\markboth{Abkürzungsverzeichnis}{Abkürzungsverzeichnis}\addcontentsline{toc}{section}{Abkürzungsverzeichnis}
	\begin{acronym}
	\acro{hmd}[HDM]{Head-Mounted-Display}
	\acro{sve}[SVE]{Shared-Virtual-Environment}
	\acro{vr}[VR]{Virtual-Reality}
	\acro{ivr}[IVR]{Immersive-Virtual-Reality}
	\acro{ipo}[IPO]{Input-Process-Output}
	\acro{fov}[FOV]{Field-of-View}
	\acro{bip}[BIP]{Break-in-Presence}
\end{acronym}

	
	%\newpage
	%\listoftables\addcontentsline{toc}{section}{Tabellenverzeichnis}
	%\newpage
	%\listoffigures\addcontentsline{toc}{section}{Abbildungsverzeichnis}
	\newpage
	\pagenumbering{arabic}
	
\section{Abgrenzung zu anderen Studien}
	\begin{itemize}
	\item{Team besteht aus 3! Personen}
	\item{Es wird die initiale Anfangsphase eines Teams untersucht}
	
	\item{Es wird geschaut, ob das geschaffene Vertrauen eine Auswirkung auf die Team-Effizienz hat}
	\item{Es wird schaut, mit welcher Kondition ( Self-Avatar vs. non Self-Avatar ) mehr Vertrauen in das Team geschaffen wurde}
	\item{Es wird geschaut, mit welcher Kondition ( Self-Avatar vs non Self-Avatar ) eine höhere Team-Effizienz wird}
	\end{itemize}		
	
	
\section*{Einleitung}\markboth{Einleitung}{Einleitung}\addcontentsline{toc}{section}{Einleitung}
	Mit voranschreitender Technologischer Entwicklung, rückt die digitale Kommunikation immer mehr in den Mittelpunkt. Unternehmen weltweit setzen schon seit langem darauf, räumliche und zeitliche Grenzen zu überwinden. Begonnen mit der Implementierung eines Telegrafennetz in den 1840er Jahren über das Telefon in den 1870er Jahren bis hin zur E-Mail 1970 und dem World-Wide-Web einige Jahre später.
	
	Seit den 1990er Jahren ist die Kommunikation mittels Computer nicht mehr wegzudenken. Der Computer steht in jedem Büro, in nahezu jedem Haushalt. Chats, E-Mails, das World-Wide-Web sowie Video- und Sprachkommunikation sind zu Standartkommikationsmittel der heutigen Zeit geworden. \citep[p. 14-16]{thurlow2004computer} \\ 
Neue Generationen von Sozialen-Netzwerksystemen werden mit der Prämisse erstellt, die Kommunikation zu entfernten Personen zu verbessern.
Einsatzfelder sind dabei :
\begin{itemize}
	\item{\textbf{Gemeinsame Arbeitsumgebungen}: Gemeinsame Arbeitsumgebungen sind zunehmend auf digitale Kommunikation angewiesen.}
	\item{\textbf{Die Mobil- und Internettelefonie:} Die Mobil- und Internettelefonie bieten zunehmend ständigen, von Raum und Zeit unabhängigen, sozialen Kontakt zu anderen Nutzern}
	\item{\textbf{Agentenbasierte Hilfsschnitstellen:} Hilfsagenten auf Websites, Charaktere in \ac{sve}'s sowie Charaktere in Computerspielen bieten \flqq quasi\frqq-sozialbeziehungen. Diese \flqq quasi\frqq-sozialen Beziehungen bieten eine neue Formen von Mensch-Maschinen-Kommunikation in \ac{sve}'s.} 
	\item{\textbf{FOIP/VOIP-Telefonkonferenzen:} Telekonferenzen ermöglichen eine simulierte Kommunikation von Angesicht zu Angesicht und weitere Soziale-Interaktionmöglichkeiten.}
	\item{\textbf{Sprach-interfaces:} Simulationen von sozialen Interaktionen mit dem Computer durch die menschlichen Sprache.}
	\item{\textbf{Soziale virtuelle 3D-Umgebungen:} Ermöglichen eine soziale Interaktion mit vollständig dargestellten Avataren.}
\end{itemize}

All diese Technologien teilen das selbe Ziel : \\ \flqq Die Verbesserung der Sozialen Präsenz, so dass der Nutzer das Gefühl hat zu einem gewissen Grad Einblicke in die kognitiven und affektiven Zustände des anderen zu haben.\frqq \citep{biocca2002defining} \citep[p.407–447]{biocca2001plugging} \\

	Aufgrund der voranschreitenden Globalisierung werden Teambuilding und Zusammenarbeit innerhalb eines Teams heute immer wichtiger für die Effizienz von Unternehmen. Dies hat zur Folge, dass sich Teams sehr häufig nicht am selben Ort befinden, wobei viele Unternehmen sich trotzdem eine effektive Gestaltung eines Teams wünschen.\citep[p.791-792]{jarvenpaa1999communication} Virtuelle Teambuildingmaßnahmen können hierbei Abhilfe schaffen. 
	
Erst seit ungefähr 10-15 Jahren sind Virtuelle Teams aus der bis dahin vorhandenen Nische in den Alltag von Unternehmen eingezogen. \citep{gilson2015virtual}

Im Jahr 2020, vor der Corona Pandemie im 2. Quartal 2020, haben 40\% aller Angestellten in Deutschland von zu Hause im \flqq Homeoffice\frqq gearbeitet. Dieser Anteil hat sich im Laufe des Jahres - Stand 03.08.2020 - auf 60\% gesteigert und es könnten theoretisch 80\% der Belegschaften von zu Hause aus arbeiten. \citep{statistaCorona2020} Durch diese Entwicklung, mussten Unternehmen sich zwangsläufig mit der Funktionsweise von virtuellen Teams beschäftigen.
	
Diese Ausarbeitung zielt auf einen Teilaspekt dieser Verbesserung ab. Genauer darauf, herauszufinden, wie sich Vertrauen zu den Personen hinter den in dem \ac{sve} genutzen Avataren \footnote{Grafikfigur, die die Onlinerepräsentation eines Nutzers in einer Virtuellen Umgebung darstellt. \citep[p.1]{neustaedter2009presenting}} auf auf die Effizienz eines Teams auswirkt, wenn sich die Teammitglieder untereinander nicht kennen. Dazu wird die Vertrauensbildung in einer Teambuildingmaßnahme sowie die Effizienz dieser anhand von \ac{ik} und \ac{nik} verglichen.	
Es wird der Generelle Hang zum Vertrauen, das Kognitive Vertrauen der Personen sowie die Team-Effektivität mit den Konditionen \ac{ik} oder \ac{nik} in einem drei-Personen Team analysiert. 

% Das hier hin?

\newpage
	\subsection{Motivation}
	Um ein gutes Arbeitsklima für zukünftige Zusammenarbeiten in einem Team zu schaffen, ist die Anfangsphase einer Teambuildingmaßnahme von großer Bedeutung. In dieser Zeit werden wichtige Richtungsweisende Grundsteine gelegt, die den Erfolg oder Misserfolg eines Teams bestimmen können. Charakterzüge der Mitglieder werden kennengelernt und es werden Beziehungen untereinander aufgebaut. \\	
	Viele Unternehmen setzen, aufgrund der wachsenden Globalisierung, auf geografisch trennte Teams um Aufgaben effektiv und effizient zu bearbeiten. Teambuilding in einem räumlich getrenntem Team spielt eine ebenso große Rolle wie Teambuildingmaßnahmen in einem Team, das die Möglichkeit hat sich in Persona kennenzulernen. \\
	In einem räumlich getrenntem Team zu arbeiten, das sich gegenseitig nicht Vertraut oder nicht richtig Zusammenarbeitet, hemmt die Performance dieses. \citep[p. 98-107]{huang1998supporting} \citep[p. 399-417]{turoff1993distributed} \\	
	Teambuilding steht jedoch nicht nur in der Kennenlernphase eines Teams im Fokus, denn es wird unter anderem besonders dann benötigt, wenn ein Team beispielsweise zu langsam arbeitet, die individuelle Leistung eines Teammitglieds nicht genügt, Konflikte entstehen oder die Gruppendynamik nicht dem soll entspricht. \citep[p. 1-3]{biech2007pfeiffer}\\
	Durch voranschreitende Forschung, ist es heutzutage möglich, dass sich viele Personen gleichzeitig in einem \glqq \ac{sve} \grqq befinden. Dadurch ist es in \ac{sve}'s möglich, auch Teambuildingmaßnahmen durchzuführen, wenn sich Teams räumlich getrennt voneinander befinden.\\
	Die Repräsentation eines Individuums innerhalb des \ac{sve} kann von \ac{sve} zu \ac{sve} unterschiedlich sein.
	
	\subsection{Ziele der Arbeit}
		Es gilt herauszufinden, welche Art der Repräsentation des Individuums einer Gruppe von Personen in einem \ac{sve} einen größereren interpersonellen Vertrauensaufbau gewährt.
Dabei wird der Fokus auf die beiden Konditionen \ac{ik} sowie \ac{nik} gelegt um zu analysieren, bei welcher dieser Konditionen eine größere Effektivitätssteigerung innerhalb des Teams vorhanden ist.
Es wird ein spezieller Fokus auf das kognitive Vertrauen in die Teammitglieder, den generellen Hang zum Vertrauen eines Individuums sowie die daraus resultierende Team-Effektivität gelegt.

	Durch diese Forschung soll es leichter möglich sein, eine Entscheidung über die Wahl und das Design einer Repräsentation eines Avatars in einer Teambuildingmaßnahme zu treffen, um Teambuilding über \ac{vr} effizienter zu gestalten.\\
	Es sind verschiedene Hypothesen aufgestellt, anhand denen es möglich ist, das kognitive Vertrauen in das Team, den generellen Hang zum Vertrauen eines Individuums, die Teameffizienz und deren Wechselwirkung zu analysieren.
%	Anhand verschiedener Faktoren, die zum Erfolgreichem messen von Teambuildingmaßnahmen ausgewählt wurden, werden die zuvor definierten Hypothesen ausgewertet.\\
	Diese Arbeit ist dem Gebiet der Virtuellen Realität und Sozialpsychologie zuzuordnen, speziell des Teambuilding in der Virtuellen Realität.
	In diesem Bereich gibt es noch nicht viel Literatur, weshalb eine Zeitgemäße Betrachtung und Analyse der Kombination von Virtueller Realität und Teambuilding als Sinnvoll erachtet wird.\\

%Repräsentationen in der virtuellen Welt haben einen großen Einfluss auf unterschiedliche Faktoren. Sie helfen dabei die andere Person in der \ac{vr} zu Lokalisieren, diese Wahrzunehmen, zu Identifizieren und zu verstehen, was eine andere Person aktuell tut. \citep[p.{pan2017impact} Daher gilt es herauszufinden, ob verschiedene Repräsentationen in einem \ac{sve} einen unterschiedlichen Einfluss auf die Team-Effektivität und das Vertrauen in das Team haben.\\


	\subsection{Inhaltlicher Aufbau der Arbeit}
	Diese Masterarbeit ist im wesentlichen in \textit{6 Kapitel} aufgeteilt.
	Im \textbf{Kapitel 1} die Grundlagen der \ac{vr} und des Teambuilding erklärt. in der \ac{vr} wird speziell auf die Punkte .... eingegangen.\\
	Im Teambuilding wird auf die Punkte ... eingegangen.\\
	Anschließend werden im \textbf{Kapitel 2} zu dem zu untersuchendem Gegenstand \textbf{3.4.5?} Hypothesen aufgestellt anhand dessen das im \textbf{Kapitel 3} beschriebene Experiment im \textbf{Kapitel 4} ausgewertet werden kann.\\
	\textbf{Kapitel 3} Beschäftigt sich mit der Vorgehensweise der Datenerhebung.
	Es werden die Teilnehmer, Abhängigen sowie die unabhängigen Variablen erläuter und die Untersuchungsmethode sowie die Aufgabe der Teilnehmer dargestellt.\\
	Das \textbf{Kapitel 4} beschäftigt sich Hauptsächlich mit der Analyse der Ergebnisse die im Kapitel 3 beschrieben wurden.\\
	\textbf{Kapitel 5} fasst alle Ergebisse zusammen.\\
	\textbf{Kapitel 6} beschäftigt hauptsächlich mit der Diskussion der Ergebnissen, der Eingesetzten Methoden und der Auswirkung der Ergebnisse auf den aktuellen Stand der Technik.\\
	
	\subsection{Verwandte Arbeiten}
	%https://www.researchgate.net/publication/320312582_AMELIO_Evaluating_the_Team-building_Potential_of_a_Mixed_Reality_Escape_Room_Game
	
	%https://www.ncbi.nlm.nih.gov/pmc/articles/PMC5730128/
	
	%https://pdfs.semanticscholar.org/5a82/6c0d065ef551ce7d8477dc5bbc475c8a9300.pdf
	
%https://www.researchgate.net/publication/323594629_Trusting_Strangers_in_Immersive_Virtual_Reality/link/5d3d981fa6fdcc370a666bb3/download
	
%https://d1wqtxts1xzle7.cloudfront.net/30602859/Presence_202004_20-_20Valencia.pdf?1361179740=&response-content-disposition=inline%3B+filename%3DOn_the_importance_of_reliable_real_time.pdf&Expires=1600803758&Signature=VH~~iWA9W1I3kVcnQySsr3qL0-2xMaNabSR76dVecHZv-icRMoouWAKZ6MDkHXfcTy56wkD31~TogcEPLHpBJgek0u8wj-Q6l2uKXvyqcXJO05r-fLGaeZ~Qxl--Mp~4C5ZyS9~DWbcAGK~40WMitvhrQZDhor-VWiVGp1wzn6~-xc1bW9BtlTJqSK4h0Q~5zeqeZSV9mXOuKN5AIQ8Qz1xLwC9KGhHqtDAc5pe8Mczf8uJFzaIfzMi5kwgGV-E1A~7upXbkuO1cktiHi2GhTADTVNsN7Ml~Bn3qNJeb4SoIvkEanhZfvuXY9SJIewThSFVImjfZrRMfTWqlhwxYnA__&Key-Pair-Id=APKAJLOHF5GGSLRBV4ZA#page=54

	\subsection{Das Framework}

Dieses Framework verdeutlicht den Einfluss vom Generellen Vertrauen und dem Vertrauen in das Team auf das zwischenmenschliche Vertrauen. 
Weiterhin verdeutlicht es, wie das zwischenmenschliche Vertrauen in Kombination mit der Avatar Verkörperung einen Einfluss auf die Team-Effektivität hat.

	\begin{figure}[H]
		\begin{footnotesize}
			\includegraphics[width=\textwidth]{Abbildungen/Framework.JPG}\\
			\caption[Abbildung 1]{Das Framework}
			\label{Framework}
		\end{footnotesize}
	\end{figure}

Grundlage für das Framework \autoref{Framework} sind verschiedene wissenschaftlich basierten Quellen, die sich mit der \ac{vr} sowie zwischenmenschlichem Vertrauen befassen. Diese beiden Teilbereiche wurden kombiniert und soll versuchen, die Team-Effektivität anhand variierendem zwischenmenschlichen Vertrauen und verschiedenen Avatar-Verkörperungen zu messen.

Wie an dem Framework zu erkennen ist, wird davon ausgegangen, dass die beiden Faktoren \flqq Generelles Vertrauen\frqq sowie \flqq Vertrauen in das Team\frqq das \flqq Zwischenmenschliche Vertrauen\frqq zu gleichen Teilen beeinflussen. 

Weiterhin hat die Kombination des \flqq Avatar-Embodiment\frqq sowie des \flqq Zwischenmenschlichem Vertrauen\frqq ebenfalls einen Einfluss zu gleichen Teilen.

Da dies jedoch das erste mal ist, dass dieses Framework theoretisch Aufgestellt wurde, kann es sein, dass die verschiedenen Faktoren nicht zu gleichen oder auch gar keinen Einfluss auf die Team-Effektivität haben.

	\newpage
\section{Grundlagen}

\subsection{Team und Gruppen}

\subsection{Abgrenzung Effektivität und Effizienz in Teambuildingmaßnahmen}
Die Begriffe Effektivität und Effizienz werden häufig als Synonyme eingesetzt. Dies ist jedoch nicht korrekt. Es gibt eine eindeutige Abgrenzung dieser, wobei beide einen eindeutigen Schwerpunkt auf den zu Analysierenden Inhalt setzen.

\paragraph{Effizienz}
Bei der Effizienz geht es darum, sein Handeln so zu optimieren, dass ein Ziel möglichst schnell und mit möglichst geringen Aufwand erreicht wird. Die Wirtschaftlichkeit steht bei der Effizienz im Vordergrund. Ergebnis und eingesetzte Mittel müssen dabei immer in einem möglichst günstigen Kosten-Nutzen-Verhältnis stehen.\\
$Effizienz = \frac{Ergebnis}{Aufwand}$ \\
Ist ein Team effektiv, wenn es sich nur über ein Telefon unterhalten würde, obwohl es sich auch in Persona treffen könnte?
Nein wäre es nicht, da der wirtschaftliche Aspekt der Telefonkosten mit eingerechnet werden muss. Somit wäre es günstiger(effizienter) sich in Persona zu treffen. Ebenfalls ist es Aufwändiger das Telefon in der Hand zu halten, Nummern zu wählen, etc.

Anders ausgedrückt : Effizienz beschreibt Mittel und Wege zur Erreichung der Effektivität.

Es müssen sich zur Erreichung von Effizienz immer folgende Fragen gestellt werden :
Gehen wir den Weg des geringsten Aufwands, um unser Ziel zu erreichen?
Tun wir die Dinge richtig?

\paragraph{Effektivität}
Bei der Effektivität geht es darum, die Dinge zu tun, die einem dem Ziel näher bringen. 
Somit arbeitet ein Team Effektiv, wenn es die richtigen Maßnahmen ergreift um dem zu erreichendem Ziel näher zu kommen.\\
$Effektivität = \frac{Ergebnis}{Ziel}$ \\
Ist ein Team effektiv, wenn es sich nur über ein Telefon unterhalten würde?
Effektiv ja, da eine Kommunikation zu den anderen Teammitgliedern stattfindet.
Effektiv ist es auch, wenn es sich in Persona trifft.

Anders ausgedrückt : Effektivität ist das Ausmaß der Erreichung angestrebter Ergebnisse/Ziele/Zwecke.

Somit kann bei dem Begriff \flqq Effektivität\frqq auch immer der Grad der Wirksamkeit berücksichtigt werden.

Somit stellen sich bei der Frage nach der Effektivität folgende Fragen: 
Bringt uns die Maßnahme dem Ziel näher? 
Tun wir die Dinge, die uns voran bringen?
Gibt es einen Maßnahme, die einen höheren Grad an Zielerreichung mit sich bringt?

%\paragraph{Abgrenzung zu dieser Ausarbeitung}
In dieser Masterarbeit wird geschaut, in welchem ausmaß Vertrauen bei unterschiedlichen Avatarkonditionen in einer Teambuildingmaßnahme gebildet wird und wie sich dieses auf die Teamleistung bezieht.

Effizienz ist wichtig, aber unter falschen Voraussetzungen Dinge effizient zu tun bringt dem Team in einer Teambuildingmaßnahme keinen Mehrwert.

So muss sich im Bezug auf diese Masterarbeit die Frage gestellt werden, welches die richtige Kondition (\ac{ik} oder \ac{nik}) in einer Teambuildingmaßnahme ist um einen höheren Grad an Effektivität zu erzielen.

	\subsection{Virtual Reality}
		\subsubsection{VR - Wozu brauchen wir Virtual Reality?}
		\subsubsection{VR - Virtuelle 3D-Welten}
Virtual-Reality ist eine Realität, die durch den Computer, geschriebenen Computercode sowie erstellte 3D-Welten, abgebildet und zum Leben erweckt wird. Dabei Spielt der Nutzer, der in diese Eintaucht, das Immersive Erlebnis dessen sowie die Interaktivität in dieser, eine Zentrale Rolle. \citep[p.6-12]{sherman2018understanding}
	Seit vielen Jahren sind \ac{sve}'s Forschungsgrundlage der Virtuellen Realität. Siehe \citep{shuffler2011there} \citep{steed1999leadership} und \citep{de2011level} \\
	\ac{sve}'s bieten die Möglichkeit, Geographisch getrennte Benutzer in einem Virtuellem Raum zu verbinden. Diese stellen Nutzern der Virtuellen Realität die Möglichkeit bereit zu kommunizieren und zu interagieren. \citep[p. 1-3]{pettifer1999designing} Eine gute Übersicht der Anwendungsgebiete eines \ac{sve} wurde von Richard Waters dargestellt. Siehe \citep{waters1997rise}.
	Zum Erfüllen dieser Studie wurde ein \ac{sve} entwickelt, welches den Nutzern ein \flqq Hand- und Kopf getrackten Avatar\frqq oder ein \flqq Hand-, Kopf und Inverskinematisch-simuliertem Torso getrackten Avatar\frqq, je nach Anwendungsfall, zur Verfügung stellt. Innerhalb des \ac{sve} können sich die Nutzer frei bewegen, andere Avatare Wahrnehmen und mit diesen interagieren.
	
	
		\subsubsection{VR - Vorbedingungen für Präsenz}
Um bestmögliche Präsenz in einer \ac{vr} zu erreichen, wird eine Schnittstelle zur Interaktion zwischen den Sinnen und der Außenwelt benötigt. Dies kann ein \ac{hmd} sein, welches ein computergeneriertes Bild erzeugt, durch das die virtuelle Welt wahrgenommen werden kann. Darüberhinaus muss das \ac{hmd} in der Lage sein, den Kopf des Benutzers frei im Raum zu Verfolgen und die gewonnenen Positionsdaten auf die \ac{vr} abzubilden. Weiterhin werden Controller benötigt, durch deren Einsatz es ebenfalls möglich ist, auch die Handbewegungen der realen Personen zu Verfolgen und diese in der \ac{vr} darzustellen. Das \ac{hmd}, die Controller, eventelle zusätzliche Körpertracker, Kopfhörer, eventuell wahrgenommener Geruch etc. definieren, zu welchem Ausmaß Sinnesmodalität in der \ac{vr} generiert werden können. Der Grad der Immersion hängt somit direkt mit der Anzahl der angesprochenen Sinnesmodalitäten zusammen. Je mehr Sinnesmodalitäten auf einmal angesprochen werden, desto mehr ist der Wahrnehmungsapparat in der Lage, die virtuelle Umgebung auf die reale Welt abzubilden. \\
Somit lässt sich sagen, dass die Voraussetzungen für Präsenz in der \ac{vr} die Korrelation zwischen den Sinneseindrücken, der Propriozeption und dem Grad der wahrgenommenen Realität der Illusion, sich in einem stabilen räumlichen Ort zu befinden, darstellt. Sind diese Voraussetzungen gegeben, kann der Nutzer einen plausiblen Vergleich zwischen realen sensorischen und virtuellen, durch Illusion erzeugten Daten, aufstellen. Vgl. \citep{slater2009we}

		\subsubsection{VR - Präsenz in Virtual Reality}
			
			Dank heutiger Technologien ist es uns möglich zu jeder Zeit mit Personen an verschiedenen Orten zu interagieren und kommunizieren. Dadurch ist Kommunikation nicht mehr nur auf die Personen in unserer unmittelbaren Umgebung beschränkt. 
			Diese Neuerung ermöglicht es, uns nicht mehr nur auf Soziale Interaktionen mit physischen Wesen zu beschränken, sondern erweitert diese auch auf Repräsentationen geschaffen aus Pixeln, E-Mails, Film oder durch das Telefon. Je nachdem wie Stark diese Repräsentation von uns Wahrgenommen wird, schafft Sie es, kraftvolle Emotionen in uns auszulösen.\citep[p. 4-6]{biocca2002defining}\\
			Nur wenn eine gewisse Präsenz dieser Repräsentation besteht, kann Vertrauensbildung funktionieren. Somit definiert das Vorhandensein des Gefühls von Präsenz den Grundbaustein für alle weiteren Schritte.\\
			Der Begriff \flqq Präsenz\frqq ist nicht genau definiert. Am ehesten trifft die Beschreibung zu, dass Präsenz das subjektive empfinden ist, an einem anderen Platz zu sein, obwohl man physikalisch eigentlich woanders ist. \citep[p. 1]{witmer1998measuring}\\
			Wenn eine Person eine andere Person in einer \ac{ivr} als Präsent wahrnimmt, werden die Wahrnehmenden, vestibulären, propriozeptiven und autonomen Nervensysteme in einen Zustand gebracht, der einem realen Zustand gleicht. Obwohl die betroffene Person weiß, dass Sie sich nicht in einer realen Lebenssituation befindet, wird diese dazu neigen, sich so zu verhalten, als ob diese in einer ist und Ähnliche Gedanken und Gefühle haben. \citep{slater2003note}

Diesbezüglich kann Präsenz als eine Art von Illusion angesehen werden, da die erzeugte Stimuli in der \ac{vr}, wie in der Realen Welt auch auf unsere Rezeptoren projeziert werden.

Somit lässt sich Präsenz in der \ac{vr} in 4 verschiedene Teilbereiche unterteilen.

\begin{itemize}
	\item{Die Illusion, sich in einem stabilen räumlichen Ort zu befinden} : Alle Stimuli zur räumlichen Wahrnehmung - wie zum Beispiel keine Restriktionen des \ac{fov} \footnote{Sichtfeld}, keine Kabel am \ac{hmd} - sollten sich möglichst wie in der realen Welt verhalten. \citep[p.47]{jerald2015vr}
	\item{Die Illusion der Selbstverkörperung} : Beschreibt das Gefühl einen Körper in der virtuellen Umgebung zu haben. Studien fanden heraus, dass durch einen virtuellen Körper die Präsenz in der \ac{vr} stark steigt. \citep[p.756]{botvinick1998rubber} Der virtuelle Körper muss nicht unserem eigentlichem ähnlich sehen. \citep[p.7]{maxwell1960psycho}
	\item{Die Illusion von körperlichen Interaktionen} : Beschreibt beispielsweise das vorhanden sein von Audio Feedback oder die Vibration des Controllers. Diese Kleinigkeiten tragen eine große Menge dazu bei, Präsenz in der \ac{vr} zu steigern. \citep[p.48]{jerald2015vr}
	\item{Die Illusion von sozialer Kommunikation} Sozialer-Realismus kann vom Physischen-Realismus abgetrennt werden. Social-Präsenz beschreibt das Gefühl, wirklich mit jemandem in einem \ac{sve} zu kommunizieren. Sei es verbal oder durch Körpersprache. Je mehr Nutzer der virtuellen Welt sich so verhalten, als ob diese real wäre, desto mehr steigt auch die Social-Präsenz. \citep[p.49]{jerald2015vr} \citep[p.12]{guadagno2007virtual}
\end{itemize}

\begin{figure}[H]
		\begin{footnotesize}
		\centering
			\includegraphics[scale= 0.4]{Abbildungen/forms_presence.JPG}
			\caption[Abbildung 1]{Forms of Presence}
			\textit{Die Verschiedenen Formen von Presence laut U.  Schultze \citep{schultze2010embodiment} }
			\label{vertical_horizontal}
		\end{footnotesize}
	\end{figure}


Präsenz ist ein Konstrukt aus Immersion und dem Nutzer. Immersion kann ein Gefühl von Präsenz erschaffen, muss es aber nicht zwangsläufig. Je Immersiver ein \textbf{System} ist, desto höher ist das Potential dieses Systems, dass der Nutzer ein Gefühl von Präsenz entwickelt.
			
Dementsprechend unterscheidet Lombard 6-Arten von Präsenz :
	\begin{itemize}
		\item \textbf{Sozialer Reichtum} : Das Medium wird als empfindlich oder persönlich wahrgenommen, wenn es zur Interaktion mit anderen Menschen verwendet wird.
		\item \textbf{Der Realismus} : Beschreibt die Wahrnehmung oder/und den Realismus der Präsenz und bis zu welchem Grad dieser als "Real" dargestellt werden kann.
		\item \textbf{Das Transportmedium} : Dies beschreibt das Gefühl "Du bist da", "Es ist da", oder/und "Wir sind Zusammen".
		\item \textbf{Die Immersion} : Beschreibt, wie Flächendeckend die Sinne des Benutzers angesprochen werden.
		\item \textbf{Der soziale Akteur innerhalb des Mediums} : Beschreibt die Reaktion auf eine Repräsentation einer Person durch ein Medium, auch wenn diese Irrational begründet ist.
		\item \textbf{Das Medium als sozialer Akteur} : Beschreibt die Situation wo der Akteur ( z.B. Computer ) selber als ein soziales Wesen wahrgenommen wird.					
				\citep{lombard1997heart}
			\end{itemize}
			
Um Vertrauen optimal aufbauen zu können, sollte das gesamte \ac{sve} so real wie möglich abgebildet sein. Diesbezüglich richtet sich diese Arbeit an nahezu alle Arten der hier genannten Präsenz.
		\subsubsection{VR - Präsenz und Co-Präsenz}
Wenn Personen sich zusammen in einer \ac{ivr} befinden und die andere Person wahrnehmen, wird dieses Gefühl mit  \grqq Co-Presence bzw. Social-Presence \glqq bezeichnet. \citep{schuemie2001research}\\
Das Gefühl, mit einer anderen Person in Verbindung zu stehen, wird als \textbf{Co-Präsenz} bezeichnet. Während des Gefühlt wird die Anwesenheit der anderen Person gespürt und es wird gespürt, dass der andere ebenfalls spürt, dass man Selbst anwesend ist. Somit kann Co-Präsenz als eine psychiologische Verbindung zum und mit dem anderem Charakterisiert werden.  \\
Fühlt sich ein Nutzer einer \ac{ivr} \flqq innerhalb\frqq des Mediums \ac{vr}, so wird dies als \textbf{Telepräsenz} bezeichnet. Telepräsenz bezeichnet das Gefühl \flqq da zu sein\frqq. Je höher das Level wahrgenommener Telepräsenz ist, desto weniger fühlt sich der Nutzer an dem Ort seines physikalischem Körpers und mehr an einem anderem Ort. \citep[p.482]{nowak2004effect}\\
Der Begriff der \flqq \textbf{Sozialen-Präsenz} \frqq beschreibt, wie stark ein Nutzer eine Person, mit der dieser nur mittels Kommunikationstechnologie kommuniziert, als \flqq real\frqq bezeichnet. Soziale-Präsenz wird dabei auf die Kommunikationstechnologie an sich bezogen. Je mehr Soziale-Präsenz vorhanden ist, desto besser ist ein Kommunikationsmittel geeignet um Informationen über den Interaktionspartner zu übermitteln. So erzeugt ein Videochat durch die zusätzliche Übertragung eines Videos mehr Soziale-Präsenz als eine Telefonkommunikation.  \citep[p.151]{gunawardena1995social}
Es wird davon ausgegangen, dass eine \ac{sve}'s mehr Soziale Präsenz erwecken kann als Beispielsweise ein Telefonat. Dies sei damit begründet, dass \ac{sve} die Eigenschaften und Interaktionen des anderen besser darstellen und einfangen kann. Desto mehr Eigenschaften einer Person dargestellt werden können, desto höher ist der wahrgenommene Realitätsgrad des anderen. \citep[p. 5-8]{biocca2002defining}
Somit reicht das gesamte Kontinuum der Präsenz von der räumlichen Komponente bis hin starken psychiologischen Beteiligungen. Dies macht es möglich, auch auf die affektiven und kognitives Zustände von Personen aufzudecken. Höhere wahrgenommene Präsenz führt dazu, dass die Person sich mehr mit der \ac{vr} engagieren kann, was zu Handlungen führt, die als verbunden und voneinander abhängig wahrgenommen werden. \citep{biocca2001criteria}
		\subsubsection{VR - Präsenz und Teambuilding}

\newpage

	\subsection{Vertrauen}
	
Der Begriff Vertrauen ist in viele Bereiche des alltäglichen Lebens eingezogen. Es ist ein Psychiologischen Konzept, dass jeden Menschen ständig begleitet.
Ein Politiker wirbt um das Vertrauen seiner wähler, Unternehmer beschreiben sich selbst als Geschäftspartner des Vertrauens, man empfiehlt den Arzt des Vertrauens etc..

Vertrauen in die \ac{vr} kann auf zwei weisen Betrachtet werden. Das Vertrauen in die Akzeptanz von \ac{vr} sowie das zwischenmenschliche Vertrauen, welches in der \ac{vr} zwischen 2 oder mehreren Personen gebildet wird.
Mangelndes Vertrauen in eine Technologie, kann Nutzer dran hindern, diese zu Nutzen. \citep{trustInVRTechnology}

Da diese Arbeit sich mit dem zwischenmenschlichem Vertrauen beschäftigt, wird das Vertrauen in die Akzeptanz der \ac{vr} nicht weiter behandelt.
Vertrauen wird in dieser Ausarbeitung als bilaterales Konstrukt zwischen einer vertrauenden Person und einer zu vertrauenden Person definiert.
% https://sci-hub.ren/10.4337/9781847202819.00008
% https://sci-hub.ren/10.4337/9781847202819.00008
Aufgrund der Vielseitigkeit von Vertrauen, gibt es unterschiedliche Definitionen. Die erste Definition von Vertrauen wurde 1967 von Rotter aufgestellt. Er definiert Vertrauen als \\ \dq die Erwartung eines Individuums oder einer Gruppe, dass man sich auf das Wort, das Versprechen, die mündliche oder schriftliche Aussage eines anderen Individuums oder einer Gruppe verlassen kann\dq{}. \citep[p.651]{rotter1967new}.\\
Die meist verbreitetste und akzeptierteste definition kommt jedoch von Meyer.\citep[p.712]{mayer1995integrative} So definiert er Vertrauen als : \\\dq die Bereitschaft einer Person, für die Handlungen einer anderen Person anfällig zu sein, basierend auf der Erwartung, dass die zu vertrauende Person eine bestimmte, für den Vertrauensgeber wichtige Handlung ausführen wird, unabhängig von der Möglichkeit, diese Person zu überwachen oder zu kontrollieren.\dq{}\\
Jede zwischenmenschliche Beziehung beginnt mit einer frühen Phase der Vertrauensbildung. Diese frühen Phase kann mit Unsicherheiten und Zweifel charakterisiert werden. Das gegenseitige Vertrauen, dass man sich anschließend schenkt, muss erst einmal ausgelotet werden. 
%\citep[p.166-168]{meyerson1996swift}

Während der frühen Phase der Vertrauensbildung entscheidet sich, ob eine Beziehung aufrecht erhalten wird oder nicht. Unterbewusst bildet sich ein Gefühl von Zuversicht und Sicherheit oder ein Gefühl von Spannung, Zweifel und Skepsis dem Interaktionspartner gegenüber. 
Dabei ist es egal, ob sich dafür entschieden wird jemandem zu Vertrauen oder nicht. Auf jeden Fall beeinflusst die Stärke des positiven oder negativen Vertrauensgefühl die Effektivität der Zusammenarbeit. Vertrauen kann es einfach oder schwierig machen mit einer anderen Person zu arbeiten und ziele in einer Gruppe oder einem Team zu erreichen.
Früher Vertrauensaufbau ist daher der Schlüssel zur erfolgreichen Zusammenarbeit. \citep[p.405-406]{bigley1998straining}
Die initiale Phase der Vertrauensbildung wirkt sich auf das Kognitive und das Affektive Vertrauen aus, welche einen starken Einfluss auf das sich entwickelnde Vertrauensmodell zu einer Person hat. In dieser Phase sind diese beiden Vertrauensarten anfällig für Veränderungen. \citep[p.461-462]{baldwin1992relational}

Meinungen und Annahmen, die sich frühzeitig bilden, prägen sich somit stark auch auf die eigene zukünftige Meinung über der zu vertrauenden Person aus.

Auch im Unternehmertum muss sich mit dem Konzept des Vertrauens beschäftigt werden um Erfolgreich arbeiten zu können. Ohne Vertrauen in ein Team oder in Unterschiedliche Personen, fällt es schwer Risiken einzugehen. Ist Vertrauen vorhanden, wird man nicht mit der Angst konfrontiert, dass andere Personen einen ausnutzen könnten. \citep[p.1152]{breuer2016does}.

Betrachtet man Vertrauen im Unternehmen auf der Ebene von Teams, setzen sich die Vertrauenden Personen sowie die zu Vertrauenden Personen aus mehreren Teammitgliedern zusammen.
 
% "a willingness of a party to be vulnerable to the actions of another party based on the expectation that the other will perform a particular action important to the trustor, irrespective of the ability to monitor or control that party"

		\subsubsection{Zwischenmenschliches Vertrauen}
%	https://sci-hub.tw/https://doi.org/10.1177/1046496408323569
Während Vertrauen in Technologie sich mit der Akzeptanz dieser Beschäftigt, beschäftigt sich zwischenmenschliches Vertrauen mit dem Vertrauen zwischen zwei oder mehreren Personen. \citep{mcknight2011trust}

Vertrauen kann nicht nur statisch und einseitig betrachtet werden. Eine Person kann nicht nur entweder \dq Vertrauen\dq oder "Nicht Vertrauen". Vertrauen ist ein dynamisches Konstrukt, welches sich mit der Zeit wandelt. Vertrauen kann eine Bildungs, Stabilisierungs und Abnehmphase besitzen. \citep[p.396]{rousseau1998not}

Viele Forschungen, die sich mit dem Thema Vertrauen beschäftigen, gehen heute davon aus, dass zwischenmenschliches Vertrauen aus einem zweidimensionalem Konstrukt besteht. \citep{johnson2005cognitive} \citep{cook1980new}

Mooradian \citep[p.524-525]{mooradian2006trusts} ist der Ansicht, dass Vertrauen als \dq Eigenschaft\dq{} oder als \dq Zustand\dq{} gesehen werden.

	\subsubsection{Vertrauen als Eigenschaft - Hang zum Vertrauen}

Wird Vertrauen als Eigenschaft betrachtet, spiegelt dies die Einstellung zum Vertrauen einer Person wieder. Diese Einstellung zum Vertrauen ist Langlebig und wird nicht all zu schnell auf oder abgebaut. Unabhängig von einer Situation in dem sich diese Person aktuell befindet. Es wird davon ausgegangen, dass diese Eigenschaft aus dem Temprament oder der Lebenserfahrung einer Person entsteht. Dieses Vertrauen ist auch das Grundlevel an Vertrauen, die eine Person in eine neue zwischenmenschliche Beziehung von beginn an einbringt.\\
Der Generelle Hang zum Vertrauen ist nicht Situationsabhängig, sondern Stellt eine längerfristige Konstante auf Basis des Grundvertrauens einer Person dar.
Grundvertrauen setzt sich dabei aus der individuellen Eigenschaft des Hangs zum Vertrauen einer einzelnen Person, sowie die Grundstimmung gegenüber Personen im Allgemeinen, zusammen. \citep{couch1996assessment} \\

In dieser Studie wird mit einem Team von drei Leuten gearbeitet. In diesem Kontext, spiegelt der Hang zum Vertrauen einer Person den generellen Vertrauensvorschuss wieder, den eine Person dem Team gewährt. 

Laut Forschungen von McKnight kann ein starkes Vertrauen auch in frühen Teambuildingphasen entstehen. Glauben Teammitglieder beispielsweise in frühen Teambuildingphasen, dass der Teambuildingprozess mit Strukturierungen und Regelungen durchwachsen ist, ergibt sich dadurch ein höheres kognitives Vertrauen und es folgt ein höheres Gesamtvertrauen in das Team. \citep[p.478-479]{mcknight1998initial}

Weiterhin, da der Hang zum Vertrauen das minimale Vertrauen einer Person widerspiegelt und bei der Neugründung eines virtuellen Teams keine Vorerfahrungen mit den Teammitgliedern besteht, ist dies der einzige Indikator der eine eventuelle Vertrauensbildung widerspiegelt.
Daraus lässt sich schließen, dass, wenn Personen ein hohes Grundvertrauen haben, auch ein eine  Kognitives Vertrauen in frühen Phasen des Teambuildings besitzen. \\ 
Daraus lässt sich schließen, dass das Grundvertrauen einer Person, sich auch in einer Art und Weise auf das Kognitive Vertrauen in das Team auswirkt. Da ein höheres Kognitives Vertrauen in das Team wünschenswert ist, wird daraus eine Hypothese definiert.
%Diese Annahme wird auch von Brown citep[p.121]{brown2004interpersonal} bestätigt und weitet diese Annahme ebenfalls auf das Vertrauen in virtuelle Teams aus.
\\
\textbf{Hypothese 1}:Ein hoher General-Trust-Score wirkt sich \textbf{positiv} auf den Cognitiven-Trust-Score aus.
\\
	\subsubsection{Vertrauen als Zustand - Kognitives und Affektives Vertrauen \newline}

Wird Vertrauen als \dq Zustand\dq{} betrachtet, so kann sich dieses Vertrauen auch im Laufe der Zeit, z.B. durch Interaktion mit einer anderen Person, ändern. Dieser \dq Zustand\dq{} des Vertrauens, spiegelt sich auch in Meyers \citep[p.712]{mayer1995integrative} vorherig erwähnter Definition von Vertrauen wieder :\\ "die Bereitschaft einer Person, für die Handlungen einer anderen Person anfällig zu sein, basierend auf der Erwartung, dass die zu vertrauende Person eine bestimmte, für den Vertrauensgeber wichtige Handlung ausführen wird, unabhängig von der Möglichkeit, diese Person zu überwachen oder zu kontrollieren."
Es wird einer Person oder einem Team ein Vertrauensvorschuss gewährt, der sich jedoch zu jederzeit ändern kann wenn dieser gebrochen wird.

%Cook und Wall beschreiben diese Zwei Dimensionen als  \begin{itemize}
%\item{ “faith in the trustworthy intentions of others”, sowie}
%\item{“confidence in the ability of others, yielding ascriptions of capability and reliability”.}
%\end{itemize}

Das Konzept des Vertrauens als Zustand lässt sich laut Lewis \citep[p.970-971]{lewis1985trust} in zwei Teile unterteilen.\newline
Laut seinen Forschungen basiert Vertrauen \dq auf einem kognitiven Prozess, der zwischen vertrauenswürdigen, misstrauischen und unbekannten Personen und Institutionen unterscheidet. In diesem Sinne wählen wir kognitiv aus, wem wir in welcher Hinsicht und unter welchen Umständen vertrauen, und wir stützen die Wahl auf das, was wir als \dq gute Gründe\dq{} ansehen, die einen Beweis für die Vertrauenswürdigkeit darstellen.\dq{}\citep[p.970]{lewis1985trust}
Somit basiert das \textbf{Kognitiv aufgebaute Vertrauen} auf einer von uns definierten Logik statt auf einer emotionalen Komponente. Diese "guten Gründe" können auch leicht gebrochen werden, indem der Vertrauensvorschuss, den wir durch das Kognitive-Vertrauen unserem Interaktionspartner vorschießen enttäuscht wird.

Das Kognitive Vertrauen kann kurzfristig aufgebaut werden und ist leicht anfällig gegen äußerliche Einflüsse. 
Weiterhin besitzt Vertrauen als \dq Zustand\dq{} noch die Affektive Komponente.\\ \dq Diese \textbf{affektive Komponente des Vertrauens} besteht in einer emotionalen Bindung zwischen allen, die an der Beziehung beteiligt sind. Wie die affektiven Bindungen der Freundschaft und der Liebe schafft Vertrauen eine soziale Situation, in der intensive emotionale Investitionen getätigt werden können, und deshalb weckt der Verrat eines persönlichen Vertrauens ein Gefühl der emotionalen Empörung bei dem Betrogenen. Der Vertrauensbruch trifft die Grundlage der Beziehung selbst, nicht nur den spezifischen Inhalt des Verrats. Diese emotionale Komponente ist bei allen Arten von Vertrauen vorhanden, aber normalerweise ist sie bei engem zwischenmenschlichem Vertrauen am intensivsten.\dq{} \citep[p.971]{lewis1985trust} \\
Ein Beispiel für Affektives-Vertrauen ist eine Liebesbeziehung zwischen zwei Personen. Das affektive Vertrauen baut sich mit der Zeit langsam auf und kann durch verschiedene Ereignisse erschüttert oder gestärkt werden. Somit es eher durch Emotionen statt durch Logik charakterisiert.
Es ergibt sich somit aus zwischenmenschlichen emotionalen Verbindungen und gegenseitiger Fürsorge, während individuelles kognitives Vertrauen auf der Überzeugung in die Fähigkeiten oder in die Zuverlässigkeit eines anderen basiert. \\

Vertraunensaufbau zwischen verschiedenen Personen beinhaltet somit zwei Komponenten. Beide Komponenten sind bei der anfänglichen Teamgründung auch von großer Bedeutung. McAllister meint in seiner Forschung dazu, dass das Kognitive Vertrauen in der Anfangsphase von größer Bedeutung ist. Die Grunderwartung der kognitiven Komponente muss erst einmal erfüllt werden bevor mehr in die Beziehung zu den einzelnen Personen eines Teams investiert wird. Je mehr ein Team zusammenarbeitet und sich kennenlernt, umso wichtiger wird die Affektive Komponente der Vertrauensbildung. \citep[p.30]{mcallister1995affect}

	\paragraph{Swift-Trust}
Swift Trust ist eine form von Vertrauen in kurzzeitig zusammenarbeitenden Teams oder Gruppen. Kurzzeitig aufgestellte Teams besitzen eine endliche Zeitspanne, haben ein klares Ziel welches diese verfolgen und ihr Erfolg wird von guter Koordination bestimmt.
Für solche Teams gibt es nicht genügend Zeit um ein langfristiges Vertrauensverhältnis durch zum Beispiel Teambuildingmaßnahmen aufzubauen, wie es traditionelle Teams in einer Unternehmensorganisation tun können.
Es unterscheidet sich dahingehend vom Kognitiven-Vertrauen dahingehend, dass Swift-Trust nur auf oberflächlichen Informationen über den anderen basiert. Es wird ein Vertrauensvorschuss in das Team gewährt, welcher sich zu einem späteren Zeitpunkt bestätigt oder nicht bestätigt \citep[p.141]{wildman2012trust}, wohingegen kognitives Vertrauen in andere Personen auf einer rationalen Grundlage, die auf der Wahrnehmung von Kompetenz oder Zuverlässigkeit beruht, basiert. \citep[p.970]{lewis1985trust}

In dieser Studie nehmen die Versuchsteilnehmer an einem 10-Minütigem Teambasiertem spiel teil. Da Affektives-Vertrauen über einen längerfristigen Zeitraum gebildet wird, wurde in dieser Studie keine Affektive-Vertrauensmessung durchgeführt. Kognitives Vertrauen hingegegen ist in der Anfangsphase einer Teamgründung von Bedeutung und wird schnell durch kognitive Entscheidungsprozesse gebildet. Durch die Analyse des Kognitiven Vertrauens soll es möglich sein festzustellen, wie sehr unterschiedliche Avatararten das Kognitive Vertrauen in das Team beeinflussen.

\textbf{Hyothese 2 H$_{1}$} :Die Konditionen \ac{ik} oder \ac{nik} haben \textbf{einen Signifikant abweichenden Einfluss} auf den Cognitive-Trust-Score. \newline

Hat nun ein Head und Hand getrackter Avatar einen Einfluss auf die wahrgenommen Fähigkeiten oder die wahrgenommene Zuverlässigkeit in das Team? Genau diese Frage stellt sich diese Hypothese 2.
Es wird davon ausgegangen, dass, egal welche Kondition des Avatars ein Probant zur Verfügung gestellt bekommt, dies keinen Einfluss auf die wahrgenommenen Fähigkeiten des Teams hat. Somit erscheint ein Team in seiner Leistungsfähigkeit und Zuverlässigkeit immer gleich, egal ob ein Avatar Menschenähnlich ist oder nicht.


%\textbf{ES GIBT NOCH MEHR VERGLEICHE, SIEHE DAZU QUELLE UNTEN}
%\textbf{HIER EINEN VERGLEICH ZWISCHEN COGNITIVE TRUST UND AFFECTIVE TRUST AUFSTELLEN. AUCH BEIM FRAGEBOGEN BEACHTEN! ERKLÄRUNG GEBEN WARUM GENAU DEN VON MC ALLISTER 1995 GENOMMEN WURDE.}
%\textbf{http://citeseerx.ist.psu.edu/viewdoc/download?doi=10.1.1.496.9380&rep=rep1&type=pdf}

	\subsubsection{Vertrauen in virtuellen Teams}


In vielen Forschungen wird behauptet, dass Vertrauen einen positiven Effekt auf die Teamperformance hat. \citep{mcallister1995affect} \citep{mayer1995integrative} \citep{dirks2002trust}
Kognitives Vertrauen lässt sich nur sehr schwer aufrechterhalten. So könnte es sein, dass ein Team einen hohen Kognitiven Vertrauenswert aufweist, dann jedoch einen Rückschlag bei der Erledigung einer Aufgabe erzielt. Dies hat eine Verringerung des Kognitiven Vertrauen des Teams zur folge. \citep[p.29-31]{mcallister1995affect}
Im Vergleich wird das affektive Vertrauen im Team bei solchen Situationen nicht verringert. Um eine Verringerung des affektiven Vertrauens zu verursachen, benötigt es eine längerfristige emotionale Kriese innerhalb des Teams. Somit stellt sich herraus, dass das affektive Vertrauen eine längerfristige, stärkere Bindung schafft, als das kognitive Vertrauen.  \citep[p.29-31]{mcallister1995affect} \\

\textit{Hypothese H3} : \textbf{Ein hoher Kognitiver-Trust-Score wirkt sich positiv auf die Team-Effektivität aus.}\\

\textit{Hypothese H5} : \textbf{Teams, mit einem generell hohem Hang zum Vertrauen erzielen auch eine hohere Team-Effektivität.}

Aktuell gibt es nicht viel Forschung, wie sich Vertrauen in Virtuellen Teams aufbauen und halten lässt. \citep[p.8-23]{duarte2006mastering} 
Bisherige Studien die einen Zusammenhang zwischen Vertrauen und Team-Effektivität darstellen, haben positive Zusammenhänge \citep{davis2000trusted}, keine Zusammenhänge \citep{hertel2004managing} sowie negative Zusammenhänge \citep{dirks1999effects} festgestellt.

Bente, Rüggenberg und Krämer fanden heraus, dass wenig Kognitives-Vertrauen zu Avataren aufgebaut werden konnte, während Face-to-Face, Telefon und Chatkommunikationen besser abschnitten. \citep[p.54-59]{bente2004social}

Vertrauen im Team wird meinst jedoch nur eindimensional gemessen, obwohl es ein zweidimensionales Konstrukt ist.
Polzer hat Beispielsweise eine Forschung über räumlich getrennte Teams und deren Vertrauensbildung durchgeführt, eine eindimensionale Affektive Vertrauensmessung. \citep[p.682]{polzer2006extending}
Prichard und Ashleigh haben 2007 ebenfalls mit einer eindimensionalen Kognitiven Vertrauensmessung herausgefunden, dass Teambuilding das Vertrauen der einzelnen Teammitglieder untereinander verstärkt. \citep[p.704]{prichard2007effects}
Dirks hat 1999 zwar die Multidimensionale Komponente aufgegriffen, jedoch stand für seine Versuchsdurchführung nur ein 10-Minütiges Zeitfenster zur Verfügung. In diesen 10-Minuten kann die Affektive-Komponente nicht gebildet bzw. gemessen werden. \citep[p.445]{mayer1995integrative} Daher wird seine Forschung ebenfalls nur als eindimensional betrachtet.
Mehrdimensionale, zuverlässige, Vertrauensforschung ist im traditionellem Sinne nur mittels Langzeitstudien möglich, da das Kognitive- und besonders das Affektives-Vertrauen eine Zeitliche Komponente benötigt. \citep{jones1998experience} \\
	\newpage
	
\subsection{Avatare}
		
\subsubsection{AVATARE - Repräsentationen von Avataren}

Das Menschliche Gehirn ist in der Lage, computergenerierte Darstellungen in \flqq Lebend und Nicht-Lebend\frqq zu kategorisieren. Einige Forschungen gehen davon aus, dass das menschliche Gehirn semantische Unterschiede im zusammenhang mit der Social-Presence feststellen kann. So kann eine menschenähnliche Form als biologisch oder nicht-Lebend erkannt werden. 

	\begin{figure}[H]
		\begin{footnotesize}
		\centering
			\includegraphics[scale= 0.4]{Abbildungen/Symmetry.JPG}
			\caption[Abbildung 1]{Social presence response to vertical and
horizontal beings}
			\textit{Menschen interpretieren symmetrische Formen um eine vertikale Achse eher als \flqq Menschlich\frqq als Formen um die horizontale Achse. \citep{biocca2002defining} }
			\label{vertical_horizontal}
		\end{footnotesize}
	\end{figure}

In \autoref{vertical_horizontal} ist zu erkennen, dass eine menschenähnliche vertikale, bilateral Symmetrische Repräsentation mehr Co-Präsenz erweckt als die horizontale bilateral Symmetrische Repräsentation. \citep[p.546-551]{thornhill1998relative}
Bilateral-Vertikale Symmetrie wird vom Menschen mit der körperlichen Gesundheit eines Menschens in Verbindung gebracht. Sogar Weibchen verschiedener Spezies neigen dazu, Partner mit einem höheren Grad an bilateraler Symmetrie auszuwählen. \citep[p. 659–669]{rhodes1998facial} \citep{biocca2002defining} \citep[p.233–242]{grammer1994human} \ \\

\begin{figure}[H]
		\begin{footnotesize}
		\centering
			\includegraphics[scale= 0.5]{Abbildungen/moving_dots.JPG}
			\caption[Abbildung 1]{Social presence response to vertical and
horizontal beings}
			\textit{Stationäre punkte, bei denen der Mensch eine lebendige Bewegung ausmachen kann, wenn diese Anfangen sich zu bewegen. Es ist sogar möglich einen Menschen zu erkennen, dessen Art der Aktivität sowie den emotionalen Zustand. \citep{biocca2002defining} \citep[p.76-89]{johansson1975visual}}

			\label{moving_dots}
		\end{footnotesize}
	\end{figure}

Auch sich bewegende Punkte können als intelligentes Wesen wahrgenommen werden. Johannson \citep[p.76-89]{johansson1975visual} führte eine Studie durch, in der die Teilnehmer dreizehn \autoref{moving_dots} sich bewegende Punkte sahen und sofort die Darstellung einer menschlichen Bewegung erkennen konnten. Als die Punkte stationär waren, ist es den Teilnehmern der Studie nicht möglich gewesen diese Punkte als menschliche Repräsentation zu erkennen. Wenige Punkte reichen aus um Informationen zu erzeugen, die Aufschluss über die Aktivität, das Geschlecht, die Bewegung, den emotionalen Zustand oder die Anzahl der Personen zu geben.

		\subsubsection{AVATARE - Selbst-Avatar und Nicht-Selbst-Avatare}

In der \ac{vr} und in einigen anderen computerbasierten Medien, kann der Nutzer sich Avatare erstellen und mit diesen interagieren. Ein Avatar bezeichnet eine Grafikfigur, die die Onlinerepräsentation eines Nutzers in einer Virtuellen Umgebung darstellt. \citep[p.1]{neustaedter2009presenting} \\
\ac{hmd}'s beeinflussen das Sichtfelds des Nutzers so stark, dass diese Ihre eigenen Körper nichtmehr sehen können. Um diesem Nachteil entgegenzuwirken, kann einem Nutzer ein virtueller Körper zur Verfügung gestellt werden. Diesen Körper nennt man Selbst-Avatar.
Es ist schwierig einen Selbst-Avatar hoher Qualität zu simulieren. Dazu wäre im Idealfall das Verfolgen und Animieren mehrerer Körperteil unabdingbar. Ist der Selbst-Avatar schlecht animiert oder es entstehen während der Nutzung Trackingfehler die der Nutzer erkennt, kann sehr leicht zu einem \ac{bip} kommen. Bei diesem bricht die gesamte Illusion der \ac{vr} für den Nutzer in sich zusammenbricht. 
Dies ist auch der Grund, weshalb relativ wenige \ac{vr}-Anwendungen den menschlichen Körper als Avatare darstellen.
Sind jedoch genügend Körperteile getracked und animiert, muss der Avatar nicht unverwechselbar menschlich aussehen um einen glaubwürdigen Selbst-Avatar zu vermitteln. Selbst grobe Avatar-Darstellungen schaffen es, ausreichende Informationen über die Glaubwürdigkeit eines menschlichen Körpers zu Vermitteln.\citep{lok2003effects}

Es wurden schon zahlreiche Experimente durchgeführt um den Einfluss von Avataren zu erforschen. So gibt es Experimente, die den Einfluss des Geschlechts \cite{slater2010first}, der Hautfarbe \cite{peck2013putting} oder des Grades des Realismus \cite{roth2016avatar}, als Inhalt ihrer Forschungen haben.

Biocca forschte Umfangreich über den Einfluss von Self-Avataren auf den Nutzer in einer \ac{vr}. \citep[421-427]{construal2014connected} 
\newline So beispielsweise, wie sich die Interaktion mit der Welt verändert, wie sich soziale Interaktionen verändern und wie Aufgaben wahrgenommen und erledigt werden. \citep{benford1995user} \citep{bowers1996talk}
Diese Studien gehen davon aus, dass ein menschenähnlicher Körper als Avatar die Self-Präsenz stark erhöht.

Yee und Bailenson fanden heraus, dass die Selbst-Präsenz Einfluss auf den Proteus-Effekt hat. Je stärker die Selbst-Präsenz eines Avatars, desto stärker ist auch der Proteus-Effekt. Der Proteus-Effekt beeinflusst den Nutzer so, dass sozialen Interaktionen des Nutzers denen der sozialen Identifikation seines Avatars ähneln. So nimmt eine Person, das Verhaltensmuster entsprechend zu dem Aussehen des Avatars ein. Je nachdem, wie der Nutzer glaubt, dass bestimmte Verhaltensweisen von ihm, aufgrund des Avatars, erwarten. \citep{ratan2015leveling} 
Diese Abbildung der Verhaltensweisen des Nutzers lässt sich auch auf den Gruppenkontext beobachten. Der Nutzer wird sich in einer Gruppe so Verhalten, wie die Gruppenkonformität es vorgibt.
Dieser Effekt kommt nicht nur bei unverwechselbar Menschenähnlichen Avataren vor, sondern kann auch bei anderen Avatararten auftreten. \citep{lok2003effects} 

Solche identitätsbezogenen Avatar-induzierten Effekte, können die Kognitiven-Einstellungen und Verhaltensweisen anderer Personen gegenüber beeinflussen.

Somit stellt sich die Frage, ob ein Avatar menschenähnlich Aussehen muss oder etwa nicht. Diese Frage stellten sich George, Eiband, Hufnagel und Hussman \cite{george2018trusting} in Ihrer Forschung und verglichen, ob sich mehr Vertrauen, zwischen einem Menschenähnlichem oder einem Roboter-Avatar aufbauen lässt.
Dazu schufen Sie ein Szenario, in dem Probanten mittels eines \ac{hmd}, ein Social-Dilemma-Scenario  \footnote{Situationen, in denen - die rationale Verfolgung von Eigeninteressen zu einer kollektiven Katastrophe führen kann, erlebten. Sie fanden keinen signifikanten unterschied in der Vertrauenswürdigkeit zwischen Menschenähnlichen und Roboter-Avataren. Jedoch wurde ein größeres Gefühl von Zusammenheit festgestellt, wenn mit einem Menschenähnlichem Avatar interagiert wurde. \citep{kerr1983motivation}}
Sie erwähnten weiterhin in Ihrer Studie, dass gute Grafik und Realistisches Verhalten durch beispielsweise Mikrogestikulationen, soziale Interaktionen sowie die Co-Präsenz fördern. \citep{george2018trusting}

Um den den Einfluss des Grades des Realismus unter Avataren zu erforschen, führten Riedl, Mohr und Kenning 2014 eine Studie zum Vertrauensaufbau unter Menschen im Vergleich zu Avataren mit menschenähnlichen Gesichtern durch. Sie fanden heraus, dass es Personen leichter fällt einzuschätzen, dass eine reale Person, im Gegensatz zu einem Avatar, vertrauenswürdig ist. Es wurde der Frontalkortex - die Gehirnregion, der dafür Verantwortlich ist einzuschätzen, wie die Gedanken und Gefühle des gegenüber sind - bei Interaktionen mit Menschen mehr angeregt, als bei Interaktionen mit Avataren.
Jedoch ist die Lernrate des Vertrauensaufbau zwischen Menschen und Avataren gleich. Somit wird Vertrauen zwischen Menschen in der gleichen Geschwindigkeit aufgebaut wie zwischen Avataren. \cite{riedl2014trusting}

Somit lässt sich feststellen, dass ein höherer Grad an Realismus den Vertrauensaufbau fördert, jedoch kein signifikanter Unterschied zwischen einem menschenähnlichem sowie Roboterähnlichem Avatar besteht. Diese Vermutung bestätigten auch Bente, Rüggenberg und Krämer \citep[p.54-59]{bente2004social} indem sie 2004 eine Studie zur Sozialen Präsenz von Avataren in einem \ac{sve} durchführten. Das \ac{sve} war ähnlich einer Videokonferenz aufgebaut. Es waren keine \ac{hmd}'s vorhanden und die Teilnehmer haben sich während des Experiments nicht gesehen. Sie verglichen die Kommunikationarten, Face-to-Face, Chat und Avatarbasierte Kommunikationsmedien untereinander, um Unterschiede in der Social-Presence sowie dem zwischenmenschlichen Vertrauen festzustellen.
Es wurde festgestellt, dass wenig Kognitives-Vertrauen während der Nutzung des \ac{sve} zu Avataren aufgebaut werden konnte, während Face-to-Face, Telefon und Chatkommunikationen besser abschnitten. Weiterhin wurde weniger Affektives-Vertrauen im \ac{sve} als bei der Telefon oder Face-to-Face kommunikation aufgebaut.
Bente, Rüggenberg und Krämer gehen davon aus, dass dies mit der Neuheit der Technologie zusammenhängt.

------------ Hier nochmal überlegen, da das vorher geschriebene hierzu nicht passt. 

Da der kognitive Vertrauensaufbau in andere Personen auf einer rationalen Grundlage, die auf der Wahrnehmung von Kompetenz oder Zuverlässigkeit beruht, basiert, lässt sich die Vermutung aufstellen, dass ein \ac{ik} sowie ein \ac{nik} Avatar einen unterschiedlichen Einfluss auf die kognitive Vertrauensbildung haben.

Je realistischer und menschenähnlicher ein Avatar aussieht, desto eher könnte kognitives Vertrauen gebildet werden, da sich die Probanten eher vorstellen können mit menschlichen Wesen zusammenzuarbeiten als mit menschenähnlichen Avataren. Diese Einschätzung teilt auch Riedl \cite{riedl2014trusting}, indem er herausfand, dass es Personen leichter fällt einzuschätzen, wie vertrauenswürdig ein Mensch ist, wenn dieser ein menschenähnliches Gesicht besitzt.
Weiterhin ist durch eine erhöhte Selbstpräsenz eine bessere Identifizierung mit dem Avatar möglich. Zusammen mit dem Protheus-Effekt könnte es in einer gemeinsamen Teamaufgabe zu einer erhöhten Kognitiven Vertrauensbildung bei \ac{ik} Avataren kommen.

Connected to My Avatar:
Effects of Avatar Embodiments on User Cognitions, Behaviors,
and Self Construal 
%
%421
%https://sci-hub.do/10.1007/978-3-319-07632-4 
%file:///B:/Chrome%20Downloads/Biocca2014_Chapter_ConnectedToMyAvatar.pdf 


%Emotional Contagion with Artificial Others. Effects
%of Culture, Physical Appearance, and Nonverbal Behavior
%on the Perception of Positive/Negative Affect in Avatars 	

%412

%https://sci-hub.do/10.1007/978-3-319-07632-4
	

%https://dl.acm.org/doi/fullHtml/10.1145/223904.223935?casa_token=B8tEKM39OVQAAAAA:VNilOxaXG3_2Bw-bEClS10xyOXLBxB8ymyo4B-d1kUoAmCgWC1MDdVKSptRADsBaGw19nzE15dwIWQ

%http://mcneilllab.uchicago.edu/pdfs/dmcn_vietri_sul_mare.pdf

%https://sci-hub.ren/https://doi.org/10.1371/journal.pone.0025759 SEITE 2


\newpage

	\subsection{Teamwork}		
Was ist ein Team? Wie unterscheidet sich ein "normales" Team von einem Team, welches räumlich voneinander getrennt ist? "Normale" Teams, haben die Möglichkeit sich jederzeit oder mit einem geringem Zeitaufwand von Angesicht zu Angesicht zu treffen, während räumlich voneinander getrennte Teams nicht diese Möglichkeit besitzen.

		\subsubsection{Was ist ein Team?}
	Ein Team wird definiert als eine kleine Gruppe von Menschen mit gleichartigen Fähigkeiten, welche sich in gleicher Weise für das gleiche Ziel und gleiche Arbeitsweisen einsetzen und dies verfolgen.\citep[p.2]{zenun2007effects} \newline
Cohen \citep[p.557]{cohen1997makes} definiert ein Teams als "eine Ansammlung von Individuen, die in ihren Aufgaben voneinander abhängig sind, die die Verantwortung für die Ergebnisse teilen, die von sich selbst und von anderen als eine intakte soziale Einheit gesehen werden."
Das Verhalten von Personen, die in einem Team arbeiten, lässt sich in \glqq Teamwork \grqq und \glqq Taskwork \grqq unterteilen. \citep[p. 541-542]{rousseau2006teamwork}
Taskwork beschreibt dabei, was für eine Aufgabe ein Team erledigt, sowie, wie die Ausführung von Kernkompetenzen in einem bestimmten Bereich aussieht. 
Teamwork beschreibt, wie ein Team gemeinsam eine Aufgabe erledigt. Dies beinhaltet, wie sich interaktive sowie voneinander abhängige Verhaltensprozesse zwischen Mitgliedern des Teams auswirken um eine Aufgabe zu erledigen. \citep[p. 357]{marks2001temporally} 

Sala \citep[p.541]{salas2008teams} definiert Teams als "die voneinander abhängigen Leistungskomponenten, die erforderlich sind, um die Leistung mehrerer Personen effektiv zu koordinieren".

Teamwork beschreibt somit die gemeinsame Arbeitsleistung um eine Aufgabe zu erreichen.

Teambuilding zielt dabei darauf ab, die Haltung oder Einstellung der Personen innerhalb eines Teams zu verbessern um das gesamte Team zu stärken, während Teamtraining drauf abzielt, spezielle Fähigkeiten einzelner Personen zu fördern um das gesamte Team zu stärken. \citep[p. 367-369]{shuffler2011there}\
		
	Die wirtschaftliche Leistung von Unternehmen hängt häufig stark von der Arbeitseffizienz gut funktionierender Teams ab. Teambuilding kann somit helfen, die wirtschaftliche Leistung zu verbessern, indem Mitgliedern eines Teams weniger Fehler durch bessere Entscheidungen erzeugen. \citep[p. 1-6]{biech2007pfeiffer} 

\subsubsection{Was ist ein virtuelles Team?}

%https://sci-hub.ren/10.1111/j.1365-2575.2009.00326.x

Virtuelle Teams teilen sehr viele Eigenschaften eines herkömmlichem Team. Als aller erstes ist ein \dq virtuelles Team\dq ein \dq Team\dq. Anschließend muss unterschieden werden, wie die virtuelle Komponente des Teams definiert werden kann und was diese von einem herkömmlichem Team unterscheidet. 

Schweizer \citep[p.270]{schweitzer2010conceptualizing} erläutert die 4 Hauptkonditionen eines virtuellen Teams in der Literatur.
Virtuelle Teams sind :
\begin{itemize}
\item zustandegekommen aufgrund von Kommunikationstechnologie. Es wird durch technische Hilfsmittel Kommuniziert, entscheidungen getroffen etc.
\item räumlich getrennt. Virtuelle Teams arbeiten \textit{nicht} am selben Arbeitsplatz
\item grenzübergreifend. Es gibt zusammenarbeit der Teammitglieder aus verschiedenen Organisationen oder Organisationseinheiten.
\item Asynchron. Virtuelle Teams arbeiten zu unterschiedlichen Zeiten/Zeitzonen oder in der selben Zeitzone in unterschiedlichen Schichten.
\end{itemize}

Die vollständige Liste von Schweizer findet sich in \textbf{Abbildung} \autoref{criteriaForVirtualTeams}

Einige Authoren nehmen in ihre Definition eines virtuellen Teams den Aspekt der zeitlichen Limitierung oder der kulturellen Diversität mit auf. So ist ein virtuelles Team für Jarvenpaa \citep[p.1-2]{jarvenpaa1999communication} nur für einen bestimmten Zeitraum aufgebaut und das Team ist kulturell divers. 

Wird jedoch logisch über die Definition von Schweizer oder Jarvenpaa nachgedacht, muss ein virtuelles Team nicht zwangsläufig aus verschiedenen Organisationen bestehen oder asynchron arbeiten. So können Beispielsweise in einem Unternehmen in der selben Zeitzone ein virtuelles Team zusammenarbeiten. Ebenso ist es virtuell, wenn es nicht kulturell Divers oder zeitlich Limitiert ist.
Somit bleibt nur noch die Komponente des zustandekommens aufgrund von Kommunikationstechnologie sowie der räumlichen Getrenntheit.

Wird nun die Definition von Zenun \citep[p.2]{zenun2007effects} eines Teams herangezogen, und die virtuelle Komponente in diese mit eingebracht, kann ein virtuelles Team als

\flqq eine kleine, auf Kommunikationstechnolgie basierende, räumlich getrennte Gruppe, von Menschen mit gleichartigen Fähigkeiten, welche sich in gleicher Weise für das gleiche Ziel und gleiche Arbeitsweisen einsetzen und dies verfolgen.\frqq 

definiert werden.

Laut der obrigen Definition eines virtuellem Team ist jedes Team ein virtuelles Team. Nur selten arbeiten Team während der gesamten Lebensdauer nach vorheriger Definition. Es wird daher davon ausgegangen, dass Virtualität als Kontinuum gesehen werden kann, bei dem jedes Team einen gewissen Grad an Virtualität besitzt und das sich stetig ändern kann. Dieses Kontinuum reicht von Face-to-Face bis zur vollständigen nur über Kommunikationstechnologie stattfindende Kommunikation \cite{martins2004virtual}.


		\subsubsection{Virtuelle Teams und Teambuilding}
%	https://sci-hub.tw/https://doi.org/10.1108/13527590110395621

Seit einigen Jahren wurde die Wichtigkeit von effektiven Teambuildingmaßnahmen in der strategischen Organisationsentwicklung erkannt. Dabei spielt der Wandel hin zu einer globalen, auf Wissen basierten Wirtschaft eine zentrale Rolle. \citep{belbin2011management} \citep[p.7]{katzenbach2015wisdom}
Wirtschaftlicher Erfolg korreliert direkt mit der Fähigkeit eines Unternehmens Teams organisieren, strukturieren und managen zu können. \citep{pasmore1993designing}
Der Erfolg eines virtuelle Teams ist somit als Nebenprodukt der oranisatorischen Fähigkeiten eines Unternehmens zu verstehen. \citep[Chapter.5]{kling1994social}

Erfolgreiches Teambuilding können die Effektivität eines virtuellen Teams steigern und dazu führen, dass Personen sich mehr mit dem Team identifizieren. \citep{kaiser2000student}

Virtuelle Teams werden häufig gebildet um räumliche oder kurzzeitige Trennungen eines Teams zu umgehen. Dabei werden computergestützte Technologien so verwendet, dass räumlich getrennte Teammitglieder ihre Aufgaben, mittels computergestützter Kommunikation, im Team koordinieren können. \citep[p. 117-119]{peters2007identifying} \citep[p. 1-2]{cascio2003leadership}
Diese jedoch zu gründen, stellt nicht die Herausforderung dar. Die eigentliche Herausforderung ergibt sich aus den Unterschiedlichen Kulturen, Entfernungen und Zeitzonen, die ein Virtuelles Team mitbringt. Wird es geschafft, Vertrauen in das virtuelle Team zu bringen, kann der eigentliche Nachteil der verschiedenen Kulturen, Entfernungen und der Zeitzonen auch zum Vorteil werden. Es wird die Kulturelle Diversität gefordert und neue Verhaltensmuster erworben, wodurch neue, kreative Sichtweisen gefördert werden. Durch diese Faktoren ist es letztendlich möglich Innovativer zu Arbeiten und zu denken. \citep{dyer1995team} \citep[p.405-416]{milliken1996searching}

Die Arbeit kann Länderübergreifend stattfinden, was ein größeres Innovationspotential für Unternehmen zur folge hat.	
Virtuelle Teams haben den Ruf teuer zu sein, die Ziele nicht verfolgen zu können, nie pünktlich und schwierig zu managen zu sein. \citep[p.243-244]{gassmann2003trends}

Gerade für virtuelle Teams ist die Anfangsphase des Teambuilding von entscheidender Bedeutung. 
Die Kommunikation in virtuellen Teams ist gerade in dieser Phase sehr Ergebnisorientiert in der Art und Weise \textit{wie} kommuniziert wird. Dieses Defizit in der Sozialen Kommunikation untereinander kann die Schlüsselfaktoren eines Erfolgreichen Teams beeinträchtigen - Soziale und Emotionale Beziehungsbildung sowie den Aufbau von Vertrauen. \citep[p.378]{ren2007applying} \\
Um die Wahrscheinlichkeit zu erhöhen, dass auch in der Anfangsphase des Teambuildings eine \glqq Soziale\grqq und keine \glqq Arbeitsnahe, Ergebnisorientierte\grqq Kommunikation stattfindet, wurde eine spielerische Umgebung geschaffen, in der sich das Team das erste mal kennenlernen kann. 
Mitglieder von virtuellen Teams haben im Gegensatz zu traditionell geformten Teams weniger Möglichkeiten sich zu sehen, zu interagieren oder Konflikte zu lösen. 
Respekt und gegenseitiges Verständnis sind die Grundbausteine um Kreativität und Innovation innerhalb eines Teams zu fördern, die Effektivität eines Teams ist eine direkte Konsequenz daraus.

Vertrauensaufbau im Team nimmt eine wichtige Rolle in virtuellen Teams ein, denn im Gegensatz zu traditionellen geformten Teams haben Teammitglieder eines virtuellen Team keine Möglichkeit durch geselliges Beisammensein oder durch physischen Kontakt Bindungen aufzubauen um das gegenseitige Vertrauen zu stärken.\citep{TrustAndTheVirtualOrganisation}  \\
Vertrauen in einem Team zu fördern ist somit eine Notwendigkeit um Wachstum und Erfolg des Teams zu bestimmen. 
\citep{glacel1997teamwork} \\
Eine idiale Teambuilding Situation ist daher in virtuellen Teams nicht möglich. 
%\textbf{(Cianni andWnuck, 1997) ( Hier einfügen oder doch als eigenen Satz stehenlassen? )}
Virtuelle Teams werden trotz der Konsequenzen der räumlichen und zeitlichen Trennung gebildet, ohne die vorherigen Beschriebenen Vorteile in kauf nehmen zu können. Die Optimale Situation wäre es, in einem schon bestehendem Team eine virtuelle Komponente hinzuzufügen um auf die Vorteile von schon vorhandenen sozialen Bindungen zugreifen zu können. \citep[p.36-37]{holton2001building}
Es sollte sichergestellt werden, dass virtuelle Teams während ihres Bestehens bestmöglich in ihrem Aufbau von Vertrauen und sozialen Beziehungen unterstützt werden, um den Erfolg des Teams möglichst zu gewährleisten.


%		\subsubsection{Aktuelle Teambuildingmaßnahmen in virtuellen Teams}

%		\subsubsection{Kooperativ vs Kompetitive}
	
		\subsubsection{Teameffektivität}

Teameffektivität ist die Fähigkeit eines Teams, so miteinander zu interagieren uns sich so zu unterstützen, dass ein zuvor definiertes Ziel des Teams erreicht wird. Immer einen Fokus auf die Teameffektivität ist dahingehend wichtig, weil viele externe Faktoren zum erreichen oder nicht erreichen eines zuvor definierten Ziels beitragen können.
\citep[p.557]{salas2005there}
	Es gibt keinen einheitlichen Standard um die Performance eines Teams zu messen. Es wird davon ausgegangen, dass die Effektivität in Gruppen anhand der von der Gruppe produzierten Ergebnissen (Quantität, Qualität, Geschwindigkeit, Kundenzufriedenheit) gemessen werden kann. Eine Gruppe hat Einfluss auf die Produktivität der einzelnen Mitglieder, von diesem Effekt profitiert die gesamte Gruppe und trägt somit zur Verbesserung der Gesamteffektivität bei. \citep[p.309]{guzzo1996teams}
	
	Training im Team kann die gesamte Performance eines Teams steigern. Dies scheint am effektivsten, wenn mehrere Charakteristika des Teamworks auf einmal angesprochen werden. Diese sollten auch experimentelle Aktivitäten beinhalten um aktiv zu lernen und zu üben. \citep[19]{mcewan2017effectiveness}
Gemeinsames Training führt zu einer Steigerung der Qualität der Ideen und Entscheidungen sowie der gesamten Teamleistung.
	Die Kommunikation untereinander wird gefördert, da die einzelnen Teammitglieder gegenseitig ihre Aufgaben kennen und dadurch eher bereit sind sich untereinander zu helfen.
	Geteiltes wissen bedeutet größeren Lerneffekt, was mit einem besseren Verständnis über das Wissen des anderen einhergeht. Somit können auch individuelle Stärken gefördert, und schwächen durch andere Teammitglieder kompensiert werden.
	Ein Team bringt ein Gefühl von Sicherheit mit sich, was zu einer erhöhten Risikobereitschaft führt. Dies erhöht die Kreativität der eingebrachten Ideen können entstehen und Teammitglieder an der Möglichkeit größere Risiken einzugehen wachsen. \citep[p. 2-4]{biech2007pfeiffer}
	Dies führt zu einer Steigerung der Qualität der Ideen und Entscheidungen sowie der gesamten Teamleistung.
Auch das soziale Identitätsgefühl hängt davon ab, ob ein generelles Gruppenverständnis besteht, eine Person sich der Gruppe zugehörig fühlt und ob man sich zu als Gruppe mit anderen Gruppen vergleicht. Gruppenzugehörigkeit ist ein wichtiger Bestandteil des Selbstverständnisses eines Individuums. \citep{sutantovicious}
		Ist gute Gruppenzugehörigkeit gegeben, stärkt dies die Gruppenproduktivität sowie die individuelle Leistungsfähigkeit. Weiterhin entstehen dadurch Effekte die zum besserem Zusammenhalt, mehr Vertrauen \citep{herbsleb2000distance}, besserer Kommunikation und Kooperation untereinander führen. \citep[p. 510]{olson2003psychological}
		
Zum Messen von Teameffektivität wird in dieser Studie die Anzahl der abgeschlossenen Runden der Teambuildingaufgabe genutzt. Schafft ein Team eine Runde, hat dieses eine Teameffektivität von 1, schafft es 10 Runden, hat es eine Teameffektivität von 10.
\\
\textbf{Hyothese 4 H$_{1}$} : Die Konditionen \ac{ik} oder \ac{nik} haben \textbf{einen Signifikant abweichenden Einfluss} auf die Team-Effektivität.
\\
In dieser Hypothese wird sich mit der Frage beschäftigt, ob ein Team Effektiver ist, je nachdem was für eine Avatar Kondition diesem zugeordnet wird. Es wird davon ausgegangen, dass die Team-Effektivität nicht aufgrund der Menschenähnlichkeit variiert.
	
%	
%		\subsubsection{Teambuilding Komponenten}
%	\textit{\textbf{HIER NOCHMAL KURZ ERKLÄREN WOZU DIE TEILBEREICHE EIGENTLICH DA}}
%	Teambuilding kann in 10 verschiedene Teilbereiche unterteilt werden. Siehe \autoref{ten_characteristics}
%	\begin{figure}[H]
%		\begin{footnotesize}
%			\includegraphics[width=\textwidth]{Abbildungen/Ten_Characteristics.JPG}
%			\caption[Abbildung 1]{Ten Characteristics of a High Performance Team \citep[p. 27]{biech2007pfeiffer}}
%			\label{ten_characteristics}
%		\end{footnotesize}
%	\end{figure}
%	Die einzelnen Teilbereiche bauen aufeinander auf. Der untere Teilbereich mit den Komponenten \textit{Clear Goals"}, \textit{Defined Roles}, \textit{Open and Clear Communication} sowie \textit{Effective Decision Making} sind die Basis des gesamten Models. Diese Komponenten sollten am Anfang des Teambuildingprozesses definiert und immer als Basis der gesamten Teamarbeit gesehen werden.\\
%	Der Teilbereich mit den Komponenten \textit{Balanced Participation}, \textit{Valued Diversity} und \textit{Managed Conflict} baut auf diese auf.\\
%	\textit{Positive Atmosphere} und \textit{Cooperative Relationships} ist der dritte Teilbereich. In diesem geht es hauptsächlich um die Zufriedenheit und die Bereicherung eines einzelnen in einem Team zu arbeiten. Jedoch ist dieser Bereich nicht mehr Ausschlaggebend um eine Aufgabe zu erledigen. Für viele Teammitglieder ist dieser Teilbereich jedoch wichtigste Ziel während der Teamarbeit.\\
%	\textit{Participative Leadership} ist der einzige Bereich der aus dem Gesamtgebilde ohne Beeinflussung der anderen Komponenten entfernt werden kann. Dies sagt uns, dass ein Teamm auch ohne einen einzelnen Teamleiter existieren kann. Vgl. \citep[p. 13-16]{biech2007pfeiffer}\\
%	Aufgrund der vielen Komponenten die ein Team erfüllen muss um effizient zu sein und ein Gestärktes Teamgefühl zu bilden, wird in dieser Ausarbeitung nur auf einen Teilbereich eingegangen.
%	
%			\paragraph{Positive Atmosphere}
%	Wird der Aufbau von Vertrauen vernachlässigt, so bildet sich kein gut funktionierendes Team.
%			\paragraph{Open and Clear Communication}
%			\paragraph{Balanced Participation}
%			\paragraph{Defined Roles}
%	
%		\subsubsection{Team-Colloboraton}
%		\subsubsection{Virtual-Teambuilding}
		
%			\paragraph{Allgemeines}

%EINGEFÜGT BEI DEFINITION EINES VIRTUELLEM TEAMS
%		Virtuelle Teams werden häufig gebildet um räumliche oder kurzzeitige Trennungen eines Teams zu umgehen. Dabei werden computergestützte Technologien so verwendet, dass räumlich getrennte Teammitglieder ihre Aufgaben, mittels computergestützter Kommunikation, im Team koordinieren können. \citep[p. 117-119]{peters2007identifying} \citep[p. 1-2]{cascio2003leadership}
%		
%		Es kann effizient Länderübergreifend gearbeitet werden, was ein größeres Innovationspotential für Unternehmen zur folge hat.	
%		Virtuelle Teams haben den Ruf teuer zu sein, die Ziele nicht verfolgen zu können, nie pünktlich und schwierig zu managen zu sein. \citep[p.243-244]{gassmann2003trends}
%		
%		Virtuelle Teams sind häufig gerade in der Anfangsphase sehr Ergebnisorientiert in der Art und Weise \textit{wie} kommuniziert wird. Dieses Defizit in der Sozialen Kommunikation untereinander kann die Schlüsselfaktoren eines Erfolgreichen Teams beeinträchtigen - Soziale und Emotionale Beziehungsbildung sowie den Aufbau von Vertrauen. \citep[p.378]{ren2007applying} \\
%		Um die Wahrscheinlichkeit zu erhöhen, dass auch in der Anfangsphase des Teambuildings eine \glqq Soziale\grqq und keine \glqq Arbeitsnahe, Ergebnisorientierte\grqq Kommunikation stattfindet, wurde eine spielerische Umgebung geschaffen, in der sich das Team das erste mal kennenlernen kann. 
%		
%Virtuelle Teams werden in dieser Arbeit als temporär, divers und räumlich gerennt, beschrieben. Die Kommunikation untereinander nur durch das World-Wide-Web statt. Weiterhin impliziert die temporäre Eigenschaft der Definition, dass sich die Teammitglieder nicht kennen.
		
		
%		\subsubsection{Einflussfaktoren im Teambuilding}


%	Z.b. Aussehen, Kulturen, etc.	
\newpage
\section{Versuchshypothesen}

Aufgrund der vorherig genannten Forschungen, wurden folgende Hypothesen definiert :
\begin{itemize}
	\item{\underline{Hyothese 1} :\\ H$_{0}$ : Ein hoher General-Trust-Score wirkt sich \textbf{nicht positiv} auf den Cognitiven-Trust-Score aus. \newline
	H$_{1}$ :Ein hoher General-Trust-Score wirkt sich \textbf{positiv} auf den Cognitiven-Trust-Score aus.}
	
Es wird davon ausgegangen, dass Personen, die leicht einen Vertrauensvorschuss gewähren, nicht immer gleichzeitig auch in die Fähigkeiten oder in die Zuverlässigkeiten des Teams glauben.

\item{\underline{Hyothese 2} :\\ H$_{0}$ : Die Konditionen \ac{ik} oder \ac{nik} haben \textbf{keinen Signifikant abweichenden Einfluss} auf den Cognitive-Trust-Score. \newline
	H$_{1}$ : Die Konditionen \ac{ik} oder \ac{nik} haben \textbf{einen Signifikant abweichenden Einfluss} auf den Cognitive-Trust-Score. .}

%	\item{\underline{Hyothese 2} :\\ H$_{0}$ : Ein Torso, Head und Hand getrackter Avatar (IK) hat \textbf{keinen Einfluss} auf den Cognitiven-Trust-Score.\newline
%	H$_{1}$ : Ein Torso, Head und Hand getrackter Avatar hat \textbf{einen Einfluss} auf den Cognitiven-Trust-Score.}
	
Hat nun ein Head und Hand getrackter Avatar einen Einfluss auf die wahrgenommen Fähigkeiten oder die wahrgenommene Zuverlässigkeit in das Team? Genau diese Frage stellt sich diese Hypothese 2.
Es wird davon ausgegangen, dass, egal welche Kondition des Avatars ein Probant zur Verfügung gestellt bekommt, dies keinen Einfluss auf die wahrgenommenen Fähigkeiten des Teams hat. Somit erscheint ein Team in seiner Leistungsfähigkeit und Zuverlässigkeit immer gleich, egal ob ein Avatar Menschenähnlich ist oder nicht.

	\item{\underline{Hyothese 3} :\\ H$_{0}$ : Ein hoher Cognitiver-Trust-Score hat \textbf{keinen Einfluss} auf die Team-Effektivität bei unterschiedlichen Avatarverkörperungen. \newline
	H$_{1}$ : Ein hoher Cognitiver-Trust-Score hat \textbf{Einfluss} auf die Team-Effektivität bei unterschiedlichen Avatarverkörperungen.}
	
Es wird davon ausgegangen, dass wenn Personen oder Teams sich untereinander Leistungsfähig oder Zuverlässiger angesehen werden, dies keinen Einfluss auf die Team-Effektivität hat. 
	
	\item{\underline{Hyothese 4} :\\ H$_{0}$ : Die Konditionen \ac{ik} oder \ac{nik} haben \textbf{keinen Signifikant abweichenden Einfluss} auf die Team-Effektivität. \newline
	H$_{1}$ : 	Die Konditionen \ac{ik} oder \ac{nik} haben \textbf{einen Signifikant abweichenden Einfluss} auf die Team-Effektivität..}
	
%	\item{\underline{Hyothese 4} :\\ H$_{0}$ : Ein Torso, Head und Hand getrackter Avatar hat \textbf{keinen Einfluss} auf die Team-Effektivität. \newline
%	H$_{1}$ : Ein Torso, Head und Hand getrackter Avatar hat \textbf{Einfluss} auf die Team-Effektivität.}	
	
In dieser Hypothese wird sich mit der Frage beschäftigt, ob ein Team Effektiver ist, je nachdem was für eine Avatar Kondition diesem zugeordnet wird. Es wird davon ausgegangen, dass die Team-Effektivität nicht aufgrund der Menschenähnlichkeit variiert.

	\item{\underline{Hyothese 5} :\\ H$_{0}$ : Teams, aufgeteilt nach Avatarverkörperungen, mit einem hohem General-Trust-Score erzielen \textbf{keine höhere Team-Effektivität} als die mit einem niedrigen General-Trust-Score. \newline
	H$_{1}$ : Teams, aufgeteilt nach Avatarverkörperungen, mit einem hohen General-Trust-Score erzielen \textbf{eine hohere Team-Effektivität} als die mit einem niedrigen General-Trust-Score.}
\end{itemize}

Sind eventuell Teams in der \ac{vr} Effektiver, nur weil diese generell leichter einen Vertrauensvorschuss gewähren? Genau dieser Frage geht Hypothese 5 nach. Es wird jedoch davon ausgegangen, dass Teams keine höhere Teameffektivität erzielen, auch wenn diese einen höheren Hang zum Vertrauen haben.

%\textbf{NEU}
%
%\begin{itemize}
%	\item{H1 : Probanten mit einem hohen General-Trust-Score erzielen auch einen hohen Cognitiven-Trust-Score.}
%	\item{H2 : Probanten, die einen Torso, Head und Hand getrackter Avatar nutzen erzielen einen höheren Cognitiven-Trust-Score.}
%	\item{H3 : Ein hoher Cognitiver-Trust-Score wirkt sich positiv auf die Team-Effektivität aus.}
%	\item{H4 : Ein Torso, Head und Hand getrackter Avatar hat einen positiven Einfluss auf die Team-Effektivität.}
%	\item{H5 : Teams, mit einem generell hohem Hang zum Vertrauen erzielen auch eine hohere Team-Effektivität.}
%\end{itemize}


Diese wurden in einer Grafik Veranschaulicht um einen besseren Gesamtüberblick zu erhalten.

\begin{figure}[H]
		\begin{footnotesize}
			\includegraphics[width=\textwidth]{Abbildungen/Versuchshypothesen.JPG}\\
			\caption{Versuchshypothesen}
			\label{Versuchshypothesen}
		\end{footnotesize}
	\end{figure}	

		
	\section{Vorgehensweise}
	
	\subsection{Beschreibung des Forschungsverlaufs}
Insgesamt nahmen 30 Teilnehmer an dem Versuch Teil. Da ein Team aus 3 Teilnehmern bestand, waren es insgesamt 10 Teams. Jedes Team bekam entweder die Kondition \ac{ik} oder \ac{nik} zugeordnet. Somit gab es 5 \ac{ik} sowie 5 \ac{nik} Teams.
Jeder Probant bekam einen zufälligen Zeitslot sowie ein anonymen Namen zugeordnet mit dem dieser an dem Versuch teilnahm. Gemäß des A/B-Testings wurden jeweils drei Personen in einem Zeitslot untergebracht um ein "Team" zu bilden. Diesem "Team" wurde entweder die Kondition \ac{ik} oder \ac{nik} zugeordnet. Somit nahmen drei Probanten, an einem Versuch zur selben Zeit mit der selben Kondition, teil. Die Probanten wurden nicht Face-To-Face vorgestellt und sahen sich während des gesamten Versuchs nur als Repräsentation eines Avatars in dem \ac{sve}. Der Zeitslot von 30 Minuten teilte sich auf in
		\begin{itemize}
			\item 5 Minuten Pre-Questionnaire
			\item 5 Minuten Videoerklärung
			\item 15 Minuten Versuchsdurchführung
			\item 15 Minuten Post-Questionnaire
		\end{itemize}
Jeder Teilnehmer bekam zu beginn seines Zeitslots einen Pre-Questionnaire ausgehändigt, den dieser selbstständig ausfüllen sollte. In diesem wurden Fragen über die \flqq Person\frqq, über eventuelle \flqq Gesundheitliche Beschwerden\frqq sowie schon vorhandene \flqq \ac{vr}-Erfahrung\frqq, gestellt.
Dieser wurde von 30 Personen ausgefüllt.
Alle Probanten mussten sich anschließend ein Video über die Erklärung des Experiments anschauen. So wurde sichergestellt, dass alle Probanten den Selben Informationsgehalt über die Art und Weise des Ablaufs des Experiments bekamen. Alle Mitglieder eines Teams starteten somit mit den selben Informationen.
Nachdem alle Probanten die Videoerklärung angeschaut hatten, begann das Experiment. Dazu loggte sich das jeweilige Team in das \ac{sve} ein und spielten den Versuch durch.
		Am Ende der Versuchsdurchführung, wurde ein Post-Questionnaire ausgeteilt. In diesem wurden Fragen über das \flqq Generelles Vertrauen\frqq, das \flqq Kognitive Vertrauen\frqq die \flqq Kommunikations-Qualität\frqq, die wahrgenommene \flqq Team-Effektivität\frqq, die \flqq Beanspruchung\frqq sowie die \flqq Präsenz\frqq, gestellt. Am Ende war es jeden Teilnehmer zusätzlich möglich ein Feedback zu geben. Der Post-Questionnaire wurde von 30 Personen ausgefüllt. 
		
Die maximale Versuchsdauer nach Start der Anwendung betrug exakt 10 Minuten (600 Sekunden). Es konnten maximal 15 Runden absolviert werden, wobei jede Runde inkrementell schwieriger wurde, da jede 3. Runde jeweils 1 Symbol, in den Pool der zu erratenden Symbole, hinzukam. 

	\subsection{Allgemeiner Versuchsaufbau}

\begin{figure}[H]
		\begin{footnotesize}
			\includegraphics[width=\textwidth]{Abbildungen/Podeste.JPG}\\
			\caption[Abbildung 1]{Die Podeste der Teilnehmer}
			\label{Framework}
		\end{footnotesize}
	\end{figure}

Jeder Probant benötigte (neben einem funktionsfähigen Computer), um an dem Versuch teilnehmen zu können, entweder ein Oculus oder HTC-Vive \ac{hmd}, sowie zwei funktionsfähige Controller.

Dem Versuchsleiter war es während der gesamten Anwendung möglich, die 3 Probanten durch einen eigenen Client zu betreuen. Alle Probanten konnten durch Ihre Anwendung durch das integrierte Mikrofon im \ac{hmd} zu dem Versuchsleiter Sprechen und diesen hören. Während die Probanten sprachen, konnten die anderen zwei Probanten diese jedoch nicht hören. Die Sprachkommunikation eines Probanten war somit nur Richtung Versuchsleiter möglich. Dies dient zur Vermeidung von einigen Störvariablen und zum Erhalt der Integrität der Anonymität.
Während der Versuchsleiter sprach, konnten jedoch alle Probanten den Versuchsleiter hören. Dies diente dazu, eventuelle offene Fragen an alle Versuchsteilnehmer weiterzugeben und den Beginn sowie das Ende der Anwendung zu kommunizieren. Der Versuchsleiter gab jedoch während des gesamten Versuchs keine Hilfestellung.

In der Anwendung war es den Probanten möglich mittels der Tasten \flqq W A S und D\frqq ihre Position und mittels den Tasten \flqq Q\frqq und \flqq E\frqq ihre Höhe, in einem gewissen Bereich, zu ändern. Dies stellte sicher, dass alle Probanten ihre Avatargröße und Position individuell an ihre Körpergröße anpassen konnten.

Da es den Probanten nicht möglich sein sollte, auf die Symbole der anderen teilnehmenden Probanten zu schauen, wurde ab einer gewissen Grenze Links und Rechts des Podest der Probanten ein \flqq Fade-To-Black\frqq Mechanismus eingebaut. Kamen die Probanten mit dem Kopf ihres Avatars in diesen Bereich, sahen diese nur noch die Farbe Schwarz und mussten sich zurückbewegen.

Waren alle Probanten bereit, wurden diese vor ein für Sie eigenes Podest teleportiert und sahen einen Countdown zwischen den 3 Podesten herunter zählen. Dieser Countdown litt den baldigen beginn einer Runde ein.
Wurde eine Runde gestartet, wurde das Podest jedes Probanten eineindeutig durch die Farbe \flqq Schwarz, Rot oder Grün\frqq markiert. Die Probanten konnten Ihre und die Farbe des Spielers an einem Viereck sowie einer runden Kugel in der jeweils zugeteilten Farbe, an den jeweiligen Podesten, erkennen. Die Farbe der einzelnen Probanten änderte sich jede Runde.

	\subsection{Der Versuch}

Der Schwarz markierte Probant war mit dem Erklären in der jeweiligen Runde an der Reihe.
Auf dem jeweiligen Podest eines Probant befanden sich verschiedene Symbole. Diese Symbole waren für alle Probanten gleich, jedoch an unterschiedlichen Positionen. Der Schwarz markierte Probant hatte an den Symbolen noch entweder die Farbe \flqq Grün oder Rot\frqq oder \flqq Grün und Rot\frqq.
Der schwarz markierte Probant versuchte nun, mittels Hand und Armbewegung, den Rot und Grün markierten Probant die Symbole, die in der jeweiligen Spielerfarbe vor Ihm markiert waren, zu erklären.
Meinte ein Probant (Rot oder Grün), dem der schwarz markierte Probant ein Symbol erklärt hat, ein Symbol erkannt zu haben, loggte dieser das Symbol durch herunterdrücken des Knopfes an seinem Podest, mit dem jeweiligen erkanntem Symbol, ein. Der Schwarz markierte Probant musste nichts einloggen oder klicken, sondern nur durch Gestikulation erklären.
Hat ein Probant sich während des einloggens verklickt oder wollte seine Meinung zum erkanntem Symbol ändern, musste das Symbol vorher, durch erneutes klicken des Symbols vor ihm, ausgeloggt werden. Anschließend konnte es erneut eingeloggt werden.
Wurden alle Symbole vom roten und grünem Probant erkannt und eingeloggt, die der schwarz markierte Probant angezeigt hat, wurde dies durch eine grüne Kugel, für alle Sichtbar, erkennbar gemacht und die Runde endete. Erschien diese grüne Kugel nicht, war noch etwas falsch und der schwarz markierte Probant musste noch einmal versuchen die korrekten Symbole den jeweiligen Mitspielern aufzuzeigen.
In der nächsten Runde wurde ein anderer Probant eineindeutig mit Schwarz, Rot oder Grün markiert.
Mit steigender Anzahl an erfolgreich bestandenen Runden, müssen mehr Symbole erfolgreich erkannt werden.

Das Ziel war es nun, so viele Symbole wie möglich gemeinsam korrekt zu bearbeiten und dadurch in höhere Runden aufzusteigen.

	\subsection{Die Avatare}
Die Avatare im \ac{sve} wurden als \flqq IK-Avatar\frqq und \flqq Non-IK-Avatar\frqq implementiert.
Für beide Avatararten wurde die neutrale Farbe Schwarz gewählt. Die Probanten hatten keinen Einfluss auf die Farbwahl ihres eigenen Avatars. Dies dient dazu die Störvariable der eventuell vorhandenen Vorurteile des Mitspielers aufgrund der Farbe des Avatars, auszuschalten. Je nach Kondition \flqq IK-Avatar\frqq oder \flqq Non-IK-Avatar\frqq bekam jeder Probant das Selbe aussehen des Avatars zugeordnet. Der eigene Avatar ist nicht für die Mitspieler sichtbar. Zur besseren Interaktion mit den Knöpfen sowie zur höheren Immersion, wurde der eigene Avatar nicht sichtbar für den Probanten dargestellt. Jeder Probant, unabhängig der zugewiesenen Kondition sah somit nur eine Repräsentation von menschlich wirkenden Händen.

	\begin{figure}[H]
		\begin{footnotesize}
			\includegraphics[width=\textwidth]{Abbildungen/Avatars.JPG}\\
			\caption[Abbildung 1]{Links: IK-Avatar, Rechts: Non-IK-Avatar}
			\label{Framework}
		\end{footnotesize}
	\end{figure}

		\paragraph{IK-Avatar}
Der Invers-Kinematisch getrackte Avatar hat ein Androgynes aussehen, um die Störvariable rund um Vorurteile aufgrund des Geschlechts der Probanten auszublenden. Der IK-Avatar besitzt keine Augen, Mund, Haare oder Beine. 
Um eine Möglichst realistische Bewegung zu erzielen, wurde die Handbewegung, die Unterarmbewegung, die Oberarmbewegung sowie die Kopf-und Brustrotation simuliert.

		\paragraph{Non-IK-Avatar}
Der Non-IK-Avatar besteht aus einem Kreis mit Mund sowie einer Repräsentation der Linken sowie der Rechten Hand. Er Besitzt keine Augen, Haare, Beine, Hals, Torso sowie Ober- und Unterarm. Der Mund dieses Avatars ist als unauffälliger Orientierungspunkt des Kopfes erhalten geblieben. Der Mund des Non-IK-Avatars bewegt sich jedoch nicht. 
	\newpage
	
	\subsection{Entwicklung der Versuchsumgebung}
	Kein Teleport
	\subsubsection{Raumerschaffung in Normcore}
	\subsubsection{Sync Mechanisms}
	\subsubsection{Aussehen der Umgebung}
	Geschlossener Raum, Gras, Lichtstrahlen, Neutrale Kacheln etc.
	\subsubsection{Spectator}
	Kein Avatar zugeordnet
	\subsubsection{Client}
	XRMultiRig
	\subsubsection{Rundenschwierigkeit}
	Wie wurden die Runden schwieriger? -> Da auf die ScriptableObjects der Runden eingehen
	\subsubsection{Avatarkonditionen}
	Unsichtbare Avatare, RPC Calls um diese zu tauschen
	\subsubsection{GameManager und PodestManager}
	\subsubsection{Gametimer funktionsweise und immer zum Spieler gedreht}
	\subsubsection{Audioübertragung}
	\subsubsection{FADE TO Black}
	\subsubsection{"Quasi"-Randomisierung des Leaders }
	Wurde das nun eigentlich implementiert? Oder wurde das auch durch die SCObjects abgedeckt?
	\subsubsection{CSV-Logwriting}
	\subsubsection{Physical Handmodel, eigener Avatar war unsichtbar}
	\subsubsection{PlayerDisconnection Handling, Player Kick Mechanism, Destroying of Network Audio Receiver}
	\subsubsection{Symbole und Podest, Drückmechanismus, Colliderplatzierung}
	Eingeloggt ist Gelb umrandet etc auch erklären, Ausloggen etc. Depth of button
	\subsubsection{PlayerHeight/Position Change WASDQE}
	\subsubsection{GlowingBall Animation, HandFingeranimation}
	\subsubsection{Automatic Position Set when Spectator Pressed Start}
	\subsubsection{2 Versions -> OVR and HTC Version}
	\subsubsection{SOUNDS}
	Player Connect, Disconnect, Button Click, Spectator Audio
	\subsubsection{UML Diagramme der Funktionsweise}
			\subsubsection{Verwendete Hardware}
Um an dem Experiment Teilnehmen zu können, benötigten die Probanten ein im vollem Umfang funktionierendes HTC-Vive, HTC-Vive2 oder ein Oculus-Rift S \ac{hmd} mit funktionsfähigen Controllern sowie einen Leistungsstarken \ac{vr}-Fähigen PC. Der Spectator, der das Experiment Remote steuern und verwaltet, benötigt einen Leistungsstarken PC auf dem die Anwendung ausführbar ist.

			\subsubsection{Verwendete Software}
Das \ac{sve} wurde mit Unity 2019.4.3f1 und der HD-Render-Pipeline programmiert. Um die Echtzeitkommunikation zwischen den einzelnen Clients zu gewährleisten, wurde das multiplayer Framework \flqq Normcore v2.0\frqq \footnote{www.Normcore.io} genutzt.
Normcore unterstützt Network-Physics, automatische Realtime-Synchronisation, Voice-Chat, XR-Compitabilität sowie persistente multiplayer Räume.	
	
	\subsection{Untersuchungsmethode}
	\subsection{Qualitative vs Quantitative Untersuchung}
			
	\subsubsection{A/B Testing}
Um die beiden Unabhängigen Variablen \flqq Non-IK-Avatar\frqq und \flqq IK-Avatar\frqq innerhalb einer Teambuildingmaßnahme zu testen, wurde die Testmethode des A/B-Testings gewählt.
Gruppe A bekam dabei die Kondition \flqq Non-IK-Avata\frqq zugeteilt, während Gruppe B die Konditiion \flqq IK-Avatar\frqq zugeteilt bekam. Diese Aufteilung, welcher Probant welche Kondition zugeteilt bekam, erfolgte nach dem Zufallsprinzip. 

	\subsection{Teilnehmerfindung}
Da das gesamte Experiment als Into-the-Wild Experiment aufgebaut wurde, wurde in verschiedenen Foren ( z.B. VRForum.de, Computerbase.de, Hardwareluxx.de, etc. ) durch eine Suchanfrage, in Form eines extra dafür angelegten Threads, Teilnehmer gesucht. Weiterhin wurden Teilnehmer durch verschiede Facebookgruppen mit einem Bezug zu \ac{vr}, sowie durch zufällige Whatsapp-Chatgruppen mit 50+ Mitgliedern, akquiriert. Da das gesamte Experiment, sowie die Versuchsanleitung auf deutscher Sprache erstellt wurde, fand die Teilnehmerfindung nur im deutschsprachigem Raum statt.
Die Voraussetzung um an dem Versuch Teilzunehmen, ist es, ein \ac{hmd}, zwei Controller sowie ein Computer mit Internetzugang zu besitzen. Der gesamte Versuch ist als ein "Into the Wild" Versuch aufgebaut, welches bedeutet, dass die genaue Zielgruppe nicht genau definiert werden kann. Die einzigen Ristriktion an die Zielgruppe ist, dass diese die zu verwendete Hardware besitzen.\\
		
		
	\subsection{Unabhängige Variablen}
Als unabhängige Variablen wurden der \flqq IK-Avatar\frqq sowie der \flqq Non-IK-Avatar\frqq gewählt. Durch diese  unabhängigen Variablen wurde im weiterem Versuchsverlauf versucht den Einfluss auf die abhängigen Variablen zu bestimmen.
Insofern wird untersucht, wie der Einfluss von dem \flqq IK-Avatar\frqq sowie dem \flqq Non-IK-Avatar\frqq auf das Generelle-Vertrauen, das Vertrauen im Team sowie auf die Team-Effektivität ist.
		\subsubsection{Avatar embodiement}
			\paragraph{Head- and Handtracking - \ac{nik}}
			\paragraph{Head-,Hand and Inversekinematic-Torso - \ac{ik}}
	
	\subsection{Abhängige Variablen}
		\subsubsection{Hang zum  Vertrauen}
Der Hang zum Vertrauen bezieht sich in dieser Studie darauf, wie sehr die Probanten generell anderen Personen einen Vertrauensvorschuss geben.
Es wird davon ausgegangen, dass Probanten, die einen höheren Hang zum Vertrauen besitzen, ebenfalls ein höheres Kognitives Vertrauen zu Tage legen. 
			Wie in \textbf{Abbildung \ref{ten_characteristics}} beschrieben \textbf{ABBILDUNG 1 BESCHREIBEN, TRUST z.B. UND EINZELNE PUNKTE}, ist die Positive Atmosphere in einem Team einer der Maßgeblichen Faktoren die zur Leistungsfähigkeit und zum \textit{\glqq wir-gefühl\grqq} beitragen.
			Die Abhängige Variable Vertrauen ...	
			%\paragraph{Swift-Trust}
			%\subsubsection{Kommunikation}
			%\subsubsection{Team Zusammenhalt}
			%\subsubsection{Teambuilding Potential}
		\subsubsection{Kognitives Vertrauen}
Das Kognitive Vertrauen bezieht sich in dieser Studie darauf, wie sehr die Probanten glauben, dass das Team die ihnen gestellte Aufgabe korrekt und mit Sorgfalt erledigen.
		\subsubsection{Vertrauen in das Team}
		\subsubsection{Team Effektivität}
			%\subsubsection{Andere Faktoren}
			
	\subsection{Residuen}
		\subsubsection{Aufbau der Umgebung}
		\subsubsection{Farbe der Avatare}
	\subsection{Geschlecht}
			BSPW. Auf die Androgynität des Avatars eingehen
		\subsubsection{Sozialpsychologische Einflussfaktoren}
			\paragraph{Gamification} $~$ \\
			\paragraph{Aussehen} $~$ \\
			Dodds fand heraus, dass ein Selbst-Avatar eine wichtiger Faktor zur Kommunikation in einem \ac{sve} ist. \citep[1-11]{dodds2011talk}
			
			Black Avatars :
			https://www.sciencedirect.com/science/article/abs/pii/S1053810013000597
			
			\paragraph{Sprache} $~$ \\	
			
Allgemeine Kommunikation findet nicht nur mit Wörtern, sondern über Körpersprache wie Gestiken oder andere nonverbale Verhaltensmuster, statt. Durch \ac{sve}'s ist es nun möglich diese reine Text oder Bildbasierte Kommunikation durch Gestikulationen eines Avatars zu ersetzen. Durch diesen zusätzlichen visuellen Aspekt ist eine ganz neue Art und Weise der Kommunikation möglich, die nun gemessen werden kann. 
Nichtverbale Kommunikation -Beispielsweise nur via Text- erzeugt keinen Mehrwert an Vertrauen, Zusammenhalt und erzielt keine ausreichend gute Kommunikation. \citep[p.81]{haslam2003social}			

Nonverbale Kommunikation kann Gesichtsausdrücke, Blicke, Bewegungen etc. umfassen. Kendon definiert im \flqq Kendon Kontinuum\frqq mit dem Begriff \flqq Gestik\frqq unwissende Gestiken (natürliche Körpersprache) bis hin zu \frqq Zeichen\flqq, welche alle durch Gestikulation erzeugten Zeichen ( z.B. O.K. Handzeichen ), beinhaltet. \citep[37]{mcneill1992hand} 
Im Rahmen dieser Arbeit, ist mit Gestikulation jedoch die gesamte Körperbewegung des Avatars gemeint.

Mehrabian zeigt in seiner Forschung, dass nonverbale Kommunikation zu 55\% dafür verantwortlich ist, ob ein gesprochenes Wort Positiv, Negativ oder Neutral interpretiert wird. Die Tonhöhe trägt zu 38\% und das eigentlich gesprochene Wort mit 7\% zur Interpretation bei. \citep[43]{mehrabian1971silent}

Da 7\% des gesprochenen Wortes und 38\% der Tonhöhe zu Interpretation der nonverbalen Kommunikation beiträgt, musste diese Einflussgröße in dieser Studie ausgeschlossen werden.

%An Investigation into Gender Role Conformity in an Online Social
%Networking Environment.......................................... 322
%
%Gesture and facial expression
%Gesture is an important part of conversation and ranges from almost sub-conscious accompaniment to speech to complete and well formed sign languages for the deaf. Support for gesture implies that we need to consider what kinds of 'limbs' are present. Facial expression also plays a key role in human interaction as the most powerful external representation of emotion, either conscious or sub-conscious. Facial expression seems strongly related to gesture. However, the granularity of detail involved is much finer and the technical problems inherent in its capture and representation correspondingly more difficult. A crude, but possibly effective approach, might be to texture map video onto an appropriate facial surface of a body image (e.g. the "Talking Heads" at the Media Lab [2]). Another approach involves capturing expression information from the human face using an array of sensors on the skin, modelling it and reproducing it on the body image (e.g. the work of ATR where they explicitly track the movement of a user's face and combine it with models of facial muscles and skin [6] and also the work of Thalmann [10] and Quéau [7]).
%This discussion of gesture and facial expression relates to a further issue, that of voluntary versus involuntary expression. Real bodies provide us with the ability to consciously express ourselves as a supplement or alternative to other forms of communication. Virtual bodies can support this by providing an appropriate set of limbs and 'strings' with which to manipulate them. The more flexible the limbs; the richer the gestural language. However, we suspect that users may find ways of gesturing with even very simple limbs. On the other hand, involuntary expression (i.e. that over which users have little control) is also important (looks of shock, anger, fear etc.). However, support for this is technically much harder as it requires automatic capture of sufficiently rich data about the user. This is the real problem we are up against with the facial expression issue - how to capture involuntary expressions.

			\paragraph{Bekanntheit}
Um die gegenseitige Bekanntheit der Probanten auszuschließen, wurde jedem Probanten zu beginn des Versuchs ein Zufälliger Name zugeordnet. Vor dem gesamten Versuch konnten sich die Probanten nicht sehen oder hören.
Hätten die Probanten sich vor dem Versuch gesehen, würde sich ein Kognitives-Vertrauen oder Kognitives-Misstrauen zu der anderen Person aufbauen. 
Kennen sich die Personen untereinander schon, wirkt sich dies auf die Vertrauensbildung in der \ac{vr} in Form von schon vorhandenem Affektiven-Vertrauen auf die Ergebnisse aus.
			
	

	\subsection{Datenerhebungsmethoden}
			
		\paragraph{Demografie-Fragebogen}
Bevor der Versuch stattfinden konnte, mussten die Probanten einen Demographie-Fragebogen ausfüllen. Dieser diente allgemeine demografische Merkmale wie z.B. Das Alter, Geschlecht und den Bildungsstand abzufragen. Weiterhin wurde der demografische Fragebogen dazu genutzt, um die bisher vorhandene \ac{vr}-Erfahrung, die PC- und Internetaffinität sowie die Videospielerfahrung der einzelnen Probanten besser einschätzen zu können.
\\Ein Auszug des Demografie-Fragebogen befindet sich in \textbf{Anhang \ref{Pre-Questionnaire}}

		\paragraph{Generalized-Trust-Scale}
Der Generalized-Trust-Scale entwickelt von Couch, Adams und Jones \citep{couch1996assessment} wurde eingesetzt um das generelle Vertrauen der einzelnen Probanten zu messen.
Dieser Fragebogen besteht aus 20 Fragen. Diese sind Beispielsweise \flqq Ich neige dazu, andere zu akzeptieren \frqq oder\flqq Meine Beziehungen zu anderen werden durch Vertrauen und Akzeptanz charakterisiert \frqq und wurden auf einer 7-Point-Likert-Scale von \flqq 1: Ich stimme gar nicht zu\frqq bis zu \flqq 7 : Ich stimme voll zu\frqq gemessen. Der Generalized-Trust-Scale ist nur ein Teilauszug des \flqq Trust-Inventory von Couch\frqq, welches noch einen \flqq Partner-Trust-Scale \frqq beinhaltet. Dieser wurde für diese Forschung jedoch nicht benötigt. 
\\Ein Auszug des Generalized-Trust-Scale befindet sich in \textbf{Anhang \ref{Post-Questionnaire - Generelles Vertrauen}}

%			
%			\textbf{The propensity to trust measure developed by Couch, Adams, & Jones (1996) was
%administered in order to gauge participants’ trusting dispositions. The Generalized Trust subscale
%of this measure was utilized for this study. It contains 20 items (e.g. “I have few difficulties
%trusting people”) and participants rated how strongly they agree with each statement on a 1
%(strongly disagree) to 7 (strongly agree) scale. See Appendix B for the full scale. The full trust
%inventory by Couch et al. (1996) also contains a Partner Trust subscale, but this was not relevant
%to the study as it pertains to trust of one’s romantic partner.}
%https://sci-hub.ren/10.1207/s15327752jpa6702_7
%\citep{couch1996assessment}

		\paragraph{Cognitive-Trust-Scale}
Der Cognitive-Trust-Scale ist ein Teilauszug des von McAllister und Daniel J. (1995) \citep[p.37]{mcallister1995affect} entwickeltem Fragebogen. Der Cognitive-Trust-Scale dient dazu, herauszufinden, wie viel Cognitives Vertrauen die Probanten während der Teambuildingmaßnahme aufgebaut haben. Dieser Teilauszug des Fragebogens umfasst 5 Fragen. Diese sind Beispielsweise \flqq Diese Person geht an ihre Arbeit mit Professionalität und Hingabe heran \frqq oder\flqq Andere Personen, die mit diesen Personen interagieren müssen, halten ihn/sie für vertrauenswürdig und wurden mittels eines 5-Point-Likert-Scale gemessen. Die Antwortmöglichkeiten des Likert-Scale erstrecken sich von \flqq 1: Ich stimme gar nicht zu \frqq bis zu\flqq 5 : Ich stimme voll zu. 
\\Ein Auszug des Cognitive-Trust-Scale befindet sich in \textbf{Anhang \ref{Post-Questionnaire - Kognitives Vertrauen}}

  

%			\textbf{After each bomb was completed (successfully or not), participants filled out a cognitive trust scale developed by Wildman et al. (2009) and based on the trust theory of Lewicki, McAllister, & Bies (1998). This 8-item scale taps into participants’ trust attitudes and each item is rated on a 5-point scale from “not at all” to “very much so”. While this measure has not yet been published, it has been validated in both lab and field samples and has shown utility in prior teamwork research Lazzara, 2013; Wildman, 2011. Appendix C contains the full scale}
%\citep{mcallister1995affect}

		\paragraph{Communication-Scale}
Gonzales-Rom, Vincente und Hernandez \citep[p.1049]{gonzalez2014climate}entwickelten 2004 einen Fragebogen um die Teamkommunikations Qualität zu messen. Dieser Fragebogen umfasst 5 Fragen und wurden mit einem 5-Point-Likert-Scale von \flqq 1:Gar nicht\frqq bis \flqq 5:Sehr\frqq. Die Fragen sind alle nach dem selben Prinzip aufgebaut und es wurde jeweils nur das Ende einer Frage abgeändert : \flqq In welchem Umfang war die Kommunikation zwischen Ihnen und Ihrem Team KLAR/EFFEKTIV/ABGESCHLOSSEN/FLÜSSIG/ZUM RICHTIGEN ZEITPUNKT/?\frqq 
\\Ein Auszug des Communication-Scale befindet sich in \textbf{Anhang \ref{Post-Questionnaire - Teamkommunikation}}
%			After the experimental portion was completed, participants completed another set of measures. First, participants completed the Communication Quality Scale developed by Gonzalez-Roma & Hernandez 2014. This scale contains 5 items that assess participants perceptions of their teams communication quality, rated on a 1 to 5 scale. The full measure is available in Appendix D
%\citep[p.1049]{gonzalez2014climate}

		\paragraph{Team-Effectivness-Scale}
Gibson \citep[p.469]{gibson2003team} entwickelte 2003 einen Fragebogen um Teameffektivität zu messen. In dieser Studie wurde ein Teilauszug dieses Fragebogens verwendet um das subjektive Außmaß der Qualität - indem das Team fehlerfreie Arbeit erledigt - zu messen. Dieser Teilauszug umfasst 5 Fragen welche mithilfe einer 7-Point-Likert-Scale gemessen werden \flqq 1:Ich stimme gar nicht zu \frqq bis \flqq 7: Ich stimme voll zu \frqq. Die Fragen Zielen dabei auf die Qualität der Teambuildingmaßnahme ab, wie z.B. \flqq Mein Team hat eine geringe Fehlerquote \frqq oder \flqq Mein Team hat eine hohe Qualität \frqq. 
\\Ein Auszug des Team-Effectivness-Scale befindet sich in \textbf{Anhang \ref{Post-Questionnaire - Team-Effektivität}}

%quality dimension is the extent to which the team produces error-free work.
%			\textbf{Next, participants completed a 5-item team effectiveness measure. This measure is
%comprised of the Quality subscale from Gibson, Zellmer-Bruhn, and Schwab's (2003) Team
%Outcome Effectiveness survey. Participants rated their agreement with each of the five items on
%a 1 to 7 scale. The team effectiveness measure is available in Appendix E.
%Last, participants indicated whether or not they were already familiar with their
%teammate. If so, they were also asked to indicate approximately how many years they had known
%their teammate, and how often they communicated with the teammate. Finally, teams were
%prompted to write a few sentences about why they thought their team performed the way it did.
%Effectiveness Measures} \citep[p.469]{gibson2003team}

		\paragraph{NASA-TLX}
Der NASA-TLX ist ein multidimensionaler Fragebogen. Dieser Fragebogen findet einen direkten Einsatz nachdem Probanten eines Experimentes eine Aufgabe erledigt haben. Der NASA-TLX dient dazu, herauszufinden, wie allgemeine Belastung auf den Probanten herauszufinden.
Es insgesamt 6 Fragen auf einer 21-Punkte Skala abgefragt. Diese beinhalten die Mentale Anforderung, die Physische Anforderung, die Zeitliche Anforderung, die Leistung, die Anstrengung, und die Frustration des Probanten. \cite{NASATLX}
Der originale NASA-TLX besitzt eine kontinuierliche Skala. Dies war jedoch nicht über einen Online-Fragebogen realisierbar, weshalb die Werteskala abgeändert wurde. Diese Werteskala umfasst im abgeänderten Fragebogen \flqq 1: Wenig \frqq über \flqq 11: Mittel \frqq bis \flqq 21: Viel \frqq, wobei die Skalenbeschriftung je nach abgefragtem Item anders beschriftet ist.
\\Ein Auszug des abgeänderten NASA-TLX befindet sich in \textbf{Anhang \ref{Post-Questionnaire - NASA-TLX}}


		\paragraph{Igroup Presence Questionnaire(IPQ)}
Der IPQ dient zu Messung des Präsenzgefühls in einer virtuellen Umgebung. Dabei misst der IPQ, inwieweit sich der Nutzer in der virtuellen Umgebung anwesend fühlt, inwieweit der Nutzer seine Aufmerksamkeit der virtuellen Umgebung schenkt und wie Real die virtuelle Umgebung dem Nutzer erschien. Der IPQ umfasst 14 Fragen welche mithilfe einer 7-Point-Likert-Scale gemessen werden.
\\Ein Auszug des Igroup Presence Questionnaire(IPQ) befindet sich in \textbf{Anhang \ref{Post-Questionnaire - IPQ}}
		
		\paragraph{Co-Präsenz-Questionnaire}
Der Co-Präsenz-Fragebogen dient dazu die Selbst gemeldete Co-Präsenz, die wahrgenommene Präsenz des \flqq anderen \frqq, die Telepräsenz sowie die Social-Präsenz zu messen. Die Selbst gemeldete Co-Präsenz sowie die wahrgenommene Präsenz des \flqq anderen \frqq wurde mittels einer 5-Point-Likert-Skala gemessen. Die Telepräsenz wurde mittels einer 7-Point-Likert-Skala gemessen. Die Social-Präsenz wird im originalen Fragebogen von Nowak und Biocca mittels einer kontinuierlichen Skala gemessen. Da dies jedoch mit dem Online-Fragebogen nicht realisierbar war, wurde eine Likert-Skala von 1 bis 10 stattdessen verwendet. \citep[p.487]{nowak2004effect}
\\Ein Auszug des Co-Präsenz-Questionnaire befindet sich in \textbf{Anhang \ref{Post-Questionnaire - Co-Präsenz}}

%\newpage
%
%\begin{table}
% \centering\footnotesize\setstretch{1}\linespread{1.5}\rmfamily
%\caption{Verwendete Metriken des Fragbogens}
%\begin{tabular}{ |p{3.5cm}|p{3.5cm}|p{3.5cm}|p{3.5cm}| }
%
%\hline
%\underline{Was wurde gemessen?} & \underline{Definition} & \underline{Metrik} & \underline{Authoren}\\
%
%\hline
%\textbf{Generelles Vertrauen} & Genereller Vertrauensvorschuss eines Individuums & Teilauszug des Trust-Inventorys - \textit{Generalized-Trust-Scale}\newline 7-Point-Likert-Scale & Couch, Adams, Jones (1996) \citep{couch1996assessment} \\
%
%\hline
%\textbf{Kognitives-Vertrauen} & Überzeugung in die Fähigkeiten oder in die Zuverlässigkeit eines anderen Individuums & \textit{Cognitive-Trust-Scale}\newline 5-Point-Likert-Scale & McAllister, Daniel J. (1995) \citep{mcallister1995affect} \\
%\hline
%
%\hline
%\textbf{Team-Kommunikation} & Wahrgenommene Kommunikationsqualität des Teams & \textit{Communication-Quality-Scale}\newline 5-Point-Likert-Scale & Gonzalez-Roma, Hernandez (2014) \citep[p.1049]{gonzalez2014climate} \\
%\hline
%
%\hline
%\textbf{Team-Effektivität} & Wahrgenommenes ausmaß der Qualität der Aufgabenerledigung des Teams & Teilauszug des \textit{Team-Effectiveness-Scale}\newline 7-Point-Likert-Scale & Gibson et. al (2003) \citep[p.469]{gibson2003team} \\
%\hline
%
%\hline
%\textbf{Subjektiv-Wahrgenommene Arbeitsbelastung} & Der NASA-TLX misst die subjektiv wahrgenommene Arbeitsbelastung sowie Effektivität & \textit{NASA-TLX} & \cite{NASATLX} \\
%\hline
%
%\hline
%\textbf{Räumliche Präsenz \newline \newline
%	Umschlossenheitsgefühl der \ac{vr} \newline \newline
%	Erfahrener Realismus 
%} & Der IPQ dient zu Messung des Präsenzgefühls in einer virtuellen Umgebung & \textit{Igroup Presence Questionnaire(IPQ)} &  \cite{IPQ}\\
%\hline
%
%\hline
%\textbf{Co-Präsenz \newline \newline
%	die wahrgenommene Präsenz des \flqq anderen \frqq \newline \newline
%	Telepresence \newline \newline
%	Social-Presence
%	} & Fragebogen von Nowak und Biocca zum messen von Co-Präsenz & \textit{Presence-Questionnaire} & Nowak und Biocca (2004) \citep[p.487]{nowak2004effect}  \\
%\hline
%
%\end{tabular}
%\end{table}

\clearpage
\newpage

\begin{table}
	\centering\setstretch{1}
		\caption{Verwendete Metriken des Fragbogens}
		\label{wasWurdeGemessen}
	\begin{tabular}{p{3.5cm} p{3.5cm} p{3.5cm} p{3.5cm}}

\underline{Was wurde gemessen?} & \underline{Definition} & \underline{Metrik} & \underline{Authoren}\\
    
    \hline
\multirow{2}{3.5cm}{Generelles-Vertrauen}
	&Genereller Vertrauensvorschuss eines Individuums & Teilauszug des Trust-Inventorys - \textit{Generalized-Trust-Scale} \newline \textit{Siehe Tabelle \ref{Post-Questionnaire - Generelles Vertrauen}} \newline &Couch, Adams, Jones (1996) \citep{couch1996assessment}\\
	&\,& 7-Point-Likert-Scale & \, \\
    
    \hline
\multirow{2}{3.5cm}{Kognitives-Vertrauen}
	&Überzeugung in die Fähigkeiten oder in die Zuverlässigkeit eines anderen Individuums & \textit{Cognitive-Trust-Scale} \newline \textit{Siehe Tabelle \ref{Post-Questionnaire - Kognitives Vertrauen}} & McAllister, Daniel J. (1995) \citep{mcallister1995affect}\\
	&\,& 5-Point-Likert-Scale & \, \\
    
    \hline
\multirow{2}{3.5cm}{Team-Kommunikation}
	&Wahrgenommene Kommunikationsqualität des Teams & \textit{Communication-Quality-Scale} \newline \textit{Siehe Tabelle \ref{Post-Questionnaire - Teamkommunikation}} & Gonzalez-Roma, Hernandez (2014) \citep[p.1049]{gonzalez2014climate}\\
	&\,& 5-Point-Likert-Scale & \, \\
    
    \hline
\multirow{2}{3.5cm}{Team-Effektivität}
	&Wahrgenommenes ausmaß der Qualität der Aufgabenerledigung des Teams & Teilauszug des \textit{Team-Effectiveness-Scale} \newline \textit{Siehe Tabelle \ref{Post-Questionnaire - Team-Effektivität}} & Gibson et. al (2003) \citep[p.469]{gibson2003team}\\
	&\,& 7-Point-Likert-Scale & \, \\
    
    \hline
\multirow{2}{3.5cm}{Subjektiv-Wahrgenommene Arbeitsbelastung}
	&Der NASA-TLX misst die subjektiv wahrgenommene Arbeitsbelastung sowie Effektivität & \textit{NASA-TLX} \newline \textit{Siehe Tabelle \ref{Post-Questionnaire - NASA-TLX}} & \cite{NASATLX}\\
    
    \hline
\multirow{2}{3.5cm}{Räumliche Präsenz \newline Umschlossenheitsgefühl \newline Erfahrener Realismus}
	&Der IPQ dient zu Messung des Präsenzgefühls in einer virtuellen Umgebung & \textit{Igroup Presence Questionnaire(IPQ)} \newline \textit{Siehe Tabelle \ref{Post-Questionnaire - IPQ}} & \cite{IPQ}\\
    
    \hline
\multirow{2}{3.5cm}{Selbst wahrgenommene Co-Präsenz \newline \newline Wahrgenommene Präsenz des \flqq anderen\frqq \newline \newline Telepresence \newline \newline Social-Presence}
	&Fragebogen von Nowak und Biocca zum messen von Co-Präsenz & \textit{Presence-Questionnaire} \newline \textit{Siehe Tabelle \ref{Post-Questionnaire - Co-Präsenz}} & Nowak und Biocca (2004) \citep[p.487]{nowak2004effect}\\

	\end{tabular}
\end{table}
\clearpage

			\paragraph{Fragenbogen}
Es wurden zwei Fragenbögen an die jeweiligen Studienteilnehmer verteilt. Der erste Fragenbogen wurde an die Probanten vor der eigentlichen Studie ausgefüllt. Ziel war es, die Einstellungen zu Teamarbeit sowie zu interpersonellem Vertrauen der Personen vor der eigentlichen Untersuchung festzustellen. Der Zweite Fragebogen wurde nach der Untersuchung ausgefüllt, um die Effektivität der verschiedenen Konditionen der Untersuchung auf Erfolg oder Misserfolg untersuchen zu können.
				Es wurden nur vollständig Ausgefüllte Fragebögen zur Datenanalyse herangezogen. Konnte nicht ermittelt werden, welche Aussage angekreuzt wurde, wurde dieser ebenfalls nicht mit in die Datenanalyse aufgenommen. Eine Außnahme war dabei, falls ein Teilnehmer im Nachhinein nach die von Ihm gewünschte Antwort befragt werden konnte.
			\paragraph{Beobachtung}
				Durch die Beobachtung während des Experiments werden Kennzahlen zur Dauer des Versuchs zwischen den einzelnen Konditionen bestimmt. Anhand dieser kann im späterem Verlauf das  Teambuilding Potential sowie die subjektive Effektivität durch gesteigertem Swift-Trust analysiert werden.
			\paragraph{Induktive Quantitative Forschungsmethodik}
				Anhand der subjektiven Betrachtungsweise einzelner Personen des Themas "Vertrauen" wurde ein quantitatives Forschungsdesign gewählt anhand die Ergebnisse induktiv interpretiert und ausgewertet wurden.


			\paragraph{Bereinigte Fragen}
Um die interne Konsistenz der einzelnen Skalen zu verbessern, wurden einige Fragen aus dem Fragebogen bereinigt. Dabei wurde versucht, das Cronbachs-Alpha $\alpha$ optimalerweise auf ($\alpha > 0.8$), akzeptablerweise ($\alpha > 0.7$) und fragwürdigerweise auf ($\alpha > 0.6$) zu setzen. Dabei wurde während der Bereinigung darauf geachtet, dass nicht zu viele Fragen entfernt wurden und trotzdem eine interne Konsistenz aufrecht gehalten wurde. Eine Tabelle über die Anzahl der bereinigten Fragen im Fragebogen, die Anzahl der Teilnehmer sowie die Alphawerte befindet sich in \textbf{Tabelle \ref{alphavalues}} \\
\textbf{\textit{Generalized-Trust-Scale}}
\begin{itemize}
	\item Ich habe etwas Schwierigkeiten, Leuten zu vertrauen
	\item Meine Erfahrungen haben mir zeigen mir, dass es besser ist, anderen zu misstrauen, bis man diese besser kennt
	\item Nur ein Narr würde den meisten Personen vertrauen
	\item Es ist besser, Fremden zu misstrauen, bis man sie besser kennt
	\item Ich neige dazu, andere beim Wort zu nehmen
	\item Ich habe kein Vertrauen in andere Personen.
	\item Selbst in schlechten Zeiten denke ich, dass am Ende alles gut wird
\end{itemize}

\textbf{\textit{Cognitive-Trust-Scale}}
\begin{itemize}
	\item Wenn die Menschen mehr über diese Personen und ihren Hintergrund wüssten, würden sie sich mehr Sorgen machen und ihre Leistung genauer beobachten
\end{itemize}

\textbf{\textit{Team-Effektivität}}
\begin{itemize}
	\item Mein Team muss ihre Arbeitsqualität verbessern
\end{itemize}

\textbf{\textit{NASA-TLX}}
\begin{itemize}
	\item Wie erfolgreich haben Sie Ihrer Meinung nach die vom Versuchsleiter (oder Ihnen selbst) gesetzten Ziele erreicht? Wie zufrieden waren Sie mit Ihrer Leistung bei der Verfolgung dieser Ziele? 
\end{itemize}

\textbf{\textit{Igroup Presence Questionnaire(IPQ)}}
\begin{itemize}
	\item Ich achtete noch auf die reale Umgebung
	\item Wie real erschien Ihnen die virtuelle Welt?
	\item Ich hatte das Gefühl, nur Bilder zu sehen. 
\end{itemize}

\textbf{\textit{Co-Präsenz}}
\begin{itemize}
	\item Ich wollte ungern persönliche Informationen mit meinen Interaktionspartnern teilen
	\item Ich wollte eine gewisse Distanz zwischen mir und den Interaktionspartnern wahren
	\item Ich wollte keine engere Beziehung mit meinen Interaktionspartnern haben
	\item Wie involvierend war das Ergebnis?
	\item Mein Interaktionspartner schuf eine gewisse Distanz zwischen uns
	\item Meine Interaktionspartner kommunizierten eher \flqq kalt\frqq als \flqq warm \frqq
\end{itemize}

\newpage
\begin{table}
	\centering\footnotesize\setstretch{0.8}
	\caption{Bereinigung der Fragebögen}
	\label{alphavalues}
\begin{tabular}{ p{3.5cm} p{3.5cm} p{3.5cm} }

\underline{Fragebogen-Skala} &  & \underline{Werte}\\
    
    \hline
\multirow{3}{3.5cm}{Generalized-Trust-Scale}
	&Anzahl der Fragen im original Fragebogen (Bereinigter Fragebogen) \newline & 20 (13)\\
    &\textit{n} \newline &30\\
    &$\alpha$ (bereinigtes $\alpha$) & .122 (.806)\\
    
    \hline
\multirow{3}{3.5cm}{Cognitive-Trust-Scale}
	&Anzahl der Fragen im original Fragebogen (Bereinigter Fragebogen) \newline & 6 (5)\\
    &\textit{n} \newline &30\\
    &$\alpha$ (bereinigtes $\alpha$) & .309 (.721)\\
    
    \hline
\multirow{3}{3.5cm}{Team-Kommunikation}
	&Anzahl der Fragen im original Fragebogen (Bereinigter Fragebogen) \newline & 5 (5)\\
    &\textit{n} \newline &30\\
    &$\alpha$ (bereinigtes $\alpha$) & .823 (.823)\\
    
    \hline
\multirow{3}{3.5cm}{Team-Effektivität}
	&Anzahl der Fragen im original Fragebogen (Bereinigter Fragebogen) \newline & 5 (4)\\
    &\textit{n} \newline &30\\
    &$\alpha$ (bereinigtes $\alpha$) & .501 (.886)\\
    
    \hline
\multirow{3}{3.5cm}{NASA-TLX}
	&Anzahl der Fragen im original Fragebogen (Bereinigter Fragebogen) \newline & 6 (5)\\
    &\textit{n} \newline &30\\
    &$\alpha$ (bereinigtes $\alpha$) & .289 (.605)\\
    
    \hline
\multirow{3}{3.5cm}{Igroup Presence Questionnaire(IPQ)}
	&Anzahl der Fragen im original Fragebogen (Bereinigter Fragebogen) \newline & 14 (11)\\
    &\textit{n} \newline &30\\
    &$\alpha$ (bereinigtes $\alpha$) & .471 (.801)\\
    
    \hline
\multirow{3}{3.5cm}{Co-Präsenz}
	&Anzahl der Fragen im original Fragebogen (Bereinigter Fragebogen) \newline & 30 (24)\\
    &\textit{n} \newline &30\\
    &$\alpha$ (bereinigtes $\alpha$) & .581 (.710)\\

\end{tabular}
\end{table}
\clearpage
\section{Auswertung/Ergebnisse}
	
	Das folgende Kapitel beschäftigt sich mit der statistischen Analyse. Jede Hypothese besitzt ein eigenes Unterkapitel in der eine auf die Hypothese zugeschnittene Auswertung stattfindet. 
	
	\subsection{Teilnehmer und Demografie}
Insgesamt haben 30 Teilnehmer an dem Versuch teilgenommen. Der Mittelwert betrug ($\bar{x}$ = 30,13) und die Standartabweichung ($\sigma$ = 7,44). Somit ergibt sich eine Spannweite von 38 Jahren. Siehe Anhang \textbf{\autoref{fig:Teilnehmer}}.\\
Von diesen 30 Teilnehmern waren 19(63,3\%) Männlich und 11(36,7\%) Weiblich. Siehe Anhang \textbf{\autoref{fig:biologischesGeschlecht}}.\\
Von den 30 Teilnehmern, besaß/en 1(3,3\%) das Fachabitur/Fachgebundene Hochschulreife, 4(13,3\%) das Abitur/Allgemeine Hochschulreife, 24(80\%) ein Abgeschlossenes Studium und 1(3,3\%) eine abgeschlossene Ausbildung. Siehe Anhang \textbf{\autoref{fig:teilnehmerBildungsstand}}.\\
Bei 10(33,3\%) war noch keine VR-Erfahrung vorhanden, während bei 20(66,6\%) schon VR-Erfahrung vorhanden war. Siehe Anhang \textbf{\autoref{fig:teilnehmerVRErfahrung}}.\\ 
Von den 20 Personen, die schon VR-Erfahrung hatte, haben 7(23,3\%) schon Erfahrungen mit VR-Experimenten oder Studien. \textbf{\autoref{fig:teilnehmerVorexperimente}} \\
Mittels einer Likert-Skala von 1-7 (1 = wenig, 7= sehr viel) wurde nach dem Außmaß der Internetnutzung im täglichem Leben gefragt. Dabei lagen 25(75\%) Personen oberhalb des 25\% Perzentil, wobei der Mittelwert ($\bar{x}$ = 6,27) und die Standartabweichung ($\sigma$ = 1,172) betrug. \textbf{\autoref{fig:teilnehmerPCAußmaß}} \\ 
Mittels einer weiteren Likert-Skala von 1-7 (1 = wenig, 7= sehr viel) wurde danach gefragt, wie Häufig die Teilnehmer Videospiele spielen. Der Mittelwert der Likert-Skala beträgt 3,5, wobei im 50\% Perzentil das Videospiel Außmaß mit "3" beziffert wurde. Der Mittelwert betrug ($\bar{x}$ = 3,1) und liegt damit ebenfalls unter dem Durchschnitt. Die Standartabweichung betrug ($\sigma$ = 1,826). \textbf{\autoref{fig:teilnehmerVideospieleAußmaß}}

	
	\subsection{Validität und Reliabilität}
		Validität : Eventuell gibt es für die Probanten der Studie einen "Neuheitseffekt", weshalb unzureichend das Vertrauen analysiert werden kann. Der "Wow"-Effekt ist eventuell sehr groß. Weiterhin kann es sein, dass durch die Tatsache, dass einige Leute schon die VR Kennen, eine andere Einstellung zu dem Thema haben und deshalb nicht mehr ganz Valide antworten.
		Weshalb ist meine Forschung Valide? Weshalb ist sie reliabel?	
		\textit{BEISPIEL : Die Forschung ist valide, da immer dieselbe Waage verwendet wurde, die exakt nach europäischen Standards wiegt. Zudem wurde die Reliabilität der Waage während der Untersuchung täglich getestet, indem ein Kilogramm Blei darauf gelegt wurde. Jedes Mal zeigte die Waage dabei ein Kilo an.}
	\subsection{Gesammelte Daten}
	
	\subsection{Analyse}
Die folgende Abbildung zeigt eine diskriptive Statistik der Hauptvariablen des Experiments.
\begin{table}[H]
	\caption{Variablen, Mittelwerte, Standartabweichungen und Anzahl der Teilnehmer}
	\label{VariableBreakdown}
	\begin{tabularx}{\textwidth}{p{3cm} | p{3.5cm} p{1.5cm} p{2cm} p{2cm} p{2cm}} 
		Was wurde gemessen? & Variablenbeschreibung & Min/Max & Mittelwert & Std. Abweichung & $N$ \\
		\hline \\
		Genereller Hang zum Vertrauen &\ac{gt} & $1/7$ & $5,1795$ & $.70857$ & $30$ \\ \\
		Kognitives Vertrauen &\ac{ct} & $1/5$ & $4,2467$ & $.61629$ & $30$ \\ \\
		Kognitives Vertrauen - IK &\ac{cti} & $1/5$ & $4,0133$ & $.0,66533$ & $15$ \\ \\
		Kognitives Vertrauen - NON IK &\ac{ctn} & $1/5$ & $4,48$ & $.47689$ & $15$ \\ \\
		Team-Kommunikation &\ac{tcn} & $1/5$ & $4,2467$ & $.61629$ & $30$ \\ \\
		Team-Kommunikation - IK &\ac{tci} & $1/5$ & $4,0133$ & $.66533$ & $15$ \\ \\
		Team-Kommunikation - NON IK &\ac{tcn} & $1/5$ & $4,48$ & $.47689$ & $15$ \\ \\
		Team-Effektivität &\ac{te} & $1/7$ & $4,9583$ & $1,40056$ & $30$ \\ \\
		Team-Effektivität - IK &\ac{tei} & $1/7$ & $4,6$ & $1,63881$ & $15$ \\ \\
		Team-Effektivität - NON IK &\ac{ten} & $1/7$ & $5,3167$ & $1,04994$ & $15$ \\ \\
		NASA-TLX &\ac{tlx} & $1/21$ & $7,16$ & $2,83203$ & $30$ \\ \\
		IPQ &IPQ & $1/7$ & $4,43$ & $.98339$ & $30$ \\ \\
		Co-Präsenz &\ac{cp} & $1/5$ & $3,6583$ & $.60790$ & $30$ \\ \\
		Co-Präsenz - IK &\ac{cpi} & $1/5$ & $3,622$ & $0,66805$ & $15$ \\ \\
		Co-Präsenz - NON IK &\ac{cpn} & $1/5$ & $3,6944$ & $.56248$ & $15$ \\ \\	
	\end{tabularx}
\end{table}
\clearpage

Einige Variablen wurden in dieser Studie auf dem Konditionslevel und andere auf dem Individual-Level gemessen.
Konditionslevel bedeutet dabei dabei, dass das die Ergebnisse zwischen der Kondition \ac{ik} sowie \ac{nik} aufgeteilt wurden, während auf dem Individual-Level alle Probanten einzeln bewertet wurden. Alle Variablen, die auf dem individuellem Level gemessen wurden, haben eine Stichprobengröße $N = 30$, während die Variablen, die auf dem Teamlevel gemessen wurden eine Stichprobengröße von $ N = 15$.

	\subsection{Test auf Normalverteilung der Daten}
Für viele Tests ist eine Voraussetzung, dass die Stichproben Normalverteilt sind. Zur Überprüfung auf Normalverteilung wurde der \textit{Kolmogoroff-Smirnov-Test} genutzt. Die Null-Hypothese dieses Tests besagt, dass eine Normalverteilung der Variablen vorliegt. Wenn ($p < \alpha = 0.05$), muss H0 verworfen werden und angenommen werden, dass die Daten nicht Normalverteilt sind.

Auf dem individuellem Level konnten mit dem Kolmogoroff-Smirnoff-Test alle Daten auf Normalverteilung mit ($p>0.05$) nachgewiesen werden. Siehe \textbf{\autoref{fig:KolSmirInd}} \\
Auf dem Teamlevel konnte bei \ac{tci} mit ($p = .039 < \alpha = 0.05$) keine Normalverteilung nachgewiesen werden. Siehe \textbf{\autoref{fig:KolSmirIndIK}} \\ Weiterhin konnte bei \ac{cpn} mit ($p = .044 < \alpha = 0.05$) keine Normalverteilung nachgewiesen werden. Siehe \textbf{\autoref{fig:KolSmirIndNONIK}}
\newpage
	\subsection{Analyse Hypothese 1}
H0 : Ein hoher General-Trust-Score wirkt sich \textbf{nicht positiv} auf den Cognitiven-Trust-Score aus.\\
\ac{gt} sowie \ac{ct} wurden im vorhinein auf Ausreißer überprüft. 

\begin{figure}[H]
\centering
		\begin{footnotesize}
			\includegraphics[scale=0.5]{Abbildungen/Post_QuestionnaireStatistiks/boxplot_cognitive_trust_ausreißer}\\
			\caption{Boxplot Kognitives Vertrauen}
			\label{fig:boxplot_cognitive_trust_ausreißer}
		\end{footnotesize}
	\end{figure}	

		\subsubsection{Test auf Ausreißer}
Es sind bei \ac{ct} Ausreißer vorhanden. Um diesen zu entfernen, wurde \ac{ct} im vorhinein Winsorisiert. Siehe \textbf{\autoref{fig:boxplot_cognitive_trust_winsorisiert}}. 
Insgesamt haben ($N=30$) Personen an dem Versuch teilgenommen. \ac{gt} ist unabhängig vom \ac{ct} der einzelnen Testperson. Daher muss eine Korrelation auf Individuallevel durchgeführt werden, bei der die einzelnen Teams und Teamzusammensetzung sowie die Avatarkonditionen ignoriert werden. 

		\subsubsection{Pearson-Korrelationsanalyse}
Um diese Hypothese zu überprüfen, wurde eine Pearson-Korrelationsanalyse durchgeführt. 

\begin{figure}[H]
\centering
		\begin{footnotesize}
			\includegraphics[scale=0.8]{Abbildungen/Post_QuestionnaireStatistiks/pearson_korrelation_general_cognitive}\\
			\caption{Pearson-Korrelation}
			\label{fig:pearson_korrelation}
		\end{footnotesize}
	\end{figure}	
Es ist eine Signifikant positive Korrelation mittlerem Effektes mit dem Pearson-Korrelationskoeffizient ($r = .452$) zwischen \ac{gt} und \ac{ct} vorhanden. Weiterhin ist eine Signifikanz mit ($p = .012 < \alpha = 0.05$) zu erkennen. Dies deutet darauf hin, unsere \textit{(Hypothese 1 / H$_{0}$)} zu verwerfen und die Alternativhypothese \textit{(Hypothese 1 / H$_{1}$)} anzunehmen.

		\subsubsection{Regressionsanalyse}
Um nun die gerichtete Annahme der positiven Auswirkung des \ac{gt} auf den \ac{ct} zu bestätigen, wurde eine Regressionsanalyse durchgeführt. Wobei als unabhängige Variable \ac{gt} und als abhängige Variable \ac{ct} gewählt wurde.

\begin{figure}[H]
\centering
		\begin{footnotesize}
			\includegraphics[scale=0.8]{Abbildungen/Post_QuestionnaireStatistiks/regression_gt_ct_modell}\\
			\includegraphics[scale=0.8]{Abbildungen/Post_QuestionnaireStatistiks/regression_gt_ct_anova}\\
			\includegraphics[scale=0.8]{Abbildungen/Post_QuestionnaireStatistiks/regression_gt_ct_koeffizienten}\\
			\includegraphics[scale=0.5]{Abbildungen/Post_QuestionnaireStatistiks/diagramm_regression_gt_ct}\\
			\caption{Pearson-Korrelation}
			\label{fig:regressionsanalyse}
		\end{footnotesize}
	\end{figure}	

Das Bestimmtheitsmaß ($R^{2} = .205$) deutet laut \citep{cohen2013statistical} auf eine geringe bis mittlere Varianzaufklärung des \ac{ct} durch \ac{gt} hin. Somit lassen sich $20,5\%$ der Varianzen unseres \ac{ct} durch den \ac{gt} erklären. \\
Der Regressionskoeffizient der Variable \ac{gt} beträgt ($r = .452$) und ist signifikant ($t(28) = 2,684; p = .012 < \alpha = 0.05$). \\
Weiterhin ist das Ergebnis der ANOVA ($F(1,28) = 7,202; p = .012 < \alpha = 0.05$) und es lässt sich \textit{(Hypothese 1 / H$_{0}$)} zum Signifikanzniveau ($\alpha = 0.05$) verwerfen. \\
Der \ac{gt} eignet sich als Prädiktor für \ac{ct}. Die geschätzte Zunahme an \ac{ct} beläuft sich auf $0.456$ Punkte \ac{ct} pro \ac{gt} ($\beta = 0.456; t(28) = 2,684; p = .012 < \alpha = 0.05$). \\
Der \ac{gt} erklärt ebenso einen signifikanten Anteil der Varianz von \ac{ct} ($R^{2} = .205; F(1,28) = 7,202; p = .012 < \alpha = 0.05$)
\newpage
	\subsection{Analyse Hypothese 2}
H$_{0}$ : Die Konditionen \ac{ik} oder \ac{nik} haben \textbf{keinen Signifikant abweichenden Einfluss} auf den Cognitive-Trust-Score. .\\
Es wird der Zusammenhang zwischen den unabhängigen Variablen \ac{ik} und \ac{nik} sowie \ac{cti} und \ac{ctn} auf Konditionslevel analysiert. Dazu wird ein T-Test durchgeführt. Die Gruppierungsvariablen sind dabei als \ac{ik} und \ac{nik} definiert.

			\paragraph{Voraussetzungen für den T-Test}
\ac{ctn} ist laut Kolmogoroff-Smirnov-Test mit ($p = .2 > \alpha = 0.05$) Normalverteit. \ac{cti} ist ebenfalls laut Kolmogoroff-Smirnov-Test mit ($p = .059 > \alpha = 0.05$) Normalverteilt. Siehe \textbf{\autoref{fig:KolSmirIndNONIK}} und \textbf{\autoref{fig:KolSmirIndIK}}.
Als nächstes wurde auf Varianzgleichheit überprüft. Der Levene-Test zeigt eine Varianzgleichheit zwischen \ac{cti} und \ac{ctn} von ($L = .392 < \alpha = 0.05$). Damit können wir davon ausgehen, dass die Gruppen gleiche Varianzen haben.

\begin{figure}[H]
\centering
		\begin{footnotesize}
			\includegraphics[scale=0.8]{Abbildungen/Post_QuestionnaireStatistiks/Varianzgleichheit_Levene_cti_ctn}
			\caption{Levene-Test auf Varianzgleichheit von \ac{cti} und \ac{ctn}}
			\label{fig:Varianzgleichheit_Levene_gt_ct}
		\end{footnotesize}
	\end{figure}	
	
		\subsubsection{T-Test}
	\begin{figure}[H]
\centering
		\begin{footnotesize}
			\includegraphics[width=\textwidth]{Abbildungen/Post_QuestionnaireStatistiks/h2_ttest_gruppenstatistik}
			\includegraphics[width=\textwidth]{Abbildungen/Post_QuestionnaireStatistiks/ttest_cti_ctn}
			\caption{T-Test zwischen \ac{gt} und \ac{ct} mit dem Faktor IK und NON-IK}
			\label{fig:ttest_gt_ct}
		\end{footnotesize}
	\end{figure}	

Der T-Wert ist kleiner Null ($T = -1,685 < 0$), was auf einen Mittelwertsunterschied zwischen \ac{ct} von \ac{ik} und \ac{nik} hinweist. Der Mittelwert $\bar(x)$ der Kondition \ac{ik} ist kleiner als der Mittelwert der Kondition \ac{nik}.
Der T-Test zeigt bei der Differenz des durchschnittlichen \ac{ct} bei der Kondition \ac{ik} ($\bar(x) = 4,3067; \sigma = .45898$) und \ac{nik} ($\bar(x) = 45733; \sigma =.40614$) jedoch keine Signifikanz ($t(28) = -1,685; p = .103 > \alpha = 0.05$).
Somit kann \textit{(Hypothese 2 / H$_{0}$)} nicht verworfen werden.



\newpage
\subsection{Analyse Hypothese 3}
H$_{0}$ : Ein hoher Cognitiver-Trust-Score hat \textbf{keinen Einfluss} auf die Team-Effektivität bei unterschiedlichen Avatarverkörperungen.

		\subsubsection{Aufstellen der kognitiven Vertrauenstabelle pro Team}
Es wird der Zusammenhang zwischen \ac{cti} sowie \ac{ctn} und \ac{tei} sowie \ac{ten} analysiert.
Jeweils 3 Teammitglieder besitzen die selbe Teameffektivität. Daher wurden die Teams zu jeweils einer Variable Zusammengefasst. Für jedes Team wird eine gemeinsamer Kognitiver-Vertrauenswert berechnet. Dieser sagt aus, wie viel Kognitives Vertrauen das gesamte Team untereinander besitzt. Dieser Kognitive Vertrauenswert ergibt sich aus der Summe der Kognitiven-Vertrauensangaben der einzelnen Personen eines Teams.
Da die individuellen Vertrauenswerte zusammengefasst wurden, wird der Ausreißer \flqq ID 1\frqq ignoriert mit in die Berechnung eingezogen. Somit müssen die Kognitiven Vertrauenswerte nicht winsorisiert werden. Alle Ergebnisse werden in \ac{ik} und \ac{nik} aufgeteilt um ein Vergleich der verschiedenen Konditionen darzustellen. Für diesen Vergleich wurde eine Pearson-Korrelationsanalyse durchgeführt.

\begin{table}[H]
	\caption{Kognitive-Vertrauenswerte - Individuell und pro Team zusammengefasst}
	\label{VariableBreakdown}
	\begin{tabularx}{\textwidth}{r | p{2cm} p{1.5cm} p{1.5cm} p{1.5cm} p{1.5cm} p{1.5cm} p{1.5cm} } 
		ID & Individueller-Kognitiver Vertrauenswert & Kondition & Erfolgreich abgeschlossene Runden & Team ID & Summe \ac{ct}\\
		\hline 
		1 & $2,00$ & 1 & 4 & Team 1 & $11,00$\\
		2 & $4,60$ & 1 & 4 & &\\
		3 & $4,40$ & 1 & 4 & & \\
		\hline 
		4 & $4,00$ & 1 & 10 & Team 2 & $12,00$\\
		5 & $4,00$ & 1 & 10 & &\\
		6 & $4,00$ & 1 & 10 & &\\
		\hline 
		7 & $5,00$ & 1 & 7 & Team 3 & $13,80$\\
		8 & $3,80$ & 1 & 7 & & \\
		9 & $5,00$ & 1 & 7 & & \\
		\hline 
		10 & $5,00$ & 1 & 11 & Team 4 & $13,80$ \\
		11 & $4,00$ & 1 & 11 & & \\
		12 & $4,80$ & 1 & 11 & & \\
		\hline 
		13 & $4,20$ & 1 & 13 & Team 5 & $12,00$ \\
		14 & $4,20$ & 1 & 13 & & \\
		15 & $3,60$ & 1 & 13 & & \\
		\hline 
		16 & $5,00$ & 2 & 9 & Team 6 & $14,60$ \\
		17 & $4,60$ & 2 & 9 & &\\
		18 & $5,00$ & 2 & 9 & &\\
		\hline 
		19 & $5,00$ & 2 & 12 & Team 7 & $14,80$ \\
		20 & $4,80$ & 2 & 12 & & \\
		21 & $5,00$ & 2 & 12 & & \\
		\hline 
		22 & $4,80$ & 2 & 8 & Team 8 & $13,00$ \\
		23 & $4,40$ & 2 & 8 & & \\
		24 & $3,80$ & 2 & 8 & & \\
		\hline 
		25 & $4,20$ & 2 & 7 & Team 9 & $13,00$ \\
		26 & $4,40$ & 2 & 7 & & \\
		27 & $4,40$ & 2 & 7 & & \\
		\hline 
		28 & $4,60$ & 2 & 9 & Team 10 & $13,20$ \\
		29 & $4,80$ & 2 & 9 & &\\
		30 & $3,80$ & 2 & 9 & & \\	
	\end{tabularx}
\end{table}

Laut Kolmogoroff-Smirnoff-Test sind sowohl die Summe der individuellen Kognitiven Vertrauenswerte des Teams (\ac{ik} $p = .200 > \alpha = 0.05$), (\ac{nik} $p = .161 > \alpha = 0.05$) sowie der Anzahl der erfolgreich abgeschlossenen Runden der Teams (\ac{ik} $p = .200 > \alpha = 0.05$), (\ac{nik} $p = .109 > \alpha = 0.05$) Normalverteilt. Die Anzahl abgeschlossener Runden pro Team wird mit "RoundsDone" bezeichnet.

		\subsubsection{Pearson-Korrelationsanalyse}
Nun wird eine Pearson-Korrelation zwischen den kognitiven Vertrauenswerten des Teams und den erfolgreich abgeschlossenen Runden durchgeführt.

\begin{figure}[H]
\centering
		\begin{footnotesize}
			\includegraphics[scale=0.8]{Abbildungen/Post_QuestionnaireStatistiks/H3_IK_Korrelation_Pearson}
			\includegraphics[scale=0.8]{Abbildungen/Post_QuestionnaireStatistiks/H3_NIK_Korrelation_Pearson}
			\caption{Pearson-Korrelation zwischen den einzelnen Teams und dem \ac{te}}
			\label{fig:Pearson_ik_nik_h3}
		\end{footnotesize}
	\end{figure}	

Es ist eine Korrelation mittlerem Effektes ($r = .268$) zwischen \ac{cti} und \ac{tei} mit dem Merkmal \ac{ik} vorhanden. Diese Korrelation ist nicht statistisch Signifikant ($p = .641 > \alpha = 0.05$)
Es ist eine positive Korrelation starken Effektes ($r = .801$) zwischen \ac{ctn} und \ac{ten} mit dem Merkmal \ac{nik} vorhanden. Diese Korrelation ist ebenfalls nicht statistisch Signifikant ($p = .103 > \alpha = 0.05$).

		\subsubsection{Regressionsanalyse}
Um die nicht statistische Signifikanz zu bestätigen, wird zusätzlich eine Regressionsanalyse durchgeführt, bei der angenommen wird, dass \ac{ct} einen \underline{positiven Einfluss} auf \ac{te} hat.
\ac{ct} wird wurde als unabhängige und \ac{te} als abhängige Variable definiert.

				\paragraph{Kondition \ac{ik}}
Das Bestimmtheitsmaß ($R^{2} = .082$) deutet laut \citep{cohen2013statistical} auf eine schwache Varianzaufklärung der \ac{tei} durch \ac{cti} hin. Somit lassen sich $8,2\%$ der Varianzen unserer \ac{tei} durch den \ac{ctn} erklären. \\
Der Regressionskoeffizient der Variable \ac{cti} beträgt ($r = .286$) und statistisch nicht signifikant ($t(3) = .516; p = .641 > \alpha = 0.05$). \\
Weiterhin ist das Ergebnis der ANOVA ($F(1,3) = .267; p = .641 > \alpha = 0.05$) und es lässt sich \textit{(Hypothese 3 / H$_{0}$)} zum Signifikanzniveau ($\alpha = 0.05$) für die Kondition \ac{ik} nicht verwerfen. \\
Der \ac{cti} eignet sich nicht als Prädiktor für die \ac{tei}.
Siehe \textbf{\autoref{fig:h3_regression}}

				\paragraph{Kondition \ac{nik}}
Das Bestimmtheitsmaß ($R^{2} = .0,801$) deutet laut \citep{cohen2013statistical} auf eine sehr starke Varianzaufklärung der \ac{ten} durch \ac{ctn} hin. Somit lassen sich $80,1\%$ der Varianzen unserer \ac{ten} durch den \ac{ctn} erklären. \\
Der Regressionskoeffizient der Variable \ac{ctn} beträgt ($r = .801$) und statistisch nicht signifikant ($t(3) = 2,316; p = .106 > \alpha = 0.05$). \\
Weiterhin ist das Ergebnis der ANOVA ($F(1,3) = 5,363; p = .103 > \alpha = 0.05$) und es lässt sich \textit{(Hypothese 3 / H$_{0}$)} zum Signifikanzniveau ($\alpha = 0.05$) nicht für die Kondition \ac{nik} verwerfen. \\
Der \ac{ctn} eignet sich nicht als Prädiktor für die \ac{ten}.
\textbf{\autoref{fig:h3_regression}}\\
Die Hypothese \textit{(Hypothese 3 / H$_{1}$)} kann somit nicht angenommen werden.

	\subsubsection{Korrektur der Analyse}
Da die Daten mit ledigtlich 5 Teams pro Kondition bei der Korrelationsanalyse sowie der Regressionsanalyse zwischen \ac{cti} sowie \ac{ik} und \ac{ctn} sowie \ac{nik} zu wenig sind, wurde Zusätzlich eine Korrelationsanalyse sowie Regressionsanalyse \textit{ohne} die Aufteilung in die Konditionen \ac{ik} und \ac{nik} durchgeführt.

			\paragraph{Pearson-Korrelationsanalyse}
Es ist eine unsignifikante positive Korrelation mittlerem Effektes mit dem Pearson-Korrelationskoeffizient ($r = .361$) zwischen \ac{ct} und \ac{te} vorhanden. Weiterhin ist keine Signifikanz ($p = .306 > \alpha = 0.05$) zu erkennen. Dies deutet darauf hin, dass die \textit{(Hypothese 3 / H$_{0}$)} nicht verworfen werden kann. Siehe \textbf{\autoref{fig:h3_both}}
			
			\paragraph{Regressionsanalyse}
Das Bestimmtheitsmaß ($R^{2} = .130$) deutet laut \citep{cohen2013statistical} auf eine mittlere Varianzaufklärung der \ac{te} durch \ac{ct} hin. Somit lassen sich $13,0\%$ der Varianzen unserer \ac{te} durch den \ac{ct} erklären. \\
Der Regressionskoeffizient der Variable \ac{ct} beträgt ($r = .361$) und ist statistisch nicht signifikant ($t(8) = 1,095; p = .306 > \alpha = 0.05$). \\
Weiterhin ist das Ergebnis der ANOVA ($F(1,8) = 1,198; p = .306 > \alpha = 0.05$) und es lässt sich \textit{(Hypothese 3 / H$_{0}$)} zum Signifikanzniveau ($\alpha = 0.05$) nicht verwerfen. \\
Der \ac{ct} eignet sich nicht als Prädiktor für die \ac{te}.
Siehe \textbf{\autoref{fig:h3_both}}

\newpage
	\subsection{Analyse Hypothese 4}
H$_{0}$ : Die Konditionen \ac{ik} oder \ac{nik} haben \textbf{keinen Signifikant abweichenden Einfluss} auf die Team-Effektivität.\\
Wie in \textit{Hypothese 3} beschrieben, sind Team-Effektivität der einzelnen Teams Normalverteilt.
Um die  \textit{Hypothese 4} auszuwerten, wird ein T-Test bei unabhängigen Stichproben für \ac{te} durchgeführt. Die Gruppierungsvariablen sind dabei \ac{ik} sowie \ac{nik}.

		\subsubsection{T-Test}
Der T-Wert ist gleich Null ($T = .000 = 0$), was auf keinen Mittelwertsunterschied zwischen \ac{ik} und \ac{nik} hinweist.
Die Varianzen sind mit ($\sigma^{2} = .126$) unterschiedlich.
Der T-Test zeigt bei der Differenz des durchschnittlichen \ac{tei} bei der Kondition \ac{ik} ($\bar(x) = 9; \sigma = 3,535553$) und \ac{nik} ($\bar(x) = 1,87083; \sigma = 1,87083$) keine Signifikanz ($t(8) = .000; p = 1 > \alpha = 0.05$).\\
Somit kann \textit{(Hypothese 4 / H$_{0}$)} für \ac{ik} und \ac{nik} nicht verworfen werden.

\begin{figure}[H]
\centering
		\begin{footnotesize}
			\includegraphics[scale=0.8]{Abbildungen/Post_QuestionnaireStatistiks/h4_ttest}
			\includegraphics[scale=0.5]{Abbildungen/Post_QuestionnaireStatistiks/h4_ttest2}
			\caption{T-Test für unabhängige Stichproben der Team-Effektivität mit den unabhängigen Variablen \ac{ik} sowie \ac{nik}}
			\label{fig:h4_ttest}
		\end{footnotesize}
	\end{figure}	
	

\newpage
	\subsection{Analyse Hypothese 5}
H$_{0}$ : Teams, aufgeteilt nach Avatarverkörperungen, mit einem hohem General-Trust-Score erzielen \textbf{keine höhere Team-Effektivität} als die mit einem niedrigen General-Trust-Score.\\
Um die Hypothese 5 auszuwerten, wird wie in \textit{Hypothese 3} eine Korrelation durchgeführt. Anschließend wird eine Regressionsanalyse zwischen \ac{gti} sowie \ac{gtn} und \ac{tei} sowie \ac{ten} durchgeführt um eine Richtung des Zusammenhangs darzustellen. Dafür muss wieder eine Tabelle aufgestellt werden, welche die Summe der individuellen Personen im Team zu einem generellen Vertrauenswert des Teams zusammenfasst.

		\subsubsection{Aufstellen der generellen Vertrauenstabelle pro Team}
\begin{table}[H]
	\caption{Kognitive-Vertrauenswerte - Individuell und pro Team zusammengefasst}
	\label{VariableBreakdown}
	\begin{tabularx}{\textwidth}{r | p{2cm} p{1.5cm} p{1.5cm} p{1.5cm} p{1.5cm} p{1.5cm} p{1.5cm} } 
		ID & Individueller genereller Vertrauenswert & Kondition & Erfolgreich abgeschlossene Runden & Team ID & Summe \ac{gt}\\
		\hline 
		1 & $4,92$ & 1 & 4 & Team 1 & $15,84$\\
		2 & $4,77$ & 1 & 4 & &\\
		3 & $6,15$ & 1 & 4 & & \\
		\hline 
		4 & $5,62$ & 1 & 10 & Team 2 & $14,62$\\
		5 & $3,46$ & 1 & 10 & &\\
		6 & $5,54$ & 1 & 10 & &\\
		\hline 
		7 & $5,92$ & 1 & 7 & Team 3 & $15,46$\\
		8 & $4,31$ & 1 & 7 & & \\
		9 & $5,23$ & 1 & 7 & & \\
		\hline 
		10 & $5,46$ & 1 & 11 & Team 4 & $14,84$ \\
		11 & $4,46$ & 1 & 11 & & \\
		12 & $4,92$ & 1 & 11 & & \\
		\hline 
		13 & $4,85$ & 1 & 13 & Team 5 & $14,23$ \\
		14 & $5,46$ & 1 & 13 & & \\
		15 & $3,92$ & 1 & 13 & & \\
		\hline 
		16 & $5,77$ & 2 & 9 & Team 6 & $17,00$ \\
		17 & $5,85$ & 2 & 9 & &\\
		18 & $5,38$ & 2 & 9 & &\\
		\hline 
		19 & $4,46$ & 2 & 12 & Team 7 & $15,46$ \\
		20 & $4,77$ & 2 & 12 & & \\
		21 & $6,23$ & 2 & 12 & & \\
		\hline 
		22 & $5,46$ & 2 & 8 & Team 8 & $15,23$ \\
		23 & $4,62$ & 2 & 8 & & \\
		24 & $5,15$ & 2 & 8 & & \\
		\hline 
		25 & $4,69$ & 2 & 7 & Team 9 & $16,92$ \\
		26 & $6,08$ & 2 & 7 & & \\
		27 & $6,15$ & 2 & 7 & & \\
		\hline 
		28 & $6,00$ & 2 & 9 & Team 10 & $15,76$ \\
		29 & $5,38$ & 2 & 9 & &\\
		30 & $4,38$ & 2 & 9 & & \\	
	\end{tabularx}
\end{table}

Laut Kolmogoroff-Smirnoff-Test sind sowohl die Summe der individuellen generellen Vertrauenswerte des Teams (\ac{ik} $p = .200 > \alpha = 0.05$), (\ac{nik} $p = .200 > \alpha = 0.05$) sowie die Anzahl der erfolgreich abgeschlossenen Runden der Teams (\ac{ik} $p = .200 > \alpha = 0.05$), (\ac{nik} $p = .161 > \alpha = 0.05$) Normalverteilt.

		\subsubsection{Pearson-Korrelationsanalyse}
Nun wird eine Pearson-Korrelation zwischen den generellen Vertrauenswerten des Teams und den erfolgreich abgeschlossenen Runden durchgeführt und in die unabhängigen Variablen \ac{ik} und \ac{nik} aufgeteilt.

\begin{figure}[H]
\centering
		\begin{footnotesize}
			\includegraphics[scale=0.8]{Abbildungen/Post_QuestionnaireStatistiks/h5_korr_gt_te_ik}
			\includegraphics[scale=0.8]{Abbildungen/Post_QuestionnaireStatistiks/h5_korr_gt_te_nik}
			\caption{Pearson-Korrelation zwischen den einzelnen Teams und dem \ac{te}}
			\label{fig:Pearson_ik_nik_h3}
		\end{footnotesize}
	\end{figure}	

Es ist eine negative Korrelation sehr starkem Effekt ($r = -.971$) zwischen \ac{gti} und \ac{tei} mit dem Merkmal \ac{ik} vorhanden. Diese Korrelation ist statistisch Signifikant ($p = .006 < \alpha = 0.05$)
Weiterhin ist eine negative Korrelation mittlerem Effekt ($r = -.433$) zwischen \ac{gtn} und \ac{ten} mit dem Merkmal \ac{nik} vorhanden. Diese Korrelation ist nicht statistisch Signifikant ($p = .467 > \alpha = 0.05$).

		\subsubsection{Regressionsanalyse}
Um die statistische Signifikanz des Zusammenhangs der Kondition \ac{ik} zwischen \ac{gti} und \ac{tei} und die nicht statistische Signifikanz des Zusammenhangs der Kondition \ac{nik} zwischen \ac{gtn} und \ac{ten} zu bestätigen, wird zusätzlich eine Regressionsanalyse durchgeführt.
\ac{gt} wird wurde als unabhängige und \ac{te} als abhängige Variable definiert.

\paragraph{Kondition \ac{ik}}
Das Bestimmtheitsmaß ($R^{2} = .971$) deutet laut \citep{cohen2013statistical} auf eine sehr starke Varianzaufklärung der \ac{tei} durch \ac{gti} hin. Somit lassen sich $97,1\%$ der Varianzen unserer \ac{tei} durch den \ac{gtn} erklären. \\
Der Regressionskoeffizient der Variable \ac{cti} beträgt ($r = -.971$) und ist statistisch signifikant ($t(3) = -7,040; p = .006 < \alpha = 0.01$). \\
Weiterhin ist das Ergebnis der ANOVA ($F(1,3) = 49,568; p = .006 < \alpha = 0.01$) und es lässt sich \textit{(Hypothese 5 / H$_{0}$)} zum Signifikanzniveau ($\alpha = 0.01$) für die Kondition \ac{ik} nicht verwerfe. \\
Der \ac{gti} eignet sich als Prädiktor für \ac{tei}. Die geschätzte Abnahme an \ac{tei} beläuft sich auf $-.971$ Punkte \ac{tei} pro \ac{gti} ($\beta = -.971; t(3) = -7,040; p = .006 < \alpha = 0.01$). \\
Siehe \textbf{\autoref{fig:h5_regression}}

\paragraph{Kondition \ac{nik}}
Das Bestimmtheitsmaß ($R^{2} = .187$) deutet laut \citep{cohen2013statistical} auf eine schwache Varianzaufklärung der \ac{ten} durch \ac{gtn} hin. Somit lassen sich $18,7\%$ der Varianzen unserer \ac{ten} durch den \ac{gtn} erklären. \\
Der Regressionskoeffizient der Variable \ac{gtn} beträgt ($r = -.433$) und ist statistisch nicht signifikant ($t(3) = -.831; p = .467 > \alpha = 0.05$). \\
Weiterhin ist das Ergebnis der ANOVA ($F(1,3) = .691; p = .467 > \alpha = 0.05$) und es lässt sich \textit{(Hypothese 5 / H$_{0}$)} zum Signifikanzniveau ($\alpha = 0.05$) nicht für die Kondition \ac{nik} verwerfen. \\
Der \ac{gtn} eignet sich nicht als Prädiktor für die \ac{ten}.\\
\textbf{\autoref{fig:h5_regression}}

Die Hypothese \textit{(Hypothese 5 / H$_{0}$)} kann somit nicht angenommen werden, da ein abnehmender Effekt auf die \ac{te} pro Team durch die Konditionen \ac{ik} und \ac{nik} festgestellt wurde.

\subsubsection{Korrektur der Analyse}
Da die Daten mit ledigtlich 5 Teams pro Kondition bei der Korrelationsanalyse sowie der Regressionsanalyse zwischen \ac{gti} sowie \ac{ik} und \ac{gtn} sowie \ac{nik} zu wenig sind, wurde zusätzlich noch eine Korrelationsanalyse sowie Regressionsanalyse \textit{ohne} die Aufteilung in die Konditionen \ac{ik} und \ac{nik} durchgeführt.

			\paragraph{Pearson-Korrelationsanalyse}
Es ist eine unsignifikante negative Korrelation mittlerem Effektes mit dem Pearson-Korrelationskoeffizient ($r = -.535$) zwischen \ac{gt} und \ac{te} vorhanden. Weiterhin ist keine Signifikanz ($p = .111 > \alpha = 0.05$) zu erkennen. Dies deutet darauf hin, dass die \textit{(Hypothese 5 / H$_{0}$)} nicht verworfen werden kann. Siehe \textbf{\autoref{fig:h5_both}}
			
			\paragraph{Regressionsanalyse}
Das Bestimmtheitsmaß ($R^{2} = .286$) deutet laut \citep{cohen2013statistical} auf eine hohe Varianzaufklärung der \ac{te} durch \ac{gt} hin. Somit lassen sich $28,6\%$ der Varianzen unserer \ac{te} durch den \ac{gt} erklären. \\
Der Regressionskoeffizient der Variable \ac{gt} beträgt ($r = -.535$) und ist statistisch nicht signifikant ($t(8) = -1,791; p = .111 > \alpha = 0.05$). \\
Weiterhin ist das Ergebnis der ANOVA ($F(1,8) = 3,206; p = .111 > \alpha = 0.05$) und es lässt sich \textit{(Hypothese 5 / H$_{0}$)} zum Signifikanzniveau ($\alpha = 0.05$) nicht verwerfen. \\
Der \ac{gt} eignet sich nicht als Prädiktor für die \ac{te}.\\
Siehe \textbf{\autoref{fig:h5_both}}


\subsection{Analyse Hypothese 6}

Auch interessant : T-Test zwischen COPRÄSENZ und WieSahenIhreMitspielerAus

\subsection{Analyse Hypothese 7}

Auch interessant : Wie wirkt sich GT und CT gemeinsam auf TE aus?

Auch interssant : Rounds Done im Verhältnis zu Team-Effectiveness-Scale

Auch interessant : NASA-TLX( WIE ANSTRENGEND WAR ES?) zur Round-Efficiency

Auch interessant : T-Test von Team-Communication bei den IK und NON IK Gruppen

Auch interessant : Team-Communication zur Rounds Done


\subsection{Berechnung der Werte für die Auswertung}
Bspw. Wie wurde CT oder GT etc. berechnet? UND WARUM BENÖTIGE ICH DIESE? WARUM Z.B. BESTANDENE TEAMRUNDEN ? -> WEIL NUR SO TEAMBUILDING ERFOLG GEMESSEN WERDEN KANN
\newpage

\clearpage
\newpage
\section{Zusammenfassung}
\newpage
\section{Diskussion}
	\subsection{Diskussion der Ergebnisse}
	\subsection{Diskussion der eingesetzten Methoden}
	\subsection{Auswirkungen auf die Gegenwart}
	\subsection{Vorschläge für zukünftige Untersuchungen}
		Social Identity und Teambuilding - Ein Avatar ist anders.
	
	\newpage
	\appendix	
\section*{Anhang}\markboth{Anhang}{Anhang}\addcontentsline{toc}{section}{Anhang}

\section{Auswertungsergebnisse}
	
	\begin{figure}[H]
	\centering
		\begin{footnotesize}
			\includegraphics[scale=0.6]{Abbildungen/Post_QuestionnaireStatistiks/Normalverteilung_30}\\
			\caption{Kolmogorov-Smirnoff Normalverteilung für Individuelle Stichproben}
			\label{fig:KolSmirInd}
		\end{footnotesize}
	\end{figure}	
	
	\begin{figure}[H]
	\centering
		\begin{footnotesize}
			\includegraphics[scale=0.6]{Abbildungen/Post_QuestionnaireStatistiks/Normalverteilung_15_IK}\\
			\caption{Kolmogorov-Smirnoff Normalverteilung für die IK-Kondition}
			\label{fig:KolSmirIndIK}
		\end{footnotesize}
	\end{figure}	
	
	\begin{figure}[H]
	\centering
		\begin{footnotesize}
			\includegraphics[scale=0.6]{Abbildungen/Post_QuestionnaireStatistiks/Normalverteilung_15_NON_IK}\\
			\caption{Kolmogorov-Smirnoff Normalverteilung für die NON-IK-Kondition}
			\label{fig:KolSmirIndNONIK}
		\end{footnotesize}
	\end{figure}	
	
	
\begin{figure}[H]
\centering
		\begin{footnotesize}
			\includegraphics[scale=0.5]{Abbildungen/Post_QuestionnaireStatistiks/boxplot_cognitive_trust_winsorisiert}\\
			\caption{Boxplot kognitives Vertrauen winsorisiert}
			\label{fig:boxplot_cognitive_trust_winsorisiert}
		\end{footnotesize}
	\end{figure}	

\begin{figure}[H]
\centering
		\begin{footnotesize}
			\includegraphics[scale=0.6]{Abbildungen/Post_QuestionnaireStatistiks/h3_regression_ik}
			\includegraphics[scale=0.6]{Abbildungen/Post_QuestionnaireStatistiks/h3_regression_nik}
			\caption{Regressionsergebnisse der Regressionen \ac{cti} und \ac{tei} sowie \ac{ctn} und \ac{ten} }
			\label{fig:h3_regression}
		\end{footnotesize}
	\end{figure}	
	
\begin{figure}[H]
\centering
		\begin{footnotesize}
			\includegraphics[scale=0.6]{Abbildungen/Post_QuestionnaireStatistiks/h3_both_korrelation}
			\includegraphics[scale=0.6]{Abbildungen/Post_QuestionnaireStatistiks/h3_both_diagramm_korr}
			\includegraphics[scale=0.6]{Abbildungen/Post_QuestionnaireStatistiks/h3_both_regression}
			\caption{Regressionsergebnisse der Regressionen \ac{ct} und \ac{te} }
			\label{fig:h3_both}
		\end{footnotesize}
	\end{figure}	
	
	
\begin{figure}[H]
\centering
		\begin{footnotesize}
			\includegraphics[scale=0.6]{Abbildungen/Post_QuestionnaireStatistiks/h5_both_korrelation}
			\includegraphics[scale=0.6]{Abbildungen/Post_QuestionnaireStatistiks/h5_both_diagramm_korr}
			\includegraphics[scale=0.6]{Abbildungen/Post_QuestionnaireStatistiks/h5_both_regression}
			\caption{Regressionsergebnisse der Regressionen \ac{gt} und \ac{te} }
			\label{fig:h5_both}
		\end{footnotesize}
	\end{figure}	

\begin{figure}[H]
\centering
		\begin{footnotesize}
			\includegraphics[scale=0.6]{Abbildungen/Post_QuestionnaireStatistiks/h3_regression_ik}
			\includegraphics[scale=0.6]{Abbildungen/Post_QuestionnaireStatistiks/h3_regression_nik}
			\caption{Regressionsergebnisse der Regressionen \ac{gti} und \ac{tei} sowie \ac{gtn} und \ac{ten} }
			\label{fig:h5_regression}
		\end{footnotesize}
	\end{figure}	

\clearpage
\newpage
	\section{Demografieauswertung}
	\begin{figure}[H]
	\centering
		\begin{footnotesize}
			\includegraphics[scale=0.6]{Abbildungen/Pre_QuestionnaireStatistiks/teilnehmerStatistik}\\
			\includegraphics[scale=0.5]{Abbildungen/Demographie/alter}\\
			\caption{Alter der Teilnehmer}
			\label{fig:Teilnehmer}
		\end{footnotesize}
	\end{figure}	
	
	\begin{figure}[H]
	\centering
		\begin{footnotesize}
			\includegraphics[scale=0.6]{Abbildungen/Pre_QuestionnaireStatistiks/teilnehmerGeschlecht}\\
			\includegraphics[scale=0.5]{Abbildungen/Demographie/teilnehmerGeschlecht}\\
			\caption{Biologisches Geschlecht der Teilnehmer}
			\label{fig:biologischesGeschlecht}
		\end{footnotesize}
	\end{figure}	
	
	\begin{figure}[H]
	\centering
		\begin{footnotesize}
			\includegraphics[scale=0.6]{Abbildungen/Pre_QuestionnaireStatistiks/teilnehmerBildungsstand}\\
			\includegraphics[scale=0.5]{Abbildungen/Demographie/teilnehmerBildungsstand}\\
			\caption{Bildungsstand der Teilnehmer}
			\label{fig:teilnehmerBildungsstand}
		\end{footnotesize}
	\end{figure}	
	
	\begin{figure}[H]
	\centering
		\begin{footnotesize}
			\includegraphics[scale=0.6]{Abbildungen/Pre_QuestionnaireStatistiks/teilnehmerVRErfahrung}\\
			\includegraphics[scale=0.5]{Abbildungen/Demographie/teilnehmerVRErfahrung}\\
			\caption{VR-Erfahrung der Teilnehmer}
			\label{fig:teilnehmerVRErfahrung}
		\end{footnotesize}
	\end{figure}	
	
	\begin{figure}[H]
	[scale=0.6]
		\begin{footnotesize}
			\includegraphics[scale=0.6]{Abbildungen/Pre_QuestionnaireStatistiks/teilnehmerVorexperimente}\\
			\includegraphics[scale=0.5]{Abbildungen/Demographie/teilnehmerVorexperimente}\\
			\caption{VR-Studien Vorerfahrungen}
			\label{fig:teilnehmerVorexperimente}
		\end{footnotesize}
	\end{figure}		
	
	\begin{figure}[H]
	\centering
		\begin{footnotesize}
			\includegraphics[scale=0.6]{Abbildungen/Pre_QuestionnaireStatistiks/teilnehmerPCAußmaß}\\
			\includegraphics[scale=0.5]{Abbildungen/Demographie/teilnehmerPCAußmaß}\\
			\caption{PC-Nutzung Außmaß sowie zugehörige Perzentile}
			\label{fig:teilnehmerPCAußmaß}
		\end{footnotesize}
	\end{figure}		
	
	\begin{figure}[H]
	\centering
		\begin{footnotesize}
			\includegraphics[scale=0.6]{Abbildungen/Pre_QuestionnaireStatistiks/teilnehmerVideospieleAußmaß}\\
			\includegraphics[scale=0.5]{Abbildungen/Demographie/teilnehmerVideospieleAußmaß}\\
			\caption{Videospieleaußmaß der Teilnehmer}
			\label{fig:teilnehmerVideospieleAußmaß}
		\end{footnotesize}
	\end{figure}		

	
	\section{Pre-Questionnaire}
	\label{Pre-Questionnaire}
	
	\begin{figure}[H]
	\centering
		\begin{footnotesize}
			\includegraphics[scale=0.6]{Abbildungen/Fragebogen/Pre-Questionnaire/PQ1}\\
		\end{footnotesize}
	\end{figure}	
	
	\begin{figure}[H]
	\centering
		\begin{footnotesize}
			\includegraphics[scale=0.6]{Abbildungen/Fragebogen/Pre-Questionnaire/PQ2}\\
		\end{footnotesize}
	\end{figure}	
	
	\begin{figure}[H]
	\centering
		\begin{footnotesize}
			\includegraphics[scale=0.6]{Abbildungen/Fragebogen/Pre-Questionnaire/PQ3}\\
		\end{footnotesize}
	\end{figure}	
	
	\begin{figure}[H]
	\centering
		\begin{footnotesize}
			\includegraphics[scale=0.6]{Abbildungen/Fragebogen/Pre-Questionnaire/PQ4}\\
		\end{footnotesize}
	\end{figure}	
	
	\begin{figure}[H]
	\centering
		\begin{footnotesize}
			\includegraphics[scale=0.6]{Abbildungen/Fragebogen/Pre-Questionnaire/PQ5}\\
		\end{footnotesize}
	\end{figure}	
	
	\begin{figure}[H]
	\centering
		\begin{footnotesize}
			\includegraphics[scale=0.6]{Abbildungen/Fragebogen/Pre-Questionnaire/PQ6}\\
		\end{footnotesize}
	\end{figure}	
	
\section{Post-Questionnaire - Konditionsabfrage}
\label{Post-Questionnaire - Konditionsabfrage}

	\begin{figure}[H]
	\centering
		\begin{footnotesize}
			\includegraphics[scale=0.6]{Abbildungen/Fragebogen/Post-Questionnaire/PQM1}
		\end{footnotesize}
	\end{figure}	

\newpage

\section{Post-Questionnaire - Generelles Vertrauen}
\label{Post-Questionnaire - Generelles Vertrauen}

	\begin{figure}[H]
	\centering
		\begin{footnotesize}
			\includegraphics[scale=0.6]{Abbildungen/Fragebogen/Post-Questionnaire/PQG1}
		\end{footnotesize}
	\end{figure}	
	
	\begin{figure}[H]
	\centering
		\begin{footnotesize}
			\includegraphics[scale=0.6]{Abbildungen/Fragebogen/Post-Questionnaire/PQG2}
		\end{footnotesize}
	\end{figure}	
	
	\begin{figure}[H]
	\centering
		\begin{footnotesize}
			\includegraphics[scale=0.6]{Abbildungen/Fragebogen/Post-Questionnaire/PQG3}
		\end{footnotesize}
	\end{figure}	
	
	\begin{figure}[H]
	\centering
		\begin{footnotesize}
			\includegraphics[scale=0.6]{Abbildungen/Fragebogen/Post-Questionnaire/PQG4}
		\end{footnotesize}
	\end{figure}	

\newpage
\section{Post-Questionnaire - Kognitives Vertrauen}	
\label{Post-Questionnaire - Kognitives Vertrauen}	

	\begin{figure}[H]
	\centering
		\begin{footnotesize}
			\includegraphics[scale=0.6]{Abbildungen/Fragebogen/Post-Questionnaire/PQC1}
		\end{footnotesize}
	\end{figure}	
	
	\begin{figure}[H]
	\centering
		\begin{footnotesize}
			\includegraphics[scale=0.6]{Abbildungen/Fragebogen/Post-Questionnaire/PQC2}
		\end{footnotesize}
	\end{figure}	

\newpage
\section{Post-Questionnaire - Teamkommunikation}	
\label{Post-Questionnaire - Teamkommunikation}	

\begin{figure}[H]
	\centering
		\begin{footnotesize}
			\includegraphics[scale=0.4]{Abbildungen/Fragebogen/Post-Questionnaire/PQKQ1}
		\end{footnotesize}
	\end{figure}	

\newpage
\section{Post-Questionnaire - Team-Effektivität}	
\label{Post-Questionnaire - Team-Effektivität}	

	\begin{figure}[H]
	\centering
		\begin{footnotesize}
			\includegraphics[scale=0.6]{Abbildungen/Fragebogen/Post-Questionnaire/PQTE1}
		\end{footnotesize}
	\end{figure}	

\newpage
\section{Post-Questionnaire - NASA-TLX}
\label{Post-Questionnaire - NASA-TLX}

	\begin{figure}[H]
	\centering
		\begin{footnotesize}
			\includegraphics[scale=0.6]{Abbildungen/Fragebogen/Post-Questionnaire/PQTLX1}
		\end{footnotesize}
	\end{figure}	
	
	\begin{figure}[H]
	\centering
		\begin{footnotesize}
			\includegraphics[scale=0.6]{Abbildungen/Fragebogen/Post-Questionnaire/PQTLX2}
		\end{footnotesize}
	\end{figure}	
	
	\begin{figure}[H]
	\centering
		\begin{footnotesize}
			\includegraphics[scale=0.6]{Abbildungen/Fragebogen/Post-Questionnaire/PQTLX3}
		\end{footnotesize}
	\end{figure}	
	
	\begin{figure}[H]
	\centering
		\begin{footnotesize}
			\includegraphics[scale=0.6]{Abbildungen/Fragebogen/Post-Questionnaire/PQTLX4}
		\end{footnotesize}
	\end{figure}	
	
	\begin{figure}[H]
	\centering
		\begin{footnotesize}
			\includegraphics[scale=0.6]{Abbildungen/Fragebogen/Post-Questionnaire/PQTLX5}
		\end{footnotesize}
	\end{figure}	
	
	\begin{figure}[H]
	\centering
		\begin{footnotesize}
			\includegraphics[scale=0.6]{Abbildungen/Fragebogen/Post-Questionnaire/PQTLX6}
		\end{footnotesize}
	\end{figure}	

\newpage
\section{Post-Questionnaire - IPQ}
\label{Post-Questionnaire - IPQ}

	\begin{figure}[H]
	\centering
		\begin{footnotesize}
			\includegraphics[scale=0.6]{Abbildungen/Fragebogen/Post-Questionnaire/PQP1}
		\end{footnotesize}
	\end{figure}	
	
	\begin{figure}[H]
	\centering
		\begin{footnotesize}
			\includegraphics[scale=0.6]{Abbildungen/Fragebogen/Post-Questionnaire/PQP2}
		\end{footnotesize}
	\end{figure}	
	
	\begin{figure}[H]
	\centering
		\begin{footnotesize}
			\includegraphics[scale=0.6]{Abbildungen/Fragebogen/Post-Questionnaire/PQP3}
		\end{footnotesize}
	\end{figure}	

\newpage
\section{Post-Questionnaire - Co-Präsenz}			
\label{Post-Questionnaire - Co-Präsenz}		

	\begin{figure}[H]
	\centering
		\begin{footnotesize}
			\includegraphics[scale=0.6]{Abbildungen/Fragebogen/Post-Questionnaire/PQCP1}
		\end{footnotesize}
	\end{figure}	
	
	\begin{figure}[H]
	\centering
		\begin{footnotesize}
			\includegraphics[scale=0.6]{Abbildungen/Fragebogen/Post-Questionnaire/PQCP2}
		\end{footnotesize}
	\end{figure}	
	
	\begin{figure}[H]
	\centering
		\begin{footnotesize}
			\includegraphics[scale=0.6]{Abbildungen/Fragebogen/Post-Questionnaire/PQCP3}
		\end{footnotesize}
	\end{figure}	
	
	\begin{figure}[H]
	\centering
		\begin{footnotesize}
			\includegraphics[scale=0.6]{Abbildungen/Fragebogen/Post-Questionnaire/PQCP4}
		\end{footnotesize}
	\end{figure}	
	
	\begin{figure}[H]
	\centering
		\begin{footnotesize}
			\includegraphics[scale=0.6]{Abbildungen/Fragebogen/Post-Questionnaire/PQCP5}
		\end{footnotesize}
	\end{figure}	
	
	\begin{figure}[H]
	\centering
		\begin{footnotesize}
			\includegraphics[scale=0.6]{Abbildungen/Fragebogen/Post-Questionnaire/PQCP6}
		\end{footnotesize}
	\end{figure}	

\section{Other}	
	
\begin{figure}[H]
		\begin{footnotesize}
			\includegraphics[scale=1]{Abbildungen/criteriaForVirtualTeams.JPG}
			\caption[Abbildung 1]{Kriterien für virtuelle Teams von \citep[p. 27]{schweitzer2010conceptualizing}}
			\label{criteriaForVirtualTeams}
		\end{footnotesize}
	\end{figure}	
	
	\begin{figure}[H]
	\centering
		\begin{footnotesize}
			\includegraphics[scale=0.6]{Abbildungen/CouchEtAl_1996_TrustScale}\\
			\caption[Abbildung 1]{Items of the Trust Inventory \citep[305-323]{couch1996assessment}}
			\label{Trust-Inventory}
		\end{footnotesize}
	\end{figure}
	
	\newpage

\bibliographystyle{dcu}
\bibliography{bibfile}
\end{document}
