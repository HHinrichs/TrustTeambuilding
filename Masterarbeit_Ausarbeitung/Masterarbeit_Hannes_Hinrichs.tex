%TODO : Mein eigenes Verwendetes System beschreiben, genauer, welche verschiedenen Interaktionen hatte ich, etc. Warum habe ich was gemacht, eventuell sogar mit UML Diagrammen arbeiten
%
%Auch weitere Hypoethesen in Sachen präsenz als zusätzliches Kapitel "Signifikanzanalyse" oder so mit hineinbringen. Nur die, die signifikant sind.
%
%Analyse mit Signifikanzniveau von 10% anschauen, damit Ergebnisse schnelller signifikant sind.
%
%Umbenennung von "VR" und "AVATARE" etc.
%
%Nicht alles so kleingliedrig
%
%Auf Verwendung von nicht parametrische Tests achten.
%
%Leseprobe senden!
%Auswertungen nicht als Screenshot sondern als Tabelle
%Gerichtete Hypothese überprüfen, schauen ob der p-Wert halbiert werden kann
\documentclass[a4paper,11pt]{article}%Schriftgröße

\usepackage{booktabs}
\usepackage{multirow}
\usepackage{rotating}
\newcommand\tabrotate[1]{\begin{turn}{90}\rlap{#1}\end{turn}}
\usepackage{varwidth}
\newcommand\tabvarwidth[2][3cm]{\begin{varwidth}[b]{#1}\centering #2\end{varwidth}}
\usepackage{tabularx}
\usepackage{amsmath}
\usepackage{subfig}
\usepackage{enumitem}
\usepackage{pslatex}
\usepackage[T1]{fontenc} 
\usepackage[utf8]{inputenc}
\usepackage[ngerman]{babel}%Veröffentlichungssprache
\usepackage{graphicx}
\usepackage{ragged2e}
\usepackage[format=plain,justification=RaggedRight,singlelinecheck=false,font={small},labelsep=space]{caption}
\usepackage{float}
\usepackage{xcolor}	
\usepackage[a4paper]{geometry}
\usepackage{fancyhdr}
\usepackage[onehalfspacing]{setspace}%Zeilenabstand
\usepackage[onehalfspacing]{setspace}%Zeilenabstand
\usepackage{fancyhdr}
\usepackage[authoryear]{natbib}
\usepackage{relsize}
%\usepackage[printonlyused, smaller, footnote]{acronym}
\usepackage[printonlyused, smaller]{acronym}
%\usepackage[printonlyused, withpage, smaller, footnote]{acronym}
\renewcommand{\\}{\vspace*{0.5\baselineskip} \newline}
\renewcommand*\MakeUppercase[1]{#1}	
\renewcommand{\familydefault}{\sfdefault}
\geometry{left=3.5cm,right=2.5cm,top=2.4cm,bottom=2cm}%Seitenränder
\pagestyle{fancy}
\renewcommand{\headrulewidth}{0pt}
\renewcommand{\footrulewidth}{0pt}
\fancyhead[R]{\footnotesize{\thepage}}
\fancyhead[L]{\footnotesize{\leftmark}}
\fancyfoot{}
\usepackage[colorlinks,pdfpagelabels,pdfstartview = FitH,bookmarksopen = true,bookmarksnumbered = true,linkcolor = black,urlcolor = black,plainpages = false,hypertexnames = false,citecolor = black] {hyperref}

\begin{document}
	\begin{titlepage}
		\begin{flushleft}
			\vspace*{-1cm}
 			\includegraphics[scale=1]{Abbildungen/TH.png}\\
			\vspace*{1cm}
		\end{flushleft}
		\begin{center}

			\noindent
			%Vergleich des Teambuildingerfolgs mittels eines nicht-vollkörpergetracktem- und vollkörpergetracktem Avatars in einer Immersive Virtual Reality anhand einer Escape-Room-Artigen Umgebung. \\
			
%			Vertrauensaufbau einer Teambuildingmaßnahme zwischen eines Hand- und Kopf getracktem Avatar und  eines Hand-, Kopf und Inverskinematisch-simuliertem Torsos getracktem Avatar in einem Shared Virtual Environment anhand einer Escape-Room-Artigen Umgebung. \\
			
%			Auswirkung des Vertrauens auf die Effizienz in einer Teambuildingmaßnahme anhand eines Hand- und Kopfgetracktem und einem Hand-, Kopf und Inverskinematisch-simuliertem Torsos getracktem Avatar ein einem Shared Virtual Environment. //
			
%			Auswirkung von Vertrauen zu Avataren auf die \mbox{Effizienz} in einer Teambuildingmaßnahme in ein einem Shared Virtual Environment.

%Auswirkung des Vertrauens einer Teambuildingmaßnahme anhand eines \dq{}IK-Avatar\dq{}und einem \dq{}Non-IK-Avatar\dq{} in einem Shared-Virtual-Environment. \\
%
%Auswirkung des Vertrauens auf die Effizienz einer Teambuildingmaßnahme anhand eines \dq{}IK-Avatar\dq{}und einem \dq{}Non-IK-Avatar\dq{} in einem Shared-Virtual-Environment. \\
%
%Welche Avatarkondition schafft in meinem Team mehr Vertrauen? Hat dieses geschaffene Vertrauen eine Auswirkung auf die Effizienz meiner Teambuildingmaßnahme? Welche Avatarkondition sollte ich also nutzen, um Vertrauen zu schaffen, sodass in einem Team ein effizienteres Arbeiten stattfinden kann?

%Auswirkung von Vertrauen auf die Anfangsphase einer virtuellen Teamgrün
\begin{huge}
Einfluss des gebildeten Vertrauens von \ac{ik} und \ac{nik} Avataren auf die Teameffektivität in einem kurzzeitig zusammenarbeitenden virtuellem Team.
%Einfluss der Vertrauensbildung zwischen \ac{ik} und \ac{nik} Avataren auf die Team-Effektivität in einem kurzzeitig zusammenarbeitenden virtuellem Team.\\
\end{huge}
\vspace{2cm} 
		 Masterarbeit zur Erlangung des akademischen Grades\\ \vspace{0.5cm} 
		 \textit{Master of Science}\\ \vspace{0.5cm} 
		 im Studiengang Medientechnologie\\
		 an der Fakultät für Informations-, Medien- und Elektrotechnik\\
		 der Technischen Hochschule Köln
		~\\
		~\\
		~\\\vspace{1cm} 
		\noindent\begin{tabular}{ll}
			vorgelegt von: & Hannes Hinrichs \\
			Matrikel-Nr.: &	11121733 \\
			Adresse: & Zülpicher Straße 19 \\
			~ &	50674 Köln \\
			~ &	hannes.hinrichs@web.de \\
			~ & ~ \\
			eingereicht bei: & Prof. Dr. Arnulph Fuhrmann \\
			Zweitgutachter/in: & Prof. Dr. Stefan Grünvogel
		\end{tabular}	
		~\\
		~\\\vspace{1cm} 
		{\today}
	\end{center}
	\end{titlepage}
	\pagenumbering{Roman}
	\pagestyle{fancy}
	\newpage
\section*{Erklärung}\markboth{Erklärung}{Erklärung}\addcontentsline{toc}{section}{Erklärung}
	Ich versichere, die von mir vorgelegte Arbeit selbstständig verfasst zu haben. Alle Stellen, die wörtlich oder sinngemäß aus veröffentlichten oder nicht veröffentlichten Arbeiten anderer oder der Verfasserin/des Verfassers selbst entnommen sind, habe ich als entnommen kenntlich gemacht. Sämtliche Quellen und Hilfsmittel, die ich für die Arbeit benutzt habe, sind angegeben. Die Arbeit hat mit gleichem Inhalt bzw. in wesentlichen Teilen noch keiner anderen Prüfungsbehörde vorgelegen.\\
	Anmerkung: In einigen Studiengängen steht die Erklärung am Ende des Textes.\\
	~\\
	~\\
	\rule{0.35\textwidth}{0.4pt} \hspace*{3cm} \rule{0.45\textwidth}{0.4pt} \newline
	Ort, Datum	\hspace*{6.3cm}	Rechtsverbindliche Unterschrift
	\newpage
\section*{Kurzfassung/Abstract}\markboth{Kurzfassung/Abstract}{Kurzfassung/Abstract}\addcontentsline{toc}{section}{Kurzfassung/Abstract}
	Virtual Reality \ac{vr} hat in den letzten Jahren aufgrund von verbesserter Technologie und sinkenden Kosten an Bedeutung gewonnen. Verschiedene Felder, wie Medizin, Wirtschaft, Training oder die Industrie greifen auf diese Technologie zurück. Voranschreitende Forschung in diesem Feld, ist aufgrund von wachsendem Interesse von großer Bedeutung. Vertrauensbildung sowie das gesamte Konstrukt von Vertrauen sind in vielen Forschungsbereichen eine wichtige Grundlage. Diese wurde im Zusammenhang mit \ac{vr} jedoch in den letzten Jahren wenig betrieben. Diese Arbeit zielt darauf ab, das Konstrukt des Vertrauens in der Virtuellen Welt besser zu verstehen und mit diesem umzugehen. Dafür wurde eine Studie der Technischen Hochschule zu Köln durchgeführt, um einen Zusammenhang der Vertrauensbildung in einer Teambuildingmaßnahme zwischen einem Menschenähnlichen Avatar und einem nicht-Menschenähnlichen Avatar, darzustellen.
	
			\paragraph{Stichwörter}
			Virtual-Reality, Vertrauen, Teamgründung,Virtuelles Team, Avatar
			
			\paragraph{Datum}
			{\today}
	\newpage
	\tableofcontents
	\newpage
	
\section*{Abkürzungsverzeichnis}\markboth{Abkürzungsverzeichnis}{Abkürzungsverzeichnis}\addcontentsline{toc}{section}{Abkürzungsverzeichnis}
	\begin{acronym}
	\acro{hmd}[HDM]{Head-Mounted-Display}
	\acro{sve}[SVE]{Shared-Virtual-Environment}
	\acro{vr}[VR]{Virtual-Reality}
	\acro{ivr}[IVR]{Immersive-Virtual-Reality}
	\acro{ipo}[IPO]{Input-Process-Output}
	\acro{fov}[FOV]{Field-of-View}
	\acro{bip}[BIP]{Break-in-Presence}
\end{acronym}

	
	%\newpage
	%\listoftables\addcontentsline{toc}{section}{Tabellenverzeichnis}
	%\newpage
	%\listoffigures\addcontentsline{toc}{section}{Abbildungsverzeichnis}
	\newpage
	\pagenumbering{arabic}
	
%\section{Abgrenzung zu anderen Studien}
%	\begin{itemize}
%	\item{Team besteht aus 3! Personen}
%	\item{Es wird die initiale Anfangsphase eines Teams untersucht}
%	
%	\item{Es wird geschaut, ob das geschaffene Vertrauen eine Auswirkung auf die Team-Effizienz hat}
%	\item{Es wird schaut, mit welcher Kondition ( Self-Avatar vs. non Self-Avatar ) mehr Vertrauen in das Team geschaffen wurde}
%	\item{Es wird geschaut, mit welcher Kondition ( Self-Avatar vs non Self-Avatar ) eine höhere Team-Effizienz wird}
%	\end{itemize}		
	

%Der Titel dieser Masterarbeit lässt sich in 2 Fragestellungen unterteilen :

%Ist die Avatarkondition \ac{ik} oder die Avatarkondition \ac{nik} effektiver um Vertrauen in einem \ac{sve} zu bilden?
%Falls es auswirkungen gibt :
%Welche Avatarkondition sollte man nutzen um frühzeitig Vertrauen zu bilden?
%Warum ist eine Avatarkondition effektiver?
%Falls es keine Auswirkungen gibt :
%Weshalb gibt es keine Auswirkungen?
%Was hat der Versuchsaufbau mit den Ergebnissen zu tun?
%

%Wirkt sich das gebildete Vertrauen der verschiedenen Avatarkonditionen auf die Effektivität eines virtuellen Teams aus?
%Wenn ja : Welche Avatarkondition ist effektiver?
%Wie viel effektiver ist eine Avatarkonditione?
%Wenn nein : Woran liegt es, dass die Teams gleich effektiv sind?

%Was bedeutet das im Allgemeinen für ein Team?

%Welche Grundlagen benötige ich daher für die Masterarbeit?
%Was ist Virtual-Reality? XXX
%Was ist ein SVE? XXX
%Was ist Presence? XXX
%Was sind Avatare? XXX Abgedeckt im Kapitel : Avatare und Nicht-Selbstavatare
%Wie unterscheiden sich NIK und IK Avatare?
%Welche Einflussfaktoren auf die Wahrnehmung zwischen Avataren gibt es?
%Wie wird Vertrauen aufgebaut?
%Was gibt es für Arten von Vertrauen?
%Was ist Effektivität?
%Wie wird Effektivität gemessen?
%Wie wird Teameffektivität gemessen?
%Was sind Einflussfaktoren auf die Teameffektivität?
%Was ist ein Team?
%Was ist ein virtuelles Team?
\section*{Einleitung}\markboth{Einleitung}{Einleitung}\addcontentsline{toc}{section}{Einleitung}
	Mit voranschreitender technologischer Entwicklung, rückt die digitale Kommunikation immer mehr in den Mittelpunkt. Unternehmen weltweit setzen schon seit langem darauf, räumliche und zeitliche Grenzen zu überwinden. Begonnen mit der Implementierung eines Telegrafennetzes in den 1840er Jahren über das Telefon in den 1870er Jahren bis hin zur E-Mail 1970 und dem World-Wide-Web einige Jahre später.
	Seit den 1990er Jahren ist die Kommunikation mittels Computer nicht mehr wegzudenken. Der Computer steht in jedem Büro, in nahezu jedem Haushalt. Chats, E-Mails, das World-Wide-Web sowie Video- und Sprachkommunikation sind zu Standartkommunikationsmittel der heutigen Zeit geworden \citep[p. 14-16]{thurlow2004computer}.
	
Neue Generationen von Sozialen-Netzwerksystemen werden mit der Prämisse erstellt, die Kommunikation zu entfernten Personen zu verbessern.
Einsatzfelder sind dabei :
\begin{itemize}
	\item{\textbf{Gemeinsame Arbeitsumgebungen}: Gemeinsame Arbeitsumgebungen sind zunehmend auf digitale Kommunikation angewiesen.}
	\item{\textbf{Die Mobil- und Internettelefonie:} Die Mobil- und Internettelefonie bieten zunehmend ständigen, von Raum und Zeit unabhängigen, sozialen Kontakt zu anderen Nutzern}
	\item{\textbf{Agentenbasierte Schnittstellen:} Hilfsagenten auf Websites, Charaktere in \ac{sve}'s sowie Charaktere in Computerspielen bieten \dq{}quasi\dq{}-sozial Beziehungen. Diese \dq{}quasi\dq{}-sozialen Beziehungen bieten eine neue Form von Mensch-Maschinen-Kommunikation in \ac{sve}'s.} 
	\item{\textbf{FOIP/VOIP-Telefonkonferenzen:} Telekonferenzen ermöglichen eine simulierte Kommunikation von Angesicht zu Angesicht und weitere Soziale-Interaktionsmöglichkeiten.}
	\item{\textbf{Sprach-interfaces:} Simulationen von sozialen Interaktionen mit dem Computer durch die menschliche Sprache.}
	\item{\textbf{Soziale virtuelle 3D-Umgebungen:} Ermöglichen eine soziale Interaktion mit vollständig dargestellten Avataren.}
\end{itemize}

All diese Technologien teilen dasselbe Ziel : \\ \dq{}Die Verbesserung der \dq{}Social-Presence\dq{}, so dass der Nutzer das Gefühl hat zu einem gewissen Grad Einblicke in die kognitiven und affektiven Zustände des anderen zu haben \dq{} \citep{biocca2002defining} \citep[p.407–447]{biocca2001plugging}.

Teambuilding und Zusammenarbeit zwischen entfernten Personen wird heute immer wichtiger für die Effizienz von Unternehmen. Mitarbeiter befinden sich sehr häufig nicht am selben Ort, wobei viele Unternehmen sich trotzdem eine effektive Gestaltung ihrer Teams wünschen \citep[p.791-792]{jarvenpaa1999communication}. \ac{vts} können hierbei Abhilfe schaffen. 
	
Erst seit ungefähr 10-15 Jahren sind \ac{vts} aus der bis dahin vorhandenen Nische in den Alltag von Unternehmen eingezogen \citep{gilson2015virtual}.

Im Jahr 2020, vor der Corona Pandemie im 2. Quartal 2020, haben 40\% aller Angestellten in Deutschland von zu Hause im \dq{}Homeoffice\dq{} gearbeitet. Dieser Anteil ist im Laufe des Jahres - Stand 03.08.2020 - auf 60\% gestiegen und es könnten theoretisch 80\% der Belegschaften von zu Hause arbeiten.\citep{statistaCorona2020}. Durch diese Entwicklung, mussten Unternehmen sich zwangsläufig mit der Funktionsweise von \ac{vts} beschäftigen.

Trifft sich ein virtuelles Team in einer \ac{vr}, können Avatare \footnote{Grafikfigur, die die Onlinerepräsentation eines Nutzers in einer virtuellen Umgebung darstellt \citep[p.1]{neustaedter2009presenting}.} zur Repräsentation des eigenen Individuums eingesetzt werden. Durch diese wird mit anderen Teilnehmern des \ac{sve} interagiert und kommuniziert.

\newpage
	\subsection{Motivation}
	Um ein gutes Arbeitsklima für zukünftige Zusammenarbeiten in einem Team zu schaffen, ist die Anfangsphase einer Teamgründung von großer Bedeutung. In dieser Zeit werden wichtige Richtungsweisende Grundsteine gelegt, die den Erfolg oder Misserfolg eines Teams bestimmen können. Charakterzüge der Mitglieder werden kennengelernt und es werden Beziehungen untereinander aufgebaut.
	
	Viele Unternehmen setzen, aufgrund der wachsenden Globalisierung, auf geografisch trennte Teams um Aufgaben effizient zu bearbeiten. Die Teambuilding in einem räumlich getrenntem Team spielt eine ebenso große Rolle wie in einem Team, das die Möglichkeit hat sich in Persona kennenzulernen.
	
	In einem räumlich getrennten Team zu arbeiten, das sich gegenseitig nicht Vertraut oder nicht richtig Zusammenarbeitet, hemmt die Performance dieses \citep[p. 98-107]{huang1998supporting} \citep[p. 399-417]{turoff1993distributed}.
	
	Teambuilding steht jedoch nicht nur in der Kennenlernphase eines Teams im Fokus, denn es wird unter anderem besonders dann benötigt, wenn ein Team beispielsweise zu langsam arbeitet, die individuelle Leistung eines Teammitglieds nicht genügt, Konflikte entstehen oder die Gruppendynamik nicht dem soll entspricht \citep[p. 1-3]{biech2007pfeiffer}.
	
	Durch voranschreitende Forschung, ist es heutzutage möglich, dass sich viele Personen gleichzeitig in einem \ac{sve} befinden. Dadurch ist es in \ac{sve}'s möglich, auch Teambuildingmaßnahmen durchzuführen, wenn sich Teams räumlich getrennt voneinander befinden.
	
	Die Repräsentation eines Individuums innerhalb eines \ac{sve} kann sich von \ac{sve} zu \ac{sve} unterscheiden.
	
	\subsection{Ziele der Arbeit}
Es gilt herauszufinden, welche Art von Repräsentation in einem \ac{sve}, in der \dq{}Kennenlernphase\dq{} eines virtuellen Teams, mehr zwischenmenschliches Vertrauen aufbaut. Dabei wird der Fokus auf die beiden Konditionen \ac{ik} sowie \ac{nik} gelegt um zu analysieren, bei welcher dieser Konditionen eine größere Effektivitätssteigerung, während der erstmaligen Zusammenarbeit, vorhanden ist.
Weiterhin wird ein spezieller Fokus auf das kognitive Vertrauen in die Teammitglieder, den generellen Hang zum Vertrauen der einzelnen Personen sowie die daraus resultierende Team-Effektivität, im Hinblick auf die unterschiedlichen Konditionen, gelegt. Dazu bestreitet ein drei-Personen Team, welches sich noch nicht vorher kennt, in einem \ac{sve} eine kooperative Aufgabe.

	Durch diese Forschung soll es leichter möglich sein, eine Entscheidung über die Darstellungsweise der Repräsentation eines Avatars in einem \ac{sve} zu treffen, um die Zusammenarbeit eines virtuellen Teams über \ac{vr} Effektiver zu gestalten.
	Es sind verschiedene Hypothesen aufgestellt, anhand denen es möglich ist, das kognitive Vertrauen in das Team, den generellen Hang zum Vertrauen einer Person, die Teameffektivität und deren Wechselwirkungen zu analysieren.
%	Anhand verschiedener Faktoren, die zum Erfolgreichem messen von Teambuildingmaßnahmen ausgewählt wurden, werden die zuvor definierten Hypothesen ausgewertet.
	Diese Arbeit ist dem Gebiet der Virtuellen Realität und Sozialpsychologie zuzuordnen, speziell der virtuellen Teams in der Virtuellen Realität.
	In diesem Bereich gibt es noch nicht viel Literatur, weshalb eine Zeitgemäße Betrachtung und Analyse der Kombination von virtueller Realität und Sozialpsychologie als Sinnvoll erachtet wird.

%Repräsentationen in der virtuellen Welt haben einen großen Einfluss auf unterschiedliche Faktoren. Sie helfen dabei die andere Person in der \ac{vr} zu Lokalisieren, diese Wahrzunehmen, zu Identifizieren und zu verstehen, was eine andere Person aktuell tut. \citep[p.{pan2017impact} Daher gilt es herauszufinden, ob verschiedene Repräsentationen in einem \ac{sve} einen unterschiedlichen Einfluss auf die Team-Effektivität und das Vertrauen in das Team haben.\\


%	\subsection{Inhaltlicher Aufbau der Arbeit}
%	Diese Masterarbeit ist im wesentlichen in \textit{6 Kapitel} aufgeteilt.
%	Im \textbf{Kapitel 1} die Grundlagen der \ac{vr} und des Teambuilding erklärt. in der \ac{vr} wird speziell auf die Punkte .... eingegangen.\\
%	Im Teambuilding wird auf die Punkte ... eingegangen.\\
%	Anschließend werden im \textbf{Kapitel 2} zu dem zu untersuchendem Gegenstand \textbf{3.4.5?} Hypothesen aufgestellt anhand dessen das im \textbf{Kapitel 3} beschriebene Experiment im \textbf{Kapitel 4} ausgewertet werden kann.\\
%	\textbf{Kapitel 3} Beschäftigt sich mit der Vorgehensweise der Datenerhebung.
%	Es werden die Teilnehmer, Abhängigen sowie die unabhängigen Variablen erläuter und die Untersuchungsmethode sowie die Aufgabe der Teilnehmer dargestellt.\\
%	Das \textbf{Kapitel 4} beschäftigt sich Hauptsächlich mit der Analyse der Ergebnisse die im Kapitel 3 beschrieben wurden.\\
%	\textbf{Kapitel 5} fasst alle Ergebisse zusammen.\\
%	\textbf{Kapitel 6} beschäftigt hauptsächlich mit der Diskussion der Ergebnissen, der Eingesetzten Methoden und der Auswirkung der Ergebnisse auf den aktuellen Stand der Technik.\\
%	
%	\subsection{Verwandte Arbeiten}
	%https://www.researchgate.net/publication/320312582_AMELIO_Evaluating_the_Team-building_Potential_of_a_Mixed_Reality_Escape_Room_Game
	
	%https://www.ncbi.nlm.nih.gov/pmc/articles/PMC5730128/
	
	%https://pdfs.semanticscholar.org/5a82/6c0d065ef551ce7d8477dc5bbc475c8a9300.pdf
	
%https://www.researchgate.net/publication/323594629_Trusting_Strangers_in_Immersive_Virtual_Reality/link/5d3d981fa6fdcc370a666bb3/download
	
%https://d1wqtxts1xzle7.cloudfront.net/30602859/Presence_202004_20-_20Valencia.pdf?1361179740=&response-content-disposition=inline%3B+filename%3DOn_the_importance_of_reliable_real_time.pdf&Expires=1600803758&Signature=VH~~iWA9W1I3kVcnQySsr3qL0-2xMaNabSR76dVecHZv-icRMoouWAKZ6MDkHXfcTy56wkD31~TogcEPLHpBJgek0u8wj-Q6l2uKXvyqcXJO05r-fLGaeZ~Qxl--Mp~4C5ZyS9~DWbcAGK~40WMitvhrQZDhor-VWiVGp1wzn6~-xc1bW9BtlTJqSK4h0Q~5zeqeZSV9mXOuKN5AIQ8Qz1xLwC9KGhHqtDAc5pe8Mczf8uJFzaIfzMi5kwgGV-E1A~7upXbkuO1cktiHi2GhTADTVNsN7Ml~Bn3qNJeb4SoIvkEanhZfvuXY9SJIewThSFVImjfZrRMfTWqlhwxYnA__&Key-Pair-Id=APKAJLOHF5GGSLRBV4ZA#page=54


%
%Grundlage für das Framework \autoref{Framework} sind verschiedene wissenschaftlich basierten Quellen, die sich mit der \ac{vr} sowie zwischenmenschlichem Vertrauen befassen. Diese beiden Teilbereiche wurden kombiniert und soll versuchen, die Team-Effektivität anhand variierendem zwischenmenschlichen Vertrauen und verschiedenen Avatar-Verkörperungen zu messen.
%
%Wie an dem Framework zu erkennen ist, wird davon ausgegangen, dass die beiden Faktoren \dq{}Generelles Vertrauen\dq{} sowie \dq{}Vertrauen in das Team\dq{} das \dq{}Zwischenmenschliche Vertrauen\dq{} zu gleichen Teilen beeinflussen. 
%
%Weiterhin hat die Kombination des \dq{}Avatar-Embodiment\dq{} sowie des \dq{}Zwischenmenschlichem Vertrauen\dq{} ebenfalls einen Einfluss zu gleichen Teilen.
%
%Da dies jedoch das erste mal ist, dass dieses Framework theoretisch Aufgestellt wurde, kann es sein, dass die verschiedenen Faktoren nicht zu gleichen oder auch gar keinen Einfluss auf die Team-Effektivität haben.

	\newpage
\section{Theoretischer Rahmen}

%\subsection{Team und Gruppen}
In diesem Kapitel werden die einzelnen Teilbereiche \dq{}Virtual-Reality\dq{}, \dq{}Avatare\dq{}, \dq{}Vertrauen\dq{} und \dq{}Teamwork\dq{} in einem grundlegenden Umfang  erläutert.

Zu Beginn der Grundlagen und ohne Kapitel sollte jedoch als aller erstes die Begrifflichkeiten \dq{}Effizienz\dq{} und \dq{}Effektivität\dq{} in Bezug auf virtuelle Teams abgegrenzt werden.

Die Begriffe Effektivität und Effizienz werden häufig als Synonyme benutzt. Dies ist jedoch nicht korrekt. Es gibt eine eindeutige Abgrenzung dieser, wobei beide einen eindeutigen Schwerpunkt auf den zu Analysierenden Inhalt setzen.

%Die Effizienz ist in dieser Arbeit nicht von bedeutung. Es geht nicht darum, herauszufinden, durch welche Avatarkondition effizienter in einem \ac{sve} eine gemeinsame Aufgabe erledigt werden kann.
%Es geht darum, zu schauen, durch welche Avatarkondition Effektiv Vertrauen aufgebaut werden kann, um so eine eventuelle Effektive steigerung der Teameffektivität durch eine bestimmte Avatarkontion auszumachen.
\paragraph{Effizienz}
Bei der Effizienz geht es darum, sein Handeln so zu optimieren, dass ein Ziel möglichst schnell und mit möglichst geringem Aufwand erreicht wird. Die Wirtschaftlichkeit steht bei der Effizienz im Vordergrund. Ergebnis und eingesetzte Mittel müssen dabei immer in einem möglichst günstigen Kosten-Nutzen-Verhältnis stehen.\\
$Effizienz = \frac{Ergebnis}{Aufwand}$\\
Ist ein Team effizient, wenn es sich nur über ein Telefon unterhalten würde, obwohl es sich auch in Persona treffen könnte?
Nein wäre es nicht, da der wirtschaftliche Aspekt der Telefonkosten mit eingerechnet werden muss. Somit wäre es günstiger (effizienter) sich in Persona zu treffen. Ebenfalls ist es Aufwändiger das Telefon in der Hand zu halten, Nummern zu wählen, etc.

Anders ausgedrückt : Effizienz beschreibt Mittel und Wege zur Erreichung der Effektivität.

Es müssen sich zur Erreichung von Effizienz immer folgende Fragen gestellt werden :
Gehen wir den Weg des geringsten Aufwands, um unser Ziel zu erreichen?
Tun wir die Dinge richtig?

\paragraph{Effektivität}
Bei der Effektivität geht es darum, die Dinge zu tun, die einem dem Ziel näher bringen. 
Somit arbeitet ein Team Effektiv, wenn es die richtigen Maßnahmen ergreift um dem zu erreichendem Ziel näher zu kommen.\\
$Effektivität = \frac{Ergebnis}{Ziel}$ \\
Ist ein Team effektiv, wenn es sich nur über ein Telefon unterhalten würde?
Effektiv ja, da eine Kommunikation zu den anderen Teammitgliedern stattfindet.
Effektiv ist es auch, wenn es sich in Persona trifft.

Anders ausgedrückt : Effektivität ist das Ausmaß der Erreichung angestrebter Ergebnisse/Ziele/Zwecke.

Somit kann bei dem Begriff \dq{}Effektivität\dq{} auch immer der Grad der Wirksamkeit berücksichtigt werden.

Somit stellen sich bei der Frage nach der Effektivität folgende Fragen: 
Bringt uns die Maßnahme dem Ziel näher? 
Tun wir die Dinge, die uns voranbringen?
Gibt es eine Maßnahme, die einen höheren Grad an Zielerreichung mit sich bringt?

%\paragraph{Abgrenzung zu dieser Ausarbeitung}
In dieser Masterarbeit wird geschaut, in welchem Ausmaß Vertrauen bei unterschiedlichen Avatar Konditionen in einer Teambuildingmaßnahme gebildet wird und wie sich dieses auf die Teamleistung bezieht.

Effizienz ist wichtig, aber unter uneffektiven Voraussetzungen Dinge effizient zu tun bringt dem Team in einer Teambuildingmaßnahme keinen Mehrwert.

So muss sich in Bezug auf diese Masterarbeit die Frage gestellt werden, welches die richtige Kondition (\ac{ik} oder \ac{nik}) in einer Teambuildingmaßnahme ist um einen höheren Grad an Effektivität zu erzielen.
	\subsection{Virtual Reality}
	\label{Virtual Reality}
%		\subsubsection{VR - Wozu brauchen wir Virtual Realitky?}
		\subsubsection{Virtuelle 3D-Welten}
Virtual-Reality ist eine Realität, die durch den Computer, geschriebenen Computercode sowie erstellte 3D-Welten, abgebildet und zum Leben erweckt wird. Dabei Spielt der Nutzer, das umschließende Erlebnis sowie die Interaktivität in der \ac{vr}, eine zentrale Rolle \citep[p.6-12]{sherman2018understanding}.
	Seit vielen Jahren sind \ac{sve}'s Forschungsgrundlage der Virtuellen Realität. Siehe \citep{shuffler2011there} \citep{steed1999leadership} und \citep{de2011level}.

\paragraph{Shared-Virtual-Environment}
	\ac{sve}'s bieten die Möglichkeit, geographisch getrennte Personen in einer virtuellen Umgebung zu verbinden. Dadurch wird den Nutzern der virtuellen Realität die Möglichkeit bereitgestellt, durch ein Avatar miteinander kommunizieren und interagieren zu können \citep[p. 1-3]{pettifer1999designing}. Eine grundlegende und frühe Übersicht der Anwendungsgebiete eines \ac{sve} wurde von Richard Waters dargestellt. Siehe \citep{waters1997rise}. Die dort aufgeführten Anwendungsgebiete haben sich aufgrund der voranschreitenden Technologie weiterentwickelt und reichen heute von der Medizintechnik über die Konstruktion bis hin zur Lernsimulation.
	Zur Durchführung und zur Datenerhebung dieser Studie wurde ein \ac{sve} entwickelt, welches den Nutzern ein \dq{}Hand- und Kopf getrackten Avatar\dq{} (\ac{ik}) oder ein \dq{}Hand-, Kopf und Inverskinematisch-simuliertem Torso getrackten Avatar\dq{} (\ac{nik}), je nach Anwendungsfall, zur Verfügung stellt. Innerhalb des \ac{sve} können sich die Nutzer frei bewegen, andere Avatare Wahrnehmen und mit diesen interagieren.
	
	
		\subsubsection{Vorbedingungen für Presence}
Um die bestmöglichste Umschlossenheit in einer \ac{vr} zu erreichen, wird eine Schnittstelle zur Interaktion zwischen den Sinnen und der Außenwelt benötigt. Dies kann ein \ac{hmd} sein, welches ein computergeneriertes Bild erzeugt, durch das die \ac{vr} wahrgenommen wird. Darüber hinaus muss das \ac{hmd} in der Lage sein, den Kopf des Benutzers frei im Raum zu verfolgen und die gewonnenen Positionsdaten auf die \ac{vr} abzubilden. Optimalerweise werden Controller benötigt, durch deren Einsatz es möglich ist, auch die Handbewegungen der realen Personen zu Verfolgen. Durch die gewonnenen Positionsdaten kann die Position der Hände des Nutzers in der \ac{vr} dargestellt werden. Das \ac{hmd}, die Controller, eventuelle zusätzliche Körpertracker, Kopfhörer, eventuell wahrgenommener Geruch etc. definieren, zu welchem Ausmaß Sinnesmodalitäten in der \ac{vr} angesprochen werden. Der Grad der Immersion hängt somit direkt mit der Anzahl der angesprochenen Sinnesmodalitäten zusammen. Je mehr Sinnesmodalitäten gleichzeitig angesprochen werden, desto mehr ist der Wahrnehmungsapparat in der Lage, die virtuelle Umgebung auf die reale Welt abzubilden.

Somit lässt sich sagen, dass die Voraussetzungen für \dq Presence\dq{} in der \ac{vr} die Korrelation zwischen den Sinneseindrücken, der Propriozeption \footnote{die Wahrnehmung des eigenen Körpers nach dessen Lage im Raum} und dem Grad der wahrgenommenen Realität der Illusion, sich in einem stabilen räumlichen Ort zu befinden, darstellt. Sind diese Voraussetzungen gegeben, kann der Nutzer einen plausiblen Vergleich zwischen realen sensorischen und virtuellen, durch Illusion erzeugten Daten, aufstellen \citep{slater2009we}.

		\subsubsection{Presence in Virtual Reality}
			
%	Dank heutiger Technologien ist es uns möglich, zu jeder Zeit mit Personen an verschiedenen Orten zu interagieren und zu kommunizieren. Kommunikation nicht mehr nur auf die Personen in unserer unmittelbaren Umgebung beschränkt. 
	Das Voranschreiten der Technologie ermöglicht es, uns nicht mehr nur auf soziale Interaktionen mit physischen Wesen zu beschränken, sondern erweitert diese auch auf Repräsentationen geschaffen aus Pixeln, die E-Mails, den Film oder durch das Telefon. Je nachdem, wie stark diese Repräsentation von uns Wahrgenommen wird, schafft Sie es, kraftvolle Emotionen in uns auszulösen \citep[p. 4-6]{biocca2002defining}.
	
Nur wenn eine gewisse \dq{}Presence\dq{} dieser Repräsentation besteht, kann \textbf{Vertrauensbildung} stattfinden. Somit definiert das Vorhandensein des Gefühls von \dq Presence \dq{} den Grundbaustein für alle weiteren Schritte zum Aufbau von Vertrauen.

Der Begriff \dq{}Presence\dq{} ist nicht genau definiert. Am ehesten trifft die Beschreibung zu, dass \dq{}Presence\dq{} das subjektive Empfinden ist, an einem anderen Platz zu sein, obwohl man physikalisch eigentlich woanders ist \citep[p. 1]{witmer1998measuring}.

	Nimmt eine Person eine andere Person in einer \ac{vr} als Präsent wahr, werden die Wahrnehmenden, vestibulären, propriozeptiven und autonomen Nervensysteme in einen Zustand gebracht, der einem realen Zustand gleicht. Obwohl die betroffene Person weiß, dass Sie sich nicht in einer realen Lebenssituation befindet, wird diese dazu neigen, sich so zu verhalten, als ob diese in realen Lebenssituation ist und ähnliche Gedanken und Gefühle haben \citep{slater2003note}.

Diesbezüglich kann \dq{}Presence\dq{} als eine Art von Illusion angesehen werden, da die erzeugten Stimuli in der \ac{vr}, wie in der realen Welt auch auf unsere Rezeptoren projiziert werden.

Somit lässt sich \dq{}Presence\dq{} in der \ac{vr} in 4 verschiedene Teilbereiche unterteilen.

\begin{itemize}
	\item{\textbf{Die Illusion, sich in einem stabilen räumlichen Ort zu befinden}}, ist der wichtigste Aspekt um Presence zu erzeugen. Alle Stimuli zur räumlichen Wahrnehmung - wie zum Beispiel keine Restriktionen des \dq{}Field-of-View\dq{} \footnote{Sichtfeld}, keine Kabel am \ac{hmd} - sollten sich möglichst wie in der realen Welt verhalten \citep[p.47]{jerald2015vr}.
	\item{\textbf{Die Illusion der Selbstverkörperung}} beschreibt das Gefühl einen Körper in der virtuellen Umgebung zu haben. Studien fanden heraus, dass durch einen v.irtuellen Körper die \dq{}Presence\dq{} in der \ac{vr} stark steigt, da der Mensch sein Leben lang sich an einen(seinen) Körper gewöhnt hat \citep[p.756]{botvinick1998rubber}. Der virtuelle Körper muss nicht unserem eigentlichen ähnlichsehen \citep[p.7]{maxwell1960psycho}.
	\item{\textbf{Die Illusion von körperlichen Interaktionen}} beschreibt beispielsweise das vorhanden sein von Audio-Feedback, die Vibration des Controllers während etwas aufgehoben wird oder visuelle Highlights. Der Nutzer einer bekommt dadurch ein Gefühl mit der \ac{vr} zu interagieren. Diese Kleinigkeiten tragen eine große Menge dazu bei, \dq{}Presence \dq{} in der \ac{vr} zu steigern \citep[p.48]{jerald2015vr}.
	\item{\textbf{Die Illusion von sozialer Kommunikation}} Sozialer-Realismus kann vom Physischen-Realismus abgetrennt werden. \dq{}Social-Presence\dq{} beschreibt das Gefühl, real mit jemandem in einem \ac{sve} zu kommunizieren. Dabei ist es irrelevant, ob der Kommunikationspartner Menschlich oder nicht menschlich ist. Sei es verbal oder durch Körpersprache. Je mehr Nutzer der virtuellen Welt sich so verhalten, als ob diese real wäre, desto mehr steigt auch die \dq{}Social-Presence\dq{} \citep[p.49]{jerald2015vr} \citep[p.12]{guadagno2007virtual}.
\end{itemize}

%\begin{figure}[H]
%		\begin{footnotesize}
%		\centering
%			\includegraphics[scale= 0.4]{Abbildungen/forms_presence.JPG}
%			\caption[Abbildung 1]{Forms of Presence}
%			\textit{Die Verschiedenen Formen von Presence laut U.  Schultze \citep{schultze2010embodiment} }
%			\label{vertical_horizontal}
%		\end{footnotesize}
%	\end{figure}


%\dq{}Presence\dq{} ist ein Konstrukt aus Immersion und dem Nutzer. Immersion kann ein Gefühl von \dq{}Presence\dq{} erschaffen, muss es aber nicht zwangsläufig. Je Immersiver ein \textbf{System} ist, desto höher ist das Potential dieses Systems, dass der Nutzer ein Gefühl von \dq Presence\dq{} entwickelt.
%			
%Dementsprechend unterscheidet Lombard 6-Arten von \dq Presence\dq{} :
%	\begin{itemize}
%		\item \textbf{Sozialer Reichtum} : Das Medium wird als empfindlich oder persönlich wahrgenommen, wenn es zur Interaktion mit anderen Menschen verwendet wird.
%		\item \textbf{Der Realismus} : Beschreibt die Wahrnehmung oder/und den Realismus der \dq Presence\dq{} und bis zu welchem Grad dieser als "Real" dargestellt werden kann.
%		\item \textbf{Das Transportmedium} : Dies beschreibt das Gefühl "Du bist da", "Es ist da", oder/und "Wir sind Zusammen".
%		\item \textbf{Die Immersion} : Beschreibt, wie Flächendeckend die Sinne des Benutzers angesprochen werden.
%		\item \textbf{Der soziale Akteur innerhalb des Mediums} : Beschreibt die Reaktion auf eine Repräsentation einer Person durch ein Medium, auch wenn diese Irrational begründet ist.
%		\item \textbf{Das Medium als sozialer Akteur} : Beschreibt die Situation wo der Akteur ( z.B. Computer ) selber als ein soziales Wesen wahrgenommen wird.					
%				\citep{lombard1997heart}
%			\end{itemize}
			
%Telepresence
%Co-Präsenz
%Selbst wahrgenommene Copräsenz
%Others Co-Präsenz
%wie im Fragebogen halt!

%https://scihubtw.tw/10.1057/jit.2010.25

		\subsubsection{Presence, Co-Presence und co.}
Wenn Personen sich zusammen in der \ac{vr} befinden und die andere Person wahrnimmt, wird dieses Gefühl mit \dq{}Co-Presence\dq{} bzw. \dq{}Social-Presence\dq{} bezeichnet. Aber auch die \dq{}Telepresence\dq{} und die \dq{}Selbstpräsenz\dq{} tragen einen wichtigen Teil zum Aufbau und Erhalt von Immersion bei \citep{schuemie2001research}.

\textbf{\dq Co-Presence\dq{}} bezeichnet das Gefühl, mit einer anderen Person in Verbindung zu stehen.
Es wird die Anwesenheit der anderen Person in der \ac{vr} gespürt und es wird wahrgenommen, dass die andere Person ebenfalls spürt, dass man Selbst in der \ac{vr} anwesend ist. 
Co-Presence dementsprechend als eine psychologische Verbindung \textit{zu und mit} dem anderem charakterisiert werden \citep[179-182]{ijsselsteijn2001presence}.

Fühlt sich ein Nutzer der \ac{vr} \dq{}innerhalb\dq{} des Mediums \ac{vr}, so wird dies als \textbf{Telepresence} bezeichnet. Telepresence bezeichnet das Gefühl \dq{}da zu sein\dq{}. Je höher das Level wahrgenommener Telepresence ist, desto weniger fühlt sich der Nutzer an dem Ort seines physikalischen Körpers und mehr an einem anderen Ort \citep[p.482]{nowak2004effect}. Es beschreibt die Illusion, durch die \textbf{Tele}kommunikation, an einem anderen, weit entfernten, realen Platz, zu sein. Der Begriff Telepresence wird seit den Forschungen von Biocca \cite[p.12]{biocca1999cyborg} im Allgemeinem \ac{vr}-spezifischen Sprachgebraucht als die Illusion des \dq{}being there\dq{}, oder einfach als \dq{}Presence\dq{} genutzt

Der Begriff der \dq{}\textbf{Social-Presence} \dq{} beschreibt, wie stark ein Nutzer eine Person, mit der dieser nur mittels Kommunikationstechnologie kommuniziert, als \dq{}real\dq{} bezeichnet. Social-Presence wird dabei auf die Kommunikationstechnologie an sich bezogen. Je mehr Social-Presence vorhanden ist, desto besser ist ein Kommunikationsmittel geeignet um Informationen über den Interaktionspartner zu vermitteln. So erzeugt ein Videochat durch die zusätzliche Übertragung eines Videos mehr Social-Presence als eine Telefonkommunikation \citep[p.151]{gunawardena1995social}.
Es wird davon ausgegangen, dass eine \ac{sve}'s mehr Social-Presence erweckt als Beispielsweise ein Telefonat. Dies sei damit begründet, dass \ac{sve} die Eigenschaften und Interaktionen des anderen besser einfängt und darstellt. Desto mehr Eigenschaften einer Person dargestellt werden können, desto höher ist der wahrgenommene Realitätsgrad des anderen \citep[p. 5-8]{biocca2002defining}.

\textbf{Selbstpräsenz} in der \ac{vr} leitet sich von der realen Wahrnehmung über unseren eigenen Körper im täglichen Leben ab. Das Gefühl der realen Selbstpräsenz kann auf den virtuellen Raum übertragen werden, wenn die Person in der \ac{vr} durch einen Avatar repräsentiert wird. Die Selbstpräsenz ist hoch, wenn ein Nutzer der \ac{vr} keinen Unterschied zwischen sich und seiner Digitalen Repräsentation wahrnimmt \citep[p.439]{schultze2010embodiment}.

Somit reicht das gesamte Kontinuum der \dq{}Presence\dq{} von der räumlichen Komponente bis hin zur starken psychologischen Beteiligungen. Dies macht es möglich, auch auf die affektiven und kognitives Zustände von Personen aufzudecken. Höhere wahrgenommene Presence führt dazu, dass die Person sich mehr mit der \ac{vr} engagieren kann, was zu Handlungen führt, die als verbunden und voneinander abhängig wahrgenommen werden \citep{biocca2001criteria}.

Um Vertrauen optimal aufbauen zu können, sollte das gesamte \ac{sve} so real wie möglich aufgebaut sein. Diesbezüglich richtet sich diese Arbeit an nahezu alle Arten der hier genannten \dq{}Presence\dq{}.

%\subsubsection{Presence und Teambuilding}

\newpage

\subsection{Avatare}
\label{Avatare}
\subsubsection{Repräsentationen von Avataren}

Das Menschliche Gehirn ist in der Lage, computergenerierte Darstellungen in \dq{}Lebend und nicht Lebend\dq{} zu kategorisieren. Einige Forschungen gehen davon aus, dass das menschliche Gehirn semantische Unterschiede im Zusammenhang mit der \dq{}Social-Presence\dq{} feststellen kann. So kann eine menschenähnliche Form als biologisch oder nicht-Lebend erkannt werden. 
Ein Avatar wird als eine Grafikfigur, die die Onlinerepräsentation eines Nutzers in einer virtuellen Umgebung darstellt, definiert \citep[p.1]{neustaedter2009presenting}. In der \ac{vr} hilft dieser dabei, die andere Person in der zu Lokalisieren, diese Wahrzunehmen, zu Identifizieren und zu verstehen, mit wem oder was die Person aktuell interagiert \citep[]{pan2017impact}.

	\begin{figure}[b!]
		\begin{footnotesize}
		\centering
			\includegraphics[scale= 0.3]{Abbildungen/Symmetry.JPG}
			\caption[Abbildung 1]{Social presence response to vertical and
horizontal beings}
			\textit{Menschen interpretieren symmetrische Formen um eine vertikale Achse eher als \dq{}Menschlich\dq{} als Formen um die horizontale Achse \citep{biocca2002defining}. }
			\label{vertical_horizontal}
		\end{footnotesize}
	\end{figure}

In \autoref{vertical_horizontal} ist zu erkennen, dass eine menschenähnliche vertikale, bilateral Symmetrische Repräsentation mehr \dq{}Co-Presence\dq{} erweckt als die horizontale bilateral Symmetrische Repräsentation \citep[p.546-551]{thornhill1998relative}.
Bilateral-Vertikale Symmetrie wird mit der körperlichen Gesundheit eines Menschen in Verbindung gebracht. Sogar Weibchen verschiedener Spezies neigen dazu, Partner mit einem höheren Grad an bilateraler Symmetrie für die Fortpflanzung auszuwählen \citep[p. 659–669]{rhodes1998facial} \citep{biocca2002defining} \citep[p.233–242]{grammer1994human}.

\begin{figure}[b!]
		\begin{footnotesize}
		\centering
			\includegraphics[scale= 0.5]{Abbildungen/moving_dots.JPG}
			\caption[Abbildung 1]{Social presence response to vertical and
horizontal beings}
			\textit{Stationäre punkte, bei denen der Mensch eine lebendige Bewegung ausmachen kann, wenn diese Anfangen sich zu bewegen. Es ist sogar möglich einen Menschen zu erkennen, dessen Art der Aktivität sowie den emotionalen Zustand \citep{biocca2002defining} \citep[p.76-89]{johansson1975visual}.}
			\label{moving_dots}
		\end{footnotesize}
	\end{figure}

Schon sich nur bewegende Punkte können als intelligente Wesen wahrgenommen werden. Johannson \citep[p.76-89]{johansson1975visual} führte eine Studie durch, in der die Teilnehmer dreizehn \autoref{moving_dots} sich bewegende Punkte sahen und sofort die Darstellung einer menschlichen Bewegung erkannten. Als die Teilnehmer Punkte sahen, die stationär waren, ist es ihnen nicht möglich gewesen, diese Punkte als menschliche Repräsentation zu erkennen. Wenige Punkte reichen aus, um Informationen zu erzeugen, die Aufschluss über die Aktivität, das Geschlecht, die Bewegung, den emotionalen Zustand oder die Anzahl der Personen zu geben.

\subsubsection{Vorwärtskinematik vs Inverse-Kinematik}
Die Kinematik beschäftigt sich mit der mathematischen Bewegung einzelner Gelenke, ohne dabei die physikalische Kraft zu berücksichtigen. \citep[p.1]{beggs1983kinematics}

Die Vorwärtskinematik beschäftigt sich mit der Frage, wie aus bekannten Positionen und Winkeln einzelner Gelenke, ein Zielpunkt in einem bestimmten Koordinatensystem bestimmt werden kann. Damit ist die Vorwärtskinematik das Gegenstück zur Inversen-Kinematik. \citep[p.119]{kucuk2006robot}
\label{inverseKinematik}
Die Inverse-Kinematik beschäftigt sich damit, herausfinden, wie Gelenke eines Roboterarms bewegt und gedreht werden müssen, um Beispielsweise einen Gegenstand von Punkt A nach Punkt B zu bewegen. Es wird anhand der Position des Zielpunkts, alle Winkel der zugehörigen Gelenke approximiert um somit die zugehörigen Rotationen dieser zu bestimmen. \citep[p.299-300]{d2001learning} Inverse-Kinematische Berechnungen können auf Arme, auf den Torso, die Füße, den Kopf, die Beine, etc., eines Avatars angewand werden um diesen zu animieren. Benutzt eine Person in einem \ac{sve} einen Inverse-Kinematisch dargestellten Self-Avatar, kann dieser die Torsorotation, Armbewegungen oder Beinbewegungen seines eigenen Avatars sehen \citep[p.1]{kim2018real}

\subsubsection{Selbst-Avatar und Nicht-Selbst-Avatare}

In der \ac{vr} und in einigen anderen computerbasierten Medien, kann der Nutzer sich Avatare erstellen und mit diesen interagieren. Ein Avatar bezeichnet eine Grafikfigur, die die Onlinerepräsentation eines Nutzers in einer virtuellen Umgebung darstellt \citep[p.1]{neustaedter2009presenting}.

\ac{hmd}'s beeinflussen das Sichtfelds des Nutzers so stark, dass diese Ihre eigenen Körper nichtmehr sehen können. Um diesem Nachteil entgegenzuwirken, kann einem Nutzer ein virtueller Körper zur Verfügung gestellt werden. Dieser Körper wird Selbst-Avatar genannt.
Es ist schwierig einen Selbst-Avatar hoher Qualität zu simulieren. Dazu wäre im Idealfall das Verfolgen und Animieren mehrerer Körperteil unabdingbar. Ist der Selbst-Avatar schlecht animiert oder es entstehen während der Nutzung Trackingfehler, die der Nutzer erkennt, kann sehr es leicht zu einem \ac{bip} kommen. Bei diesem \ac{bip} bricht die gesamte Illusion der \ac{vr} für den Nutzer in sich zusammen. 
Dies ist auch der Grund, weshalb relativ wenige \ac{vr}-Anwendungen einen menschlichen Körper als Avatar darstellen.
Sind jedoch genügend Körperteile getracked und animiert, muss der Avatar nicht unverwechselbar menschlich aussehen um einen glaubwürdigen Selbst-Avatar zu vermitteln. Selbst grobe Avatar-Darstellungen schaffen es, ausreichende Informationen über die Glaubwürdigkeit eines menschlichen Körpers zu Vermitteln \citep{lok2003effects} .

Biocca forschte Umfangreich über den Einfluss von Self-Avataren auf den Nutzer in der \ac{vr} \citep[421-427]{construal2014connected}.
\newline So beispielsweise, wie sich die Interaktion mit der Welt verändert, wie sich soziale Interaktionen verändern und wie Aufgaben wahrgenommen und bearbeitet werden \citep{benford1995user} \citep{bowers1996talk}.
Bioccas Studien gehen davon aus, dass ein menschenähnlicher Körper als Avatar die Self-Presence stark erhöht. Damit die Self-Presence noch stärker zu erhöht wird, kann ein Inverse-Kinematisch simulierter Avatar als Self-Avatar genutzt werden\citep[421-427]{construal2014connected}.

Yee und Bailenson fanden heraus, dass die Selbst-Presence Einfluss auf den Proteus-Effekt hat. Je höher die Selbst-Presence einer Person in der \ac{vr}, desto stärker wirkt auch der Proteus-Effekt. Der Proteus-Effekt beeinflusst die Person so, dass sozialen Interaktionen der Person denen der sozialen Identifikation seines Avatars ähneln. So nimmt eine Person, das Verhaltensmuster entsprechend zu dem Aussehen des Avatars ein. Je nachdem, wie die Person glaubt, dass bestimmte Verhaltensweisen von dieser, aufgrund des Avatars, erwartet werden \citep{ratan2015leveling}.
Diese Abbildung der Verhaltensweisen lässt sich auch im Bezug auf Gruppen beobachten. Die Person wird sich in einer Gruppe so Verhalten, wie die Gruppenkonformität es vorgibt.
Dieser Effekt kommt nicht nur bei unverwechselbar menschenähnlichen Avataren vor, sondern kann auch bei anderen Avatararten auftreten. Solche identitätsbezogenen Avatar-induzierten Effekte, können die \dq{}Kognitiven-Einstellungen\dq{} (siehe Kapitel \textit{\nameref{Vertrauen}}) und Verhaltensweisen anderer Personen gegenüber, beeinflussen \citep{lok2003effects}.

Es wurden zahlreiche weitere Forschungen durchgeführt, um den Einfluss von Avataren zu erforschen. So gibt es Experimente, die den Einfluss des Geschlechts \cite{slater2010first}, der Hautfarbe \cite{peck2013putting} oder des Grades des Realismus \cite{roth2016avatar}, als Inhalt ihrer Forschungen haben.

Es stellt sich die Frage, ob ein Avatar menschenähnlich Aussehen sollte. Dieser Frage gingen George, Eiband, Hufnagel und Hussman \cite{george2018trusting} in Ihrer Forschung nach und verglichen, ob sich mehr Vertrauen zwischen einem menschenähnlichen oder einem roboterartigen Avatar aufbauen lässt.
Dazu schufen Sie ein Szenario, in dem Personen mittels eines \ac{hmd}, ein Social-Dilemma-Scenario \footnote{Situationen, in denen - die rationale Verfolgung von Eigeninteressen zu einer kollektiven Katastrophe führen kann} erlebten. Sie fanden keinen signifikanten Unterschied in der Vertrauenswürdigkeit zwischen menschenähnlichen und roboterartigen-Avataren. Jedoch wurde ein größeres Gefühl von Gemeinsamkeit festgestellt, wenn mit einem menschenähnlichen Avatar interagiert wurde \citep{kerr1983motivation}.
George, Eiband, Hufnagel und Hussman erwähnten weiterhin in Ihrer Studie, dass gute Grafik und realistisches Verhalten durch beispielsweise Mikrogestikulationen, soziale Interaktionen und den Aufbau von \dq{}Co-Presence\dq{} unterstützen \citep{george2018trusting}.

Dodds fand heraus, dass ein Selbst-Avatar eine wichtiger Faktor zur Erhöhung der Kommunikation in einem \ac{sve} darstellt \citep[1-11]{dodds2011talk}.
\label{AvatarTrust}
Um den Einfluss des Grades des Realismus unter Avataren zu erforschen, führten Riedl, Mohr und Kenning \cite{riedl2014trusting} eine Studie zum Vertrauensaufbau unter Menschen im Vergleich zu Avataren mit menschenähnlichen Gesichtern, durch. Sie fanden heraus, dass es Personen leichter fällt einzuschätzen, dass eine reale Person, im Gegensatz zu einem Avatar mit menschenähnlichem Gesicht, vertrauenswürdig ist. Es wurde der Frontalkortex - die Gehirnregion, der dafür verantwortlich ist einzuschätzen, wie die Gedanken und Gefühle des gegenüber sind - bei Interaktionen mit Menschen mehr angeregt, als bei Interaktionen mit Avataren mit menschenähnlichen Gesichtern.
Jedoch ist die Geschwindigkeit des Vertrauensaufbau zwischen Menschen und Avataren gleich. Vertrauen zwischen Menschen wird in der gleichen Geschwindigkeit aufgebaut wie zwischen Menschen und Avataren \cite{riedl2014trusting}.

Somit lässt sich feststellen, dass ein höherer Grad an Realismus den Vertrauensaufbau fördert, jedoch kein signifikanter Unterschied zwischen einem menschenähnlichen sowie Roboterähnlichen Avatar besteht. Diese Vermutung bestätigten auch Bente, Rüggenberg und Krämer \citep[p.54-59]{bente2004social} indem sie eine Studie zur Social-Presence von Avataren in einem \ac{sve} durchführten. Das \ac{sve} war ähnlich einer Videokonferenz aufgebaut. Es waren keine \ac{hmd}'s vorhanden und die Teilnehmer haben sich während des Experiments nicht gesehen. Sie verglichen die Kommunikationsarten, Face-to-Face, Chat und avatarbasierte Kommunikationsmedien untereinander, um Unterschiede in der Social-Presence sowie dem zwischenmenschlichen Vertrauen festzustellen.
Es wurde festgestellt, dass wenig kognitives Vertrauen während der Nutzung des \ac{sve} zu Avataren aufgebaut werden konnte, während Face-to-Face, Telefon und Chatkommunikationen besser abschnitten. Weiterhin wurde weniger affektives Vertrauen im \ac{sve}, als bei der Nutzung eines Telefons oder während der Face-to-Face Kommunikation, aufgebaut.
Bente, Rüggenberg und Krämer \citep[p.54-59]{bente2004social} gehen davon aus, dass dies mit der Neuheit der Technologie zusammenhängt.

\paragraph{Avatare und Kommunikation}
Allgemeine Kommunikation findet nicht nur mit Wörtern, sondern auch über Körpersprache wie Gestiken oder andere nonverbale Verhaltensmuster, statt. Durch \ac{sve}'s ist es nun möglich diese reine Text oder Bildbasierte Kommunikation durch Gestikulationen eines Avatars zu ersetzen. Durch diesen zusätzlichen visuellen Aspekt ist eine ganz neue Art und Weise der Kommunikation möglich, die auch gemessen werden kann. 
Nichtverbale Kommunikation -Beispielsweise nur via Text- erzeugt keinen Mehrwert an Vertrauen, Zusammenhalt und erzielt keine ausreichend gute Kommunikation \citep[p.81]{haslam2003social}.
Nonverbale Kommunikation kann Gesichtsausdrücke, Blicke, Bewegungen etc. umfassen. Kendon definiert im \dq{}Kendon Kontinuum\dq{} mit dem Begriff \dq{}Gestik\dq{} unwissende Gestiken (natürliche Körpersprache) bis hin zu \dq{} Zeichen\flqq, welche alle durch Gestikulation erzeugten Zeichen (z.B. O.K. Handzeichen ), beinhaltet \citep[37]{mcneill1992hand}.
Im Rahmen dieser Arbeit, ist mit Gestikulation jedoch die gesamte Körperbewegung des Avatars gemeint.
Mehrabian zeigt in seiner Forschung, dass nonverbale Kommunikation zu 55\% dafür verantwortlich ist, ob ein gesprochenes Wort Positiv, Negativ oder Neutral interpretiert wird. Die Tonhöhe trägt zu 38\% und das eigentlich gesprochene Wort mit 7\% zur Interpretation bei \citep[43]{mehrabian1971silent}.
Da 7\% des gesprochenen Wortes und 38\% der Tonhöhe zu Interpretation der nonverbalen Kommunikation beiträgt, musste diese Einflussgröße in dieser Studie ausgeschlossen werden.

Da in dieser Studie lediglich die beiden Controller sowie das \ac{hmd} getrackt werden, wurde sich dazu entschieden, keine Darstellung eines Selbst-Avatars zu wählen. Somit wurde die Möglichkeit ausgeschlossen einen \ac{bip} durch falsche Avatardarstellungen zu erzeugen. Stattdessen sehen die Nutzer des \ac{sve} menschenähnliche Hände \underline{ohne} ihre Extremitäten.

Die Avatare, die die \dq{}anderen\dq{} teilnehmenden Personen repräsentieren, können jedoch einen Körper besitzen.
Welchen Einfluss in das Vertrauen in ein Team, das Aussehen der \dq{}anderen\dq{} Avatare auf eine Person im \ac{sve} besitzt, ist eine wesentliche Fragestellung dieser Arbeit. Auf das Aussehen und die Unterschiede der \dq{}anderen\dq{} Avatare wird im Kapitel \ref{IKNIK} genauer eingegangen.

\paragraph{Uncanny-Valley}
Je menschenähnlicher ein Roboter (u.a. auch ein Avatar) aussieht und je mehr menschenähnliche Bewegungen ein Roboter ausführt, desto mehr positive Gefühle bringt dieser den Personen entgegen, die mit diesem in Berührung kommen. Ab einem bestimmten Punkt des Realismus schlägt dieses Phänomen jedoch um und der Roboter wirkt viel mehr verstörend und ruft  Ablehnung, Verstörung und Angst hervor. Dieses Phänomen und dieser exakte Grad des Realismus wird als Uncanny-Valley Effekt bezeichnet. In \textbf{Abbildung \textit{\ref{UncannyValley}}} ist eine beschreibung dieses Effektes zu sehen \citep[p.352-353]{gast2011unheimliche}.

\begin{figure}[H]
		\begin{footnotesize}
		\begin{center}
			\includegraphics[scale=.3]{Abbildungen/Uncanny-Valley.JPG}	
			\caption{Der Uncanny-Valley Effekt beschreibt das Gefühl des unbehagens ab einem gewissen Realitätsgrad}
			\label{Uncanny-Valley}
		\end{center}
		\end{footnotesize}
	\end{figure}	


%Connected to My Avatar:
%Effects of Avatar Embodiments on User Cognitions, Behaviors,
%and Self Construal 
%
%421
%https://sci-hub.do/10.1007/978-3-319-07632-4 
%file:///B:/Chrome%20Downloads/Biocca2014_Chapter_ConnectedToMyAvatar.pdf 


%Emotional Contagion with Artificial Others. Effects
%of Culture, Physical Appearance, and Nonverbal Behavior
%on the Perception of Positive/Negative Affect in Avatars 	

%412

%https://sci-hub.do/10.1007/978-3-319-07632-4
	

%https://dl.acm.org/doi/fullHtml/10.1145/223904.223935?casa_token=B8tEKM39OVQAAAAA:VNilOxaXG3_2Bw-bEClS10xyOXLBxB8ymyo4B-d1kUoAmCgWC1MDdVKSptRADsBaGw19nzE15dwIWQ

%http://mcneilllab.uchicago.edu/pdfs/dmcn_vietri_sul_mare.pdf

%https://sci-hub.ren/https://doi.org/10.1371/journal.pone.0025759 SEITE 2


\newpage

	\subsection{Vertrauen}
	\label{Vertrauen}
Der Begriff \textbf{Vertrauen} ist in viele Bereiche des alltäglichen Lebens eingezogen. Er beschreibt ein psychologisches Konzept, das jeden Menschen ständig begleitet.
Ein Politiker wirbt um das Vertrauen seiner Wähler, Unternehmer beschreiben sich selbst als Geschäftspartner des Vertrauens oder es wird der \dq{}Arzt des Vertrauens\dq{} empfohlen.

Vertrauen im Zusammenhang mit dem Begriff \dq{}\ac{vr}\dq{} kann auf zwei unterschiedliche Art und Weisen betrachtet werden. Das Vertrauen in die \ac{vr}-Technologie sowie das zwischenmenschliche Vertrauen, das in der \ac{vr} zwischen 2 oder mehreren Personen gebildet wird.
%\citep[p.10]{trustInVRTechnology}
%Mangelndes Vertrauen in eine Technologie, kann Nutzer dran hindern, diese zu Nutzen. \citep{trustInVRTechnology}

Da diese Arbeit sich mit dem zwischenmenschlichen Vertrauen beschäftigt, wird das Vertrauen in die Akzeptanz der \ac{vr} nicht weiter behandelt.
Vertrauen wird in dieser Ausarbeitung als bilaterales Konstrukt zwischen einer vertrauenden Person und einer zu vertrauenden Person definiert.
% https://sci-hub.ren/10.4337/9781847202819.00008
% https://sci-hub.ren/10.4337/9781847202819.00008
Aufgrund der Vielseitigkeit von Vertrauen, gibt es unterschiedliche Definitionen. \newline
Eine der ersten Definitionen von Vertrauen wurde 1967 von Rotter aufgestellt. Er definiert Vertrauen als 
\\
\dq{}die Erwartung eines Individuums oder einer Gruppe, dass man sich auf das Wort, das Versprechen, die mündliche oder schriftliche Aussage eines anderen Individuums oder einer Gruppe verlassen kann\dq{} \citep[p.651]{rotter1967new}.
\\
Die jedoch meist verbreitetste und akzeptierteste Definition stammt von Meyer \citep[p.712]{mayer1995integrative}. So definiert er Vertrauen als 
\\ 
\dq{}die Bereitschaft einer Person, für die Handlungen einer anderen Person anfällig zu sein, basierend auf der Erwartung, dass die zu vertrauende Person eine bestimmte, für den Vertrauensgeber wichtige Handlung ausführen wird, unabhängig von der Möglichkeit, diese Person zu überwachen oder zu kontrollieren.\dq{}
\\
Jede zwischenmenschliche Beziehung beginnt mit einer frühen Phase der Vertrauensbildung. Diese frühe Phase kann von Unsicherheiten und Zweifel geprägt sein. Das gegenseitige Vertrauen, das man sich anschließend schenkt, muss anfänglich erst einmal ausgelotet werden. 
%\citep[p.166-168]{meyerson1996swift}

Während der frühen Phase der Vertrauensbildung entscheidet sich, ob eine Beziehung aufrechterhalten wird oder nicht. Unterbewusst bildet sich ein Gefühl von Zuversicht und Sicherheit oder ein Gefühl von Spannung, Zweifel und Skepsis dem Interaktionspartner gegenüber. 
Dabei ist es egal, ob sich dafür entschieden wird, jemandem zu vertrauen, oder nicht. Auf jeden Fall beeinflusst die Stärke des positiven oder negativen Vertrauensgefühls die Effektivität der Zusammenarbeit. Vertrauen kann es einfach oder schwierig machen, mit einer anderen Person zu arbeiten und Ziele in einer Gruppe oder einem Team zu erreichen.
Früher Vertrauensaufbau ist daher der Schlüssel zur erfolgreichen Zusammenarbeit \citep[p.405-406]{bigley1998straining}.
Die initiale Phase der Vertrauensbildung wirkt sich auf das kognitive und das affektive Vertrauen aus, welche einen starken Einfluss auf das sich entwickelnde Vertrauensmodell zu einer Person haben. In dieser Phase sind diese beiden Vertrauensarten anfällig für Veränderungen \citep[p.461-462]{baldwin1992relational}.
Meinungen und Annahmen, die sich frühzeitig bilden, prägen sich somit auch stark auch auf die zukünftige Meinung über die zu vertrauenden Person aus.

Auch im Unternehmertum muss sich mit dem Konzept des Vertrauens beschäftigt werden, um erfolgreich arbeiten zu können. Ohne Vertrauen in ein Team oder in unterschiedliche Personen, fällt es schwer, Risiken einzugehen. Ist Vertrauen vorhanden, wird man nicht mit der Angst konfrontiert, dass andere Personen einen ausnutzen könnten \citep[p.1152]{breuer2016does}.

Betrachtet man Vertrauen im Unternehmen auf der Ebene von Teams, setzen sich die vertrauenden Personen sowie die zu vertrauenden Personen aus mehreren Teammitgliedern zusammen.
 
% "a willingness of a party to be vulnerable to the actions of another party based on the expectation that the other will perform a particular action important to the trustor, irrespective of the ability to monitor or control that party"

		\subsubsection{Zwischenmenschliches Vertrauen}
%	https://sci-hub.tw/https://doi.org/10.1177/1046496408323569
Während Vertrauen in die \ac{vr}-Technologie sich mit der Akzeptanz der Technologie an sich beschäftigt, beschäftigt sich zwischenmenschliches Vertrauen mit dem Vertrauen zwischen zwei oder mehreren Personen \citep{mcknight2011trust}.
Vertrauen wird nicht statisch und einseitig betrachtet. Eine Person kann nicht nur \dq vertrauen\dq oder \dq nicht vertrauen\dq. Vertrauen ist ein dynamisches Konstrukt, welches sich mit der Zeit verändert. Es kann in eine Bildungs-, Stabilisierungs- und Abnehmphase unterteilt werden \citep[p.396]{rousseau1998not}.

Viele Forschungen, die sich mit dem Thema Vertrauen beschäftigen, gehen heute davon aus, dass zwischenmenschliches Vertrauen aus einem zweidimensionalen Konstrukt besteht \citep{johnson2005cognitive} \citep{cook1980new}. So ist Mooradian \citep[p.524-525]{mooradian2006trusts} der Ansicht, dass Vertrauen als \dq{}Eigenschaft\dq{} oder als \dq{}Zustand\dq{} gesehen wird.

	\subsubsection{Vertrauen als Eigenschaft }
\label{Vertrauen als Eigenschaft}
Wird Vertrauen als Eigenschaft betrachtet, spiegelt dies die Einstellung zum Vertrauen einer Person wieder. Diese Einstellung zum Vertrauen ist langlebig und wird nicht all zu schnell auf oder abgebaut. Unabhängig von einer Situation, in der sich diese Person befindet, wird davon ausgegangen, dass diese Eigenschaft aus dem Temperament oder der Lebenserfahrungeiner Person entsteht. Dieses Vertrauen ist das Grundlevel an Vertrauen, die eine Person in eine neue zwischenmenschliche Beziehung von Anfang an einbringt. \citep[p.11]{couch1996assessment}. Es ist jedoch nicht all zu bekannt, wie generelles Vertrauen genau gebildet wird \citep[p.409]{stolle2002trusting}.

Generelles Vertrauen impliziert, dass den meisten Personen vertraut werden kann, oder dass man im Fall von generellem Misstrauen, Personen nicht vertrauen kann. \citep[p.409]{stolle2002trusting}

Der \textbf{generelle Hang zum Vertrauen} ist nicht situationsabhängig, sondern stellt eine längerfristige Konstante auf Basis des Grundvertrauens einer Person dar. Grundvertrauen setzt sich dabei aus der individuellen Eigenschaft des Hangs zum Vertrauen einer einzelnen Person, sowie die Grundstimmung gegenüber Personen im Allgemeinen, zusammen \citep[p.11]{couch1996assessment}.
%Da in dieser Studie mit einem Team von drei Leuten gearbeitet wird, spiegelt der Hang zum Vertrauen einer Person den generellen Vertrauensvorschuss wider, den eine Person dem Team gewährt.

	\subsubsection{Vertrauen als Zustand}
\label{Vertrauen als Zustand}
Wird Vertrauen als \dq{}Zustand\dq{} betrachtet, so kann sich dieses Vertrauen im Laufe der Zeit, z.B. durch Interaktion mit einer anderen Person, ändern. Dieser \dq Zustand\dq{} des Vertrauens, spiegelt sich auch in Meyers \citep[p.712]{mayer1995integrative} vorher erwähnter Definition von Vertrauen wider :\\ \dq die Bereitschaft einer Person, für die Handlungen einer anderen Person anfällig zu sein, basierend auf der Erwartung, dass die zu vertrauende Person eine bestimmte, für den Vertrauensgeber wichtige Handlung ausführen wird, unabhängig von der Möglichkeit, diese Person zu überwachen oder zu kontrollieren.\dq{} \\
Es wird einer Person oder einem Team ein Vertrauensvorschuss gewährt, der sich jedoch zu jederzeit verändern kann, wenn dieser gebrochen wird.

%Cook und Wall beschreiben diese Zwei Dimensionen als  \begin{itemize}
%\item{ “faith in the trustworthy intentions of others”, sowie}
%\item{“confidence in the ability of others, yielding ascriptions of capability and reliability”.}
%\end{itemize}

Das Konzept des Vertrauens als Zustand lässt sich laut Lewis \citep[p.970-971]{lewis1985trust} in zwei Teile unterteilen.\newline
Laut seinen Forschungen basiert Vertrauen \dq auf einem kognitiven Prozess, der zwischen vertrauenswürdigen, misstrauischen und unbekannten Personen und Institutionen unterscheidet. In diesem Sinne wählen wir kognitiv aus, wem wir in welcher Hinsicht und unter welchen Umständen vertrauen, und wir stützen die Wahl auf das, was wir als \dq gute Gründe\dq{} ansehen, die einen Beweis für die Vertrauenswürdigkeit darstellen \dq{}\citep[p.970]{lewis1985trust}.
Somit basiert das \textbf{kognitiv aufgebaute Vertrauen} auf einer von uns definierten Logik statt auf einer emotionalen Komponente. Diese \dq{}guten Gründe\dq{} können auch leicht gebrochen werden, indem der Vertrauensvorschuss, den wir durch das kognitive Vertrauen unserem Interaktionspartner vorschießen, gebrochen wird.
Das kognitive Vertrauen kann kurzfristig aufgebaut werden und ist leicht anfällig gegen äußerliche Einflüsse. 

Weiterhin besitzt Vertrauen als \dq{}Zustand\dq{} noch eine affektive Komponente.\\ \dq Diese \textbf{affektive Komponente des Vertrauens} besteht in einer emotionalen Bindung zwischen allen, die an der Beziehung beteiligt sind. Wie die affektiven Bindungen der Freundschaft und der Liebe schafft Vertrauen eine soziale Situation, in der intensive emotionale Investitionen getätigt werden können, und deshalb weckt der Verrat eines persönlichen Vertrauens ein Gefühl der emotionalen Empörung bei dem Betrogenen. Der Vertrauensbruch trifft die Grundlage der Beziehung selbst, nicht nur den spezifischen Inhalt des Verrats. Diese emotionale Komponente ist bei allen Arten von Vertrauen vorhanden, aber normalerweise ist sie bei engem zwischenmenschlichem Vertrauen am intensivsten \dq{} \citep[p.971]{lewis1985trust}
\\
Ein Beispiel für Affektives-Vertrauen ist eine Liebesbeziehung zwischen zwei Personen. Das affektive Vertrauen baut sich mit der Zeit langsam auf und kann durch verschiedene Ereignisse erschüttert oder gestärkt werden. Affektives Vertrauen kann eher durch Emotionen statt durch Logik charakterisiert werden.
\textbf{Affektives Vertrauen} ergibt sich aus \textit{zwischenmenschlichen emotionalen Verbindungen und gegenseitiger Fürsorge}, während individuelles \textbf{kognitives Vertrauen} auf der \textit{Überzeugung in die Fähigkeiten oder in die Zuverlässigkeit eines anderen basiert}. \citep[p.30]{mcallister1995affect}.

Vertrauensaufbau zwischen verschiedenen Personen beinhaltet somit zwei Komponenten. Beide Komponenten sind bei der anfänglichen Teamgründung von großer Bedeutung. 
McAllister meint in seiner Forschung dazu, dass die Grunderwartung der kognitiven Komponente erst einmal erfüllt werden muss bevor mehr in die Beziehung zu den einzelnen Personen eines Teams investiert wird. Je mehr ein Team zusammenarbeitet und sich kennenlernt, umso wichtiger wird die affektive Komponente der Vertrauensbildung \citep[p.30]{mcallister1995affect}.

Laut Forschungen von McKnight kann ein starkes Vertrauen in frühen Teamgründungsphasen entstehen. Glauben Teammitglieder beispielsweise in frühen Teamgründungsphasen, dass der Teamgründungsprozess mit Strukturierungen und Regelungen durchwachsen ist, ergibt sich dadurch ein höheres kognitives Vertrauen und es folgt ein höheres Gesamtvertrauen in das Team \citep[p.478-479]{mcknight1998initial}.
%Ich glaube, ich habe die Aufgaben des Versuchsleiters gut erledigt -> Auf Kognitives Vertrauen analisieren Pro person! Hier ist einseitige Signifikanz in der Korrelation von Wie Erfolgreich haben Sie ihrer Meinung nach die vom Versuchleiter ... mit Cog_Trust_Windsorizing vorhanden!

Da der kognitive Vertrauensaufbau in andere Personen auf einer rationalen Grundlage, die auf der Wahrnehmung von Kompetenz oder Zuverlässigkeit beruht, basiert, lässt sich die Vermutung aufstellen, dass ein \ac{ik} sowie ein \ac{nik} Avatar einen unterschiedlichen Einfluss auf die kognitive Vertrauensbildung haben.

Je realistischer und menschenähnlicher ein Avatar aussieht, desto mehr kognitives Vertrauen könnte in diesen gebildet werden, da sich die teilnehmenden Personen eher vorstellen können mit realen Menschen zusammenzuarbeiten als nur mit Repräsentationen eines Individuums aus Pixeln. Diese Einschätzung teilt auch Riedl \cite{riedl2014trusting}, indem er herausfand, dass es Personen leichter fällt einzuschätzen, wie vertrauenswürdig ein Mensch ist, wenn dieser ein menschenähnliches Gesicht besitzt.
Weiterhin ist durch eine erhöhte Self-Presence eine bessere Identifizierung seiner Selbst und dem Avatar möglich. 

%Hat nun ein Head und Hand getrackter Avatar einen Einfluss auf die wahrgenommen Fähigkeiten oder die wahrgenommene Zuverlässigkeit in das Team? Genau diese Frage stellt sich diese Hypothese 2.
%Es wird davon ausgegangen, dass, egal welche Kondition des Avatars ein Probant zur Verfügung gestellt bekommt, dies keinen Einfluss auf die wahrgenommenen Fähigkeiten des Teams besitzt. Somit erscheint ein Team in seiner Leistungsfähigkeit und Zuverlässigkeit immer gleich, egal ob ein Avatar Menschenähnlich ist oder nicht.
%\textbf{ES GIBT NOCH MEHR VERGLEICHE, SIEHE DAZU QUELLE UNTEN}
%\textbf{HIER EINEN VERGLEICH ZWISCHEN COGNITIVE TRUST UND AFFECTIVE TRUST AUFSTELLEN. AUCH BEIM FRAGEBOGEN BEACHTEN! ERKLÄRUNG GEBEN WARUM GENAU DEN VON MC ALLISTER 1995 GENOMMEN WURDE.}
%\textbf{http://citeseerx.ist.psu.edu/viewdoc/download?doi=10.1.1.496.9380&rep=rep1&type=pdf}

\subsubsection{Vertrauen und Avatare}
TODO
	\subsubsection{Vertrauen und virtuelle Teams}
	\label{Vertrauen und virtuelle Teams}
	
Vorstellungen in uns selbst und in andere sind bis zu einem gewissen Grad kategorisierbar. Das denken in verschiedenen Kategorien, erlaubt es uns, die Gesamtheit der Eigenheiten von Personen zu reduzieren um diese besser verstehen zu können. Identifiziert eine Person sich mit einer anderen oder fühlt sich dieser nahe, wird Vertrauen gebildet. Dies können beispielsweise Familienmitglieder oder Bekannte sein. Fällt eine Person in eine andere Kategorie, wird kein Vertrauen gebildet
\citep[397-403]{stolle2002trusting}
Brewer entwickelte diese Theorie weiter und meint, dass Menschen denjenigen Personen denen sie sich zugehörig fühlen und zu denen diese eine Gruppenidentität aufbauen, mehr Vertrauen und Anerkennung schenken als zu denjenigen, zudem sie dies nicht tun. Die Kategorisierung erhöht gleichzeitig die wahrgenommene Ähnlichkeit zwischen Personen der gemeinsamen Gruppe, was wiederum auf Verständnis der anderen Personen stößt, da diese die Situation ähnlich wahrnehmen.
Gemeinsame Kriterien für die Gruppenidentität können Verhaltensähnlichkeiten, geographische Nähe oder die gemeinsame Interaktion sein. Diese Gemeinsamkeiten können als Stellvertreter des zwischenmenschlichen Vertrauen dienen und es lässt sich von dem \dq{}persönlichem\dq{} Vertrauen auf ein \dq{} Vertrauen in der Gruppe \dq{} schließen, dass nur auf einer gemeinsamen Kategorie und der Gruppenidentität beruht.
In solch einer Gruppe ist die psychiologische Distanz reduziert und die Orientierung eher auf Gruppenergebnisse statt auf individuelle Ergebnisse gerichtet \citep[355-360]{brewer1981}.
Vertrauen wird in einer Gruppe im allgemeinen als Hoch angenommen. Weiterhin wird ein Fehler in einer Gruppe häufig nicht als Fehler eines Individuums angesehen, sondern als Verletzung der Gruppennorm, da alle Personen sich zu gleichen Teilen zugehörig fühlen. Die Gruppe als konsistente Einheit bietet verschiedene Mechanismen, um Vertrauen zu stärken und misstrauen abzuwehren. Diese Gruppendynamik lässt sich auf Gruppen mit Eng verbundenen Mitgliedern betrachten, sowie möglicherweise auf Gruppen mit Außenseitern \citep[397-403]{stolle2002trusting}.

%Irgendwie ist das gesamte Kapitel sehr komisch ...
In vielen Forschungen wird angenommen, dass starkes interpersonelles Vertrauen die Teamperformance positiv beeinflusst \citep{mcallister1995affect} \citep{mayer1995integrative} \citep{dirks2002trust}.
Kognitives Vertrauen in Teams lässt sich nur sehr schwierig aufrechterhalten. So könnte es sein, dass ein Team einen hohen kognitiven Vertrauenswert aufweist, dann jedoch mit einem Rückschlag bei der Bearbeitung einer Aufgabe konfrontiert wird und dies eine Verringerung des kognitiven Vertrauen des Teams zur folge hat \citep[p.29-31]{mcallister1995affect}.
Im Vergleich dazu, wird das affektive Vertrauen im Team durch gemeinsame Rückschläge nicht nachhaltig verringert. Es benötigt eine längerfristige emotionale Krise innerhalb des Teams, um eine Verringerung des affektiven Vertrauens zu verursachen. Es wird angenommen, dass das affektive Vertrauen eine längerfristige, stärkere Bindung, als das kognitive Vertrauen, schafft \citep[p.29-31]{mcallister1995affect}.

Aktuell gibt es nicht viel Forschung, wie sich Vertrauen in \ac{vts} aufbauen und halten lässt \citep[p.8-23]{duarte2006mastering}.
Bisherige Studien die einen Zusammenhang zwischen Vertrauen und Team-Effektivität darstellen, haben positive Zusammenhänge \citep{davis2000trusted}, keine Zusammenhänge \citep{hertel2004managing} sowie negative Zusammenhänge \citep{dirks1999effects} festgestellt.

Vertrauen im Team wird meinst jedoch nur eindimensional gemessen, obwohl es ein zweidimensionales Konstrukt ist.
Polzer führte eine Forschung über räumlich getrennte Teams und deren Vertrauensbildung mittels einer eindimensionale Affektive Vertrauensmessung durch. Sie fanden heraus, dass geografisch getrennte Teams eher zu Konflikten neigen als nicht geografisch getrennte Teams. Die geografische Getrenntheit hat negative Auswirkungen auf das Vertrauen zwischen den einzelnen Teammitgliedern \citep[p.682]{polzer2006extending}.
Prichard und Ashleigh führten 2007 eine Studie mit einer eindimensionalen kognitiven Vertrauensmessung durch und haben herausgefunden, dass Teambuilding das Vertrauen der einzelnen Teammitglieder untereinander verstärkt \citep[p.704]{prichard2007effects}.
Dirks führte 1999 eine Studie zur Vertrauensbildung in Teams durch. In dieser Studie wurde die multidimensionale Komponente der Vertrauensbildung aufgegriffen, jedoch stand für seine Versuchsdurchführung nur ein 10-Minütiges Zeitfenster zur Verfügung. In diesen 10-Minuten Versuchsdurchführung stellte er fest, dass die affektive Komponente nicht gebildet und dadurch nicht gemessen werden konnte \citep[p.445]{mayer1995integrative}. Daher wird seine Forschung ebenfalls nur als eindimensional betrachtet.
Mehrdimensionale, zuverlässige, Vertrauensforschung ist im traditionellem Sinne nur mittels Langzeitstudien möglich, da das Kognitive- und besonders das Affektives-Vertrauen eine Zeitliche Komponente benötigt \citep{jones1998experience}.

Vertrauensaufbau im Team nimmt eine wichtige Rolle in \ac{vts} ein, denn im Gegensatz zu traditionellen geformten Teams haben Teammitglieder eines virtuellen Team keine Möglichkeit durch geselliges Beisammensein oder durch physischen Kontakt Bindungen aufzubauen um das gegenseitige Vertrauen zu stärken \citep{TrustAndTheVirtualOrganisation}.

Vertrauen in einem Team zu fördern ist somit eine Notwendigkeit um Wachstum und Erfolg des Teams zu bestimmen \citep{glacel1997teamwork}.


%\paragraph{Swift-Trust}
%Swift Trust ist eine form von Vertrauen in kurzzeitig zusammenarbeitenden Teams oder Gruppen. Kurzzeitig aufgestellte Teams besitzen eine endliche Zeitspanne, haben ein klares Ziel welches diese verfolgen und ihr Erfolg wird von guter Koordination bestimmt.
%Für solche Teams gibt es nicht genügend Zeit um ein langfristiges Vertrauensverhältnis durch zum Beispiel Teambuildingmaßnahmen aufzubauen, wie es traditionelle Teams in einer Unternehmensorganisation tun können.
%Es unterscheidet sich dahingehend vom Kognitiven-Vertrauen dahingehend, dass Swift-Trust nur auf oberflächlichen Informationen über den anderen basiert. Es wird ein Vertrauensvorschuss in das Team gewährt, welcher sich zu einem späteren Zeitpunkt bestätigt oder nicht bestätigt \citep[p.141]{wildman2012trust}, wohingegen kognitives Vertrauen in andere Personen auf einer rationalen Grundlage, die auf der Wahrnehmung von Kompetenz oder Zuverlässigkeit beruht, basiert. \citep[p.970]{lewis1985trust}

%Hier am besten noch mehr Schreiben ...
% vielleicht das ? https://www.researchgate.net/publication/256018364_Trust_and_Team_Effectiveness_A_Meta-Analysis_of_Critical_Contingencies_and_Mediated_Mechanisms
\newpage
	\subsection{Teams}		
	\label{Teamwork}
Was ist ein Team? Wie unterscheidet sich ein "normales" Team von einem Team, welches räumlich voneinander getrennt ist? "Normale" Teams, haben die Möglichkeit sich jederzeit oder mit einem geringem Zeitaufwand von Angesicht zu Angesicht zu treffen, während räumlich voneinander getrennte Teams nicht diese Möglichkeit besitzen.

\subsubsection{Teams}
\label{team}
	Ein Team wird als eine kleine Gruppe von Menschen mit gleichartigen Fähigkeiten, welche sich in gleicher Weise für das gleiche Ziel und gleiche Arbeitsweisen einsetzen und dies verfolgen, definiert\citep[p.2]{zenun2007effects}.
	
Cohen \citep[p.557]{cohen1997makes} definiert ein Teams als "eine Ansammlung von Individuen, die in ihren Aufgaben voneinander abhängig sind, die die Verantwortung für die Ergebnisse teilen, die von sich selbst und von anderen als eine intakte soziale Einheit gesehen werden."

Das Verhalten von Personen, die in einem Team arbeiten, lässt sich in \glqq Teamwork \grqq und in \glqq Taskwork \grqq unterteilen. \citep[p. 541-542]{rousseau2006teamwork}
Taskwork beschreibt dabei, was für eine Aufgabe ein Team erledigt, sowie, wie die Ausführung von Kernkompetenzen in einem bestimmten Bereich aussieht. 
Teamwork beschreibt, wie ein Team gemeinsam eine Aufgabe erledigt. Dies beinhaltet, wie sich interaktive sowie voneinander abhängige Verhaltensprozesse zwischen Mitgliedern des Teams auswirken um eine Aufgabe zu erledigen\citep[p. 357]{marks2001temporally}.

Sala \citep[p.541]{salas2008teams} definiert ein Team als "die voneinander abhängigen Leistungskomponenten, die erforderlich sind, um die Leistung mehrerer Personen effektiv zu koordinieren".

Teamwork beschreibt somit die gemeinsame Arbeitsleistung um eine Aufgabe zu erreichen.

\textbf{Teambuilding} zielt dabei darauf ab, die Haltung oder Einstellung der Personen innerhalb und untereinander eines Teams zu verbessern um das gesamte Team zu verbessern, während \textbf{Teamtraining} drauf abzielt, spezielle Fähigkeiten einzelner Personen zu fördern um das gesamte Team punktuell zu stärken \citep[p. 367-369]{shuffler2011there}.
		
Die wirtschaftliche Leistung von Unternehmen hängt häufig stark von der Arbeitseffizienz gut funktionierender Teams ab. Teambuilding kann somit helfen, die wirtschaftliche Leistung zu verbessern, indem Mitgliedern eines Teams weniger Fehler durch bessere Entscheidungen erzeugen \citep[p. 1-6]{biech2007pfeiffer}.

\subsubsection{Virtuelle Teams}
\label{vts}
%https://sci-hub.ren/10.1111/j.1365-2575.2009.00326.x

\ac{vts} teilen viele Eigenschaften von herkömmlichen Teams. Ein \dq virtuelles Team\dq ist ein \dq Team\dq. Es muss jedoch unterschieden werden, wie die virtuelle Komponente des Teams definiert wird und weshalb diese Komponente ein herkömmliches Team zu einem virtuellem Team macht. 
\label{AnforderungenLautSchweitzer}
Schweizer \citep[p.270]{schweitzer2010conceptualizing} erläutert die 4 größten Kennzeichen eines \ac{vts} in der Literatur.
\ac{vts} sind :
\begin{itemize}
\item Zustandegekommen aufgrund von Kommunikationstechnologie.Durch technische Hilfsmittel wird Kommuniziert, es werden Entscheidungen getroffen oder Informationen ausgetauscht.
\item Räumlich getrennt. \ac{vts} arbeiten \textit{nicht} am selben Arbeitsplatz
\item Grenzübergreifend. Es gibt Zusammenarbeit der Teammitglieder aus verschiedenen Organisationen oder Organisationseinheiten.
\item Asynchron. \ac{vts} arbeiten zu unterschiedlichen Zeiten/Zeitzonen oder in der selben Zeitzone in unterschiedlichen Schichten.
\end{itemize}

Die vollständige Liste von Schweizer findet sich in \textbf{\autoref{criteriaForVirtualTeams}}.

Einige Autoren nehmen in ihre Definition eines \ac{vts} den Aspekt der zeitlichen Limitierung oder der kulturellen Diversität auf. So ist ein virtuelles Team für Jarvenpaa \citep[p.1-2]{jarvenpaa1999communication} nur für einen bestimmten Zeitraum aufgebaut und es ist zusätzlich kulturell divers. 

Wird jedoch aus heutiger Sicht über die Definition von Schweizer oder Jarvenpaa nachgedacht, muss ein virtuelles Team nicht zwangsläufig aus verschiedenen Organisationen bestehen oder asynchron arbeiten. In vielen Unternehmen ist es auch möglich, dass ein virtuelles Team in der selben Zeitzone und ohne kulturelle Diversität oder zeitliche Limitierung zusammen arbeitet.
Somit bleiben nur noch die Komponenten des Zusammenfindens mithilfe von Kommunikationstechnologie sowie der räumlichen Trennung.

Wird nun die Definition von Zenun \citep[p.2]{zenun2007effects} eines Teams herangezogen, und die virtuelle Komponente in diese mit eingebracht, kann ein virtuelles Team als \\
\dq{}eine kleine, auf Kommunikationstechnolgie basierende, räumlich getrennte Gruppe, von Menschen mit gleichartigen Fähigkeiten, welche sich in gleicher Weise für das gleiche Ziel und gleiche Arbeitsweisen einsetzen und dies verfolgen\dq{} \\
definiert werden.

Laut obriger Definition eines virtuellem Teams, ist fast jedes Team ein virtuelles Team. Nur selten arbeiten Teams während der gesamten Lebensdauer nach vorheriger Definition. Es wird daher davon ausgegangen, dass Virtualität als Kontinuum gesehen werden kann, bei dem jedes Team einen gewisses Maß an Virtualität besitzt und die Möglichkeit besitzt, sich zu ändern. Dieses Kontinuum reicht von Face-to-Face bis zur vollständigen, nur über Kommunikationstechnologie stattfindende Kommunikation \cite{martins2004virtual}.
\subsubsection{Virtuelle Teams und Teambuilding}
%	https://sci-hub.tw/https://doi.org/10.1108/13527590110395621

Seit einigen Jahren wird die Wichtigkeit von effektiven Teambuildingmaßnahmen in der strategischen Organisationsentwicklung erkannt. Dabei spielt der Wandel hin zu einer globalen, auf Wissen basierten Wirtschaft eine zentrale Rolle \citep{belbin2011management} \citep[p.7]{katzenbach2015wisdom}.
Wirtschaftlicher Erfolg korreliert direkt mit der Fähigkeit eines Unternehmens, Teams organisieren, strukturieren und managen zu können \citep{pasmore1993designing}.
Der Erfolg eines \ac{vts} ist somit als Nebenprodukt der organisatorischen Fähigkeiten eines Unternehmens zu verstehen \citep[p.5]{kling1994social}.

Erfolgreiches Teambuilding kann die Effektivität eines \ac{vts} steigern und dazu führen, dass Personen sich mehr mit dem Team identifizieren \citep{kaiser2000student}.

\ac{vts} werden häufig gebildet, um räumliche oder kurzzeitige Trennungen eines Teams zu umgehen. Dabei werden computergestützte Technologien so verwendet, dass räumlich getrennte Teammitglieder ihre Aufgaben mittels computergestützter Kommunikation, im Team koordinieren können \citep[p. 117-119]{peters2007identifying} \citep[p. 1-2]{cascio2003leadership}.
Das virtuelle Team zu gründen, stellt nicht die Herausforderung dar. Die eigentliche Herausforderung ergibt sich aus den Unterschiedlichen Kulturen, Entfernungen und Zeitzonen, die ein virtuelles Team mitbringt. Wird es geschafft, Vertrauen in das virtuelle Team zu bringen, kann der eigentliche Nachteil der verschiedenen Kulturen, Entfernungen und der Zeitzonen auch zum Vorteil werden. Es wird die Kulturelle Diversität gefordert und neue Verhaltensmuster erworben, wodurch neue, kreative Sichtweisen gefördert werden. Durch diese Faktoren ist es letztendlich möglich, Innovativer zu Arbeiten und zu denken \citep{dyer1995team} \citep[p.405-416]{milliken1996searching}.

%Die Arbeit kann Länderübergreifend stattfinden, was ein größeres Innovationspotential für Unternehmen zur folge hat.	
%\ac{vts} haben den Ruf teuer zu sein, die Ziele nicht verfolgen zu können, nie pünktlich und schwierig zu managen zu sein. \citep[p.243-244]{gassmann2003trends}

Gerade für \ac{vts} ist die Anfangsphase der Teamgründung von entscheidender Bedeutung. 
Die Kommunikation in \ac{vts} ist gerade in dieser Phase sehr Ergebnisorientiert in der Art und Weise \textit{wie} kommuniziert wird. Dieses Defizit in der sozialen Kommunikation kann die Schlüsselfaktoren eines erfolgreichen Teams beeinträchtigen - soziale und emotionale Beziehungsbildung sowie den Aufbau von Vertrauen \citep[p.378]{ren2007applying}.

Um die Wahrscheinlichkeit zu erhöhen, dass auch in der Anfangsphase des Teambuildings eine \glqq Soziale\grqq und keine \glqq Arbeitsnahe, Ergebnisorientierte\grqq Kommunikation stattfindet, wurde eine spielerische Umgebung geschaffen, in der sich das Team das erste mal kennenlernen kann. 
Mitglieder von \ac{vts} haben im Gegensatz zu traditionell geformten Teams weniger Möglichkeiten sich zu sehen, zu interagieren oder Konflikte zu lösen. 
Respekt und gegenseitiges Verständnis sind die Grundbausteine um Kreativität und Innovation innerhalb eines Teams zu fördern, die Effektivität eines Teams ist eine direkte Konsequenz daraus.


Eine ideale Teamgründung ist daher in \ac{vts} nicht möglich. 
%\textbf{(Cianni andWnuck, 1997) ( Hier einfügen oder doch als eigenen Satz stehenlassen? )}
\ac{vts} werden trotz der Konsequenzen der räumlichen und zeitlichen Trennung gebildet, ohne die vorherigen Beschriebenen Vorteile in kauf nehmen zu können. Die Optimale Situation wäre es, in einem schon bestehendem Team eine virtuelle Komponente hinzuzufügen um auf die Vorteile von schon vorhandenen sozialen Bindungen zugreifen zu können \citep[p.36-37]{holton2001building}.
Es sollte sichergestellt werden, dass \ac{vts} während ihres Bestehens bestmöglich in ihrem Aufbau von Vertrauen und sozialen Beziehungen unterstützt werden, um den Erfolg des Teams möglichst zu gewährleisten.
%		\subsubsection{Aktuelle Teambuildingmaßnahmen in virtuellen Teams}
%		\subsubsection{Kooperativ vs Kompetitive}
		\subsubsection{Teameffektivität}
\label{Teameffektivität}
\dq{}Teameffektivität\dq{} ist die Fähigkeit eines Teams, so miteinander zu interagieren uns sich so zu unterstützen, dass ein zuvor definiertes Ziel des Teams erreicht wird. Da viele externe Faktoren zum erreichen oder nicht erreichen eines zuvor definierten Ziels beitragen können, ist es wichtig, dass das Team immer einen Fokus auf seine Teameffektivität hat \citep[p.557]{salas2005there}.
	Es gibt keinen einheitlichen Standard um die Performance eines Teams zu messen. Es wird davon ausgegangen, dass die Effektivität in Gruppen anhand der von der Gruppe produzierten Ergebnissen (Quantität, Qualität, Geschwindigkeit, Kundenzufriedenheit) gemessen werden kann. Eine Gruppe hat Einfluss auf die Produktivität der einzelnen Mitglieder, von diesem Effekt profitiert die gesamte Gruppe und trägt somit zur Verbesserung der Gesamteffektivität bei \citep[p.309]{guzzo1996teams}.
	
Training im Team kann die gesamte Performance eines Teams steigern. Dies scheint am effektivsten, wenn mehrere Charakteristika des Teamworks auf einmal angesprochen werden. Diese sollten auch experimentelle Aktivitäten beinhalten um aktiv zu lernen und zu üben \citep[19]{mcewan2017effectiveness}.
Gemeinsames Training führt zu einer Steigerung der Qualität der Ideen und Entscheidungen sowie der gesamten Teamleistung.
Die Kommunikation untereinander wird gefördert, da die einzelnen Teammitglieder gegenseitig ihre Aufgaben kennen und dadurch eher bereit sind sich untereinander zu helfen.
Geteiltes wissen bedeutet größeren Lerneffekt, was mit einem besseren Verständnis über das Wissen der anderen Teammitglieder einhergeht. Somit können auch individuelle Stärken gefördert, und Schwächen durch andere Teammitglieder kompensiert werden.
Ein Team bringt ein Gefühl von Sicherheit mit sich, was zu einer erhöhten Risikobereitschaft führt. Dies erhöht die Kreativität der eingebrachten Ideen können entstehen und Teammitglieder an der Möglichkeit größere Risiken einzugehen wachsen \citep[p. 2-4]{biech2007pfeiffer}.
Dies führt zu einer Steigerung der Qualität der Ideen und Entscheidungen sowie der gesamten Teamleistung.
Auch das soziale Identitätsgefühl hängt davon ab, ob ein generelles Gruppenverständnis besteht, eine Person sich der Gruppe zugehörig fühlt und ob man sich zu als Gruppe mit anderen Gruppen vergleicht. Gruppenzugehörigkeit ist ein wichtiger Bestandteil des Selbstverständnisses eines Individuums \citep{sutantovicious}.
	
Ist gute Gruppenzugehörigkeit gegeben, stärkt dies die Gruppenproduktivität sowie die individuelle Leistungsfähigkeit. Weiterhin entstehen dadurch Effekte die zum besserem Zusammenhalt, mehr Vertrauen \citep{herbsleb2000distance}, besserer Kommunikation und Kooperation untereinander führen \citep[p. 510]{olson2003psychological}.

Zum Messen der \textit{wahrgenommenen Teameffektivität} der einzelnen Individuen, wird in dieser Studie ein Teilauszug des \textit{Team-Effectiveness-Scale} \textit{Siehe Tabelle \ref{Post-Questionnaire - Team-Effektivität}} \citep[p.469]{gibson2003team} genutzt. Zum Messen der \textit{realen Teameffektivität} wird in dieser die Anzahl der abgeschlossenen Runden \ac{tRound} des durchgeführten Experiments genutzt. Schafft ein Team eine Runde, hat dieses eine reale Teameffektivität von 1, schafft es 10 Runden, hat es eine reale Teameffektivität von 10.
%	
%		\subsubsection{Teambuilding Komponenten}
%	\textit{\textbf{HIER NOCHMAL KURZ ERKLÄREN WOZU DIE TEILBEREICHE EIGENTLICH DA}}
%	Teambuilding kann in 10 verschiedene Teilbereiche unterteilt werden. Siehe \autoref{ten_characteristics}
%	\begin{figure}[H]
%		\begin{footnotesize}
%			\includegraphics[width=\textwidth]{Abbildungen/Ten_Characteristics.JPG}
%			\caption[Abbildung 1]{Ten Characteristics of a High Performance Team \citep[p. 27]{biech2007pfeiffer}}
%			\label{ten_characteristics}
%		\end{footnotesize}
%	\end{figure}
%	Die einzelnen Teilbereiche bauen aufeinander auf. Der untere Teilbereich mit den Komponenten \textit{Clear Goals"}, \textit{Defined Roles}, \textit{Open and Clear Communication} sowie \textit{Effective Decision Making} sind die Basis des gesamten Models. Diese Komponenten sollten am Anfang des Teambuildingprozesses definiert und immer als Basis der gesamten Teamarbeit gesehen werden.\\
%	Der Teilbereich mit den Komponenten \textit{Balanced Participation}, \textit{Valued Diversity} und \textit{Managed Conflict} baut auf diese auf.\\
%	\textit{Positive Atmosphere} und \textit{Cooperative Relationships} ist der dritte Teilbereich. In diesem geht es hauptsächlich um die Zufriedenheit und die Bereicherung eines einzelnen in einem Team zu arbeiten. Jedoch ist dieser Bereich nicht mehr Ausschlaggebend um eine Aufgabe zu erledigen. Für viele Teammitglieder ist dieser Teilbereich jedoch wichtigste Ziel während der Teamarbeit.\\
%	\textit{Participative Leadership} ist der einzige Bereich der aus dem Gesamtgebilde ohne Beeinflussung der anderen Komponenten entfernt werden kann. Dies sagt uns, dass ein Teamm auch ohne einen einzelnen Teamleiter existieren kann. Vgl. \citep[p. 13-16]{biech2007pfeiffer}\\
%	Aufgrund der vielen Komponenten die ein Team erfüllen muss um effizient zu sein und ein Gestärktes Teamgefühl zu bilden, wird in dieser Ausarbeitung nur auf einen Teilbereich eingegangen.
%	
%			\paragraph{Positive Atmosphere}
%	Wird der Aufbau von Vertrauen vernachlässigt, so bildet sich kein gut funktionierendes Team.
%			\paragraph{Open and Clear Communication}
%			\paragraph{Balanced Participation}
%			\paragraph{Defined Roles}
%	
%		\subsubsection{Team-Colloboraton}
%		\subsubsection{Virtual-Teambuilding}
		
%			\paragraph{Allgemeines}

%EINGEFÜGT BEI DEFINITION EINES VIRTUELLEM TEAMS
%		\ac{vts} werden häufig gebildet um räumliche oder kurzzeitige Trennungen eines Teams zu umgehen. Dabei werden computergestützte Technologien so verwendet, dass räumlich getrennte Teammitglieder ihre Aufgaben, mittels computergestützter Kommunikation, im Team koordinieren können. \citep[p. 117-119]{peters2007identifying} \citep[p. 1-2]{cascio2003leadership}
%		
%		Es kann effizient Länderübergreifend gearbeitet werden, was ein größeres Innovationspotential für Unternehmen zur folge hat.	
%		\ac{vts} haben den Ruf teuer zu sein, die Ziele nicht verfolgen zu können, nie pünktlich und schwierig zu managen zu sein. \citep[p.243-244]{gassmann2003trends}
%		
%		\ac{vts} sind häufig gerade in der Anfangsphase sehr Ergebnisorientiert in der Art und Weise \textit{wie} kommuniziert wird. Dieses Defizit in der Sozialen Kommunikation untereinander kann die Schlüsselfaktoren eines Erfolgreichen Teams beeinträchtigen - Soziale und Emotionale Beziehungsbildung sowie den Aufbau von Vertrauen. \citep[p.378]{ren2007applying} \\
%		Um die Wahrscheinlichkeit zu erhöhen, dass auch in der Anfangsphase des Teambuildings eine \glqq Soziale\grqq und keine \glqq Arbeitsnahe, Ergebnisorientierte\grqq Kommunikation stattfindet, wurde eine spielerische Umgebung geschaffen, in der sich das Team das erste mal kennenlernen kann. 
%		
%\ac{vts} werden in dieser Arbeit als temporär, divers und räumlich gerennt, beschrieben. Die Kommunikation untereinander nur durch das World-Wide-Web statt. Weiterhin impliziert die temporäre Eigenschaft der Definition, dass sich die Teammitglieder nicht kennen.
		
		
%		\subsubsection{Einflussfaktoren im Teambuilding}


%	Z.b. Aussehen, Kulturen, etc.	
\newpage
%Allgemein : 
%Ausreißer bei ID1 vorhanden. Wie damit umgehen? Entfernen? Drin lassen?
%Filtern? Filtern bei H1, H2, H4 möglich! Somit CT Werte größer 2.4 nutzen.
%H3 und H5 benötigen alle Werte!

%Die variablen MÜSSEN in IK und NIK aufgeteilt werden, da diese nicht gemeinsam betrachtet werden können. Grund : Der gebildete CT wert ist von den variablen abhängig. Eine vermischung der CT werte (CTI und CTN) würde die Datenanalyse unzulässig zulässig machen.
% Ist dies der Fall bei H1, H5?

%------------
%Hypothese 1

%HYPOTHESE KORRELATION
%\textbf{H1$_{1}$} : Es besteht ein Zusammenhang zwischen dem General-Trust-Score (\ac{gt}) einer Person und dem Cognitive-Trust-Score (\ac{ct}) einer Person. \newline
%
%%KORRELATION
%GT ordinalskaliert -> CT ordinalskaliert > SPEARMAN-Korrelation zulässig?
%
%%Voraussetzung für Pearson-Korrelationsanalyse: 
%Die Variablen sind mindestens intervallskaliert -> Hier ordinalskaliert
%Die Variablen sind normalverteilt -> JA
%Der untersuchte Zusammenhang zwischen den Variablen muss linear sein -> JA (durch LOSSLES Punktdiagramm ) ( auch mit herausgefilterter variable ID1!)
%
%%Ab einer Likert Skala ab 6 können ordinale Skalenniveaus als metrische variablen interpretiert werden??? Könnte ich dann vielleicht trotz der ordinal skalierten Daten eine regression durchführen?
%
%%% HYPOTHESE REGRESSION
%Daher wird die \textbf{Hypothese  H1a$_{1}$} definiert.\\
%\textbf{H1a$_{1}$} : Die Höhe des Cognitive-Trust-Score \ac{ct} hängt von der höhe des General-Trust-Scores ab.
%
%%REGRESSION
%GT ordinalskaliert -> CT ordinalskaliert > einfache lineare Regression zulässig? Eventuell ordinale regression? Wie analysiere ich sonst den kausalen zusammenhang von GT und CT?
%
%%Regression voraussetungen :
%Normalverteilung JA
%Linearität JA Augenscheinlich (durch LOSSLES)
%Heteroskedastizität Wahrscheinlich NICHT bei gegenüberstellung von SRESID ,*ZPRED
%Unabhängigkeit der Daten JA

%-------------------
%Hypothese 2

%HYPOTHESE
%\textbf{H2$_{1}$} : Der erzielte Cognitive-Trust-Score (\ac{ct}) unterscheidet sich bei den Konditionen \ac{ik} und \ac{nik} signifikant voneinander. \\

%T-Test überhaupt zulässig? CT ist ordinalskaliert! Da CT Score anhand LIKERT-SCALE gemessen wurde.
%Stattdessen Mann-Whitney-U-Test nutzen?

% T TEST ZULÄSSIG, WENN :
%Die beiden Gruppen bzw. Stichproben müssen unabhängig sein JA
%Die Variablen müssen intervallskaliert sein NEIN
%Die Variablen müssen normalverteilt sein JA
%Die Varianz innerhalb der Gruppen sollte ähnlich sein JA

% MANN-WHITNEY-U TEST ZULÄSSIG, WENN:
%Unabhängigkeit der Messungen JA
%Die unabhängige Variable ist nominalskaliert und hat zwei Ausprägungen. JA > IK, NIK
%Die abhängige Variable ist mindestens ordinalskaliert. JA, da CT Likert Scale
%Die Verteilungsform der beiden Gruppen ist (etwa) gleich. JA, DA CTI und CTN Normalverteilt ( auch mit herausgefilterter variable ID1!)

%-------------------------

%Hypothese 3
%Allgemein hierzu :
%Darf ich ordinalskalierte Daten addieren? Ich denke : Im Team ja, warum nicht?

%Was denke ich denn? Ich denke : 
%Teams mit der Kondition IK bilden einen höheren Cognitiven Trust Score und schließen dadurch mehr Runden erfolgreich ab als die Teams mit der Kondition NIK.

%Skalenniveaus : 
%CT ist ordinalskaliert
%Anz. abgeschlossener Runden ist metrisch 
%-> SPEARMAN UND PEARSON Korrelation NICHT möglich.
% Kann man eventuell die metrische variable auf eine ordinale "downgraden"?

%Wie ist es mir nun möglich, eine vermutete Kausalität CT > ROUNDS festzustellen?
%Die variablen MÜSSEN in IK und NIK aufgeteilt werden, da diese nicht gemeinsam betrachtet werden können. Grund : Der gebildete CT  ist von den Konditionen abhängig. Eine vermischung der CT werte (CTI und CTN) würde die Datenanalyse unzulässig zulässig machen.

%Korrelation IK, NIK -> Fishers Z Transformation -> Konfidenzintervalle vergleichen; bei überschneidung nicht signifikanter unterschied
%Fishers Z-Transformation : 
%https://blogs.gwu.edu/weissba/teaching/calculators/fishers-z-transformation/
%Nur was habe ich dadurch gewonnen, wenn ich weiss, dass die beiden Korrelationen sich signifikant unterscheiden? Hypothese immernoch nicht bewiesen!

%Eventuell Moderationsanalyse?

%Eventuell Chi² Test?
%Stichprobe > 50? NEIN
%Variable ordinalskaliert? JA
%Grupierung für Kreuztabelle? JA
% Laut https://statistik-dresden.de/archives/6026 ist diese jedoch nur mit intervallskalierbaren daten möglich.. ist es nicht, da CT ordinal skaliert ...

%Eventuell ANOVA?
%Hypothese für zweifaktorieller Varianzanalyse
%Es gibt einen Unterschied zwischen Teams mit hohen CT-Score und niedrigem CT-Score bei der anzahl abgeschlossener Runden in Abhängigkeit der Kondition IK und NIK.

%------------------------------------

%Hypothese 4
%H4$_{1}$ : Die reale Teameffektivität \ac{tRound} unterscheidet sich bei den Konditionen \ac{ik} und \ac{nik} signifikant voneinander. \\

%T-Test überhaupt zulässig? RoundsDone ist metrisch!
%Stattdessen Mann-Whitney-U-Test nutzen?

% T TEST ZULÄSSIG, WENN :
%Die beiden Gruppen bzw. Stichproben müssen unabhängig sein JA
%Die Variablen müssen intervallskaliert sein NEIN
%Die Variablen müssen normalverteilt sein JA
%Die Varianz innerhalb der Gruppen sollte ähnlich sein JA

% MANN-WHITNEY-U TEST ZULÄSSIG, WENN:
%Unabhängigkeit der Messungen JA
%Die unabhängige Variable ist nominalskaliert und hat zwei Ausprägungen. JA > IK, NIK
%Die abhängige Variable ist mindestens ordinalskaliert. JA, da TE metrisch
%Die Verteilungsform der beiden Gruppen ist (etwa) gleich. JA, DA TEI und TEN Normalverteilt ( auch mit herausgefilterter variable ID1!)

%-------------------------------------------------

%Hypothese 5
%\textbf{H5$_{1}$} Es besteht ein Zusammenhang zwischen dem General-Trust-Score (\ac{gt}) eines Teams und der Anzahl der abgeschlossenen Runden eines Teams. \\

%Gedanke : Ich MUSS ohne die Aufteilung in GT analysieren, weil GT unabhängig von IK und NIK.
%Kann ich überhaupt H5 OHNE aufteilung in IK und NIK analysieren? Weil : GT ist zwar unabhängig von IK und NIK, aber TE hängt (eventuell) von CT und (eventuell) von IK ab.

%KORRELATION
%GT ordinalskaliert -> Rounds Done metrisch > SPEARMAN- Korrelation überhaupt zulässig? Ich denke nicht, da ordinalskaliert nicht mit metrisch gilt. Kann man hier metrische daten "abstufen" zu ordinalskalierten?

%Regression ?
%\textbf{H5a$_{1}$} : Die höhe der abgeschlossenen Runden \ac{tRounds} hängt von der Höhe des General-Trust-Scores \ac{gt} ab.
%GT ordinalskaliert -> Abg. Runden metrisch > einfache lineare Regression zulässig? Eventuell ordinale regression? Wie analysiere ich sonst den kausalen zusammenhang von GT und CT?

%Was sind die Alternativen wenn ich einen Zusammenhang feststellen möchte? Ordinale Regression?

%Regression voraussetungen :
%Normalverteilung JA
%Linearität JA Augenscheinlich (durch LOSSLES)
%Heteroskedastizität Wahrscheinlich NICHT bei gegenüberstellung von SRESID ,*ZPRED BEI IK UND NIK UND OHNE AUFTEILUNG
%Unabhängigkeit der Daten JA

\section{Versuchshypothesen}
In diesem Kapitel werden auf Basis der vorherigen Grundlagen verschiedene Hypothesen definiert.
Die \textit{Abbildung \ref{Versuchshypothesen}} verdeutlicht den Zusammenhang zwischen den verschiedenen Versuchshypothesen sowie die Ebene auf der diese Analysiert werden.
\subsection{Das Framework}

\paragraph{Analyseebenen}
Da die teilnehmenden Personen als Team arbeiten und unterschiedliche Teams unterschiedliche Konditionen aufweisen, sind einige Zusammenhänge auf \textit{Individualebene}, einige auf \textit{Konditionsebene} und einige auf \textit{Teamebene} zu betrachten.
%Da die teilnehmenden Personen als Team arbeiten, sind einige Zusammenhänge und Unterschiede auf Konditionsebene (\ac{ik} und \ac{nik}) und einige auf Teamebene zu betrachten.

Die \textbf{Individualebene} sagt etwas über die einzelnen Personen aus. Dadurch können alle teilnehmenden Personen individuell betrachtet werden. Die Betrachtung ist unabhängig vom Team oder den verschiedenen Avatarkonditionen. 
%Wie viel Vertrauen bildet die einzelne Person während des Versuchs? Wie hoch ist die generelle Hang zum Vertrauen einzelner Personen? Wie immersiv ist die \ac{vr}-Erfahrung während des Experiments einer einzelnen Person?

Die \textbf{Konditionsebene} unterscheidet zwischen den Konditionen \ac{ik} und \ac{nik}. Die Konditionsebene ordnet den einzelnen teilnehmenden Personen die Kondition zu, die diese im \ac{sve} als \textit{andere} Avatare wahrnehmen. Auf dieser Ebene ist es möglich, zwei verschiedene Gruppen desselben Versuchs miteinander zu vergleichen. 
%Welche Kondition absolviert mehr Runden? Welche Kondition baut mehr kognitives Vertrauen auf? Welche Kondition schafft mehr Immersion?

Die Konditionsebene kann in einzelne Teams von jeweils 3 Personen aufgeteilt werden. Diese Aufteilung wird als \textbf{Teamebene} bezeichnet und betrachtet das gesamte Team als Einheit. Jedes Teammitglied besitzt dieselben Konditionen. Wird das Team auf Teamebene betrachtet, ist es möglich Aussagen über das Team zu treffen. 
%Bildet ein Team einer Kondition besonders viel kognitives Vertrauen? Unterscheidet sich die Anzahl der abgeschlossenen Runden pro Team durch die verschiedenen Avatar Konditionen?

\textit{Abbildung \ref{DifferentLevels}} zeigt die Hierarchie der verschiedenen Ebenen.

Das Framework in der Abbildung \ref{Versuchshypothesen}) zeigt die einzelnen Komponenten, auf die in dieser Masterarbeit eingegangen werden. 

\paragraph{Generelles Vertrauen}
Es wird der Zusammenhang zwischen dem generellen Vertrauen (\ac{gt}) und dem kognitiven Vertrauen (\ac{ct}) auf \textit{Konditionsebene} analysiert.
Es wird auf \textit{Teamebene} analysiert, ob ein \textit{Zusammenhang} zwischen dem generellen Vertrauen und der Anzahl an abgeschlossenen Runden (\ac{tRound} besteht.

\paragraph{Kognitives Vertrauen}
Es wird der \textit{Zusammenhang} zwischen dem gebildeten kognitive Vertrauen (\ac{ct}) und der Anzahl der abgeschlossenen Runden (\ac{tRound}) auf \textit{Teamebene} analysiert.

\paragraph{Avatar Embodiment}
Es wird der \textit{Unterschied} des gebildeten dem kognitiven Vertrauen (\ac{ct}) auf \textit{Konditionsebene} analysiert.
Es wird der \textit{Unterschied} der Anzahl an abgeschlossenen Runden (\ac{tRound}) auf \textit{Teamebene} analysiert.


\begin{figure}[H]
		\begin{footnotesize}
		\centering
			\includegraphics[scale=0.4]{Abbildungen/DifferentLevels.JPG}
			
			\caption[Abbildung 1]{Hierarchie der Individualebene, Konditionsebene und Teamebene}
			\label{DifferentLevels}
		\end{footnotesize}
	\end{figure}

Anhand dieses des Frameworks in \textit{Abbildung \ref{Versuchshypothesen}}, wurden Hypothesen entwickelt, die im folgenden genauer definiert und erklärt werden.

\begin{figure}[H]
		\begin{footnotesize}
			\includegraphics[width=\textwidth]{Abbildungen/Versuchshypothesen_02.JPG}
			
			\caption{Framework inklusive Versuchshypothesen. Aufgeteilt und dargestellt nach verschiedenen Ebenen}
			\label{Versuchshypothesen}
		\end{footnotesize}
	\end{figure}	

\newpage

Da dieses Framework nicht validiert ist, kann es sein, dass die verschiedenen Faktoren nicht zu gleichen oder auch gar keinen Einfluss auf die Team-Effektivität haben.

%\paragraph{Die Analyseebenen}
%Einige Variablen wurden in dieser Studie auf der Individualebene, auf der Konditionsebene oder auf der Teamebene gemessen.
%Konditionsebene bedeutet dabei, dass das die Ergebnisse zwischen der Kondition \ac{ik} sowie \ac{nik} aufgeteilt wurden, während auf der Individualebene alle teilnehmenden Personen einzeln bewertet wurden. Die Teamebene teilt die Konditionen \ac{ik} und \ac{nik} zusätzlich noch in Teams von 3 Personen auf.  Alle Variablen, die auf der Individualebene gemessen wurden, haben eine Stichprobengröße $N = 30$, die auf dem Konditionsebene $ N = 15$ und die auf der Teamebene von $N = 5$.
%Die \textit{Tabelle \ref{MittelwerteUndCo}} zeigt eine Übersicht über die gemessenen Variablen, deren Mittelwerte, Standartabweichungen und die Anzahl der Teilnehmer der Messung.

\subsection{Die Versuchshypothesen}
\label{VersuchshypothesenAuflistung}

\paragraph{Hypothese 1}:
%Wie im Kapitel \textbf{\nameref{Vertrauen}} beschrieben, bildet sich affektives Vertrauen nur über einen längerfristigen Zeitraum. Daher findet während der Versuchsdurchführung keine Messung des affektiven Vertrauens statt. 
Kognitives Vertrauen wird in der Anfangsphase einer Teamgründung frühzeitig gebildet (vgl. Kapitel \ref{Vertrauen als Zustand}).
Eine Analyse des kognitiven-Vertrauens auf Konditionseben, soll festgestellen, ob sich unterschiedliche Avatar Konditionen auf die Wahrnehmung einer Person, in Kompetenzen, Fähigkeiten und Zuverlässigkeiten des \ac{vt}, im \ac{sve}, auswirken.
Daher wird \textbf{Hypothese 2} wie folgt definiert:\\
%H2$_{0}$ : Der erzielte Cognitive-Trust-Score (\ac{ct}) unterscheidet sich bei den Konditionen \ac{ik} und \ac{nik} nicht voneinander.\newline
\textbf{H1$_{1}$} : Die Mittelwerte der erzielten kognitive Vertrauenswerte (\ac{ct}) unterscheiden sich bei den Konditionen \ac{ik} und \ac{nik} signifikant voneinander. \newline
\textbf{H1$_{0}$} : Die Mittelwerte der erzielten kognitive Vertrauenswerte (\ac{ct}) unterscheiden sich bei den Konditionen \ac{ik} und \ac{nik} nicht voneinander. \\

%Es wird davon ausgegangen, dass durch unterschiedliche Avatar Konditionen signifikant unterschiedliche kognitive Vertrauenswerte im \ac{sve} gebildet werden. Somit würde eine Person mit der Avatarkondition \ac{ik} einen signifikant unterschiedlichen kognitiven Vertrauenswert, im Vergleich zu einer Person mit der Avatarkondition \ac{nik}, bilden.

%H2$_{0}$ geht davon aus, dass die Kondition, in der eine Person im \ac{sve} die \dq{}anderen\dq{} teilnehmenden Personen wahrnimmt, ein Signifikant abweichenden Einfluss Somit erscheint ein \ac{vt} in seiner Leistungsfähigkeit und Zuverlässigkeit für eine Person immer gleich, egal ob die Avatare der \dq{}anderen\dq{} die Kondition \ac{ik} oder \ac{nik} besitzen.
\paragraph{Hypothese 2}:

%Der Hang zum Vertrauen ist das Mindestvertrauen, dass eine Person einer anderen gewährt. 
%Es wird davon ausgegangen, dass sich die einzelnen Teammitglieder bei der Neugründung eines \ac{vt} im \ac{sve} nicht kennen.
%Der Hang zum Vertrauen kann daher als Hilfsmittel eingesetzt werden, um eine Zunahme oder Abnahme, des kognitiven Vertrauens eines Team, zu analysieren.
%Es lässt sich vermuten, dass Personen, die ein hohes Grundvertrauen besitzen, auch ein hohes kognitives Vertrauen bilden. Besitzt eine Person ein niedriges Grundvertrauen, bildet diese auch weniger kognitives Vertrauen (\textit{Siehe \nameref{Vertrauen als Zustand}}). 
Diese Hypothese wird auf Konditionsebene untersucht, um festzustellen, ob sich unterschiedliche Zusammenhänge aufgrund einer Avatarkondition zwischen dem \ac{gt} und dem \ac{ct} bilden.
Es besteht keine Vermutung darüber, ob eine bestimmte Avatarkondition einen stärkeren Zusammenhang erzielt. \newline
Da ein höheres kognitives Vertrauen in ein \ac{vt}, während der Neugründung im \ac{sve}, wünschenswert ist, wird aus der vorherigen Erkenntnis die \textbf{Hypothese H1$_{1}$} definiert.\\
% HYPOTHESE KOLLERATION

\textbf{H2$_{1}$} : Je höher die erzielten generelle Vertrauenswert \ac{gt} einer Person desto höher ist der erzielte kognitive Vertrauenswert \ac{ct} einer Person. \newline
\textbf{H2$_{0}$} : Es besteht ein kein Zusammenhang zwischen dem erzielten generellen Vertrauenswert \ac{gt} einer Person und dem erzielten kognitiven Vertrauenswert \ac{ct} einer Person.

%Es wird vermutet, dass die Höhe des Cognitive-Trust-Score \ac{ct} von der Höhe des General-Trust-Score \ac{gt} abhängt wird.
%% HYPOTHESE REGRESSION
%Daher wird die \textbf{Hypothese  H1a$_{1}$} definiert.\\
%\textbf{H1a$_{1}$} : Die Höhe des Cognitive-Trust-Score \ac{ct} hängt von der höhe des General-Trust-Scores \ac{gt} ab.
%H1$_{0}$ : Steigt der General-Trust-Score einer Person (\ac{gt}), steigt der Cognitive-Trust-Score einer Person (\ac{ct}) nicht. \newline
%H1$_{1}$ : Steigt der General-Trust-Score einer Person(\ac{gt}), steigt auch der Cognitive-Trust-Score einer Person(\ac{ct}).) \\

%H1$_{0}$ : Je höher General-Trust-Score einer Person (\ac{gt}), desto höher der Cognitive-Trust-Score einer Person (\ac{ct}). \newline

% HYPOTHESE ZUR LINEAREN REGRESSION AUFSTELLEN !!!!

%------ALT
%H1$_{0}$ : Es besteht kein Zusammenhang zwischen dem General-Trust-Score (\ac{gt}) einer Person und dem Cognitive-Trust-Score (\ac{ct}) einer Person. \newline
%H1$_{1}$ : Je höher General-Trust-Score einer Person (\ac{gt}) ist, desto höher ist der Cognitive-Trust-Score einer Person (\ac{ct}). \newline
%----

%H1$_{0}$ : Ein hoher General-Trust-Score (\ac{gt}) wirkt sich \textbf{nicht positiv} auf den Cognitiven-Trust-Score (\ac{ct}) aus. \newline
%H1$_{1}$ : Ein hoher General-Trust-Score (\ac{gt}) wirkt sich \textbf{positiv} auf den Cognitiven-Trust-Score (\ac{ct}) aus.
%------
%Es gibt keinen Zusammenhang zwischen dem General-Trust-Score und dem Cognitive-Trust-Score.
%Je höher der General-Trust-Score eine Person ist, desto höher ist auch der erzielte Cognitive-Trust-Score.

%H1$_{0}$ geht davon aus, dass Personen, die leicht einen Vertrauensvorschuss gewähren, nicht immer gleichzeitig auch in die Fähigkeiten oder in die Zuverlässigkeit des Teams glauben.
%Dies kann sich natürlich aber während des Versuchs ändern! Schlechte Teams sollten auch ein schlechteres Cognitives Vertrauen besitzen!
\paragraph{Hypothese 3}:
%NEU NEU NEU!
%Der Zusammenhang zwischen gebildetem kognitiven Vertrauen und abgeschlossenen Runden ist bei Personen mit der Avatarkondition IK größer als bei denen mit der Avatarkondition NIK.
%Fisher Z- Transformation !
%Hier weil F(p) > F(p) gerichtete Korrelationsannahme. Daher mit Signifikanzniveau von 0.1 testen?
%H0 : Die Korrelationen sind gleich ; H1 : Die Korrelationen unterscheiden sich.
%
% VGL : https://www.youtube.com/watch?v=SVuniaJXOoM&ab_channel=drfg2008
Ein starkes zwischenmenschliches Vertrauen in traditionell geformten Teams beeinflusst die reale Teameffektivität zum positiven. Daher wird davon ausgegangen, dass dies auch auf \ac{vts} zutrifft. 
%Es wird das kognitive Vertrauen analysiert, da das affektive Vertrauen vernachlässigt werden kann. 
Je größer das kognitive Vertrauen eines Teams, desto mehr können die einzelnen Teammitglieder ihre Ressourcen in die positiven Aspekte der Teamarbeit einbringen. Es findet eine effektivere Teamarbeit statt. Es werden keine defensiven Haltungen eingenommen, die darauf abzielen könnten, sich vor möglichem Schaden durch andere zu schützen und es kann sich auf die Erreichung des Ziels fokussiert werden. 
Das gesamte Team wird als Einheit betrachtet, da Gruppenzugehörigkeits und Identifikantionseffekte berücksichtigt werden müssen. Besitzt das Team ein hohes wechselseitiges kognitives Vertrauen, könnte sich dies auf die erfolgreich abgeschlossenen Runden eines Teams auswirken. 
Es wird die \textbf{Hypothese 3} wie folgt aufgestellt:\\
\textbf{H3$_{1}$} : Der Zusammenhang des kognitive Vertrauenswert von Teams \ac{teamCog} und der realen Teameffektivität von Teams \ac{tRound} mit der Kondition \ac{ik} ist stärker als der von Teams mit der Kondition \ac{nik}. \newline
\textbf{H3$_{0}$} : Der Zusammenhang des \ac{teamCog} und der realen Teameffektivität von Teams \ac{tRound} mit der Kondition \ac{ik} unterscheiden sich nicht von den Teams mit der Kondition \ac{nik}. \\
%
%Was denke ich denn? Ich denke : 
%Teams mit der Kondition IK bilden einen höheren Cognitiven Trust Score und schließen dadurch mehr Runden erfolgreich ab als mit der Kondition NIK.
%Es wird vermutet, dass ein Zusammenhang zwischen der Summe des gebildetem kognitiven Vertrauen der Teams sowie den erfolgreich abgeschlossenen Runden der Teams bei unterschiedlichen Avatar Konditionen besteht.
%
%Falls sich dieser Effekt bewahrheitet, kann analysiert werden, bei welcher Avatarkondition dieser Effekt stärker auftritt (\textit{Siehe \nameref{Vertrauen und virtuelle Teams}}).
%Es wird vermutet, dass die Teams mit der Kondition \ac{ik} 
%
%Es gibt keinen Signifikanten unterschied in der stärke des Zusammenhangs der CT Werte und realen Teameffektivität zwischen den Teams der verschiedenen Konditionen.
%
%Korrelation IK, NIK -> Fishers Z Transformation -> Konfidenzintervalle vergleichen; bei überschneidung nicht signifikanter unterschied
%Fishers Z-Transformation : 
%https://blogs.gwu.edu/weissba/teaching/calculators/fishers-z-transformation/
%H30 Der Zusammenhang des CTs der Teams und der Rounds der Teams unterscheidet sich nicht zwischen den Konditionen IK und NIK.
%
%H31 Der Zusammenhang des CTs der Teams und der Rounds der Teams ist Signifikanten stärker in den IK Kondition als in den NIK Kondition.
%
%Gibt es unterschiede des CT der einzelen Teams zwischen den Konditionen IK und NIK.
%
%Hypothese für zweifaktorieller Varianzanalyse
%%Es gibt einen Unterschied zwischen Teams mit hohen CT-Score und niedrigem CT-Score bei der realen Teameffektivität in Abhängigkeit der Kondition IK und NIK.
%
%% Der CT der IK Gruppe hängt mit der anzahl der Runden zusammen. Der CT der NIK Gruppe hängt nicht mit der anzahl der Runden zusammen. 
%% Es gibt ein Unterschied in dem Zusammenhang zwischen dem CT und Anzahl Runden zwischen den Gruppen. Ungerichtete Unterschiedshypothese.
%
%Daher wird folgende Hypothese definiert :\\
%H3$_{0}$ : Teams mit einem hohen Cognitive-Trust-Score \ac{ct} schließen weniger oder gleich viele Runden \ac{tRound} erfolgreich ab als Teams mit einem geringerem Cognitive-Trust-Score.
%H3$_{1}$ : Teams mit einem hohen Cognitive-Trust-Score schließen mehr Runden erfolgreich ab als Teams mit einem geringerem Cognitive-Trust-Score.
%
%H3a$_{0}$ : Teams mit einem hohen Cognitive-Trust-Score und der Kondition \ac{ik} schließen weniger oder gleich viele Runden erfolgreich ab als Teams mit einem hohen Cognitive-Trust-Score und der Kondition \ac{nik}.
%H3a$_{1}$ : Teams mit einem hohen Cognitive-Trust-Score und der Kondition \ac{ik} schließen mehr Runden erfolgreich ab als Teams mit einem hohen Cognitive-Trust-Score und der Kondition \ac{nik}.
%
%---ALT
%	H3$_{0}$ : Ein hoher Cognitive-Trust-Score im Team (\ac{teamCogIK} und \ac{teamCogNIK}) hat \textbf{keinen Einfluss} auf die auf die Anzahl der erfolgreich abgeschlossenen Runden (\ac{tRound}) bei unterschiedlichen Avatar Verkörperungen (\ac{tRoundIK} und \ac{tRoundNIK}). \newline
%	H3$_{1}$ : Ein hoher Cognitive-Trust-Score im Team (\ac{teamCogIK} und \ac{teamCogNIK}) hat \textbf{Einfluss} auf die auf die Anzahl der erfolgreich abgeschlossenen Runden (\ac{tRound}) bei unterschiedlichen Avatar Verkörperungen (\ac{tRoundIK} und \ac{tRoundNIK}).
%--------
%Es werden weniger oder gleich viele Runden im Team erfolgreich abgeschlossen, wenn der Cognitive-Trust-Score im Team steigt.
%Es werden mehr Runden im Team erfolgreich abgeschlossen, wenn der Cognitive-Trust-Score im Team steigt.
%
% (\ac{tRound}) bei unterschiedlichen Avatar Verkörperungen (\ac{tRoundIK} und \ac{tRoundNIK})
 %
% ---
%H3$_{0}$ : Teams mit einem hohen Cognitive-Trust-Score schließen weniger oder gleich viele Runden erfolgreich ab als Teams mit einem geringerem Cognitive-Trust-Score.
%H3$_{1}$ : Teams mit einem hohen Cognitive-Trust-Score schließen mehr Runden erfolgreich ab als Teams mit einem geringerem Cognitive-Trust-Score.
%
%	-----------------H3a
%H3a$_{0}$ : Teams mit einem hohen Cognitive-Trust-Score und der Kondition \ac{ik} schließen weniger oder gleich viele Runden erfolgreich ab als Teams mit einem hohen Cognitive-Trust-Score und der Kondition \ac{nik}.
%H3a$_{1}$ : Teams mit einem hohen Cognitive-Trust-Score und der Kondition \ac{ik} schließen mehr Runden erfolgreich ab als Teams mit einem hohen Cognitive-Trust-Score und der Kondition \ac{nik}.
%
%Die Anzahl der abgeschlossenen Runden ist höher bei Teams mit der Kondition \ac{ik}
%
%Steigt der General-Trust-Score einer Person(\ac{gt}), steigt auch der Cognitive-Trust-Score einer Person(\ac{ct}).)	
%%	Es wird davon ausgegangen, dass wenn Personen oder Teams sich untereinander Leistungsfähig oder Zuverlässiger ansehen, dies keinen Einfluss auf die Team-Effektivität hat. 
%\newpage
\paragraph{Hypothese 4}:
Die unterschiedlichen Avatarkonditionen könnten einen Einfluss auf die Anzahl der Abgeschlossenen Runden eines Teams - und somit auf die reale Teameffektivität - haben (\textit{Siehe \nameref{Teameffektivität}}). Um dies zu untersuchen, wurde \textbf{\textit{Hypothese H4$_{1}$}} aufgestellt :\\
H4$_{1}$ : Die Mittelwerte der realen Teameffektivität \ac{tRound} unterscheidet sich bei den Konditionen \ac{ik} und \ac{nik} signifikant voneinander. \newline
H4$_{0}$ : Die Mittelwerte der realen Teameffektivität \ac{tRound} unterscheidet sich bei den Konditionen \ac{ik} und \ac{nik} nicht voneinander. \\
%%----- ALT
%	H4$_{0}$ : Die Konditionen \ac{ik} oder \ac{nik} haben \textbf{keinen signifikant abweichenden Einfluss} auf die Anzahl der erfolgreich abgeschlossenen Runden \ac{tRound} eines Teams.\newline
%	H4$_{1}$ : Die Konditionen \ac{ik} oder \ac{nik} haben \textbf{einen signifikant abweichenden Einfluss} auf die Anzahl der erfolgreich abgeschlossenen Runden \ac{tRound}.
%	
%----
%	\item{\underline{Hyothese 4} :\\ H$_{0}$ : Ein Torso, Head und Hand getrackter Avatar hat \textbf{keinen Einfluss} auf die Team-Effektivität. \newline
%	H$_{1}$ : Ein Torso, Head und Hand getrackter Avatar hat \textbf{Einfluss} auf die Team-Effektivität.}	
%Es mit der Frage beschäftigt, ob ein Team effektiver (mehr Runden abschließt) ist, je nachdem welche Avatar Kondition diesem zugeordnet wird. Es wird davon ausgegangen, dass die Anzahl der abgeschlossenen Runden sich zwischen den beiden Konditionen unterscheidet.
%Es wird davon ausgegangen, dass die Team-Effektivität  aufgrund der Menschenähnlichkeit variiert.
%Es wird davon ausgegangen, dass die Anzahl der abgeschlossenen Runden sich zwischen den beiden Konditionen unterscheidet.
\paragraph{Hypothese 5}:
Damit die reale Teameffektivität (\textit{Siehe \nameref{Teameffektivität}}) im Zusammenhang mit dem generellen Vertrauen betrachtet werden kann, wurde \textbf{\textit{Hypothese 5}} aufgestellt:\\
\textbf{H5$_{1}$} : Der Zusammenhang der generelle Vertrauenswert eines Teams \ac{teamGen} und der realen Teameffektivität eines Teams \ac{tRound} mit der Kondition \ac{ik} ist stärker als von Teams mit der Kondition \ac{nik}. \newline
\textbf{H5$_{0}$} : Der Zusammenhang der generelle Vertrauenswert eines Teams \ac{teamGen} und der realen Teameffektivität eines Teams \ac{tRound} mit der Kondition \ac{ik} unterscheidet sich nicht von den Teams mit der Kondition \ac{nik}. \\
%\textbf{H5$_{1}$} : Es besteht ein Zusammenhang zwischen dem generellen Vertrauenswert (\ac{gt}) eines Teams und der Anzahl der abgeschlossenen Runden eines Teams \ac{tRound}. \\
%
%Es wird ein vermutet, dass die Höhe der abgeschlossenen Runden von der Höhe des General-Trust-Scores abhängt.
%\textbf{H5a$_{1}$} : Die höhe der abgeschlossenen Runden \ac{tRound} hängt von der Höhe des General-Trust-Scores \ac{gt} ab.
%
%
%-------- ALT
%	H5$_{0}$ : Teams, aufgeteilt nach Avatar Verkörperungen (\ac{ik}, \ac{nik}), mit einem hohem General-Trust-Score (\ac{gt}) \textbf{schließen nicht mehr Runden ab}, (\ac{tRound}) als die mit einem niedrigen General-Trust-Score \ac{gt}.\newline
%	H5$_{1}$ :  Teams, aufgeteilt nach Avatar Verkörperungen (\ac{ik}, \ac{nik}), mit einem hohem General-Trust-Score (\ac{gt}) \textbf{schließen mehr Runden ab} (\ac{tRound}) als die mit einem niedrigen General-Trust-Score \ac{gt}.
%--------
%H5$_{0}$ : Teams, aufgeteilt nach Avatar Verkörperungen (\ac{ik}, \ac{nik}), mit einem hohem General-Trust-Score (\ac{gt}) \textbf{schließen nicht mehr Runden ab}, (\ac{tRound}) als die mit einem niedrigen General-Trust-Score \ac{gt}.


%	H5a$_{0}$ : FÜR REGRESSION
%Diese Hypothese beschäftigt sich somit mit der Frage, ob Teams in der \ac{vr} Effektiver sind wenn diese eher dazu bereit sind, anderen Personen Vertrauensvorschuss gewähren. Es wird davon ausgegangen, dass einen Zusammenhang zwischen dem General-Trust-Score Teams \ac{teamGen}  sowie der Anzahl der abgeschlossenen Runden \ac{tRound} eines Teams existiert.\\
%Weiterhin wird vermutet, dass die Höhe der abgeschlossenen Runden von der Höhe des General-Trust-Scores abhängt.

%\paragraph{Hypothese 6}:
%%Wenn man nun eine Stufe runter gehen möchte, könnte ich einfach auch nur überprüfen, ob unterschiedliche Avatarkonditionen einen unterschiedlichen Einfluss auf die gebildete Co-Präsenz besitzt.
%Da die teilnehmenden Personen entweder die Avatarkondition \ac{ik} oder die Avatarkondition \ac{nik} während des Versuchs als die \dq{}anderen\dq{} Personen im Raum wahrnehmen, wird der unterschied der wahrgenommenen Co-Präsenz durch die Avatarkonditionen \ac{ik} und \ac{nik} analysiert.
%
%Es wird die \textbf{\textit{Hypothese 6}} aufgestellt :
%
%H6$_{1}$ : Es besteht ein signifikanter unterschied bei der wahrgenommenen Co-Präsenz zwischen der Avatarkondition \ac{ik} und \ac{nik}.
%
%%NEU NEU NEU!
%%Der Zusammenhang zwischen gebildetem kognitiven Vertrauen und abgeschlossenen Runden ist bei Personen mit der Avatarkondition IK größer als bei denen mit der Avatarkondition NIK.
%%Fisher Z- Transformation !
%%Hier weil F(p) > F(p) gerichtete Korrelationsannahme. Daher mit Signifikanzniveau von 0.1 testen?
%%H0 : Die Korrelationen sind gleich ; H1 : Die Korrelationen unterscheiden sich.
%% VGL : https://www.youtube.com/watch?v=SVuniaJXOoM&ab_channel=drfg2008
%
%Da die teilnehmenden Personen entweder die Avatarkondition \ac{ik} oder die Avatarkondition \ac{nik} während des Versuchs als die \dq{}anderen\dq{} Personen im Raum wahrnehmen, wird der Zusammenhang an gebildeter Co-Präsenz und gebildetem kognitiven Vertrauen zwischen den unterschiedlichen Avatarkonditionen analysiert.
%
%Es wird die \textbf{\textit{Hypothese 6}} aufgestellt :
%
%H6$_{1}$ : Der Zusammenhang zwischen der wahrgenommen Co-Präsenz und gebildeten kognitiven Vertrauen ist bei Personen mit der Avatarkondition IK \textbf{größer} als bei denen mit der Avatarkondition NIK.
%
%\paragraph{Hypothese 7}:
%Die Avatare mit der Kondition \ac{ik} werden als Inverse-Kinematische, menschenähnliche Avatare dargestellt. Die Avatare mit der Kondition \ac{nik} werden als nicht-Inverse-Kinematische, nicht menschenähnliche Avatare dargestellt. Um nun den Einfluss der unterschiedlichen Avatarko

\newpage
\section{Vorgehensweise}
\subsection{Der Versuch}
	
Das folgende Kapitel beschreibt den Ablauf des Versuchs, die Methodik sowie den Versuchsaufbau.

\subsubsection{Untersuchungsmethode}
Um die beiden Unabhängigen Variablen \dq{}Non-IK-Avatar\dq{} und \dq{}IK-Avatar\dq{} innerhalb einer Teambuildingmaßnahme zu testen, wurde das A/B-Testing in kombination einer Quantitativen Forschung gewählt.
Gruppe A bekam dabei die Kondition \dq{}Non-IK-Avata\dq{} zugeteilt, während Gruppe B die Konditiion \dq{}IK-Avatar\dq{} zugeteilt bekam. Diese Aufteilung, welcher Probant welche Kondition zugeteilt bekam, erfolgte nach dem Zufallsprinzip. 

	\subsubsection{Teilnehmerfindung}
Der gesamte Versuch wurde als Into-the-Wild Experiment aufgebaut, was bedeutet, dass die genaue Zielgruppe nicht definiert werden kann. Die einzigen Ristriktion an die Zielgruppe ist, dass diese die zu verwendete Hardware besitzen. Es wurde in verschiedenen Foren ( z.B. VRForum.de, Computerbase.de, Hardwareluxx.de, etc.) durch eine Suchanfrage, in Form eines extra dafür angelegten Threads, Teilnehmer gesucht die an der Studie teilnehmen wollen. Weiterhin wurden Teilnehmer durch verschieden sozialen Netzwerken mit einem Bezug zu \ac{vr}, sowie durch zufällige Whatsapp-Chatgruppen mit 50 oder mehr Mitgliedern, akquiriert. Da der gesamte Versuch, die Fragebögen sowie das Erklärvideo auf deutscher Sprache erstellt wurde, fand die Teilnehmerfindung nur im deutschsprachigem Raum statt.

	\subsubsection{Versuchsablauf}
Insgesamt nahmen 30 Personen an dem Forschungsversuch teil, wobei ein Team aus 3 Personen bestand. Somit gab es insgesamt 10 Teams. Jedes Team bekam entweder die Kondition \ac{ik} oder \ac{nik} zugeordnet. Es gab 5 Teams mit der Kondition \ac{ik} sowie 5 Teams mit der Kondition \ac{nik}.
Jede teilnehmende Person bekam einen zufälligen Zeitslot und einen anonymen Namen zugeordnet mit dem die Person an dem Versuch teilnahm. Gemäß des A/B-Testing wurden jeweils drei Personen in einem Zeitslot untergebracht um ein \dq{}Team\dq{} zu bilden.
Insgesamt nahmen somit drei Person, an einem Versuch zur selben Zeit mit der selben Kondition, teil. Die teilnehmenden Personen wurden sich untereinander \textit{nicht} Face-To-Face vorgestellt und sahen sich während des gesamten Versuch nur als Repräsentation eines Avatars in dem \ac{sve}. 
Ein Zeitslot von 30 Minuten teilte sich auf in
		\begin{itemize}
			\item 5 Minuten Pre-Questionnaire
			\item 5 Minuten Videoerklärung
			\item 15 Minuten Versuchsdurchführung
			\item 15 Minuten Post-Questionnaire
		\end{itemize}
Jede teilnehmende Person bekam zu Beginn seines Zeitslots einen Pre-Questionnaire ausgehändigt, den dieser selbstständig ausfüllen sollte.
Alle teilnehmende Personen mussten sich anschließend ein Erklärvideo über das Experiments anschauen, indem alle relevanten Mechaniken und Funktionsweisen sowie der grobe Spielablauf erklärt wurden. Durch das vorherige Erklärvideo wurde sichergestellt, dass alle teilnehmende Person den selben Informationsgehalt über die Art und Weise des Ablaufs des Experiments besaßen. Alle Mitglieder eines Teams starteten somit auf einheitlichen Wissensstand.
Nachdem alle Personen das Videoerklärung angeschaut hatten, begann das Experiment. Dazu starteten die jeweiligen teilnehmenden Personen die Anwendung. Es wurde sich automatisch in das \ac{sve} eingeloggt. Die teilnehmenden Personen spielten gemeinsam den Versuch als Team durch.
	Am Ende der Versuchsdurchführung, wurde ein Post-Questionnaire ausgeteilt.  Am Ende war es jeden Teilnehmer zusätzlich möglich, ein Feedback über die Erfahrung zu geben. Der Pre- sowie der Post-Questionnaire wurde insgesamt von 30 Personen ausgefüllt. 
Die maximale Versuchsdauer nach Start der Anwendung betrug exakt 10 Minuten (600 Sekunden). Es konnten maximal 15 Runden absolviert werden, wobei jede Runde inkrementell schwieriger wurde, da jede 3. Runde jeweils 1 Symbol, in den Pool der zu erratenden Symbole, hinzukam. 

	\subsubsection{Allgemeiner Versuchsaufbau}

\begin{figure}[h]
		\begin{footnotesize}
			\includegraphics[width=\textwidth]{Abbildungen/Podeste.JPG}
			
			\caption[Abbildung 1]{Die Podeste der Teilnehmer}
			\label{Podeste}
		\end{footnotesize}
	\end{figure}

Jede teilnehmende Person benötigte (neben einem funktionsfähigen Computer), um an dem Versuch teilnehmen zu können, entweder ein Oculus, HTC-Vive oder Windows Mixed-Reality\ac{hmd}, sowie zwei funktionsfähige Controller.

Dem Versuchsleiter war es während der gesamten Anwendung möglich, die 3 Probanten durch einen separaten Spectator-Client zu betreuen. Alle teilnehmende Person konnten in der Anwendung durch das integrierte Mikrofon im \ac{hmd} zu dem Versuchsleiter sprechen und diesen hören. Wenn eine teilnehmende Person sprach, konnten die anderen zwei teilnehmenden Personen diese nicht hören. Die Sprachkommunikation einer teilnehmenden Person war somit nur Richtung Versuchsleiter möglich um Störvariablen zu vermeiden und die Integrität der Anonymität zu erhalten.
Während der Versuchsleiter sprach, konnten jedoch alle teilnehmenden Personen den Versuchsleiter hören. Dies diente dazu, eventuelle offene Fragen an alle Versuchsteilnehmer weiterzugeben und den Beginn sowie das Ende des Versuchs zu kommunizieren. Der Versuchsleiter gab jedoch während des gesamten Dauer des Versuchs keine Hilfestellung.

%In der Anwendung war es den Probanten möglich mittels der Tasten \dq{}W A S und D\dq{} ihre Position und mittels den Tasten \dq{}Q\dq{} und \dq{}E\dq{} ihre Höhe, in einem gewissen Bereich, zu ändern. Dies stellte sicher, dass alle Probanten ihre Avatargröße und Position individuell an ihre Körpergröße anpassen konnten.
%
%Da es den Probanten nicht möglich sein sollte, auf die Symbole der anderen teilnehmenden Probanten zu schauen, wurde ab einer gewissen Grenze Links und Rechts des Podest der Probanten ein \dq{}Fade-To-Black\dq{} Mechanismus eingebaut. Kamen die Probanten mit dem Kopf ihres Avatars in diesen Bereich, sahen diese nur noch die Farbe Schwarz und mussten sich zurückbewegen.

Waren alle teilnehmenden Personen bereit, wurden diese vor ein für Sie eigenes Podest (\textit{Abbildung \ref{Podeste}}) teleportiert und sahen einen Countdown zwischen den 3 Podesten herunter zählen. Durch diesen Countdown wurde der baldige beginn einer Runde eingeleitet.
Wurde eine Runde gestartet, wurde das Podest jeder teilnehmenden Personen eindeutig durch die Farbe \dq{}Schwarz, Rot oder Grün\dq{} markiert. Die teilnehmenden Personen konnten Ihre und die Farbe der anderen Spieler an einem Viereck sowie einer runden Kugel in der jeweils zugeteilten Farbe, an den jeweiligen Podesten, erkennen. Die Farbe der einzelnen teilnehmenden Personen änderte sich jede Runde.

	\subsubsection{Detaillierter Versuchsablauf}

Zu Beginn jeder neuen Runde ist der Schwarz markierte Spieler mit dem Erklären der Symbole an der Reihe.
Auf auf Podesten der andersfarbigen Spieler befinden sich ebenfalls Symbole, welche jedoch unterschiedlich angeordnet sind. Die Symbole sind bei dem schwarz markierten Spieler entweder in der Farbe \dq{}Grün oder Rot\dq{} oder \dq{}Grün und Rot\dq{} markiert. Die Symbole der anderen Spieler sind in keiner Farbe markiert. Siehe \textit{Abbildung \ref{Podeste}}
Der schwarz markierte Spieler versucht nun, mittels Hand und Armbewegung, den Rot und Grün markierten Spieler die Symbole, die in der jeweiligen Spielerfarbe vor Ihm markiert sind, zu erklären.
Meint ein Spieler (Rot oder Grün), dem der schwarz markierte Spieler ein Symbol erklärt hat, ein Symbol erkannt zu haben, loggt dieser das Symbol durch herunterdrücken des Knopfes an seinem Podest, mit dem jeweiligen erkanntem Symbol, ein. Der schwarz markierte Spieler muss nichts einloggen oder klicken, sondern nur mittels Gestikulation die Symbole erklären.
Hat ein Spieler sich während des einloggens der Symbole verklickt oder möchte seine Meinung zum erkanntem Symbol ändern, muss das Symbol vorher, durch erneutes klicken des Symbols vor ihm, ausgeloggt werden. Anschließend kann es erneut eingeloggt werden.
Werden alle Symbole vom roten und grünem Spieler erkannt und eingeloggt, die der schwarz markierte Spieler gestikuliert hat, wird dies durch eine grüne Kugel (\textit{Abbildung \ref{RoundFinished}}), für alle Sichtbar zwischen den einzelnen Podesten, erkennbar gemacht und die Runde endet. Erscheint diese grüne Kugel nicht, ist noch ein Symbol falsch eingeloggt und der schwarz markierte Spieler muss noch einmal versuchen, die korrekten Symbole den jeweiligen Mitspielern aufzuzeigen.
In der nächsten Runde wird ein anderer Spieler eineindeutig mit Schwarz, Rot oder Grün markiert.
Mit steigender Anzahl an erfolgreich bestandenen Runden müssen mehr Symbole erfolgreich erkannt werden.

Das Ziel ist es nun, so viele Symbole wie möglich gemeinsam korrekt zu bearbeiten um dadurch in höhere Runden aufzusteigen.
Für einen detaillierter grafischen Ablauf, siehe \textit{Abbildung \ref{DetaillierterVersuchsablauf}}.

	\begin{figure}[h]
		\begin{footnotesize}
		\centering
			\includegraphics[scale=0.30]{Abbildungen/DetaillierterVersuchsablauf.JPG}
			
			\caption[Abbildung 1]{Auf dieser Grafik ist der detaillierte Versuchsablauf von links oben nach rechts unten dargestellt}
			\label{DetaillierterVersuchsablauf}
		\end{footnotesize}
	\end{figure}

	\subsubsection{Die Avatare}
\label{IKNIK}
Die Avatare sind in dieser Studie als \dq{}IK-Avatar\dq{} und \dq{}Non-IK-Avatar\dq{} implementiert. Als Grundlage der visualisierung der Avatare dient eine Studie von Kristoffer Waldow \cite{waldow2019investigating}, in der über den Einfluss des Aussehens von Avataren in einem Augmented-Reality \ac{sve} geforscht wurde. Lediglich die Farbe der Avatare wurde von Grün auf Schwarz geändert.
Mehr zu Avataren befindet sich im Kapitel \ref{inverseKinematik}.
%In dieser Studie wurde für die beide Avatararten die Farbe Schwarz gewählt. Die teilnehmenden Personen hatten keinen Einfluss auf die Farbwahl der Avatare. Dies dient dazu die Störvariable der eventuell vorhandenen Vorurteile der teilnehmenden Personen aufgrund der Farbe des Avatars, auszuschalten.
Je nach Kondition \dq{}IK-Avatar\dq{} oder \dq{}Non-IK-Avatar\dq{} bekam jede teilnehmenden Personen dasselbe aussehen des Avatars zugeordnet. Zur besseren Interaktion mit den Knöpfen sowie zur Vermeidung eines \ac{bip}, wurde der eigene Avatar nicht sichtbar für die teilnehmenden Personen dargestellt. Jeder teilnehmende Person, unabhängig der zugewiesenen Kondition, sah somit nur eine Repräsentation von menschlich wirkenden Händen ohne einen Körper. Für die teilnehmenden Personen waren die unterschiedlichen Avatararten somit nur durch das Aussehen der \dq{}anderen\dq{} teilnehmenden Personen zu erkennen. Das Aussehen der Avatare befindet sich in \textit{Abbildung \ref{AvatareAussehen}}.

	\begin{figure}[H]
		\begin{footnotesize}
		\centering
			\includegraphics[scale=0.6]{Abbildungen/Avatars.JPG}
			
			\caption[Abbildung 1]{Links: IK-Avatar, Rechts: Non-IK-Avatar}
			\label{AvatareAussehen}
		\end{footnotesize}
	\end{figure}
		\paragraph{IK-Avatar}
Der Invers-Kinematisch dargestellte Avatar besitzt laut der Studie von Waldow \citep[p.251]{waldow2019investigating} ein neutrales Erscheinungsbild. Der Avatar besitzt keine Augen, Mund, Haare oder Beine. Lediglich der Oberkörper, der Kopf sowie die Arme sind zu erkennen. Dieser Avatar besitzt somit keine Beine und schwebt somit mit dem Torso über den Boden.
Um Nebeneffekte der Interaktion zu vermeiden, werden die Handbewegungen, die Unterarmbewegungen, die Oberarmbewegungen sowie die Kopf-und Torsorotation der Avatare des \ac{ik}-Avatar simuliert. Die Positions- und Rotationsdaten für den Kopf sowie des Oberkörpers werden über das \ac{hmd} gewonnen. Die Positions- und Rotationsdaten für die Arme werden über die beiden Controller gewonnen. Der Oberkörper des Avatars folgt nicht direkt der Rotation des Kopfes. Der Kopf des Avatars besitzt einen horizontalen Bewegungsbereich von 90$^\circ$. Wenn der Kopf diesen Bereich verlässt, rotiert der Oberkörper in Richtung der aktuellen Blickrichtung des Avatar. Zusammen mit menschlichen Motion-Capture-Daten einer Leerlaufanimation, wurde so eine plausible menschliche Bewegung erzeugt \citep[p.251]{waldow2019investigating}.

		\paragraph{Non-IK-Avatar}
Der \ac{nik}-Avatar besteht aus einem Kreis mit Mund sowie einer Repräsentation der linken sowie der rechten Hand. Er besitzt keine Augen, Haare, Beine, Hals, Torso sowie Ober- und Unterarme. Der Mund dieses Avatar ist als unauffälliger Orientierungspunkt des Kopfes erhalten geblieben. Durch den fehlenden Oberkörper und die fehlenden Beine, schwebt dieser Avatar über dem Boden. Der Mund des Non-IK-Avatars bewegt sich jedoch nicht. Der Kopf ist frei beweglich und unabhängig von den Händen. Die Positions- und Rotationsdaten für den Kopf werden über das \ac{hmd} gewonnen. Die Positions- und Rotationsdaten für die Hände werden über die beiden Controller gewonnen.
\newpage

Die \textit{Abbildung \ref{AvatareImEinsatz}} zeigt beide Avatar Konditionen \ac{ik} (a) und \ac{nik} (b) während der Versuchsdurchführung im Spectatorview.

\begin{figure}[h]
  \centering
  \subfloat[][]{\includegraphics[width=0.4\linewidth]{Abbildungen/Podeste_IK_Avatars.jpg}}
  \qquad
  \subfloat[][]{\includegraphics[width=0.4\linewidth]{Abbildungen/Podeste_Non_IK_Avatars.jpg}}
  \caption{Avatar Konditionen während der Versuchsdurchführung im Spectatorview}
  \label{AvatareImEinsatz}
\end{figure}

\begin{figure}[H]
		\begin{footnotesize}
		\centering
			\includegraphics[scale=0.2]{Abbildungen/RoundSuccsessful.JPG}
			
			\caption[Abbildung 1]{Grüne Kugel bei erfolgreich abgeschlossener Runde}
			\label{RoundFinished}
		\end{footnotesize}
	\end{figure}
	
\paragraph{Erwarteter Vertrauensaufbau}
Der zu erwartete Vertrauensaufbau des \dq{}Erklärende Teilnehmer\dq{} einer Runde, bezieht sich bei dieser Aufgabe darauf, dass er den \dq{}Erratenden Teilnehmern\dq{} vertraut, die Symbol Zuverlässig und ohne größere Probleme zu erraten.
Der zu erwartende Vertrauensaufbau der \dq{}Erratenden Teilnehmern\dq{} bezieht sich darauf, dass diese es dem \dq{}Erklärende Teilnehmer\dq{} Zutrauen, zuverlässig die Symbole zu erklären.

\paragraph{Erwartete Teameffektivität}
Je mehr kognitives Vertrauen sich durch die einzelnen Runden aufbaut, desto einfacher fällt es den teilnehmenden Personen, eine größere Anzahl an Runden abzuschließen. Andersrum, je weniger kognitives Vertrauen sich durch die einzelnen Runden aufbaut, desto schwieriger fällt es den teilnehmenden Personen, eine größere Anzahl an Runden abzuschließen.

\newpage
	\subsection{Anforderungen an die Versuchsumgebung}
\label{TechnischeAnforderungen}
Das System besitzt folgende \textbf{technische Anforderungen} die umgesetzt werden müssen :
\begin{itemize}
\item \textbf{Online-Fähigkeit und Unterstützung mehrerer HMD's}: Da die Teilnehmer auch von zu Hause aus an dem Versuch teilnehmen können sollen, muss die gesamte Anwendung mehrere HMD's unterstützen sowie auf sehr vielen verschiedenen Computermodellen zum Einsatz kommen können. Die Gesamte Anwendung muss Online-Fähig sein. Alle relevanten Statuszustände (Runde, Bewegungen und Positionen der Avatare, Rundenzeit, Status des jeweiligen Zustands des Podests eines Spielers etc.) müssen Synchronisiert sein.
\item \textbf{Geringe Latenz}: Das gesamte System muss Latenzfrei sein um die Bewegungen und Interaktionen der Avatare für die Nutzer nachvollziehbar zu machen.
\item \textbf{Steuerung und Verwaltung der Anwendung durch einen Spectator}: Damit der Versuchsleiter so wenig wie möglich in den Versuch eingreift, muss die Anwendung von Außen über einen eigenen Client gesteuert werden können.
\item \textbf{Interaktionsmöglichkeiten}: Es muss eine Übertragung der aktuellen Zustände der Knöpfe auf den Podesten zu allen Teilnehmern in dem \ac{sve} stattfinden.
\item \textbf{Sprachübertragung}: Es muss eine Sprachübertragung vom einem Client zu allen Nutzern und von einem Nutzer zum Client stattfinden können.
\item \textbf{Avatararten}: Es müssen die beiden Avatararten \ac{ik} sowie \ac{nik} im System vorhanden sein.
\end{itemize}

%Weiterhin besitzt das System noch \textbf{zusätzliche Anforderungen} :
%
%\begin{itemize}
%\item \textbf{Teamfähigkeit} Das Team muss eine Aufgabe gemeinsam lösen. Jedes Teammitglied muss zu gleichen Teilen involviert sein. Fehlt ein Teammitglied, soll die Aufgabe nicht lösbar sein.
%\item \textbf{Inkrementelle Erhöhung der Schwierigkeit} Um einen Vergleich zwischen der realen Teameffektivität und der unterschiedlichen Gruppen herzustellen, mussten die Runden inkrementell schwieriger werden. Dies wurde dadurch gelöst, dass die teilnehmenden Personen jede drei Runden ein Symbol mehr gestikulieren  oder erraten mussten.
%\item \textbf{Neutrale Avatare} Die Avatare sollen so neutral wie möglich gestaltet sein. 
%\item \textbf{Vermeidung von Bekanntheit} Um die gegenseitige Bekanntheit  teilnehmenden Personen auszuschließen, wurde jedem Teilnehmer zu beginn des Versuchs ein Zufälliger Name zugeordnet. Vor dem gesamten Versuch konnten sich die teilnehmenden Personen nicht sehen oder hören.
%Können die teilnehmenden Personen sich vor dem Versuch gesehen, würde sich ein Kognitives-Vertrauen oder Kognitives-Misstrauen zu der anderen Person aufbauen. 
%Kennen sich die Personen untereinander schon, wirkt sich dies auf die Vertrauensbildung in der \ac{vr} in Form von schon vorhandenem Affektiven-Vertrauen auf die Ergebnisse aus.
%\item \textbf{Keine verbale Kommunikation} Da die Stimme der teilnehmenden Personen ein Residuum ist, wurde die verbale Kommunikation von Versuchsteilnehmer zu Versuchsteilnehmer nicht gestattet. 
%\end{itemize}

\subsubsection{Technik der Versuchsumgebung}
Um den Versuch durchzuführen, wurde ein \ac{sve} entwickelt, in dem sich die 3 Teammitglieder gegenseitig sehen und miteinander interagieren können. 
Das \ac{sve} ist mit Unity 2019.4.3f1 und der HD-Render-Pipeline entwickelt worden. Um die Echtzeitkommunikation zwischen den einzelnen Clients zu gewährleisten, wurde das Multiplayer-Framework \dq{}Normcore v2.0\dq{} \footnote{www.Normcore.io} genutzt.
Normcore unterstützt Network-Physics simulationen, automatische Realtime-Synchronisation, Voice-Chat, XR-Compitabilität sowie persistente multiplayer Räume.	

\paragraph{Normcore Datastore / Sync Mechanismus}
Normcore besitzt das Konzept eines globalen Datenspeicher.  Alle Zustände, egal ob die Position eines Spielers, einzelne bool, integer oder float Variablen, werden in einem globalen Datenspeicher gespeichert. Werden Objekte in der Anwendung bewegt, synchronisiert Normcore diese Änderung der Position und/oder Rotation automatisch mit allen anderen Mitspielern.

\paragraph{Verwendete Hardware}
Um an dem Experiment Teilnehmen zu können, benötigten die teilnehmenden Personen ein in vollem Umfang funktionierendes HTC-Vive, HTC-Vive2, Windows-Mixed-Reality oder ein Oculus-Rift S \ac{hmd} mit funktionsfähigen Controllern sowie einen Leistungsstarken \ac{vr}-fähigen PC. Der Spectator, der das Experiment von außerhalb steuert und verwaltet, benötigt einen Leistungsstarken PC auf dem die Anwendung ohne \ac{hmd} ausführbar ist.

\paragraph{VR-Client}
Jeder Versuchsteilnehmer benötigt eine Client Version der Umgebung um an dem Versuch teilzunehmen. Die Client Version gibt es in drei verschiedenen Versionen. Diese verschiedene Versionen unterstützen die HTC-Vive \ac{hmd}'s, die Oculus Rift \ac{hmd}'s sowie die Windows Mixed-Reality \ac{hmd}'s. Beim Start des Clients gelangt der Probant in das \ac{sve} und verbindet sich automatisch mit dem Server (Normcore-Raum).
Der Client sieht seine beiden Hände, jedoch keine der beiden Avatar Konditionen \ac{ik} oder \ac{nik}, sondern menschenähnliche Hände.

\paragraph{Spectator}
Der Spectator besitzt ein eigenen Client, dieser eigene Client hat keine eigene Repräsentation des Körpers und benötigt daher auch kein \ac{hmd} um an der Session teilzunehmen. Dem Spectator ist es möglich sich frei mit der Kamera in der Umgebung zu bewegen.
Der Spectator besitzt die Möglichkeit das Spiel zu starten, die Nummer der zu beginnenden Runde einzustellen, einen Restart durchzuführen, die verbundenen Spieler anzuzeigen und herauszuwerfen, die Avatarkondition der Clients zwischen \ac{ik} und \ac{nik} zu wechseln sowie das Spiel zu verlassen.
Je nachdem, wie viele Spieler in dem entsprechendem Raum sind, werden diese dem Spectator angezeigt. So sieht der Spectator "Player 1" bis \dq{}Player 3\dq{} bei einem vollem Team auf seinem Bildschirm. Der Spectator besitzt außerdem die Möglichkeit Spieler herauszuwerfen, um die Session aufzuräumen und für die nächsten Teams vorzubereiten.
Weiterhin besitzt der Spectator noch einige Funktionen um sich Informationen über das aktuelle Spielgeschehen zu machen. So bekommt dieser die gesamte vergangene Zeit angezeigt, in welcher Runde sich die Mitspieler aktuell befinden, die aktuelle Effizienz des Teams, ob der \dq{}Grüne\dq{} oder \dq{}Rote\dq{} Spieler aktuell mit ihrer Eingabe korrekt sind und wie viele verschiedene Symbole der "Schwarze" Spieler aktuell erklären muss. Der Spectator-Client kann in \textit{Abbildung \ref{SpectatorView}}  gesehen werden.

\begin{figure}[H]
		\begin{footnotesize}
		\centering
			\includegraphics[scale=0.3]{Abbildungen/Versuchsumgebung/SpectatorView.JPG}
			
			\caption[Abbildung 1]{Der Spectator-Client mit den verschiedenen Funktionen zum verwalten des laufenden Versuchs}
			\label{SpectatorView}
		\end{footnotesize}
	\end{figure}
	
\paragraph{GameManager}
Der GameManager übernimmt alle wichtigen zu Funktionalitäten zur Steuerung in der Anwendung und der Netzwerkkommunikation.
Diese sind unter anderem :
\begin{itemize}[itemsep=0cm]
\item Steuerung und Entscheidungslogik der Runden
\item Netzwerkmanagement und RPC Handling
\item Start, Reset, Überprüfungsmechanismen sowie Zeitmessungen der einzelnen Runden
\item Interaktionen und Verwaltung von globalen Datastore-Variablen
\end{itemize}

Die Netzwerkkommunikation zwischen den einzelnen Clients sowie dem Spectator wird vom GameManager verwaltet. Der Spectator sowie der Client besitzen GameManager mit unterschiedlichen funktionalitäten.
Der GameManager des Clients übernimmt das für den Clienten relevante RPC-basierte Netzwerkhandling. Gleichzeitig lauscht dieser auf Änderungen von relevanten Variablen im globalen Datastore.
Eine wesentliche Aufgabe ist es zu überprüfen, ob eine Runde vom GameManager des Spectators als beendet markiert wurde und leitet gegebenenfalls die Initialisierung einer neuen Runde ein.

Der GameManager des Spectators gibt Hauptsächlich  Information über den Start, Stop oder den Reset der einzelnen Runden an die GameManager der Clienten weiter. Weiterhin steuert dieser aufgrund seiner besonderen Rolle ebenfalls die Spectator-to-Client Netzwerkinteraktionen wie die Audioübertragung, den Kickmechanismus, die Rundenauswahl sowie das Aussehen der Avatare der Spieler.
In \textit{Abbildung \ref{GameManagerClientSpectator}} wird ein grober Überblick über die Zusammenhänge und Funktionalitäten des GameManagers im \ac{sve} gegeben.

Die GameManager der Clients steht in Verbindung mit den einzelnen Podesten der Spieler. Der PodestManager eines jeweiligen Clients verwaltet einzelne Podeste, welche wiederum  einzelne Knöpfe eines Podests verwalten. 
Drückt ein Spieler ein Knopf mit einem Symbol ein- oder auszuloggen, sendet der Knopf eine Information über seinen Zustand an den Podestmanager des Clients. Dieser Überprüft wiederum, ob die zur Zeit gedrückten Knöpfe mit den zu drückenden Knöpfen der Runde übereinstimmen. Falls die gedrückten Knöpfe mit den zu drückenden Knöpfen dieser Runde übereinstimmen, wird eine boolean-Variable in einem globalen Datastore umgeschaltet, der den GameManager des Spectators darüber informiert, dass ein Teilnehmer die korrekten Knöpfe dieser Runde gedrückt hat. Haben der \dq{}Rote und Grüne\dq{} Spieler die korrekten Knöpfe gedrückt und dadurch eine Flag im globalen Datastore über die jeweilige Korrektheit gesetzt, registriert der GameManager des Spectators dies und setzt ein Flag über das Erfolgreiche abschließen der Runde. Die Clients registrieren dies und leiten die Startsequenz der nächsten Runde ein. 

\begin{figure}[H]
		\begin{footnotesize}
		\centering
			\includegraphics[width=\textwidth]{Abbildungen/GameManagerClientSpectator.jpg}
			
			\caption[Abbildung 1]{Grober Überblick über die Zusammenhänge der Funktionalitäten der GameManager im \ac{sve}}
			\label{GameManagerClientSpectator}
		\end{footnotesize}
	\end{figure}

\paragraph{Audioübertragung}
Die Audioübertragung wird von den GameManager des Clients sowie des Spectators verwaltet. Sobald der Teilnehmer mit dem mit dem Normcore-Netzwerk verbunden ist, wird das Mikrofon des \ac{hmd}'s aktiviert. Es werden unterschiedliche MessageID's für Spectator und Client für die Netzwerkpakete festgelegt. Die serialisierten Audiodaten werden mittels RCPMessage an alle verbundenen Teilnehmer gesendet. Der Spectator ließt nur die Netzwerknachrichten mit der MessageID der Clients und der Client ließt nur die Netzwerknachrichten mit der MessageID des Spectators.

\textit{Abbildung \ref{ClientRCPMessageReceived}} zeigt Beispielhaft den Programmcode des Spectator für das eingehen einer RCP-Nachricht. Wird ein RCP-Paket erkannt, wird dieses entpackt und die MessageID in den ersten 32bits der RCP-Message ausgelesen. Je nach MessageID, dient das RCP-Paket einem anderen Zweck. Es kann eine Audio-Message, Kick-Message  oder eine Message über den aktuellen schwarz makierten Spieler sein. Da der Spectator nur Audiostreams des Clients empfängt, muss dieser nur RCP-Messages mit einer MessageID von \dq{}2000\dq{} verarbeiten. 
Wird die RCP-Message entpackt und es befindet sich die MessageID für Audiostreams vom Client  in der Nachricht, wird dieser vom \dq{}NetworkAudioReceiver\dq{} des Spectator verarbeitet.  Falls kein \dq{}NetworkAudioReceiver\dq{} vorhanden, wird ein neuer \dq{}NetworkAudioReceiver\dq{} erstellt. Loggt sich ein neuer Client in das \ac{sve} ein und sendet einen Audiostream, wird für diesen Client ein neuer \dq{}NetworkAudioReceiver\dq{} erstellt.

\begin{figure}[H]
		\begin{footnotesize}
		\centering
			\includegraphics[width=\textwidth]{Abbildungen/ClientRCPMessageReceived.jpg}
			
			\caption[Abbildung 1]{Programmcode zum empfangen von Client RCP-Messages des Spectator}
			\label{ClientRCPMessageReceived}
		\end{footnotesize}
	\end{figure}
	
\paragraph{Normcore Variable-Sync}
Um die variablen im globalen Datenspeicher zu ändern, muss für jeden Datentyp eine Sync-Klasse geschrieben werden. Diese Sync-Klasse informiert den globalen Datenspeicher (in \textit{Abbildung \ref{boolValueChanged} \dq{}Model\dq{} genannt}) darüber, dass eine Variable geändert wurde. Der globale Datenspeicher informiert alle auf diese Variable zugreifenden Systeme darüber, dass eine Variable im globalen Datenspeicher geändert wurde und diese sich updaten sollen. Ein Auszug des Variable-Sync mechanismus für eine bool-Variable ist in \textit{Abbildung \ref{boolSync}} zu sehen.

\begin{figure}[H]
		\begin{footnotesize}
			\includegraphics[scale=.75]{Abbildungen/boolValueChanged.jpg}
			
			\caption[Abbildung 1]{Programmcode zum Update einer globalen Variable}
			\label{boolSync}
		\end{footnotesize}
	\end{figure}

\newpage	
%\paragraph{PlayerDisconnection Handling, Player Kick Mechanism, Destroying of Network Audio Receiver}	
	%\paragraph{Symbole und Podest, Drückmechanismus, Colliderplatzierung}
%Eingeloggt ist Gelb umrandet etc auch erklären, Ausloggen etc. Depth of button
\subsubsection{Optik der Versuchsumgebung}
\paragraph{Aussehen der Umgebung}
Die teilnehmenden Personen befinden sich in einem rechteckigen Raum mit 4 Fenstern vor denen einige Low-Poly Bäume stehen. Durch die 4 Fenster und den davor platzierten Bäumen sollte eine gewisse Offenheit erzeugt werden, damit die teilnehmenden Personen sich nicht in der \ac{vr} eingeschlossen fühlen. Weiterhin wurde der Raum in einer schlichten gräulichen Farbe gehalten um nicht all zu viel Ablenkung in der zu erzeugen. Die Spieler sollten sich so gut es geht nur auf den eigentlichen Versuch konzentrieren können. Die gesamte Umgebung ist möglichst Performant entwickelt worden, um diese auf möglichst vielen System einzusetzen. Siehe für die Versuchsumgebung von Außen \textit{Siehe Abbildung \ref{Versuchsumgebung}} 

\begin{figure}[H]
		\begin{footnotesize}
		\centering
			\includegraphics[scale=0.4]{Abbildungen/Versuchsumgebung/Raum.JPG}
			
			\caption[Abbildung 1]{Die Versuchsumgebung von Außen}
			\label{Versuchsumgebung}
		\end{footnotesize}
	\end{figure}

\paragraph{Avatar Konditionen}
Betritt ein Versuchsteilnehmer das \ac{sve}, bekommt dieser als Selbstrepräsentation seine eigenen, menschenähnlichen Hände zu sehen. Die Versuchsteilnehmer sehen nicht ihre eigene Repräsentation der Konditionen \ac{ik} oder \ac{nik}. Es wird nur die jeweilige Avatarkondition der Mitspieler gesehen.
Falls schon ein anderer Spieler im \ac{sve} anwesend ist oder zu einem späterem Zeitpunkt hinzukommt, ist dieser als die vom Spectator jeweilig ausgewählte Avatarkondition zu sehen. 

\paragraph{Fade-to-Black und Positionsmanagement}
	Da alle Mitspieler sich in dem \ac{sve} frei bewegen können, müssen diese beim Start des Spiels vor ein Podest platziert werden. Startet der Spectator das Spiel, überprüft ein im globalen Datenspeicher liegendes Byte, wobei die Bits des Bytes als Boolean genutzt wurden, welche Podeste noch nicht belegt sind. Alle Mitspieler werden dementsprechend auf die freien Positionen vor den Podesten teleportiert.
Damit es dem roten sowie dem grünen Spieler nicht einfach möglich ist, auf die markierten Symbole des schwarzen Spielers zu schauen, indem der jeweilige Spieler dort hin läuft, wurde ein Fade-To-Black Mechanismus implementiert. Entfernt sich ein Spieler in einem bestimmten Radius von seinem Podest, wird sein Bild ab diesem Zeitpunkt zunehmend schwärzer, je weiter der Spieler sich entfernt. Die Schwärze verschwindet, desto näher der Spieler sich dem maximal erlaubtem Radius wieder nähert.

%\paragraph{Physical Handmodel, eigener Avatar war unsichtbar}
\paragraph{Animationen, Shadergraph und VFX-Graph}
Die teilnehmenden Personen können in dem \ac{sve} lediglich virtuelle Hände ihrer eigenen Körperrepräsentation sehen. Da die Hauptinteraktion das Drücken der Knöpfe des zugeteilten Podestes darstellt, soll dies für die teilnehmenden Personen so einfach wie möglich gestaltet sein. Die teilnehmende Person soll sofort verstehen, wie diese mit seinem Avatar einen Knopf drückt. Die virtuellen Hände besitzen das Aussehen echter Hände mit einem grau-hellgrauem Karo-Muster Überzug. Drückt eine teilnehmende Person den Triggerbutton seines Controllers, schließen sich alle Finger zu einer Faust.

Drückt ein Spieler einen Knopf mit Symbolen, bekommt dieser durch eine gelbe Umrandung des Knopfes die Bestätigung, dass dieser den Knopf eingeloggt hat. Drückt dieser diesen erneut, verschwindet die gelbe Umrandung. Diese Umrandungen Leuchten im Ping-Pong-Effekt auf und wurden mit Unity Shader-Graph erstellt.

Ist eine Runde noch nicht abgeschlossen, sehen die Teilnehmer zwischen den Podesten einen leuchtenden Ball, welcher aus Unitys VFX-Graph Partikelsystem besteht. Wird eine Runde beendet, wechselt die Farbe des Partikelsystems von Rot auf Grün und es wirkt eine Kraft im Zentrum des Partikelsystems auf die einzelnen Partikel ein und schleudert diese auseinander.

\subsubsection{Allgemeines der Versuchsumgebung}

\paragraph{Rundenschwierigkeit}
Ein ScriptableObject beinhaltet die Regeln sowie den Ablauf der einzelnen Runden. 
Durch diese Sammlung von Regeln und Abläufen, bekommt jedes Team jeweils in der selben Reihenfolge die selben Zeichen angezeigt um ein Vergleich der einzelnen Teams zu ermöglichen. Es beinhaltet somit die allgemeinen "Rundeneinstellungen" des Spiels. Es gibt vor, in welcher Runde die Spieler wie viele Symbole zu erklären und zu erraten haben, welche Knöpfe mit Symbolen gedrückt werden müssen und wer der Schwarze, Rote und Grüne Spieler der jeweiligen Runde ist. Die Regeln wurden so definiert, dass der erklärende Spieler spätestens jede 3. Runde wieder mit dem erklären an der Reihe ist. Die anderen Runden ist dieser der erratende Spieler. Jede 3. Runde wird die Anzahl der zu erratenden und dadurch auch der zu erklärenden Symbole um 1. erhöht. Somit mussten in Runde 1-3 1 Symbol, 4-6 2 Symbole, 7-9 3 Symbole, 10-12 4 Symbole und Runde 13-15 5 Symbole erraten und erklärt werden.
Wer zu welcher Runde der Schwarze, Grüne oder Rote Spieler ist, und welche Symbole welche Runde gedrückt werden mussten, wurde einmalig Pseudozufällig festgelegt und in den "RoundRules" festgelegt.

\paragraph{CSV-Logwriting}
Der GameManager des Spectator sammelt alle relevanten Daten über die Clients, während die einzelnen Runden von den Teilnehmern durchgeführt werden. Es wird inkrementell ein CSV-Log pro Durchgang fortgeschrieben. Dieses Log umfasst unter anderem das Startdatum, die Startzeit, die Spielzeit, die Rundenzeit, die Rundennummer %einen Effizienzwert berechnet durch $1-\frac{timeSinceGameStarts}{OverallGameTime}$ ( welcher jedoch nicht verwendet wurde ),
sowie welche Avatarkondition in diesen Durchlauf verwendet wurde.

\paragraph{Gametimer Funktionsweise}
Die jeweiligen Teams haben 600 Sekunden Zeit, um so viele Symbole wie möglich, richtig zu erkennen. Um sich auf die startende Runde vorzubereiten, zählt ein Timer von 10 herunter. Anschließend beginnt die Runde. Dieser Spiele-Timer ist von jedem Mitspieler zwischen den drei Podesten gut sichtbar und immer zu den einzelnen Versuchsteilnehmern hingedreht. Die 10 Sekunden zwischen den einzelnen Runden, werden nicht Gesamtzeit hinzugezählt. Schaffen die Versuchsteilnehmer 10 Runden in 600 Sekunden zu absolvieren, dauert der Versuch somit insgesamt \\
$600 + 10 * 10 = 700$.

\paragraph{PlayerHeight/Position Change}
Die unterschiedlichen Körpergrößen und unterschiedlichen Zimmergrößen der Versuchsteilnehmer müssen berücksichtigt werden.
Falls eine Person zu klein oder zu groß für das Podest ist, kann dieser seine Spielerkamera inklusive Avatar mit den Tasten Q und E bis zu einem bestimmten Schwellenwert höher oder niedriger stellen. Durch diesen Mechanismus ist es auch sitzenden Personen möglich, an dem Versuch teilzunehmen.
Falls der Raum zu klein oder aus anderen Gründen im realem Raum nicht genügend Platz vorhanden ist, um die Knöpfe auf dem Podest zu betätigen, können sich die Spieler mit den Tasten W, A, S und D vorwärts, links, rechts und rückwärts in Blickrichtung bewegen.

\paragraph{Sounds}
Ist ein Teilnehmer den Raum beigetreten, wurde ein selbstaufgenommener Sound "Player Connected" abgespielt. Hat ein Disconnect eines Teilnehmers stattgefunden, wurde der Sound "Player Disconnected" abgespielt.
Um ein Feedback bei erfolgreichen Buttonklicks bereitzustellen, wurde ein "Klick" sound beim drücken eines Buttons abgespielt.
Alle Clients konnten mit dem Spectator sprechen. Dieser konnte alle Clients hören und zu allen auf einmal sprechen. Clients konnten keine Clients hören oder zu diesen Sprechen.

\newpage	
\subsection{Methodik}

%			\
%			\paragraph{Sprache} $~$ \\	
%			
%Allgemeine Kommunikation findet nicht nur mit Wörtern, sondern über Körpersprache wie Gestiken oder andere nonverbale Verhaltensmuster, statt. Durch \ac{sve}'s ist es nun möglich diese reine Text oder Bildbasierte Kommunikation durch Gestikulationen eines Avatars zu ersetzen. Durch diesen zusätzlichen visuellen Aspekt ist eine ganz neue Art und Weise der Kommunikation möglich, die nun gemessen werden kann. 
%Nichtverbale Kommunikation -Beispielsweise nur via Text- erzeugt keinen Mehrwert an Vertrauen, Zusammenhalt und erzielt keine ausreichend gute Kommunikation. \citep[p.81]{haslam2003social}			
%
%Nonverbale Kommunikation kann Gesichtsausdrücke, Blicke, Bewegungen etc. umfassen. Kendon definiert im \dq{}Kendon Kontinuum\dq{} mit dem Begriff \dq{}Gestik\dq{} unwissende Gestiken (natürliche Körpersprache) bis hin zu \dq{} Zeichen\flqq, welche alle durch Gestikulation erzeugten Zeichen ( z.B. O.K. Handzeichen ), beinhaltet. \citep[37]{mcneill1992hand} 
%Im Rahmen dieser Arbeit, ist mit Gestikulation jedoch die gesamte Körperbewegung des Avatars gemeint.
%
%Mehrabian zeigt in seiner Forschung, dass nonverbale Kommunikation zu 55\% dafür verantwortlich ist, ob ein gesprochenes Wort Positiv, Negativ oder Neutral interpretiert wird. Die Tonhöhe trägt zu 38\% und das eigentlich gesprochene Wort mit 7\% zur Interpretation bei. \citep[43]{mehrabian1971silent}
%
%Da 7\% des gesprochenen Wortes und 38\% der Tonhöhe zu Interpretation der nonverbalen Kommunikation beiträgt, musste diese Einflussgröße in dieser Studie ausgeschlossen werden.
%
%			Dodds fand heraus, dass ein Selbst-Avatar eine wichtiger Faktor zur Kommunikation in einem \ac{sve} ist. \citep[1-11]{dodds2011talk}
%			
%			Black Avatars :
%			https://www.sciencedirect.com/science/article/abs/pii/S1053810013000597
%
%%An Investigation into Gender Role Conformity in an Online Social
%%Networking Environment.......................................... 322
%%
%%Gesture and facial expression
%%Gesture is an important part of conversation and ranges from almost sub-conscious accompaniment to speech to complete and well formed sign languages for the deaf. Support for gesture implies that we need to consider what kinds of 'limbs' are present. Facial expression also plays a key role in human interaction as the most powerful external representation of emotion, either conscious or sub-conscious. Facial expression seems strongly related to gesture. However, the granularity of detail involved is much finer and the technical problems inherent in its capture and representation correspondingly more difficult. A crude, but possibly effective approach, might be to texture map video onto an appropriate facial surface of a body image (e.g. the "Talking Heads" at the Media Lab [2]). Another approach involves capturing expression information from the human face using an array of sensors on the skin, modelling it and reproducing it on the body image (e.g. the work of ATR where they explicitly track the movement of a user's face and combine it with models of facial muscles and skin [6] and also the work of Thalmann [10] and Quéau [7]).
%%This discussion of gesture and facial expression relates to a further issue, that of voluntary versus involuntary expression. Real bodies provide us with the ability to consciously express ourselves as a supplement or alternative to other forms of communication. Virtual bodies can support this by providing an appropriate set of limbs and 'strings' with which to manipulate them. The more flexible the limbs; the richer the gestural language. However, we suspect that users may find ways of gesturing with even very simple limbs. On the other hand, involuntary expression (i.e. that over which users have little control) is also important (looks of shock, anger, fear etc.). However, support for this is technically much harder as it requires automatic capture of sufficiently rich data about the user. This is the real problem we are up against with the facial expression issue - how to capture involuntary expressions.
%
%			
%	

\subsubsection{Allgemeines}
			\paragraph{Induktive Quantitative Forschungsmethodik}
				Anhand der subjektiven Betrachtungsweise einzelner Personen des Themas Vertrauen, wurde ein quantitatives Forschungsdesign gewählt anhand die Ergebnisse induktiv interpretiert und ausgewertet werden.
			\paragraph{Between-Subject Forschungsdesign}
Wie groß das gebildete Vertrauen und die dadurch resultierende reale Teameffektivität ist, soll durch die beiden Avatarkonditionen \ac{ik} sowie \ac{nik} analysiert werden. Da der kognitive Vertrauensaufbau und die Teameffektivität anhand eines inkrementell Schwieriger werdenden Spiels gemessen wird, können die Versuchsteilnehmer nicht mit beiden Avatarkonditionen an dem Versuch teilnehmen. Die teilnehmenden Personen würden bei dem zweiten Durchlauf schon wissen, wie das Spiel funktioniert und die Messwerte der kognitive Vertrauensaufbau sowie die reale Teameffektivität beeinflussen. Daher wurde sich für ein Betwee-Subject Forschungsdesign entschieden, bei dem jeder Versuchsteilnehmer nur einmal entweder mit der Kondition \ac{ik} oder \ac{nik} am Versuch teilnimmt.
			\paragraph{Unabhängige Variablen}
%Die unabhängigen Variablen sind der \dq{}\ac{ik}-Avatar\dq{} sowie der \dq{}\ac{ik}-Avatar\dq{}. Durch diese beiden Variablen wurde im weiterem Versuchsverlauf versucht den Einfluss auf die abhängigen Variablen zu bestimmen.
Für die \textbf{Hypothese 1} wurde als unabhängie Variable das \textit{generelle Vertrauen} und als abhängige Variable das \textit{kognitive Vertrauen gewählt.}
%Für die \textbf{Hypothese 2} wurde als unabhängie Variable die \textit{Avatarkondition} und als abhängige Variable das \textit{kognitive Vertrauen gewählt.}
%Für die \textbf{Hypothese 3} wurde als unabhängie Variable das \textit{kognitive Vertrauen im Team} und als abhängige Variable die \textit{reale Teameffektivität} gewählt.
Für die \textbf{Hypothese 4} wurde als unabhängie Variable die \textit{Avatarkondition} und als abhängige Variable die \textit{reale Teameffektivität} gewählt.
%Für die \textbf{Hypothese 5} wurde als unabhängie Variable das \textit{generelle Vertrauen} und als abhängige Variable die \textit{reale Teameffektivität} gewählt.

Die abhängigen Variablen werden wie folgt bezeichnet :
\begin{itemize}
\item \textbf{Hang zum Vertrauen} Der Hang zum Vertrauen bezieht sich in dieser Studie darauf, wie sehr die teilnehmenden Personen  anderen Personen einen Vertrauensvorschuss gewähren. Der Hang zum Vertrauen wird auch als \textit{generelles Vertrauen} \ac{gt} bezeichnet\citep[p.30]{mcallister1995affect} (\textit{vgl. Kapitel \nameref{Vertrauen}}).
\item \textbf{Kognitives Vertrauen} Das kognitive Vertrauen \ac{ct} bezieht sich auf die \textit{Überzeugung in die Fähigkeiten oder in die Zuverlässigkeit eines anderen}. \citep[p.30]{mcallister1995affect} (\textit{vgl. Kapitel \nameref{Vertrauen}}).
\item \textbf{Reale Team-Effektivität} Die Teameffektivität \ac{tRound} wird anhand der Anzahl der abgeschlossenen Runden im Team gemessen (\textit{vgl. Kapitel \nameref{Avatare}}).
\end{itemize}

Ein Unterschied der subjektiven Empfindungsdaten zwischen den folgenden abhängigen Variablen und der unabhängigen Variable der Avatarkondition analysiert : Wahrgenommene Teameffektivität (\ac{te}), Presence (\ac{ipq}), Subjektiv wahrgenommene Arbeitsbelastung (\ac{tlx}), Team-Kommunikation (\ac{tc}) sowie Co-Presence( \ac{cp})
				
%	\paragraph{Residuuen}
%\begin{itemize}
%\item \textbf{Kommunikation mit dem Spectator} Da die teilnehmenden Personen mit dem Spectator in Verbindung stehen, kann es sein, dass die teilnehmenden Personen durch eine plötzlich auftretende Stimme einen \ac{bip} erfahren.
%\item \textbf{Farbe und Geschlecht der Avatare} Personen verbinden mit verschiedenen Farben verschiedene Eigenschaften. Es wurde durch die Farbe schwarz so neutral wie möglich zu bleiben.
%\item \textbf{Grad der Gamification} Einige Spieler könnten das gesamte Experiment \ac{vr} als Spiel zur Unterhaltung empfinden, was das Ergebnis beeinflussen könnte. 
%\end{itemize}				

			\subsubsection{Datenerhebungsmethoden}
Es wurden zwei Fragenbögen an die teilnehmenden Personen verteilt. Der erste Fragenbogen wurde an die teilnehmenden Personen vor der eigentlichen Studie ausgefüllt. In diesem wurde die Datenschutzerklärung ausgefüllt sowie Fragen über die \dq{}Person\dq{}, über eventuelle \dq{}Gesundheitliche Beschwerden\dq{} sowie schon vorhandene \dq{}\ac{vr}-Erfahrung\dq{}, gestellt. Der Zweite Fragebogen wurde nach der Erfahrung im \ac{sve} ausgefüllt. In diesem wurden Fragen über das \dq{}generelles Vertrauen\dq{}, das \dq{}kognitive Vertrauen\dq{} die \dq{}Kommunikations-Qualität\dq{}, die wahrgenommene \dq{}Team-Effektivität\dq{}, die \dq{}Beanspruchung\dq{}, die \dq{}Presence\dq{}, sowie der \dq{}Co-Presence\dq{} gestellt.
Es wurden nur vollständig Ausgefüllte Fragebögen zur Datenanalyse herangezogen.\\
Die Fragen der Likert-Scales wurden als intervallskaliert angenommen, da nur so das subjektive empfinden der einzelnen Testpersonen hinreichend analysiert werden kann.
				
		\paragraph{Demografie-Fragebogen}
Bevor der Versuch stattfinden konnte, mussten die teilnehmenden Personen einen Demographie-Fragebogen ausfüllen. Dieser diente allgemeine demografische Merkmale wie z.B. Das Alter, Geschlecht und den Bildungsstand abzufragen. Weiterhin wurde der demografische Fragebogen dazu genutzt, um die bisher vorhandene \ac{vr}-Erfahrung, die PC- und Internetaffinität sowie die Videospielerfahrung der einzelnen Probanten besser einschätzen zu können.
Ein Auszug des Demografie-Fragebogen befindet sich in \textbf{Anhang \ref{Pre-Questionnaire}}

		\paragraph{General-Trust-Scale}
Der General-Trust-Scale entwickelt von Couch, Adams und Jones \citep{couch1996assessment} wurde eingesetzt um den  \textit{generellen Vertrauenswert} der einzelnen teilnehmenden Personen zu messen.
Dieser Fragebogen besteht aus 20 Fragen. Diese sind Beispielsweise \dq{}Ich neige dazu, andere zu akzeptieren \dq{} oder\dq{}Meine Beziehungen zu anderen werden durch Vertrauen und Akzeptanz charakterisiert \dq{} und wurden auf einer 7-Point-Likert-Scale von \dq{}1: Ich stimme gar nicht zu\dq{} bis zu \dq{}7 : Ich stimme voll zu\dq{} gemessen. Der General-Trust-Scale ist nur ein Teilauszug des \dq{}Trust-Inventory von Couch\dq{}, welches noch einen \dq{}Partner-Trust-Scale \dq{} beinhaltet. Dieser wurde für diese Forschung jedoch nicht benötigt. 
Ein Auszug des General-Trust-Scale befindet sich in \textbf{Anhang \ref{Post-Questionnaire - Generelles Vertrauen}}

%			
%			\textbf{The propensity to trust measure developed by Couch, Adams, & Jones (1996) was
%administered in order to gauge participants’ trusting dispositions. The Generalized Trust subscale
%of this measure was utilized for this study. It contains 20 items (e.g. “I have few difficulties
%trusting people”) and participants rated how strongly they agree with each statement on a 1
%(strongly disagree) to 7 (strongly agree) scale. See Appendix B for the full scale. The full trust
%inventory by Couch et al. (1996) also contains a Partner Trust subscale, but this was not relevant
%to the study as it pertains to trust of one’s romantic partner.}
%https://sci-hub.ren/10.1207/s15327752jpa6702_7
%\citep{couch1996assessment}

		\paragraph{Cognitive-Trust-Scale}
Der Cognitive-Trust-Scale ist ein Teilauszug des von McAllister und Daniel J \citep[p.37]{mcallister1995affect} entwickeltem Fragebogen. Der Cognitive-Trust-Scale dient dazu, herauszufinden, wie viel kognitives Vertrauen die teilnehmenden Personen während des Versuchs aufbauen. Die Messgröße lässt sich als \dq{}\textit{kognitiver Vertrauenswert}\dq{} bezeichnen. Dieser Teilauszug des Fragebogens umfasst 5 Fragen. Diese sind Beispielsweise \dq{}Diese Person geht an ihre Arbeit mit Professionalität und Hingabe heran\dq{} oder \dq{}Andere Personen, die mit diesen Personen interagieren müssen, halten ihn/sie für vertrauenswürdig und wurden mittels eines 5-Point-Likert-Scale gemessen. Die Antwortmöglichkeiten des Likert-Scale erstrecken sich von \dq{}1: Ich stimme gar nicht zu \dq{} bis zu\dq{}5 : Ich stimme voll zu. 
Ein Auszug des Cognitive-Trust-Scale befindet sich in \textbf{Anhang \ref{Post-Questionnaire - Kognitives Vertrauen}}

  

%			\textbf{After each bomb was completed (successfully or not), participants filled out a cognitive trust scale developed by Wildman et al. (2009) and based on the trust theory of Lewicki, McAllister, & Bies (1998). This 8-item scale taps into participants’ trust attitudes and each item is rated on a 5-point scale from “not at all” to “very much so”. While this measure has not yet been published, it has been validated in both lab and field samples and has shown utility in prior teamwork research Lazzara, 2013; Wildman, 2011. Appendix C contains the full scale}
%\citep{mcallister1995affect}

		\paragraph{Communication-Scale}
Gonzales-Rom, Vincente und Hernandez \citep[p.1049]{gonzalez2014climate} entwickelten 2004 einen Fragebogen um die  \textit{Qualität von Teamkommunikation} zu messen. Dieser Fragebogen umfasst 5 Fragen und wurden durch einen 5-Point-Likert-Scale von \dq{}1:Gar nicht\dq{} bis \dq{}5:Sehr\dq{}. Die Fragen sind alle nach dem selben Prinzip aufgebaut und es wurde jeweils nur das Ende einer Frage abgeändert : \dq{}In welchem Umfang war die Kommunikation zwischen Ihnen und Ihrem Team KLAR/EFFEKTIV/ABGESCHLOSSEN/FLÜSSIG/ZUM RICHTIGEN ZEITPUNKT/?\dq{} 
Ein Auszug des Communication-Scale befindet sich in \textbf{Anhang \ref{Post-Questionnaire - Teamkommunikation}}
%			After the experimental portion was completed, participants completed another set of measures. First, participants completed the Communication Quality Scale developed by Gonzalez-Roma & Hernandez 2014. This scale contains 5 items that assess participants perceptions of their teams communication quality, rated on a 1 to 5 scale. The full measure is available in Appendix D
%\citep[p.1049]{gonzalez2014climate}

		\paragraph{Team-Effectivness-Scale}
Gibson \citep[p.469]{gibson2003team} entwickelte 2003 einen Fragebogen um die \textit{wahrgenommene Teameffektivität} zu messen. In dieser Studie wurde ein Teilauszug dieses Fragebogens verwendet um das \textit{subjektive Außmaß der wahrgenommenen Teameffektivität} zu messen. Dieser Teilauszug umfasst 5 Fragen die mithilfe einer 7-Point-Likert-Scale gemessen werden \dq{}1:Ich stimme gar nicht zu \dq{} bis \dq{}7: Ich stimme voll zu \dq{}. Die Fragen Zielen dabei auf die Qualität der der Teameffektivität ab, wie z.B. \dq{}Mein Team hat eine geringe Fehlerquote \dq{} oder \dq{}Mein Team hat eine hohe Qualität \dq{}. 
Ein Auszug des Team-Effectivness-Scale befindet sich in \textbf{Anhang \ref{Post-Questionnaire - Team-Effektivität}}

%quality dimension is the extent to which the team produces error-free work.
%			\textbf{Next, participants completed a 5-item team effectiveness measure. This measure is
%comprised of the Quality subscale from Gibson, Zellmer-Bruhn, and Schwab's (2003) Team
%Outcome Effectiveness survey. Participants rated their agreement with each of the five items on
%a 1 to 7 scale. The team effectiveness measure is available in Appendix E.
%Last, participants indicated whether or not they were already familiar with their
%teammate. If so, they were also asked to indicate approximately how many years they had known
%their teammate, and how often they communicated with the teammate. Finally, teams were
%prompted to write a few sentences about why they thought their team performed the way it did.
%Effectiveness Measures} \citep[p.469]{gibson2003team}

		\paragraph{NASA-TLX}
Der NASA-TLX ist ein multidimensionaler Fragebogen. Dieser Fragebogen findet einen direkten Einsatz nachdem Probanten eines Experimentes eine Aufgabe erledigt haben. Der NASA-TLX dient dazu, herauszufinden, wie allgemeine Belastung auf den Probanten herauszufinden.
Es insgesamt 6 Fragen auf einer 21-Punkte Skala abgefragt. Diese beinhalten die Mentale Anforderung, die Physische Anforderung, die Zeitliche Anforderung, die Leistung, die Anstrengung, und die Frustration des Probanten \cite{NASATLX}.
Der originale NASA-TLX besitzt eine kontinuierliche Skala. Dies war jedoch nicht über einen Online-Fragebogen realisierbar, weshalb die Werteskala abgeändert wurde. Diese Werteskala umfasst im abgeänderten Fragebogen \dq{}1: Wenig \dq{} über \dq{}11: Mittel \dq{} bis \dq{}21: Viel \dq{}, wobei die Skalenbeschriftung je nach abgefragtem Item anders beschriftet ist.
Ein Auszug des abgeänderten NASA-TLX befindet sich in \textbf{Anhang \ref{Post-Questionnaire - NASA-TLX}}


		\paragraph{Igroup Presence Questionnaire(IPQ)}
Der IPQ dient zu  \textit{Messung des Presence-Gefühls} in einer virtuellen Umgebung. Dabei misst der IPQ, inwieweit sich der Nutzer in der virtuellen Umgebung anwesend fühlt, inwieweit der Nutzer seine Aufmerksamkeit der virtuellen Umgebung schenkt und wie Real die virtuelle Umgebung dem Nutzer erschien. Der IPQ umfasst 14 Fragen welche mithilfe einer 7-Point-Likert-Scale gemessen werden.
\\Ein Auszug des Igroup Presence Questionnaire(IPQ) befindet sich in \textbf{Anhang \ref{Post-Questionnaire - IPQ}}
		
		\paragraph{Co-Presence-Questionnaire}
Der Co-Presence-Fragebogen dient dazu die  \textit{Selbst gemeldete Co-Presence, die wahrgenommene Presence des \dq{}anderen \dq{}, die Telepresence sowie die Social-Presence} zu messen. Die Selbst gemeldete Co-Presence sowie die wahrgenommene Presence des \dq{}anderen \dq{} wurde mittels einer 5-Point-Likert-Skala gemessen. Die Telepresence wurde mittels einer 7-Point-Likert-Scale gemessen. Die Social-Presence wird im originalen Fragebogen von Nowak und Biocca mittels einer kontinuierlichen Skala gemessen. Da dies jedoch mit dem Online-Fragebogen nicht realisierbar war, wurde stattdessen eine Likert-Skala von 1 bis 10 verwendet \citep[p.487]{nowak2004effect}.
Ein Auszug des Co-Presence-Questionnaire befindet sich in \textbf{Anhang \ref{Post-Questionnaire - Co-Presence}}

%\newpage
%
%\begin{table}
% \centering\footnotesize\setstretch{1}\linespread{1.5}\rmfamily
%\caption{Verwendete Metriken des Fragbogens}
%\begin{tabular}{ |p{3.5cm}|p{3.5cm}|p{3.5cm}|p{3.5cm}| }
%
%\hline
%\underline{Was wurde gemessen?} & \underline{Definition} & \underline{Metrik} & \underline{Authoren}\\
%
%\hline
%\textbf{Generelles Vertrauen} & Genereller Vertrauensvorschuss eines Individuums & Teilauszug des Trust-Inventorys - \textit{Generalized-Trust-Scale}\newline 7-Point-Likert-Scale & Couch, Adams, Jones (1996) \citep{couch1996assessment} \\
%
%\hline
%\textbf{Kognitives-Vertrauen} & Überzeugung in die Fähigkeiten oder in die Zuverlässigkeit eines anderen Individuums & \textit{Cognitive-Trust-Scale}\newline 5-Point-Likert-Scale & McAllister, Daniel J. (1995) \citep{mcallister1995affect} \\
%\hline
%
%\hline
%\textbf{Team-Kommunikation} & Wahrgenommene Kommunikationsqualität des Teams & \textit{Communication-Quality-Scale}\newline 5-Point-Likert-Scale & Gonzalez-Roma, Hernandez (2014) \citep[p.1049]{gonzalez2014climate} \\
%\hline
%
%\hline
%\textbf{Team-Effektivität} & Wahrgenommenes ausmaß der Qualität der Aufgabenerledigung des Teams & Teilauszug des \textit{Team-Effectiveness-Scale}\newline 7-Point-Likert-Scale & Gibson et. al (2003) \citep[p.469]{gibson2003team} \\
%\hline
%
%\hline
%\textbf{Subjektiv-Wahrgenommene Arbeitsbelastung} & Der NASA-TLX misst die subjektiv wahrgenommene Arbeitsbelastung sowie Effektivität & \textit{NASA-TLX} & \cite{NASATLX} \\
%\hline
%
%\hline
%\textbf{Räumliche Presence \newline \newline
%	Umschlossenheitsgefühl der \ac{vr} \newline \newline
%	Erfahrener Realismus 
%} & Der IPQ dient zu Messung des Presencegefühls in einer virtuellen Umgebung & \textit{Igroup Presence Questionnaire(IPQ)} &  \cite{IPQ}\\
%\hline
%
%\hline
%\textbf{Co-Presence \newline \newline
%	die wahrgenommene Präsenz des \dq{}anderen \dq{} \newline \newline
%	Telepresence \newline \newline
%	Social-Presence
%	} & Fragebogen von Nowak und Biocca zum messen von Co-Präsenz & \textit{Presence-Questionnaire} & Nowak und Biocca (2004) \citep[p.487]{nowak2004effect}  \\
%\hline
%
%\end{tabular}
%\end{table}

\clearpage
\newpage
\begin{table}
		\centering\footnotesize\setstretch{1.2}
		\caption{Verwendete Metriken des Fragbogens}
		\label{wasWurdeGemessen}
	\begin{tabular}{p{3.5cm} p{3.5cm} p{3.5cm} p{3.5cm}}

\underline{Was wurde gemessen?} & \underline{Definition} & \underline{Metrik} & \underline{Authoren}\\
    
    \hline
\multirow{2}{3.5cm}{Generelles Vertrauen}
	&Genereller Vertrauensvorschuss eines Individuums & Teilauszug des Trust-Inventorys - \textit{Generalized-Trust-Scale} \newline \textit{Siehe Tabelle \ref{Post-Questionnaire - Generelles Vertrauen}} \newline &Couch, Adams, Jones (1996) \citep{couch1996assessment}\\
	&\,& 7-Point-Likert-Scale & \, \\
    
    \hline
\multirow{2}{3.5cm}{Kognitives-Vertrauen}
	&Überzeugung in die Fähigkeiten und in die Zuverlässigkeit des Teams& \textit{Cognitive-Trust-Scale} \newline \textit{Siehe Tabelle \ref{Post-Questionnaire - Kognitives Vertrauen}} & McAllister, Daniel J. (1995) \citep{mcallister1995affect}\\
	&\,& 5-Point-Likert-Scale & \, \\
    
    \hline
\multirow{2}{3.5cm}{Team-Kommunikation}
	&Wahrgenommene Kommunikationsqualität des Teams & \textit{Communication-Quality-Scale} \newline \textit{Siehe Tabelle \ref{Post-Questionnaire - Teamkommunikation}} & Gonzalez-Roma, Hernandez (2014) \citep[p.1049]{gonzalez2014climate}\\
	&\,& 5-Point-Likert-Scale & \, \\
    
    \hline
\multirow{2}{3.5cm}{Wahrgenommene Team-Effektivität}
	&Wahrgenommenes ausmaß der Qualität der Aufgabenerledigung des Teams & Teilauszug des \textit{Team-Effectiveness-Scale} \newline \textit{Siehe Tabelle \ref{Post-Questionnaire - Team-Effektivität}} & Gibson et. al (2003) \citep[p.469]{gibson2003team}\\
	&\,& 7-Point-Likert-Scale & \, \\
    
    \hline
\multirow{2}{3.5cm}{Abgeschlossene Runden des Teams} & Messung der Abgeschlossenen Runden des Teams während des Versuchs & Erhoben durch die Anwendung &  \\    
    
    \hline
\multirow{2}{3.5cm}{Subjektiv-Wahrgenommene Arbeitsbelastung}
	&Der NASA-TLX misst die subjektiv wahrgenommene Arbeitsbelastung sowie Effektivität & \textit{NASA-TLX} \newline \textit{Siehe Tabelle \ref{Post-Questionnaire - NASA-TLX}} & \cite{NASATLX}\\
    
    \hline
\multirow{2}{3.5cm}{Räumliche Presence \newline Umschlossenheitsgefühl \newline Erfahrener Realismus}
	&Der IPQ dient zu Messung des Presence Gefühls in einer virtuellen Umgebung & \textit{Igroup Presence Questionnaire(IPQ)} \newline \textit{Siehe Tabelle \ref{Post-Questionnaire - IPQ}} & \cite{IPQ}\\
    
    \hline
\multirow{2}{3.5cm}{Selbst wahrgenommene Co-Presence \newline \newline Wahrgenommene Presence des \dq{}anderen\dq{} \newline \newline Telepresence \newline \newline Social-Presence}
	&Fragebogen von Nowak und Biocca zum messen von Co-Presence & \textit{Presence-Questionnaire} \newline \textit{Siehe Tabelle \ref{Post-Questionnaire - Co-Presence}} & Nowak und Biocca (2004) \citep[p.487]{nowak2004effect}\\
	
	\end{tabular}
\end{table}
\clearpage			

%			\paragraph{Bereinigte Fragen}
%Um die interne Konsistenz der einzelnen Skalen zu verbessern, wurden einige Fragen aus dem Fragebogen bereinigt. Dabei wurde versucht, das Cronbachs-Alpha $\alpha$ optimalerweise auf ($\alpha > 0.8$), akzeptablerweise ($\alpha > 0.7$) und fragwürdigerweise auf ($\alpha > 0.6$) zu setzen. Dabei wurde während der Bereinigung darauf geachtet, dass nicht zu viele Fragen entfernt wurden und trotzdem eine interne Konsistenz aufrecht gehalten wurde. Eine Tabelle über die Anzahl der bereinigten Fragen im Fragebogen, die Anzahl der Teilnehmer sowie die Cronbachs-Alpha $\alpha$  befindet sich in \textbf{Tabelle \ref{alphavalues}}.\newline
%Die folgenden Fragen wurden aus den Fragebögen bereinigt : \\
%\textbf{\textit{Generalized-Trust-Scale}}
%\begin{itemize}
%	\item Ich habe etwas Schwierigkeiten, Leuten zu vertrauen
%	\item Meine Erfahrungen haben mir zeigen mir, dass es besser ist, anderen zu misstrauen, bis man diese besser kennt
%	\item Nur ein Narr würde den meisten Personen vertrauen
%	\item Es ist besser, Fremden zu misstrauen, bis man sie besser kennt
%	\item Ich neige dazu, andere beim Wort zu nehmen
%	\item Ich habe kein Vertrauen in andere Personen.
%	\item Selbst in schlechten Zeiten denke ich, dass am Ende alles gut wird
%\end{itemize}
%
%\textbf{\textit{Cognitive-Trust-Scale}}
%\begin{itemize}
%	\item Wenn die Menschen mehr über diese Personen und ihren Hintergrund wüssten, würden sie sich mehr Sorgen machen und ihre Leistung genauer beobachten
%\end{itemize}
%
%\textbf{\textit{Wahrgenommene Team-Effektivität}}
%\begin{itemize}
%	\item Mein Team muss ihre Arbeitsqualität verbessern
%\end{itemize}
%
%\textbf{\textit{NASA-TLX}}
%\begin{itemize}
%	\item Wie erfolgreich haben Sie Ihrer Meinung nach die vom Versuchsleiter (oder Ihnen selbst) gesetzten Ziele erreicht? Wie zufrieden waren Sie mit Ihrer Leistung bei der Verfolgung dieser Ziele? 
%\end{itemize}
%
%\textbf{\textit{Igroup Presence Questionnaire(IPQ)}}
%\begin{itemize}
%	\item Ich achtete noch auf die reale Umgebung
%	\item Wie real erschien Ihnen die virtuelle Welt?
%	\item Ich hatte das Gefühl, nur Bilder zu sehen. 
%\end{itemize}
%
%\textbf{\textit{Co-Presence}}
%\begin{itemize}
%	\item Ich wollte ungern persönliche Informationen mit meinen Interaktionspartnern teilen
%	\item Ich wollte eine gewisse Distanz zwischen mir und den Interaktionspartnern wahren
%	\item Ich wollte keine engere Beziehung mit meinen Interaktionspartnern haben
%	\item Wie involvierend war das Ergebnis?
%	\item Mein Interaktionspartner schuf eine gewisse Distanz zwischen uns
%	\item Meine Interaktionspartner kommunizierten eher \dq{}kalt\dq{} als \dq{}warm \dq{}
%\end{itemize}
%
%\newpage
%\begin{table}
%	\centering\footnotesize\setstretch{0.8}
%	\caption{Alphawerte und Anzahl der Fragen nach der Bereinigung des Fragebogens}
%	\label{alphavalues}
%\begin{tabular}{ p{3.5cm} p{3.5cm} p{3.5cm} }
%
%\underline{Fragebogen-Skala} &  & \underline{Werte}\\
%    
%    \hline
%\multirow{3}{3.5cm}{Generalized-Trust-Scale}
%	&Anzahl der Fragen im original Fragebogen (Bereinigter Fragebogen) \newline & 20 (13)\\
%    &\textit{n} \newline &30\\
%    &$\alpha$ (bereinigtes $\alpha$) & .122 (.806)\\
%    
%    \hline
%\multirow{3}{3.5cm}{Cognitive-Trust-Scale}
%	&Anzahl der Fragen im original Fragebogen (Bereinigter Fragebogen) \newline & 6 (5)\\
%    &\textit{n} \newline &30\\
%    &$\alpha$ (bereinigtes $\alpha$) & .309 (.721)\\
%    
%    \hline
%\multirow{3}{3.5cm}{Team-Kommunikation}
%	&Anzahl der Fragen im original Fragebogen (Bereinigter Fragebogen) \newline & 5 (5)\\
%    &\textit{n} \newline &30\\
%    &$\alpha$ (bereinigtes $\alpha$) & .823 (.823)\\
%    
%    \hline
%\multirow{3}{3.5cm}{Team-Effektivität}
%	&Anzahl der Fragen im original Fragebogen (Bereinigter Fragebogen) \newline & 5 (4)\\
%    &\textit{n} \newline &30\\
%    &$\alpha$ (bereinigtes $\alpha$) & .501 (.886)\\
%    
%    \hline
%\multirow{3}{3.5cm}{NASA-TLX}
%	&Anzahl der Fragen im original Fragebogen (Bereinigter Fragebogen) \newline & 6 (5)\\
%    &\textit{n} \newline &30\\
%    &$\alpha$ (bereinigtes $\alpha$) & .289 (.605)\\
%    
%    \hline
%\multirow{3}{3.5cm}{Igroup Presence Questionnaire(IPQ)}
%	&Anzahl der Fragen im original Fragebogen (Bereinigter Fragebogen) \newline & 14 (11)\\
%    &\textit{n} \newline &30\\
%    &$\alpha$ (bereinigtes $\alpha$) & .471 (.801)\\
%    
%    \hline
%\multirow{3}{3.5cm}{Co-Presence}
%	&Anzahl der Fragen im original Fragebogen (Bereinigter Fragebogen) \newline & 30 (24)\\
%    &\textit{n} \newline &30\\
%    &$\alpha$ (bereinigtes $\alpha$) & .581 (.710)\\
%
%\end{tabular}
%\end{table}


\clearpage
\section{Statistische Auswertung}
Das folgende Kapitel beschäftigt sich mit der statistischen Analyse. Jede Hypothese besitzt ein eigenes Unterkapitel in der eine auf die Hypothese zugeschnittene Auswertung stattfindet.
Alle Ergebnisse wurden mit \textit{IBM SPSS Statistics} und \textit{Microsoft Excel} statistisch ausgewertet. 
Es wurde für alle hier aufgeführten Tests eine Irrtumswahrscheinlichkeit von $\alpha = 0.05$ festgelegt.

Die \textit{Tabelle \ref{VariableBreakdown1} und die Tabelle \ref{VariableBreakdown2}} zeigen die verwendeten Methoden und Ergebnisse zur Auswertung der einzelnen Hypothesen im Überblick.

Die \textit{Tabelle \ref{MittelwerteUndCo}} zeigt eine diskriptive Statistik der Hauptvariablen des Experiments.

%\paragraph{Extremwerte}
%Es wurde ein Extremwert bei einer Person mit der Kondition \ac{gti} erkannt. Dieser Extremwert wurde von einer teilnehmenden Person gebildet, die sich in dem Team mit der \textit{geringsten Anzahl} an abgeschlossenen Runden ($\ac{tRound} = 4$) befand. Weiterhin wurde bei dieser Person eine wahrgenommene Teameffektivität von ($\ac{te} = 1$) angegeben. Daher wird der Extremwert als plausibel betrachtet und nicht als Ausreißer definiert.

	\subsection{Teilnehmer und Demografie}
Insgesamt haben 30 Teilnehmer an dem Versuch teilgenommen. Der Mittelwert betrug ($\bar{x}$ = 30,13) und die Standartabweichung ($\sigma$ = 7,44). Somit ergibt sich eine Spannweite von 38 Jahren. 
%Siehe Anhang \textbf{\autoref{fig:Teilnehmer}}.
Von diesen 30 Teilnehmern waren 19(63,3\%) Männlich und 11(36,7\%) Weiblich. 
%Siehe Anhang \textbf{\autoref{fig:biologischesGeschlecht}}.

Von den 30 Teilnehmern, besaß/en 1(3,3\%) das Fachabitur/Fachgebundene Hochschulreife, 4(13,3\%) das Abitur/Allgemeine Hochschulreife, 24(80\%) ein Abgeschlossenes Studium und 1(3,3\%) eine abgeschlossene Ausbildung. 
%Siehe Anhang \textbf{\autoref{fig:teilnehmerBildungsstand}}.

Bei 10(33,3\%) war noch keine VR-Erfahrung vorhanden, während bei 20(66,6\%) schon VR-Erfahrung vorhanden war. 
%Siehe Anhang \textbf{\autoref{fig:teilnehmerVRErfahrung}}.

Von den 20 Personen, die schon VR-Erfahrung hatte, haben 7(23,3\%) schon Erfahrungen mit VR-Experimenten oder Studien. 
%\textbf{\autoref{fig:teilnehmerVorexperimente}}

Mittels einer Likert-Skala von 1-7 (1 = wenig, 7= sehr viel) wurde nach dem Außmaß der Internetnutzung im täglichem Leben gefragt. Dabei lagen 25(75\%) Personen oberhalb des 25\% Perzentil, wobei der Mittelwert ($\bar{x}$ = 6,27) und die Standartabweichung ($\sigma$ = 1,172) betrug.
%\textbf{\autoref{fig:teilnehmerPCAußmaß}}
 
Mittels einer weiteren Likert-Skala von 1-7 (1 = wenig, 7= sehr viel) wurde danach gefragt, wie Häufig die Teilnehmer Videospiele spielen. Der Mittelwert der Likert-Skala beträgt 3,5, wobei im 50\% Perzentil das Videospiel Außmaß mit "3" beziffert wurde. Der Mittelwert betrug ($\bar{x}$ = 3,1) und liegt damit ebenfalls unter dem Durchschnitt. Die Standartabweichung betrug ($\sigma$ = 1,826).
%\textbf{\autoref{fig:teilnehmerVideospieleAußmaß}}


%	\subsection{Voraussetzungen}
	
%\paragraph{Anzahl abgeschlossener Runden}
%Die maximale Anzahl an abgeschlossenen Runden beträgt 13. Die minimale Anzahl 4. Um in einem Wertebereich zwischen 0 und 1 zu bleiben, wurde die Anzahl an abgeschlossenen Runden im Vorhinein normalisiert.
%Siehe \textbf{\autoref{fig:boxplot_cognitive_trust_winsorisiert}}. 
%\ac{gt} ist unabhängig vom \ac{ct} der einzelnen Testperson. Es muss eine Spearman-Korrelation auf Individualebene mit $N=30$ Personen durchgeführt werden, bei der die einzelnen Teams und Teamzusammensetzungen sowie die Avatar Konditionen ignoriert werden. 
%	
%\paragraph{Test auf Normalverteilung der Daten}
%Für viele Tests ist eine Voraussetzung, dass die Stichproben Normalverteilt sind. Zur Überprüfung auf Normalverteilung wurde der \textit{Kolmogoroff-Smirnov-Test} genutzt. Die Null-Hypothese dieses Tests besagt, dass eine Normalverteilung der Variablen vorliegt. Wenn ($p < \alpha = 0.05$), muss H0 verworfen werden und angenommen werden, dass die Daten nicht Normalverteilt sind.
%
%Bei kleinen Stichproben kann ebenfalls die Skewness und die Kurtosis genutzt werden, um auf Normalverteilung zu überprüfen. 
%Die Schiefe und die Kurtosis sollten "nahe Null" sein, damit man von einer normalverteilten Variable ausgehen kann.\\
%$ z_{Skewness} = \frac{S-0}{\sigma_{Skewness}} $ \\
%$ z_{Kurtosis} = \frac{K-0}{\sigma_{Kurtosis}} $ \\
%$\sigma_{Kurtosis}$ sowie $\sigma_{Skewness}$ werden von SPSS ausgegeben. $S$ ist die Skewness, $K$ ist die Kurtosis und ${\sigma_{(Skewness/Kurtosis)}}$ ist die Standardabweichung.
%Um " nahe Null " zu bestimmen, können die mit vorheriger Formel bestimmten Werte mit den folgenden Grenzwerten verglichen werden:
%Für kleine Stichproben (n < 50) : ($z < |+/-1.96|$) ist \textbf{signifikant} mit $p<0.05$. \newline
%Für mittlere Stichproben (50 < n < 100) : ($z < |+/-2.58|$) ist \textbf{signifikant} mit $p<0.01$. \newline
%Für mittlere Stichproben (n > 100) : ($z < |+/-3.29|$) ist \textbf{signifikant} mit $p<0.001$. \newline
%Ist der jeweilige z-Wert größer als dieser Grenzwert muss die Nullhypothese verworfen werden und von nicht normalverteilen Daten ausgegangen werden \citep[p.184]{field2013discovering}.
%
%\paragraph{T-Test}
%Um zu überprüfen, ob sich zwei  Gruppen unterscheiden, muss ein t-Test für unabhängige Variablen durchgeführt werden.
%Dieser hat die Voraussetzung, dass die Stichproben \textbf{unabhängig}, \textbf{intervallskaliert}, \textbf{Normalverteilt} sowie eine \textbf{ähnlich gleiche Varianz} besitzen.
%
%\paragraph{Mann-Whitney-U-Test}
%Dieser Test ist im Gegensatz zum t-Test eher unanspruchsvoll. Die Messungen der einen Gruppe dürfen nicht durch Messungen der anderen Gruppe beeinflusst sein. Somit müssen die Messungen \textbf{unabängig} sein. Die unabhängige Variable ist \textbf{nominalskaliert} und besitzt \textbf{zwei Ausprägungen}. Die \textbf{abhängige Variable} ist mindestens \textbf{ordinalskaliert}  und die Gruppen müssen etwa die gleiche Verteilung besitzen \citep{eid2017statistik}.

\newpage
	
	\subsection{Auswertung Hypothese 1}
H1$_{0}$ : Die Mittelwerte der erzielten kognitive Vertrauenswerte (\ac{ct}) unterscheiden sich bei den Konditionen \ac{ik} und \ac{nik} nicht voneinander.

\paragraph{Allgemeines}
Es wird auf \textit{Konditionsebene} analysiert, ob die teilnehmenden Personen aufgrund der unterschiedlichen Avatarkonditionen einen unterschiedlichen kognitiven Vertrauenswert bilden. Dazu wird der Mann-Whitney-U-Test durchgeführt. \\
Der \ac{ct} ist laut Shapiro-Wilk-Test mit ($p = .023 < \alpha = 0.05$) \textit{nicht normalverteit}.
%Es wurde auf Varianzgleichheit überprüft. Der Levene-Test zeigt eine Varianzgleichheit zwischen \ac{cti} und \ac{ctn} ($L = .889; p=.354 > \alpha = 0.05$). Damit kann davon ausgegangen werden, dass die Varianzen sich \textbf{nicht signifikant} voneinander unterscheiden.\\
%\paragraph{Mittelswerte und Standardabweichungen}
Der Cognitive-Trust-Score (\ac{cti}) beträgt im Mittel ($\tilde x = 4,1238$) mit einer Standardabweichung von ($\sigma = .67194$).\newline 
Der Cognitive-Trust-Score (\ac{ctn}) beträgt  im Mittel ($\tilde x = 4,5905$) mit einer Standardabweichung von ($\sigma = .49152$).

\paragraph{Mann-Whitney-U Test}
%Aufgrund des geringen p-Wertes des \ac{cti} im Kolmogoroff-Smirnov-Test mit ($p = .185 > \alpha = 0.05$) wurde zusätzlich noch der Mann-Whitney-U-Test durchgeführt.
Die Verteilungsform der beiden Konditionen unterscheiden sich laut Kolmogorov-Smirnov ($p = .181 > \alpha = 0.05$) \textbf{nicht signifikant} voneinander. Es gab \textbf{einen signifikanten Unterschied} der Mittelwerte des \ac{ct} zwischen ($\ac{ik} : \tilde x = 4,1238$) und ($\ac{nik} : \tilde x = 4,5905)$, ($U = 62,500 , Z = -2.083, p = .037 < \alpha = 0.05, r =-.3803$). \\

	\begin{figure}[H]
		\begin{footnotesize}
		\centering
			\includegraphics[scale=0.3]{Abbildungen/AuswertungDiagramme/H1_Mittelwerte.png}
			\caption[Abbildung 1]{Mittelwerte und Standardabweichung der kognitiven Vertrauenswerte. Links die Kondition \ac{ik} und Rechts die Kondition \ac{nik}}
			\label{H1_Mittelwerte}
		\end{footnotesize}
	\end{figure}
\newpage

\subsection{Auswertung Hypothese 2}
%H1$_{0}$ : Steigt der General-Trust-Score einer Person (\ac{gt}), steigt der Cognitive-Trust-Score einer Person (\ac{ct}) nicht.
H2$_{0}$ : Es besteht ein kein Zusammenhang zwischen dem erzielten generellen Vertrauenswert \ac{gt} einer Person und dem erzielten kognitiven Vertrauenswert \ac{ct} einer Person.

\paragraph{Allgemeines}
Der kognitive Vertrauenswert muss für die \textit{Hypothese 1} auf \textit{Konditionsebene} analysiert werden, da der \ac{ct} aufgrund der unterschiedlichen Avatarkonditionen gebildet wurde. Somit ist es nicht möglich, den Zusammenhang des \ac{gt} und \ac{ct} unabhängig der Konditionsebene zu betrachten.
%Auf dem individuellem Level konnten mit dem Kolmogoroff-Smirnoff-Test alle Daten auf Normalverteilung mit ($p>0.05$) nachgewiesen werden. Siehe \textbf{\autoref{fig:KolSmirInd}} \\
%Auf dem Teamlevel konnte bei \ac{tci} mit ($p = .039 < \alpha = 0.05$) keine Normalverteilung nachgewiesen werden. Siehe \textbf{\autoref{fig:KolSmirIndIK}} \\ Weiterhin konnte bei \ac{cpn} mit ($p = .044 < \alpha = 0.05$) keine Normalverteilung nachgewiesen werden. Siehe \textbf{\autoref{fig:KolSmirIndNONIK}}
Der Shapiro-Wilk-Test zeigt, dass \ac{gti} mit  $p = .078 > \alpha = 0.05$ \textit{normalverteilt} und \ac{gtn} mit $p = .020 < \alpha = 0.05$ \textit{nicht normalverteilt} ist. \\
Aus einem weiteren Shapiro-Wilk-Test geht hervor, dass \ac{cti} mit  $p = .356 > \alpha = 0.05$ \textit{normalverteilt} und \ac{ctn} mit $p = .054 > \alpha = 0.05$ \textit{normalverteilt} ist.
Das Lowess-Verfahren deutet Augenscheinlich auf keinen linearen Zusammenhang zwischen \ac{gti} und \ac{cti} sowie \ac{gtn} und \ac{ctn} hin. Daher wird die \textit{Linearitätsannahme nicht angenommen}.
%Für \ac{gt} ist $ z_{Skewness} = -0,985 $ und $z_{Kurtosis} = -.416$ und somit ist \ac{gt} normalverteilt.
%Für \ac{ct} ist $ z_{Skewness} = -.257 $ und $z_{Kurtosis} = -1,764$ und somit ist \ac{ct} normalverteilt.
%Voraussetzung für Pearson-Korrelationsanalyse: 
%Die Variablen sind mindestens intervallskaliert -> Hier ordinalskaliert
%Die Variablen sind normalverteilt -> JA
%Der untersuchte Zusammenhang zwischen den Variablen muss linear sein -> JA
%Ab einer Likert Skala ab 6 können ordinale Skalenniveaus als metrische variablen interpretiert werden???
%\paragraph{Pearson-Korrelationsanalyse}
%Es ist eine positive Korrelation starken Effektes (vgl. \cite{cohen2013statistical}) mit dem Pearson-Korrelationskoeffizient ($r = .522$) zwischen \ac{gti} und \ac{cti} vorhanden. Die Pearson-Korrelation ist \textbf{nicht signifikant} ($p = .055 > \alpha = 0.05$).\\
%Weiterhin ist eine positive Korrelation moderaten Effektes (vgl. \cite{cohen2013statistical}) mit dem Pearson-Korrelationskoeffizient ($r = .306$) zwischen \ac{gtn} und \ac{ctn} vorhanden. Die Pearson-Korrelation ist \textbf{nicht signifikant} ($p = .267 > \alpha = 0.05$).
\paragraph{Spearman-Korrelationsanalyse}
Es ist eine positive Korrelation starken Effektes (vgl. \cite{cohen2013statistical}) mit dem Spearman-Korrelationskoeffizient ($r = .441$) zwischen \textbf{\ac{gti} und \ac{cti}} vorhanden. Die Spearman-Korrelation ist \textbf{nicht signifikant} ($p = .100 > \alpha = 0.05$).\\
Es ist eine positive Korrelation moderaten Effektes (vgl. \cite{cohen2013statistical}) mit dem Spearman-Korrelationskoeffizient ($r = .378$) zwischen \textbf{\ac{gtn} und \ac{ctn}} vorhanden. Die Spearman-Korrelation ist \textbf{nicht signifikant} ($p = .165 > \alpha = 0.05$).

\paragraph{Fisher-Z-Transformation für unabhängige Stichproben}
Es wird eine Fisher-Z-Transformation für unabhängige Stichproben durchgeführt.
Die beiden Korrelationen unterscheiden sich laut Fisher-Z-Wert für unabhängige Stichproben ($Z=.186$) \textbf{nicht signifikant} voneinander ($p = .426 > \alpha = 0.05$).


\begin{figure}[h]
  \centering
  \subfloat[][]{\includegraphics[width=0.45\linewidth]{Abbildungen/AuswertungDiagramme/H_2_KorrIK.png}}
  \qquad
  \subfloat[][]{\includegraphics[width=0.45\linewidth]{Abbildungen/AuswertungDiagramme/H_2_KorrNIK.png}}
  \caption{Korrelationen der generellen Vertrauenswerte mit den kognitiven Vertrauenswerten. Links (a) für die Kondition \ac{ik} und Rechts (b) für die Kondition \ac{nik}}
  \label{AvatareImEinsatz}
\end{figure}
%Hier nun eventuell noch Fishers-Z-Transformation?

%Laut Pagan-Breusch liegt homoskedastizität vor.
%Jedoch durch das augenscheinliche lowess verfahren liegt keine linearität vor..

%\paragraph{Spearman-Korrelationsanalyse}
%Um diese Hypothese zu überprüfen, wurde eine Spearman-Korrelationsanalyse durchgeführt. 
%
%%\begin{figure}[H]
%%\centering
%%		\begin{footnotesize}
%%			\includegraphics[scale=0.8]{Abbildungen/Post_QuestionnaireStatistiks/pearson_korrelation_general_cognitive}\\
%%			\caption{Pearson-Korrelation}
%%			\label{fig:pearson_korrelation}
%%		\end{footnotesize}
%%	\end{figure}	
%	
%Es ist eine \textbf{signifikant} ($p = .022 < \alpha = 0.05$) positive Korrelation mit dem Pearson-Korrelationskoeffizient ($r = .417$) zwischen \ac{gt} und \ac{ct} vorhanden. 
%Das generelle Vertrauen einer Person korreliert  moderat positiv (laut Cohen \cite{cohen2013statistical}) mit dem kognitiven Vertrauen der Personen.
%%Dies deutet darauf hin, die \textit{(Hypothese 1 / H$_{0}$)} zu verwerfen und die Alternativhypothese \textit{Hypothese 1 / H$_{1}$} anzunehmen.

%\paragraph{Einfache Lineare-Regressionsanalyse}
%Um die vermutete gerichtete Annahme der neutralen Auswirkung des \ac{gt} auf den \ac{ct} zu überprüfen, wurde eine Einfache lineare-Regressionsanalyse durchgeführt. Dabei wurde als unabhängige Variable \ac{gt} und als abhängige Variable \ac{ct} gewählt wurde.
%%
%%\begin{figure}[H]
%%\centering
%%		\begin{footnotesize}
%%			\includegraphics[scale=0.8]{Abbildungen/Post_QuestionnaireStatistiks/regression_gt_ct_modell}\\
%%			\includegraphics[scale=0.8]{Abbildungen/Post_QuestionnaireStatistiks/regression_gt_ct_anova}\\
%%			\includegraphics[scale=0.8]{Abbildungen/Post_QuestionnaireStatistiks/regression_gt_ct_koeffizienten}\\
%%			\includegraphics[scale=0.5]{Abbildungen/Post_QuestionnaireStatistiks/diagramm_regression_gt_ct}\\
%%			\caption{Pearson-Korrelation}
%%			\label{fig:regressionsanalyse}
%%		\end{footnotesize}
%%	\end{figure}	
%
%%Die Einfache lineare-Regressionsanalyse ist mit \ac{ct} als abhängige und \ac{gt} als unabhängige Variable signifikant ($F(1,28) = 7,202; p = .012 < \alpha = 0.05$).
%
%Das Bestimmtheitsmaß ($R^{2} = .160$) deutet laut \citep{cohen2013statistical} auf eine geringe bis mittlere Varianzaufklärung des \ac{ct} durch \ac{gt} hin. Der \ac{gt} erklärt somit einen signifikanten Anteil der Varianz vom \ac{ct} ($R^{2} = .160; F(1,28) = 5,325; p = .029 < \alpha = 0.05$). Es lassen sich $16,0\%$ der Varianz unseres \ac{ct} durch den \ac{gt} erklären.
%
%Der Regressionskoeffizient der Variable \ac{gt} beträgt $\beta = .400$ und ist \textbf{signifikant} ($t(28) = 2,308; p = .029 < \alpha = 0.05$).
%
%Der \ac{gt} eignet sich als Prädiktor für den \ac{ct}. Die geschätzte Zunahme an \ac{ct} beläuft sich auf $0.400$ Punkte \ac{ct} pro \ac{gt} ($\beta = 0.400; t(28) = 2,308; p = .029 < \alpha = 0.05$).
%
%Aufgrund der Zunahme des \ac{ct} durch den \ac{gt} lässt sich \textit{Hypothese 1 / H$_{0}$} zum Signifikanzniveau $\alpha = 0.05$ verwerfen.

\newpage
\subsection{Auswertung Hypothese 3}
\textbf{H3$_{0}$} : Der Zusammenhang des \ac{teamCog} und der realen Teameffektivität von Teams \ac{tRound} mit der Kondition \ac{ik} unterscheiden sich nicht von den Teams mit der Kondition \ac{nik}.
%NEU NEU NEU!
%Der Zusammenhang zwischen gebildetem kognitiven Vertrauen und abgeschlossenen Runden ist bei Personen mit der Avatarkondition IK größer als bei denen mit der Avatarkondition NIK.
%Fisher Z- Transformation !
%Hier weil F(p) > F(p) gerichtete Korrelationsannahme. Daher mit Signifikanzniveau von 0.1 testen?
%H0 : Die Korrelationen sind gleich ; H1 : Die Korrelationen unterscheiden sich.
%
% VGL : https://www.youtube.com/watch?v=SVuniaJXOoM&ab_channel=drfg2008
%
%DIESE HYPOTHESE?
%Es besteht kein Zusammenhang zwischen dem Cognitiver-Trust-Score im Team und der Anzahl der erfolgreich abgeschlossenen Runden im Team
%H3$_{0}$ : Ein hoher Cognitiver-Trust-Score im Team (\ac{teamCogIK} und \ac{teamCogNIK}) hat \textbf{keinen Einfluss} auf die auf die Anzahl der erfolgreich abgeschlossenen Runden (\ac{tRound}) bei unterschiedlichen Avatar Verkörperungen (\ac{tRoundIK} und \ac{tRoundNIK}).
%
\paragraph{Vertrauenstabelle pro Team}
%Es wird der Zusammenhang zwischen \ac{cti} und \ac{tRoundIK} sowie \ac{ctn} und \ac{tRoundNIK} auf Teamebene analysiert.
Während jedem Versuchsdurchlauf sieht jedes der 3 Teammitglieder die gleiche Avatarkondition. Da die 3 Teammitglieder des Teams die gleiche Anzahl an Runden abgeschlossen haben, besitzen diese die gleiche Teameffektivität. Die Anzahl der abgeschlossenen Runden pro Team und pro Kondition wurde zu jeweils einer Variable für jedes Team zusammengefasst : \ac{tRoundIK} sowie \ac{tRoundNIK}. Für jedes Team wird eine gemeinsamer kognitiver Vertrauenswert berechnet. Dieser sagt aus, wie viel kognitives Vertrauen das gesamte Team untereinander besitzt. Dieser kognitive Vertrauenswert ergibt sich aus der Summe der kognitiven Vertrauenswerte der einzelnen Personen eines Teams. Die kognitiven Vertrauenswerte der Teams der unterschiedlichen Konditionen werden zu den Variablen \ac{teamCogIK} sowie \ac{teamCogNIK} für jedes Team zusammengefasst. Eine Übersicht dieser definierten Werte befindet sich auf \textbf{Tabelle \ref{TeamCogTabelle}})

Es wird eine Spearman-Korrelation zwischen \ac{tRound} und \ac{teamGen} auf textit{Teamebene} durchgeführt, wobei die Teams in verschiedenen Konditionen aufgeteilt werden. Damit anschließend überprüft werden kann, ob sich die beide Korrelationen signifikant voneinander unterscheiden, wird eine Fisher-Z-Transformation für unabhängige Stichproben durchgeführt.\\
%Somit müssen die kognitiven Vertrauenswerte nicht winsorisiert werden. 
%Alle Ergebnisse werden in \ac{ik} und \ac{nik} aufgeteilt um ein Vergleich der verschiedenen Konditionen darzustellen. Kondition 1 definiert die Kondition \ac{ik} und Kondition 2 definiert die Kondition \ac{nik}.
%
Der \ac{teamCogIK} ist laut Shapiro-Wilk-Test mit ($p = .017 < \alpha = 0.05$) nicht normalverteit. \newline
Der \ac{teamCogNIK} ist laut Shapiro-Wilk-Test mit ($p = .713 < \alpha = 0.05$) normalverteit. \newline
Der \ac{tRoundIK} ist laut Shapiro-Wilk-Test mit ($p = .844 > \alpha = 0.05$) normalverteit. \newline
Der \ac{tRoundNIK} ist laut Shapiro-Wilk-Test mit ($p = .453 > \alpha = 0.05$) normalverteit. \\
Das Lowess-Verfahren deutet Augenscheinlich auf keinen linearen Zusammenhang zwischen \ac{teamCogIK} und \ac{tRoundIK} sowie \ac{teamCogNIK} und \ac{tRoundNIK} hin. Daher wird die \textit{Linearitätsannahme nicht angenommen}.

Aufgrund der geringen Stichprobenumfangs ($\ac{teamCogIK}, \ac{teamCogNIK} : N=5$) und der nicht eindeutigen Linearität durch das Lowess-Verfahren, wurde sich für eine Spearman-Korrelation entschieden.

\paragraph{Spearman-Korrelation}
Es ist eine positive Korrelation moderaten Effektes (vgl. \cite{cohen2013statistical}) mit dem Spearman-Korrelationskoeffizient ($r = -.205$) zwischen \textbf{\ac{teamCogIK} und \ac{tRoundIK}} vorhanden. Die Spearman-Korrelation ist \textbf{nicht signifikant} ($p = .741 > \alpha = 0.05$).\\
Weiterhin ist eine positive Korrelation moderaten Effektes (vgl. \cite{cohen2013statistical}) mit dem Spearman-Korrelationskoeffizient ($r = .462$) zwischen \textbf{\ac{teamCogNIK} und \ac{tRoundNIK}} vorhanden. Die Spearman-Korrelation ist \textbf{nicht signifikant} ($p = .434 > \alpha = 0.05$).

\paragraph{Fisher-Z-Transformation für unabhängige Stichproben}
Es wird eine Fisher-Z-Transformation für unabhängige Stichproben durchgeführt, obwohl die Stichprobe mit ($N=5$) sehr klein ist.
Die beiden Korrelationen unterscheiden sich laut Fisher-Z-Wert für unabhängige Stichproben ($Z=-0.708$) \textbf{nicht signifikant} voneinander ($p = .24 > \alpha = 0.05$).
%
%\paragraph{Normalverteilung}
%Laut Kolmogoroff-Smirnoff-Test sind sowohl die Summe der individuellen kognitiven Vertrauenswerte des Teams (\ac{teamCogIK} $p = .149 > \alpha = 0.05$), (\ac{teamCogNIK} $p = .109 > \alpha = 0.05$) sowie der Anzahl der erfolgreich abgeschlossenen Runden der Teams (\ac{tRoundIK} $p = .200 > \alpha = 0.05$), (\ac{tRoundNIK} $p = .161 > \alpha = 0.05$) Normalverteilt.
%
%\paragraph{Mittelwerte und Standardabweichungen}
%Der Cognitive-Trust-Score pro Team (\ac{teamCogIK}) beträgt der Kondition \ac{ik} im Mittel ($\tilde x = 12,6$) mit einer Standardabweichung von ($\sigma = 1,122$).\newline 
%Der Cognitive-Trust-Score pro Team (\ac{teamCogNIK}) beträgt der Kondition \ac{nik} im Mittel ($\tilde x = 13,72$) mit einer Standardabweichung von ($\sigma = .9011$). \newline 
%Die Anzahl der Abgeschlossenen Runden pro Team (\ac{tRoundIK}) beträgt der Kondition \ac{ik} im Mittel ($\tilde x = 9$) mit einer Standardabweichung von ($\sigma = 3,535$).\newline 
% Die Anzahl der Abgeschlossenen Runden pro Team (\ac{tRoundNIK}) beträgt der Kondition \ac{nik} im Mittel ($\tilde x = 9$) mit einer Standardabweichung von ($\sigma = 1,870$). 
%
%\paragraph{Spearman-Korrelationsanalyse}
%Die Stichprobengröße von ($N=5$) ist sehr klein. Es ist kein linearer Zusammenhang zu erkennen. Es wird eine Spearman-Korrelationsanalyse durchgeführt. Diese korreliert die kognitiven Vertrauenswerte des Teams und die erfolgreich abgeschlossenen Runden der Teams für beide Konditionen.
%
%%\begin{figure}[H]
%%\centering
%%		\begin{footnotesize}
%%			\includegraphics[scale=0.8]{Abbildungen/Post_QuestionnaireStatistiks/H3_IK_Korrelation_Spearman}
%%			\includegraphics[scale=0.8]{Abbildungen/Post_QuestionnaireStatistiks/H3_NIK_Korrelation_Spearman}
%%			\caption{Spearman-Korrelation zwischen den einzelnen Teams und dem \ac{te}}
%%			\label{fig:Pearson_ik_nik_h3}
%%		\end{footnotesize}
%%	\end{figure}	
%
%Es ist eine Korrelation mittlerem Effektes ($r = .316$) zwischen \ac{teamCogIK} und \ac{tRoundIK} vorhanden. Diese Korrelation ist statistisch \textbf{nicht signifikant} ($p = .604 > \alpha = 0.05$)
%Es ist eine positive Korrelation starken Effektes ($r = .947$) zwischen \ac{teamCogNIK} und \ac{tRoundNIK} vorhanden. Diese Korrelation ist statistisch \textbf{signifikant} ($p = .014 < \alpha = 0.05$). 
%
\begin{table}[H]
	\centering\footnotesize\setstretch{1.2}
	\caption{Tabelle des kognitiven Vertrauen und der erfolgreich abgeschlossenen Runden - Individuell und pro Team zusammengefasst. Kondition 1 definiert die Kondition \ac{ik} und Kondition 2 definiert die Kondition \ac{nik}.}
	\label{TeamCogTabelle}
	\begin{tabularx}{\textwidth}{r | p{2cm} p{1.5cm} p{3cm} p{1.5cm} p{1.5cm} p{1.5cm} p{1.5cm} } 
		ID & Individueller-Kognitiver Vertrauenswert & Kondition & Erfolgreich abgeschlossene Runden \newline(\ac{teamCogIK},\ac{teamCogNIK}) & Team ID & Summe \ac{ct} (\ac{tRoundIK}, \ac{tRoundNIK})\\
		\hline 
		1 & $2,57$ & 1 & 4 & Team 1 & $11,85$\\
		2 & $4,71$ & 1 & 4 & &\\
		3 & $4,57$ & 1 & 4 & & \\
		\hline 
		4 & $4,14$ & 1 & 10 & Team 2 & $11,71$\\
		5 & $3,86$ & 1 & 10 & &\\
		6 & $3,71$ & 1 & 10 & &\\
		\hline 
		7 & $5,00$ & 1 & 7 & Team 3 & $13,28$\\
		8 & $3,71$ & 1 & 7 & & \\
		9 & $4,57$ & 1 & 7 & & \\
		\hline 
		10 & $4,43$ & 1 & 11 & Team 4 & $13,29$ \\
		11 & $3,86$ & 1 & 11 & & \\
		12 & $5,00$ & 1 & 11 & & \\
		\hline 
		13 & $4,14$ & 1 & 13 & Team 5 & $11,71$ \\
		14 & $4,43$ & 1 & 13 & & \\
		15 & $3,14$ & 1 & 13 & & \\
		\hline 
		16 & $5,14$ & 2 & 9 & Team 6 & $15,00$ \\
		17 & $5,00$ & 2 & 9 & &\\
		18 & $4,86$ & 2 & 9 & &\\
		\hline 
		19 & $5,14$ & 2 & 12 & Team 7 & $14,71$ \\
		20 & $4,71$ & 2 & 12 & & \\
		21 & $4,86$ & 2 & 12 & & \\
		\hline 
		22 & $4,57$ & 2 & 8 & Team 8 & $12,28$ \\
		23 & $4,00$ & 2 & 8 & & \\
		24 & $3,71$ & 2 & 8 & & \\
		\hline 
		25 & $4,29$ & 2 & 7 & Team 9 & $13,57$ \\
		26 & $4,71$ & 2 & 7 & & \\
		27 & $4,57$ & 2 & 7 & & \\
		\hline 
		28 & $4,71$ & 2 & 9 & Team 10 & $13,28$ \\
		29 & $5,00$ & 2 & 9 & &\\
		30 & $3,57$ & 2 & 9 & & \\	
	\end{tabularx}
\end{table}

%\paragraph{Lineare-Regressionsanalyse}
%Um die Vermutung zu überprüfen, dass ein hoher Cognitiver-Trust-Score im Team  \textbf{keinen Einfluss} auf die auf die Anzahl der erfolgreich abgeschlossenen Runden bei unterschiedlichen Avatar Verkörperungen hat, wurde eine Lineare Regressionsanalyse durchgeführt.
%%Es wird angenommen, dass das \ac{teamCog} einen \underline{positiven Einfluss} auf die \ac{tRound} hat.
%\ac{teamCog} wird als unabhängige und \ac{tRound} als abhängige Variable definiert.
%
%\paragraph{Kondition \ac{ik}}
%Die Einfache lineare-Regressionsanalyse ist mit \ac{teamCogIK} als unabhängige und \ac{tRoundNIK} als abhängige Variable statistisch \textbf{nicht signifikant} (($F(1,3) = .111; p = .761 > \alpha = 0.05$)).
%
%Das Bestimmtheitsmaß ($R^{2} = .036$) deutet laut \citep{cohen2013statistical} auf eine schwache Varianzaufklärung der \ac{tRoundIK} durch \ac{teamCogIK} hin. Somit lassen sich $3,6\%$ der Varianzen unserer \ac{tRoundIK} durch den \ac{teamCogNIK} erklären.
%
%Der Regressionskoeffizient der Variable \ac{teamCogIK} beträgt ($\beta = .189$) und ist statistisch \textbf{nicht signifikant} ($t(3) = .333; p = .761 > \alpha = 0.05$).
%
%Der \ac{teamCogIK} eignet sich nicht als Prädiktor für die \ac{tRoundIK}.
%%Siehe \textbf{\autoref{fig:h3_regression}}
%
%\textit{(Hypothese 3 / H$_{0}$)} lässt sich zum Signifikanzniveau ($\alpha = 0.05$) für die Kondition \ac{ik} nicht verwerfen.
%
%\paragraph{Kondition \ac{nik}}
%Die Einfache lineare-Regressionsanalyse ist mit \ac{teamCogIK} als unabhängige und \ac{tRoundNIK} als abhängige Variable statistisch \textbf{nicht signifikant} (($F(1,3) = 5,363; p = .103 > \alpha = 0.05$)).
%
%Das Bestimmtheitsmaß ($R^{2} = .641$) deutet laut \citep{cohen2013statistical} auf eine starke Varianzaufklärung der \ac{tRoundNIK} durch \ac{teamCogNIK} hin. Somit lassen sich $64,1\%$ der Varianzen unserer \ac{tRoundNIK} durch den \ac{teamCogNIK} erklären.
%
%Der Regressionskoeffizient der Variable \ac{teamCogNIK} beträgt ($\beta = .801$) und ist statistisch \textbf{nicht signifikant} ($t(3) = 2,316; p = .103 > \alpha = 0.05$).
%
%Der \ac{teamCogNIK} eignet sich nicht als Prädiktor für die \ac{tRoundNIK}.
%
%\textit{(Hypothese 3 / H$_{0}$)} lässt sich zum Signifikanzniveau ($\alpha = 0.05$) für die Kondition \ac{nik} nicht verwerfen.
%
%\paragraph{Hypothese 3a : Analyse ohne die Aufteilung in \ac{ik} und \ac{nik} }
%Da die Stichprobe mit lediglich 5 Teams pro Kondition bei der Korrelationsanalyse sowie der Linearen-Regressionsanalyse zwischen \ac{teamCogIK} und \ac{tRoundIK} sowie \ac{teamCogNIK} und \ac{tRoundNIK} zu klein ist, wurde zusätzlich eine Spearman-Korrelationsanalyse sowie Einfache lineare-Regressionsanalyse \textit{ohne} die Aufteilung in die Konditionen \ac{ik} und \ac{nik} durchgeführt und somit folgende Hypothese aufgestellt :
%
%H3a$_{0}$ : Ein hoher Cognitiver-Trust-Score des Teams, hat \textbf{keinen Einfluss} auf die realen Teameffektivität im Team (\ac{tRound}).
%
%Laut Kolmogoroff-Smirnoff-Test sind, ohne die Aufteilung in die Konditionen \ac{ik} und \ac{nik}, sowohl die Summe der individuellen kognitiven Vertrauenswerte des Teams \ac{teamCog} ($p = .200 > \alpha = 0.05$) sowie der Anzahl der erfolgreich abgeschlossenen Runden der Teams \ac{tRound} ($p = .200 > \alpha = 0.05$) Normalverteilt.
%
%Der Cognitive-Trust-Score der Teams (\ac{teamCog} beträgt im Mittel ($\tilde x = 13,16$) mit einer Standardabweichung von ($\sigma = 1,126$).
%Die Anzahl der Abgeschlossenen Runden (\ac{tRound} beträgt im Mittel ($\tilde x = 9$) mit einer Standardabweichung von ($\sigma = 2,666$). 
%
%			\paragraph{Spearman-Korrelationsanalyse}
%Es ist eine positive Korrelation mittlerem Effektes mit dem Spearman-Korrelationskoeffizient ($r = .265$) zwischen \ac{teamCog} und \ac{tRound} vorhanden. Die Spearman-Korrelation ist statistisch \textbf{nicht signifikant} ($p = .460 > \alpha = 0.05$)
%%Dies deutet darauf hin, dass die \textit{(Hypothese 3a / H3a$_{0}$)} nicht verworfen werden kann. Siehe \textbf{\autoref{fig:h3_both}}
%			
%			\paragraph{Lineare-Regressionsanalyse}
%Die Einfache lineare-Regressionsanalyse ist mit \ac{teamCog} als unabhängige und \ac{tRound} als abhängige Variable statistisch \textbf{nicht signifikant} ($F(1,8) = .855; p = .382 > \alpha = 0.05$).
%
%Das Bestimmtheitsmaß ($R^{2} = .097$) deutet laut \citep{cohen2013statistical} auf eine schwache Varianzaufklärung der \ac{tRound} durch \ac{teamCog} hin. Somit lassen sich $9,7\%$ der Varianzen unserer \ac{tRound} durch den \ac{teamCog} erklären.
%
%Der Regressionskoeffizient der Variable \ac{teamCog} beträgt ($\beta = .311$) und ist statistisch \textbf{nicht signifikant} ($t(8) = .924; p = .382 > \alpha = 0.05$).
%
%Der \ac{teamCog} eignet sich nicht als Prädiktor für die \ac{tRound}.
%
%\textit{(Hypothese 3 / H$_{0}$)} lässt sich zum Signifikanzniveau ($\alpha = 0.05$)  nicht verwerfen.
\newpage
	\subsection{Auswertung Hypothese 4}
H4$_{0}$ : Die Mittelwerte der realen Teameffektivität \ac{tRound} unterscheidet sich bei den Konditionen \ac{ik} und \ac{nik} nicht voneinander.

\paragraph{Allgemeines}

Es wird auf \textit{Konditionsebene} analysiert, ob die teilnehmenden Personen aufgrund der unterschiedlichen Avatarkonditionen eine unterschiedliche Teameffektivität besitzen.\\
Die reale Teameffektivität \ac{tRound} ist laut Shapiro-Wilk-Test ($p = .113 > \alpha = 0.05$) normalverteit.\\
Es wurde auf Varianzgleichheit überprüft. Der Levene-Test zeigt eine keine Varianzgleichheit zwischen \ac{tRoundIK} und \ac{tRoundNIK} ($L = 10,182; p=.003 < \alpha = 0.05$). Damit kann davon ausgegangen werden, dass die Varianzen sich \textbf{signifikant} voneinander unterscheiden. Es wird ein Mann-Whitney-U Test durchgeführt. \\
Die reale Teameffektivität (\ac{tRoundIK}) beträgt bei Kondition \ac{ik} im Mittel ($\tilde x = 9,0000$) mit einer Standardabweichung von ($\sigma = 3,2732$).\newline 
Die reale Teameffektivität(\ac{tRoundNIK}) beträgt der Kondition \ac{nik} im Mittel ($\tilde x = 9,000$) mit einer Standardabweichung von ($\sigma = 1,7320$).

%Aufgrund der Varianzgleichheit hier doch ein Mann-Whitney-U-Test?
%\paragraph{T-Test für unabhängige Stichproben}
%Der T-Wert für unabhängige Stichproben ist kleiner Null ($T = 1,000 > 0$), was auf keinen Mittelwertsunterschied zwischen \ac{tRoundIK} und \ac{tRoundNIK} hinweist. Der Mittelwert $\bar(x) = 9,000$ der Kondition \ac{ik} ist kleiner als der Mittelwert $\bar(x) = 9,000$ der Kondition \ac{nik}.
%Der T-Test für unabhängige Stichproben ergibt bei der Analyse der Differenz des durchschnittlichen \ac{tRound} bei der Kondition \ac{ik} ($\bar(x) = 9,000; \sigma =  3,2732$) und \ac{nik} ($\bar(x) = 9,000; \sigma = 1,7320$) \textbf{keine Signifikanz} ($t(21,270) = .00; p = 1.000 > \alpha = 0.05$).

\paragraph{Mann-Whitney-U-Test}
Die Verteilungen der beiden Gruppen \ac{tRoundIndIK} und \ac{tRoundIndNIK} unterschieden sich laut Kolmogorov-Smirnov ($p = .181 > \alpha = 0.05$) \textbf{nicht signifikant} von einander. Es gab keinen signifikanten Unterschied der Mittelwerte des \ac{tRoundInd} zwischen ($\ac{tRoundIndIK} : \tilde x = 9$) und ($\ac{tRoundIndNIK} : \tilde x = 9)$, ($U = 103,500, Z = -.377, p = .706 > \alpha = 0.05, r = -,0602$). \\

	\begin{figure}[H]
		\begin{footnotesize}
		\centering
			\includegraphics[scale=0.3]{Abbildungen/AuswertungDiagramme/H4_Mittelwerte.png}	
			\caption[Abbildung 1]{Mittelwerte und Standardabweichung der Teameffektivität. Links die Kondition \ac{ik} und Rechts die Kondition \ac{nik}}
			\label{H1_Mittelwerte}
		\end{footnotesize}
	\end{figure}

%\paragraph{Mann-Whitney-U Test}
%%Aufgrund des geringen p-Wertes des \ac{cti} im Kolmogoroff-Smirnov-Test mit ($p = .185 > \alpha = 0.05$) wurde zusätzlich noch der Mann-Whitney-U-Test durchgeführt.
%Die Verteilungsform der beiden Gruppen \ac{ik} und \ac{nik} unterscheiden sich laut Kolmogorov-Smirnov ($p = .925 > \alpha = 0.05$) \textbf{nicht signifikant} voneinander. Es gab keinen signifikanten Unterschied der Mittelwerte der Anzahl an Abgeschlossenen Runden (\ac{tRound} zwischen ($\ac{ik} : \tilde x = 4.20$) und ($\ac{nik} : \tilde x = 4.573)$, ($U = 74,000 , Z = -1.615, p = .106 > \alpha = 0.05, r =-.294$).
% T TEST ZULÄSSIG, WENN :
%Die beiden Gruppen bzw. Stichproben müssen unabhängig sein JA
%Die Variablen müssen intervallskaliert sein NEIN
%Die Variablen müssen normalverteilt sein JA
%Die Varianz innerhalb der Gruppen sollte ähnlich sein JA

%H4$_{0}$ : Die Konditionen \ac{ik} oder \ac{nik} haben \textbf{keinen signifikant abweichenden Einfluss} auf die Anzahl der erfolgreich abgeschlossenen Runden \ac{tRound} eines Teams.
%
%Wie in \textit{Hypothese 3} beschrieben, ist \ac{tRound} der einzelnen Teams Normalverteilt.
%Um die  \textit{Hypothese 4} auszuwerten, wird ein T-Test mit 2 unabhängigen Stichproben für \ac{tRound} durchgeführt. Die Gruppierungsvariablen sind dabei \ac{ik} sowie \ac{nik}.
%
%Die abgeschlossenen Runden pro Team (\ac{tRoundIK}) betragen bei der Kondition \ac{ik} im Mittel ($\tilde x = 9$) mit einer Standardabweichung von ($\sigma = 3,535$).\newline 
%Der abgeschlossenen Runden pro Team (\ac{tRoundIK}) betragen bei der Kondition \ac{nik} im Mittel ($\tilde x = 9$) mit einer Standardabweichung von ($\sigma = 1,870$). \newline 
%Es ist kein Mittelwertsunterschied festzustellen.
%
%		\subsubsection{T-Test}
%Der T-Wert ist gleich Null ($T = .000 = 0$), was auf keinen Mittelwertsunterschied zwischen \ac{ik} und \ac{nik} hinweist.
%Die Varianzen sind mit ($L= 2.909; p = .126 > \alpha = 0.05$) gleich.
%Der T-Test zeigt bei der Differenz des durchschnittlichen \ac{tRound} bei der Kondition \ac{ik} ($\bar(x) = 9; \sigma = 3,5355$) und \ac{nik} ($\bar(x) = 9; \sigma = 1,8708$) keine Signifikanz ($t(8) = .000; p = 1 > \alpha = 0.05$).
%
%%\begin{figure}[H]
%%\centering
%%		\begin{footnotesize}
%%			\includegraphics[scale=0.8]{Abbildungen/Post_QuestionnaireStatistiks/h4_ttest}
%%			\includegraphics[scale=0.5]{Abbildungen/Post_QuestionnaireStatistiks/h4_ttest2}
%%			\caption{T-Test für unabhängige Stichproben der Team-Effektivität mit den unabhängigen Variablen \ac{ik} sowie \ac{nik}}
%%			\label{fig:h4_ttest}
%%		\end{footnotesize}
%%	\end{figure}	
%Um dies zu Bestätigen und die Stichprobe zu vergrößern, wird ein weiterer T-Test durchgeführt, der nicht auf Teamebene sondern auf Konditionsebene misst. Somit erhöht sich die Stichprobe auf \ac{ik} = 15 und \ac{nik} = 15.
%
%Der T-Wert ist gleich Null ($T = .000 = 0$), was auf keinen Mittelwertsunterschied zwischen \ac{ik} und \ac{nik} hinweist.
%Die Varianzen sind mit ($L= 10,182; p = .003 < \alpha = 0.05$) nicht gleich.
%Der T-Test (Welch-Test) zeigt bei der Differenz des durchschnittlichen \ac{tRound} bei der Kondition \ac{ik} ($\bar(x) = 9; \sigma = 3,2732$) und \ac{nik} ($\bar(x) = 9; \sigma = 1,7320$) keine Signifikanz ($t(21,270) = .000; p = 1 > \alpha = 0.05$).
%
%Somit kann \textit{(Hypothese 4 / H4$_{0}$)} für \ac{ik} und \ac{nik} nicht verworfen werden.
%\subsubsection{Mann-Whitney-U-Test}
%Es wurde zusätzlich durch den Mann-Whitney-U-Test überprüft, ob sich \ac{tRound} nach Konditionen \ac{ik} oder \ac{nik} unterscheidet. Die Verteilungen der beiden Gruppen \ac{tRoundIK} und \ac{tRoundNIK} unterschieden sich laut Kolmogorov-Smirnov ($p = .181 > \alpha = 0.05$) \textbf{nicht signifikant} von einander. Es gab keinen signifikanten Unterschied der Mediane des \ac{tRound} zwischen ($\ac{tRoundIK} : \tilde x = 10$) und ($\ac{tRoundNIK} : \tilde x = 9)$, ($U = 103,500, Z = -.377, p = .734 > \alpha = 0.05, r = -0,068$).

\newpage
	\subsection{Auswertung Hypothese 5}
% Die Grundgesamtheut Gt auf linearität und normalvertielung überprüfen
% Die tRoundsIK auf normalvertielung und linearität überpürfen
% Die tRoundsNIK auf normalverteilung und linearität überprüfen
% Damit Korrelationsart belegen

%\textbf{H5$_{1}$} : Es besteht ein Zusammenhang zwischen dem General-Trust-Score (\ac{gt}) eines Teams und der Anzahl der abgeschlossenen Runden eines Teams \ac{tRound}. \\

%Je höher der General-Trust-Score eines Teams, desto höher ist die Anzahl der abgeschlossenen Runden einer Teams.

%Zweites : Je höher der Individuelle Vertrauenswert, desto höher ist die Anzahl der abgeschlossen Runden.
% DIE HIER EVENTUELL BENUTZEN

% Der generelle Trust Score v  der generelle trust score von teams korreliert mit der anzahl abgerschlossener runden von teams mit der kondition ik stärker als der generelle trust score von teams ohne die kondition ik.
%Je höher der Hang zum Vertrauen der Personen im Team, desto höher ist die Anzahl der abgeschlossenen Runden des Teams.

\textbf{H5$_{0}$} : Der Zusammenhang der generelle Vertrauenswert eines Teams \ac{teamGen} und der realen Teameffektivität eines Teams \ac{tRound} mit der Kondition \ac{ik} unterscheidet sich nicht von den Teams mit der Kondition \ac{nik}. \\

%Besitzt nun Korr. A eine signifikanz und Korr B. nicht, weiss ich, dass die Kondition einen Einfluss auf die anz. abg. Runden besitzt. (Trotzdem sollte ich Fishers-Z-Transformation machen.
%Besitzt nun Korr. A eine signifikanz und Korr B. auch, kann ich schauen, ob diese sich signifikant unterscheiden. (Ich MUSS eine Fisher-Z-Transformation machen)
%Besitzt nun Korr. A keine signifikanz und Korr B. auch nicht, hat der generelle Trust Score keine Auswirkung auf die anz. abgeschlossenen Runden und es ist egal welche Kondition ich laut dieser Hypothese nehme. ( Ich müsste keine Fisher-Z-Transformation machen)
Um die Hypothese 5 wird auf \textit{Teamebene} analysiert. Dazu wird eine Tabelle (siehe \textbf{Tabelle \ref{TeamGenTabelle}}) aufgestellt, die den generellen Vertrauenswerte \ac{gt} der 3 Personen eines Versuchsdurchlauf zu einem generellen Vertrauenswert des Teams zusammenfasst. Diese Summe des generellen Vertrauenswert eines Teams wird im weiterem Verlauf \ac{teamGen} genannt.
Jedes der 3 Teammitglieder des Teams hat die gleiche Anzahl an Runden abgeschlossen. Daher besitzt diese 3 Teammitglieder als Team die gleiche Teameffektivität. Die Anzahl der abgeschlossenen Runden pro Team und pro Kondition wurde zu jeweils einer Variable für jedes Team zusammengefasst : \ac{tRoundIK} sowie \ac{tRoundNIK}.
Es wird eine Spearman-Korrelation zwischen \ac{tRound} und \ac{teamGen} durchgeführt. Damit überprüft werden kann, ob sich die beide Korrelationen signifikant voneinander unterscheiden, wird eine Fisher-Z-Transformation für unabhängige Stichproben duchgeführt.\\
%Darauffolgend wird auf Konditionsebene eine Korrelation durchgeführt um die Stichprobengröße zu vergrößern. Es folgt eine Einfache lineare-Regressionsanalyse zwischen \ac{tRound} und \ac{teamGen} um eine eventuelle Vorhersage des Zusammenhangs festzustellen. Kondition 1 definiert die Kondition \ac{ik} und Kondition 2 definiert die Kondition \ac{nik}.
%
%Um die Hypothese 5 auszuwerten, wird eine Tabelle aufgestellt, die die individuellen generellen Vertrauenswerte  der Personen zu einem generellen Vertrauenswert des Teams zusammenfasst. Die durch die Tabelle erstellte Variable der Summe der individuellen generellen Vertrauenswerte wird im weiterem Verlauf \ac{teamGen} genannt.
%Es wird eine Spearman-Korrelation zwischen \ac{tRound} und \ac{teamGen} auf Teamebene durchgeführt, wobei die Teams in verschiedenen Konditionen aufgeteilt werden.
%Darauffolgend wird auf Konditionsebene eine Korrelation durchgeführt um die Stichprobengröße zu vergrößern. Es folgt eine Einfache lineare-Regressionsanalyse zwischen \ac{tRound} und \ac{teamGen} um eine eventuelle Vorhersage des Zusammenhangs festzustellen. Kondition 1 definiert die Kondition \ac{ik} und Kondition 2 definiert die Kondition \ac{nik}.
%
Der \ac{teamGenIK} ist laut Shapiro-Wilk-Test mit ($p = .677 > \alpha = 0.05$) normalverteit. \newline
Der \ac{teamGenNIK} ist laut Shapiro-Wilk-Test mit ($p = .943 > \alpha = 0.05$) normalverteit. \newline
Der \ac{tRoundIK} ist laut Shapiro-Wilk-Test mit ($p = .844 > \alpha = 0.05$) normalverteit. \newline
Der \ac{tRoundNIK} ist laut Shapiro-Wilk-Test mit ($p = .453 > \alpha = 0.05$) normalverteit. \\
Das Lowess-Verfahren deutet Augenscheinlich auf keinen linearen Zusammenhang zwischen \ac{teamGenIK} und \ac{tRoundIK} sowie \ac{teamGenNIK} und \ac{tRoundNIK} hin. Daher wird die \textit{Linearitätsannahme nicht angenommen}.

Aufgrund der geringen Stichprobenumfangs ($tGenIK, tGenNIK : N=5$) und der nicht eindeutigen Linearität durch das Lowess-Verfahren, wurde sich für eine Spearman-Korrelation entschieden.

\paragraph{Spearman-Korrelation}
Es ist eine positive Korrelation schwachen Effektes (vgl. \cite{cohen2013statistical}) mit dem Spearman-Korrelationskoeffizient ($r = .200$) zwischen \textbf{\ac{teamGenIK} und \ac{tRoundIK}} vorhanden. Die Spearman-Korrelation ist \textbf{nicht signifikant} ($p = .747 > \alpha = 0.05$).\\
Weiterhin ist eine positive Korrelation moderaten Effektes (vgl. \cite{cohen2013statistical}) mit dem Spearman-Korrelationskoeffizient ($r = .359$) zwischen \textbf{\ac{teamGenNIK} und \ac{tRoundNIK}} vorhanden. Die Spearman-Korrelation ist \textbf{nicht signifikant} ($p = .553 > \alpha = 0.05$).

\paragraph{Fisher-Z-Transformation für unabhängige Stichproben}
Es wird eine Fisher-Z-Transformation für unabhängige Stichproben durchgeführt, obwohl die Stichprobe mit ($N=5$) sehr klein ist.
Die beiden Korrelationen unterscheiden sich laut Fisher-Z-Wert für unabhängige Stichproben ($Z=-.173$) \textbf{nicht signifikant} voneinander ($p = .431 > \alpha = 0.05$).

\begin{table}[H]
	\centering\footnotesize\setstretch{1.2}
	\caption{Tabelle des individuellen generellen Vertrauen und der erfolgreich abgeschlossenen Runden - Individuell und pro Team zusammengefasst. Kondition 1 definiert die Kondition \ac{ik} und Kondition 2 definiert die Kondition \ac{nik}.}
	\label{TeamGenTabelle}
	\begin{tabularx}{\textwidth}{r | p{2cm} p{1.5cm} p{3cm} p{1.5cm} p{1.5cm} p{1.5cm} p{1.5cm}} 
		ID & Individueller genereller Vertrauenswert & Kondition & Erfolgreich abgeschlossene Runden \newline(\ac{tRoundIK}, \ac{tRoundNIK}) & Team ID & Summe \ac{gt} (\ac{teamCogIK}, \ac{teamCogNIK})\\
		\hline 
		1 & $4,55$ & 1 & 4 & Team 1 & $13,95$\\
		2 & $4,8$ & 1 & 4 & &\\
		3 & $4,6$ & 1 & 4 & & \\
		\hline 
		4 & $4,6$ & 1 & 10 & Team 2 & $13,45$\\
		5 & $3,9$ & 1 & 10 & &\\
		6 & $4,95$ & 1 & 10 & &\\
		\hline 
		7 & $4,8$ & 1 & 7 & Team 3 & $13,35$\\
		8 & $4$ & 1 & 7 & & \\
		9 & $4,55$ & 1 & 7 & & \\
		\hline 
		10 & $4,95$ & 1 & 11 & Team 4 & $14,15$ \\
		11 & $4,5$ & 1 & 11 & & \\
		12 & $4,7$ & 1 & 11 & & \\
		\hline 
		13 & $4,3$ & 1 & 13 & Team 5 & $13,80$ \\
		14 & $4,95$ & 1 & 13 & & \\
		15 & $4,55$ & 1 & 13 & & \\
		\hline 
		16 & $4,9$ & 2 & 9 & Team 6 & $14,15$ \\
		17 & $4,85$ & 2 & 9 & &\\
		18 & $4,4$ & 2 & 9 & &\\
		\hline 
		19 & $4,6$ & 2 & 12 & Team 7 & $14,60$ \\
		20 & $4,3$ & 2 & 12 & & \\
		21 & $5,7$ & 2 & 12 & & \\
		\hline 
		22 & $4,85$ & 2 & 8 & Team 8 & $13,95$ \\
		23 & $4,5$ & 2 & 8 & & \\
		24 & $4,6$ & 2 & 8 & & \\
		\hline 
		25 & $4,5$ & 2 & 7 & Team 9 & $14,35$ \\
		26 & $4,95$ & 2 & 7 & & \\
		27 & $4,9$ & 2 & 7 & & \\
		\hline 
		28 & $4,9$ & 2 & 9 & Team 10 & $14,2$ \\
		29 & $4,9$ & 2 & 9 & &\\
		30 & $4,4$ & 2 & 9 & & \\	
	\end{tabularx}
\end{table}

%Auf dem individuellem Level konnten mit dem Kolmogoroff-Smirnoff-Test alle Daten auf Normalverteilung mit ($p>0.05$) nachgewiesen werden. Siehe \textbf{\autoref{fig:KolSmirInd}} \\
%Auf dem Teamlevel konnte bei \ac{tci} mit ($p = .039 < \alpha = 0.05$) keine Normalverteilung nachgewiesen werden. Siehe \textbf{\autoref{fig:KolSmirIndIK}} \\ Weiterhin konnte bei \ac{cpn} mit ($p = .044 < \alpha = 0.05$) keine Normalverteilung nachgewiesen werden. Siehe \textbf{\autoref{fig:KolSmirIndNONIK}}


%Der Shapiro-Wilk-Test zeigt, dass \ac{gti} mit  $p = .877 > \alpha = 0.05$ \textit{nicht normalverteilt} und \ac{gtn} mit $p = .184 > \alpha = 0.05$ ebenfalls \textit{nicht normalverteilt} ist. \\
%Weiterhin geht aus dem Shapiro-Wilk-Test hervor, dass \ac{cti} mit  $p = .052 > \alpha = 0.05$ \textit{nicht normalverteilt} und \ac{ctn} mit $p = .047 > \alpha = 0.05$ \textit{normalverteilt} ist. 


%Für \ac{gt} ist $ z_{Skewness} = -0,985 $ und $z_{Kurtosis} = -.416$ und somit ist \ac{gt} normalverteilt.
%Für \ac{ct} ist $ z_{Skewness} = -.257 $ und $z_{Kurtosis} = -1,764$ und somit ist \ac{ct} normalverteilt.
%Voraussetzung für Pearson-Korrelationsanalyse: 
%Die Variablen sind mindestens intervallskaliert -> Hier ordinalskaliert
%Die Variablen sind normalverteilt -> JA
%Der untersuchte Zusammenhang zwischen den Variablen muss linear sein -> JA
%Ab einer Likert Skala ab 6 können ordinale Skalenniveaus als metrische variablen interpretiert werden???
%\paragraph{Pearson-Korrelationsanalyse}
%Es ist eine positive Korrelation starken Effektes (vgl. \cite{cohen2013statistical}) mit dem Pearson-Korrelationskoeffizient ($r = .522$) zwischen \ac{gti} und \ac{cti} vorhanden. Die Pearson-Korrelation ist \textbf{nicht signifikant} ($p = .055 > \alpha = 0.05$).\\
%Weiterhin ist eine positive Korrelation moderaten Effektes (vgl. \cite{cohen2013statistical}) mit dem Pearson-Korrelationskoeffizient ($r = .306$) zwischen \ac{gtn} und \ac{ctn} vorhanden. Die Pearson-Korrelation ist \textbf{nicht signifikant} ($p = .267 > \alpha = 0.05$).
%
%\paragraph{Pearson-Korrelationsanalyse}
%Es ist eine positive Korrelation starken Effektes (vgl. \cite{cohen2013statistical}) mit dem Spearman-Korrelationskoeffizient ($r = .471$) zwischen \textbf{\ac{gti} und \ac{cti}} vorhanden. Die Spearman-Korrelation ist \textbf{nicht signifikant} ($p = .077 > \alpha = 0.05$).\\
%Weiterhin ist eine positive Korrelation moderaten Effektes (vgl. \cite{cohen2013statistical}) mit dem Spearman-Korrelationskoeffizient ($r = .239$) zwischen \textbf{\ac{gtn} und \ac{ctn}} vorhanden. Die Spearman-Korrelation ist \textbf{nicht signifikant} ($p = .390 > \alpha = 0.05$).

%
%Laut Kolmogoroff-Smirnoff-Test sind sowohl die Summe der individuellen generellen Vertrauenswerte des Teams (\ac{teamGen} $p = .200 > \alpha = 0.05$), sowie die Anzahl der erfolgreich abgeschlossenen Runden der Teams (\ac{tRound} $p = .200 > \alpha = 0.05$) Normalverteilt.
%Zu dem selben Ergebnis kommt auch der Shapiro-Wilk-Test mit (\ac{teamGen} $p = .548 > \alpha = 0.05$) und (\ac{tRound} $p = .953 > \alpha = 0.05$).
%
%Für \ac{tRound} ist $ z_{Skewness} = -0,448 $ und $z_{Kurtosis} = 0.0217$. Somit ist tRound Normalverteilt.
%Für \ac{teamGen} ist $ z_{Skewness} = 0,676 $ und $z_{Kurtosis} = -0.2653$. Somit ist teamGen Normalverteilt.
%
%Der General-Trust-Score pro Team (\ac{teamGenIK}) beträgt der Kondition \ac{ik} im Mittel ($\tilde x = 14,998$) mit einer Standardabweichung von ($\sigma = 0,648$).\newline 
%Der General-Trust-Score pro Team (\ac{teamGenNIK}) beträgt der Kondition \ac{nik} im Mittel ($\tilde x = 16,074$) mit einer Standardabweichung von ($\sigma = .830$). \newline 
%Die Anzahl der Abgeschlossenen Runden pro Team (\ac{tRoundIK}) beträgt der Kondition \ac{ik} im Mittel ($\tilde x = 9$) mit einer Standardabweichung von ($\sigma = 3,535$).\newline 
% Die Anzahl der Abgeschlossenen Runden pro Team (\ac{tRoundNIK}) beträgt der Kondition \ac{nik} im Mittel ($\tilde x = 9$) mit einer Standardabweichung von ($\sigma = 1,870$). 
%
%		\paragraph{Spearman-Korrelationsanalyse}
%Aufgrund der geringen Stichprobengröße von ($N=5$) wird eine Spearman-Korrelation zwischen den generellen Vertrauenswerten des Teams \ac{teamGen} und den erfolgreich abgeschlossenen Runden durchgeführt und in die unabhängigen Variablen \ac{ik} und \ac{nik} aufgeteilt.
%%
%%\begin{figure}[H]
%%\centering
%%		\begin{footnotesize}
%%			\includegraphics[scale=0.8]{Abbildungen/Post_QuestionnaireStatistiks/h5_korr_spearman_ik}
%%			\includegraphics[scale=0.8]{Abbildungen/Post_QuestionnaireStatistiks/h5_korr_spearman_nik}
%%			\caption{Pearson-Korrelation zwischen den einzelnen Teams und dem \ac{te}}
%%			\label{fig:Pearson_ik_nik_h3}
%%		\end{footnotesize}
%%	\end{figure}	
%
%Es ist eine negative Korrelation sehr starkem Effekt ($r = -.900$) zwischen \ac{teamGenIK} und \ac{tRoundIK} vorhanden. Diese Korrelation ist statistisch \textbf{signifikant} ($p = .037 < \alpha = 0.05$)
%Weiterhin ist eine negative Korrelation schwachen Effekt ($r = -.103$) zwischen \ac{teamGenNIK} und \ac{tRoundNIK} vorhanden. Diese Korrelation ist statistisch \textbf{nicht signifikant} ($p = .870 > \alpha = 0.05$).
%
%		\paragraph{Lineare-Regressionsanalyse}
%Um die statistische Signifikanz des Zusammenhangs zwischen \ac{tRoundIK} und \ac{teamGenIK} und die nicht statistische Signifikanz des Zusammenhangs der Kondition \ac{nik} zwischen \ac{tRoundNIK} und \ac{teamGenNIK} zu bestätigen, wird zusätzlich eine Einfache lineare-Regressionsanalyse durchgeführt.
%\ac{teamGen} wird wurde als unabhängige und \ac{tRound} als abhängige Variable definiert.
%
%\paragraph{Kondition \ac{ik}}
%Das Bestimmtheitsmaß ($R^{2} = .971$) deutet laut \citep{cohen2013statistical} auf eine sehr starke Varianzaufklärung der \ac{tRoundIK} durch \ac{teamGenIK} hin. Somit lassen sich $97,1\%$ der Varianzen unserer \ac{tRoundIK} durch den \ac{teamGenIK} erklären.
%
%Der Regressionskoeffizient der Variable \ac{teamGenIK} beträgt ($r = -.971$) und ist statistisch \textbf{signifikant} ($t(3) = -7,040; p = .006 < \alpha = 0.01$). \\
%Weiterhin ist das Ergebnis der ANOVA ($F(1,3) = 49,568; p = .006 < \alpha = 0.01$) und es lässt sich \textit{(Hypothese 5 / H5$_{0}$)} zum Signifikanzniveau ($\alpha = 0.01$) für die Kondition \ac{ik} verwerfen.
%
%Der \ac{teamGenIK} eignet sich als Prädiktor für \ac{tRoundIK}. Die geschätzte Abnahme an \ac{tRoundIK} beläuft sich auf $-.971$ Punkte \ac{tRoundIK} pro \ac{teamGenIK} ($\beta = -.971; t(3) = -7,040; p = .006 < \alpha = 0.01$). Siehe \textbf{\autoref{fig:h5_regression}}
%
%\paragraph{Kondition \ac{nik}}
%Das Bestimmtheitsmaß ($R^{2} = .187$) deutet laut \citep{cohen2013statistical} auf eine schwache Varianzaufklärung der \ac{tRoundNIK} durch \ac{teamGenNIK} hin. Somit lassen sich $18,7\%$ der Varianzen unserer \ac{tRoundNIK} durch den \ac{teamGenNIK} erklären.
%
%Der Regressionskoeffizient der Variable \ac{teamGenNIK} beträgt ($r = -.433$) und ist statistisch \textbf{nicht signifikant} ($t(3) = -.831; p = .467 > \alpha = 0.05$). \\
%Weiterhin ist das Ergebnis der ANOVA ($F(1,3) = .691; p = .467 > \alpha = 0.05$) und es lässt sich \textit{(Hypothese 5 / H5$_{0}$)} zum Signifikanzniveau ($\alpha = 0.05$) für die Kondition \ac{nik} nicht verwerfen.
%
%\ac{teamGenNIK} eignet sich nicht als Prädiktor für die \ac{tRoundNIK}. Siehe \textbf{\autoref{fig:h5_regression}}
%
%Die Hypothese \textit{(Hypothese 5 / H$_{0}$)} kann somit nicht angenommen werden, da ein abnehmender Effekt auf die \ac{tRound} pro Team durch die Konditionen \ac{ik} und \ac{nik} festgestellt wurde.
%
%\paragraph{Hypothese 5a} Analyse ohne die Aufteilung in \ac{ik} und \ac{nik}
%H5a$_{0}$ : Teams mit einem hohem General-Trust-Score (\ac{gt}) \textbf{schließen nicht mehr Runden ab} (\ac{tRound} als die mit einem niedrigen General-Trust-Score \ac{gt}.
%
%Da die Stichprobengröße mit lediglich 5 Teams pro Kondition bei der Korrelationsanalyse sowie der Einfache lineare-Regressionsanalyse zu wenig sind, wurde zusätzlich noch eine Korrelationsanalyse sowie Einfache lineare-Regressionsanalyse \textit{ohne} die Aufteilung in die Konditionen \ac{ik} und \ac{nik} durchgeführt.


%Laut Kolmogoroff-Smirnoff-Test sind sowohl die Summe der individuellen generellen Vertrauenswerte des Teams (\ac{teamGen} $p = .200 > \alpha = 0.05$), sowie die Anzahl der erfolgreich abgeschlossenen Runden der Teams (\ac{tRound} $p = .200 > \alpha = 0.05$) Normalverteilt.
%Zu dem selben Ergebnis kommt auch der Shapiro-Wilk-Test mit (\ac{teamGen} $p = .548 > \alpha = 0.05$) und (\ac{tRound} $p = .953 > \alpha = 0.05$).
%
%Der generelle Trust-Score der Teams (\ac{teamGen} beträgt im Mittel ($\tilde x = 15,536$) mit einer Standardabweichung von ($\sigma = .902$).
%Die Anzahl der Abgeschlossenen Runden (\ac{tRound} beträgt im Mittel ($\tilde x = 9$) mit einer Standardabweichung von ($\sigma = 2,666$). 
%
%			\paragraph{Spearman-Korrelationsanalyse}
%Es ist eine negative Korrelation mittlerem Effektes mit dem Pearson-Korrelationskoeffizient ($r = -.627$) zwischen \ac{teamGen} und \ac{tRound} vorhanden. Diese Korrelation ist statistisch \textbf{nicht signifikant} ($p = .052 > \alpha = 0.05$). Dies deutet darauf hin, dass die \textit{(Hypothese 5a / H5a$_{0}$)} nicht verworfen werden kann. Siehe \textbf{\autoref{fig:h5_both}}
%			
%			\paragraph{Lineare-Regressionsanalyse}
%Das Bestimmtheitsmaß ($R^{2} = .286$) deutet laut \citep{cohen2013statistical} auf eine hohe Varianzaufklärung der \ac{tRound} durch \ac{teamGen} hin. Somit lassen sich $28,6\%$ der Varianzen unserer \ac{tRound} durch den \ac{teamGen} erklären.
%
%Der Regressionskoeffizient der Variable \ac{teamGen} beträgt ($r = -.535$) und ist statistisch \textbf{nicht signifikant} ($t(8) = -1,791; p = .111 > \alpha = 0.05$). \\
%Weiterhin ist das Ergebnis der ANOVA ($F(1,8) = 3,206; p = .111 > \alpha = 0.05$) und es lässt sich \textit{(Hypothese 5a / H5a$_{0}$)} zum Signifikanzniveau ($\alpha = 0.05$) nicht verwerfen. \\
%Der \ac{teamGen} eignet sich nicht als Prädiktor für die \ac{tRound}.

%\newpage
%\paragraph{Analyse weiterer Variablen}
%Zusätzlich wurden alle weiteren, durch den Fragebogen erhobenen Variablen auf signifikante Mittelwertsunterschiede und Mediansunterschiede sowie auf signifikante Korrelationen überprüft.
%
%\paragraph{Signifikante unterschiede des Medians}
%Es sind signifikante Mittelwertsunterschiede bei der Variable \ac{tc} durch den t-Test sowie dem Mann-Whitney-U-Test zu erkennen.
%Es wurde durch den Mann-Whitney-U-Test überprüft, ob sich \ac{tc} nach Konditionen \ac{ik} oder \ac{nik} unterscheidet. Die Verteilungen der beiden Gruppen \ac{tci} und \ac{tcn} unterschieden sich laut Kolmogorov-Smirnov ($p = .925 > \alpha = 0.05$) \textbf{nicht signifikant} von einander. Es gab einen signifikanten Unterschied der Mediane des \ac{tc} zwischen ($\ac{tci} : \tilde x = 12,23$) und ($\ac{tcn} : \tilde x = 18,77)$, ($U = 63,500, Z = -2,062, p = .039 < \alpha = 0.05$).
%
%\paragraph{Signifikante Korrelationen}
%Es wurden signifikante Zusammenhänge mittels Spearman-Rho-Korrelation auf \underline{Individualebene} zwischen folgenden Variablen gefunden :\newline
%\acs{te} und \acs{tc} mit ($r = .695, p = .000 < \alpha = 0.05$)\newline
%\acs{te} und \acs{ct} mit ($r = .588, p = .001 < \alpha = 0.05$)\newline
%\acs{te} und \acs{cp} mit ($r = -.369, p = .045 < \alpha = 0.05$)\newline
%\acs{tc} und \acs{ct} mit ($r = .436, p = .016 < \alpha = 0.05$)\newline
%\acs{scp} und \acs{ipq} mit ($r = .585, p = .001 < \alpha = 0.05$)\newline
%\acs{tp} und \acs{ipq} mit ($r = .723, p = .000 < \alpha = 0.05$)\newline
%\acs{sp} und \acs{cp} mit ($r = .869, p = .000 < \alpha = 0.05$)\newline
%
%Es wurden signifikante Zusammenhänge mittels Spearman-Rho-Korrelation auf \underline{Konditionsebene} zwischen folgenden Variablen gefunden :
%
%\underline{\ac{ik}} :\newline
%\acs{tei} und \ac{tci} mit ($r = .767, p = .001 < \alpha = 0.05$)\newline
%\acs{scpi} und \acs{ipqi} mit ($r = .515, p = .050 < \alpha = 0.05$)\newline
%\acs{tpi} und \acs{ipqi} mit ($r = .747, p = .001 < \alpha = 0.05$)\newline
%\acs{spi} und \acs{cpi} mit ($r = .884, p = .000 < \alpha = 0.05$)\newline
%\acs{tlxi} und \acs{cpi} mit ($r = .574, p = .048 < \alpha = 0.05$)\newline
%\acs{tlxi} und \acs{tci} mit ($r = .652, p = .008 < \alpha = 0.05$)\newline
%\acs{tlxi} und \acs{tei} mit ($r = .555, p = .032 < \alpha = 0.05$)\newline
%\newline
%\underline{\ac{nik}} :\newline
%\acs{ten} und \acs{tcn} mit ($r = .635, p = .011 < \alpha = 0.05$)\newline
%\acs{ten} und \acs{ctn} mit ($r = .767, p = .001 < \alpha = 0.05$)\newline
%\acs{ten} und \acs{cpn} mit ($r = -.647, p = .009 < \alpha = 0.05$)\newline
%\acs{tcn} und \acs{ctn} mit ($r = .630, p = .012 < \alpha = 0.05$)\newline
%\acs{scpn} und \acs{ipqn} mit ($r = .681, p = .005 < \alpha = 0.05$)\newline
%\acs{tpn} und \acs{ipqn} mit ($r = .583, p = .022 < \alpha = 0.05$)\newline
%\acs{spn} und \acs{cpn} mit ($r = .884, p = .000 < \alpha = 0.05$)\newline
%%

\begin{table}[H]
	\centering\footnotesize\setstretch{1}
	\caption{Übersicht zur Auswertung der Hypothesen 1 - 3}
	\label{VariableBreakdown1}
	\begin{tabularx}{\textwidth}{p{3cm} | p{3.5cm} | p{3.5cm} | p{3.5cm}} 
		Was wurde gemessen? & Hypothese 1 & Hypothese 2 & Hypothese 3 \\
		\hline
		Shapiro-Wilk-Test
		&
		$\underline{\ac{ct}}:$\newline$p=.23<\alpha=0.05^*$ \newline		
		& 
		$\underline{\ac{gti}}:$\newline$p=.078>\alpha=0.05$\newline 
		$\underline{\ac{gtn}}:$\newline$p=.020<\alpha=0.05^*$ \newline
		$\underline{\ac{cti}}:$\newline$p=.356>\alpha=0.05$ \newline 
		$\underline{\ac{cti}}:$\newline$p=.054<\alpha=0.05$ \newline
		& 
		$\underline{\ac{teamCogIK}}:$\newline $p=.017<\alpha=0.05^*$ \newline 
		$\underline{\ac{teamCogNIK}}:$\newline $p=.713>\alpha=0.05$ \newline
		$\underline{\ac{tRoundIK}}:$\newline $p=.844>\alpha=0.05$ \newline 
		$\underline{\ac{tRoundNIK}}:$\newline $p=.453>\alpha=0.05$ \\ 
	
%		\hline 
%		
%		T-Test 
%		&  
%		&
%%		& $t(28)=-1,685$\newline$p=.103>\alpha=0.05$
%		&  \\ 

		\hline 		
		
		Mann-Whitney-U 
		& 
		Kolm. Smirnov:\newline$p=.181>\alpha=.05$\newline
		Mann-Whitney-U :\newline
		$U=62,500$\newline $Z=-2,083$ \newline 
		$p=.037<\alpha=.05$ \newline 
		$r=-.3803$
		& 
		&  \\
		
		\hline 				
		
%		Varianzgleichheit
%		&  
%		& $L=.759$\newline$p=.392>\alpha=0.05$
%		&   \\ 
%
%		\hline 		
		
%		Pearson-Korrelation 
%		& $r = .452$\newline$p=.012<\alpha=0.05$
%		& 
%		&  \\
%		
%		\hline 		
		
		Spearman-Roh Korrelation 
		&
		& $\underline{\ac{gti}}$ und \newline $\underline{\ac{cti}}:$\newline
		$r=.441$\newline
		$p=.100>\alpha=0.05$\newline
		$\underline{\ac{gtn}}$ und $\underline{\ac{ctn}}:$\newline
		$r=.378$\newline
		$p=.165>\alpha=0.05$\newline
		& $\underline{\ac{teamCogIK}}$ und $\underline{\ac{tRoundIK}}:$\newline
		$r=.-.205$\newline$p=.741>\alpha=0.05$\newline
		$\underline{\ac{teamCogNIK}}$ und $\underline{\ac{tRoundNIK}}:$\newline
		$r=.462$\newline
		$p=.434>\alpha=0.05$ \\
		
		\hline 		
		
%		ANOVA 
%		& $F(1,28)=7,202$\newline$p=.012<\alpha=0.05$
%		&  
%		& $\underline{\ac{teamCogIK}}:$\newline
%		$F(1,3)=0,267$\newline$p=.641>\alpha=0.05$\newline
%		$\underline{\ac{teamCogNIK}}:$\newline
%		$F(1,3)=5.363$\newline$p=.103>\alpha=0.05$\\ 
%		
%		\hline 
			
		Fishers-Z-Transformation für unabh. Stichp.
		&
%		& $R^{2}=.205$\newline$\beta=.452$\newline$t(28)=2,684$\newline$p=.012<\alpha=0.05$
		&  
		& $\underline{r=.-.205; N=5}$ und $\underline{r=.741; N=5} :$\newline
		$Z=-.708$\newline
		$p=.240>\alpha=0.05$ \\
		
		\hline 
	\end{tabularx}
\end{table}		

\begin{table}[H]
	\centering\footnotesize\setstretch{1}
	\caption{Übersicht zur Auswertung der Hypothesen 3 - 5}
	\label{VariableBreakdown2}
	\begin{tabularx}{\textwidth}{p{3cm} | p{3.5cm} | p{3.5cm}} 
		Was wurde gemessen? & Hypothese 4 & Hypothese 5  \\
		\hline
		Shapiro-Wilk-Test
		&
		$\underline{\ac{tRound}}:$\newline$p=.113>\alpha=0.05$\newline 
		& 
		$\underline{\ac{teamGenIK}}:$\newline$p=.677>\alpha=0.05$\newline 
		$\underline{\ac{teamGenNIK}}:$\newline$p=.943>\alpha=0.05$\newline 
		$\underline{\ac{tRoundIK}}:$\newline$p=.844>\alpha=0.05$\newline 
		$\underline{\ac{tRoundNIK}}:$\newline$p=.453>\alpha=0.05$\\ 
	
		\hline 
		
%		T-Test 
%		& $t(21,270)=.00$\newline$p=1.00>\alpha=0.05$
%		& \\
%		& $t(28)=-1,685$\newline$p=.103>\alpha=0.05$
%		&  \\ 
%
%		\hline 		
		
		Mann-Whitney-U 
		& 
		Kolm. Smirnov:\newline$p=.181>\alpha=.05$\newline
		Mann-Whitney-U :\newline
		$U=103,500$\newline 
		$Z=-.377$ \newline 
		$p=-.706>\alpha=.05$ \newline 
		$r=-.0602$
		& 
%		Kolm. Smirnov:\newline$p=.925>\alpha=.05$)\newline 
%		$U=74,000$\newline $Z=-1.615$ \newline 
%		$p=.106>\alpha=.05$ \newline 
%		$r=-.294$\\
%		
		\\
		\hline 				
		
		Varianzgleichheit
		&  $L=10,182$\newline$p=.003<\alpha=0.05$
		& \\
%		&   \\ 

		\hline 		
		
%		Pearson-Korrelation 
%		& $r = .452$\newline$p=.012<\alpha=0.05$
%		& 
%		&  \\
%		
%		\hline 		
		
		Spearman-Roh Korrelation 
		&
		&$\underline{\ac{teamGenIK}}$ und \newline $\underline{\ac{tRoundIK}}:$\newline
		$r=.200$\newline
		$p=.747>\alpha=0.05$\newline
		$\underline{\ac{teamGenNIK}}$ und $\underline{\ac{tRoundNIK}}:$\newline
		$r=.359$\newline
		$p=.553>\alpha=0.05$ \\
		
		\hline 		
		
%		ANOVA 
%		& $F(1,28)=7,202$\newline$p=.012<\alpha=0.05$
%		&  
%		& $\underline{\ac{teamCogIK}}:$\newline
%		$F(1,3)=0,267$\newline$p=.641>\alpha=0.05$\newline
%		$\underline{\ac{teamCogNIK}}:$\newline
%		$F(1,3)=5.363$\newline$p=.103>\alpha=0.05$\\ 
%		
%		\hline 
			
		Fishers-Z-Transformation für unabh. Stichp.
		&
%		& $R^{2}=.205$\newline$\beta=.452$\newline$t(28)=2,684$\newline$p=.012<\alpha=0.05$
		&  $\underline{r=.200; N=5}$ und $\underline{r=.359; N=5} :$\newline
		$Z=-.173$\newline
		$p=.431>\alpha=0.05$ \\
		
		\hline 	
	
%	\begin{tabularx}{\textwidth}{p{3cm} | p{3.5cm} | p{4cm} | p{3.5cm}} 
%		Was wurde gemessen? & Hypothese 3a & Hypothese 4 & Hypothese 5 \\
%		\hline \\
%		Kolmogoroff-Smirnov-Test 
%		& $\underline{\ac{teamCog}}:$\newline$p=.200>\alpha=0.05$\newline 
%		$\underline{\ac{tRound}}:$\newline$p=.200>\alpha=0.05$ 
%		&
%		& $\underline{\ac{teamGen}}:$\newline$p=.200>\alpha=0.05$\newline 
%		$\underline{\ac{tRound}}:$\newline$p=.200>\alpha=0.05$ \\
%	
%		\hline 
%		
%		T-Test 
%		&  
%		& $\underline{Teamebene}:$\newline $t(8)=0.000$\newline$p=1>\alpha=0.05$ \newline
%		$\underline{Konditionsebene}:$\newline $t(21,270)=0.000$\newline$p=1>\alpha=0.05$
%		&  \\ 
%
%		\hline 		
%		
%		Mann-Whitney-U 
%		& 
%		& Kolm. Smirnov:\newline($p=.181>\alpha=.05$)\newline $U=103.500\newline Z=-.377 \newline p=.734>\alpha=.05 \newline r=-.068$
%		&  \\		
%		
%		\hline 		
%		
%		Varianzgleichheit
%		&  
%		& $\underline{Teamebene}:$\newline $L=2,909$\newline$p=.126>\alpha=0.05$\newline
%		$\underline{Konditionsebene}:$\newline $L=10,182$\newline$p=.003<\alpha=0.05$\newline
%		& \\ 
%
%		\hline 		
%		
%%		Pearson-Korrelation 
%%		& 
%%		& 
%%		&  \\
%		
%		\hline 	
%			
%		Spearman-Roh Korrelation 
%		& $r = .265$\newline$p=.460>\alpha=0.05$
%		& 
%		& $\underline{\ac{teamGenNIK}}:$\newline
%		$r = -.900$\newline$p=.037<\alpha=0.05$
%		$\underline{\ac{teamGenIK}}:$\newline
%		$r = -.103$\newline$p=.870>\alpha=0.05$\\
%		
%		\hline 
%		ANOVA 
%		& $F(1,8)=1,198$\newline$p=.306>\alpha=0.05$
%		&  
%		& $\underline{\ac{teamGenIK}}:$\newline
%		$F(1,3)=49,568$\newline$p=.006<\alpha=0.05$\newline
%		$\underline{\ac{teamGenNIK}}:$\newline
%		$F(1,3)=.691$\newline$p=.467>\alpha=0.05$\\ 
%		
%		\hline 
%			
%		Lineare-Regressionsanalyse
%		& $R^{2}=.130$\newline$\beta=.361$\newline$t(8)=1,095$\newline$p=.306>\alpha=0.05$
%		&  
%		&  $\underline{\ac{teamGenIK}}:$\newline
%		$R^{2}=.971$\newline$\beta=-.971$\newline$t(3)=-7,040$\newline$p=.006<\alpha=0.05$\newline 
%		$\underline{\ac{teamGenNIK}}:$\newline
%		$R^{2}=.187$\newline$\beta=-.433$\newline$t(3)=-.831$\newline$p=.467>\alpha=0.05$\newline \\ 		
%		
	\end{tabularx}
\end{table}		

%\begin{table}[H]
%	\centering\footnotesize\setstretch{1}
%	\caption{Auswertung Hypothese 5a -}
%	\label{VariableBreakdown3}
%	\begin{tabularx}{\textwidth}{p{3cm} | p{3.5cm} | p{4cm} | p{3.5cm}} 
%		Was wurde gemessen? & Hypothese 5a &  &  \\
%		\hline \\
%		Kolmogoroff-Smirnov-Test 
%		& $\underline{\ac{teamGen}}:$\newline$p=.200>\alpha=0.05$\newline 
%		$\underline{\ac{tRound}}:$\newline$p=.200>\alpha=0.05$ \newline \newline 
%		&
%		& \\
%	
%		\hline 
%		
%%		T-Test 
%		&  
%		& 
%		&  \\ 
%
%		\hline 		
%		
%		Mann-Whitney-U 
%		& 
%		& 
%		&  \\		
%		
%		\hline 		
%		
%		Varianzgleichheit
%		&  
%		& 
%		& \\ 
%
%		\hline 		
%		
%		Pearson-Korrelation 
%		& 
%		& 
%		&  \\
%		
%		\hline 	
%			
%		Spearman-Roh Korrelation 
%		& $r = -.627$\newline$p=.052>\alpha=0.05$\newline \newline 
%		& 
%		& \\
%		
%		\hline 
%		ANOVA 
%		& $F(1,8)=3,206$\newline$p=.111>\alpha=0.05$ \newline \newline 
%		&  
%		& \\ 
%		
%		\hline 
%			
%		Lineare-Regressionsanalyse
%		& $R^{2}=.286$\newline$\beta=-.535$\newline$t(8)=-1,791$\newline$p=.111>\alpha=0.05$  \newline \newline 
%		&  
%		&  \\ 		
%		
%	\end{tabularx}
%\end{table}
In der folgenden Tabelle wird beschrieben, ob die einzelnen Hypothesen Signifikanz aufweisen.

%\subsection{Analyse Hypothese 7}
%

% WO IST MEHR WAHRGENOMENE TEAMEFFEKTIVITÄT VORHANDEN??

%Unterschiedliche wahrgenommene Teameffektivität bei unterschiedlichen Avatar Konditionen.

%Auch interessant : T-Test zwischen COPRÄSENZ und WieSahenIhreMitspielerAus
%
%Auch interessant : Wie wirkt sich gt und CT gemeinsam auf TE aus? ( Auch im Team )
%
%Auch interssant : Rounds Done im Verhältnis zu Team-Effectiveness-Scale
%Auch interessant : Wie ist das Verhältnis von gedachtem Effektivitätswert zu eigentlichem tRounds?
%
%
%Auch interessant : NASA-TLX( WIE ANSTRENGEND WAR ES?) zur Round-Efficiency
%
%Auch interessant : T-Test von Team-Communication bei den IK und NON IK Gruppen
%	Hier signifikanten unterschied der Mittelwerte gefunden!
%	
%Auch interessant : Team-Communication zur Rounds Done


%AUch interessant :   Schlechte Teams sollten auch ein schlechteres Cognitives Vertrauen besitzen! Ist das der Fall??

%-------------------------------------------------------------------------------
% WAR MEHR PRÄSENZ BEI IK ODER NIK AVATAR VORHANDEN?
%KAPITEL : VR - Präsenz in Virtual Reality

%KAPITEL : Vertrauen als Zustand - Kognitives und Affektives Vertrauen
%Ich glaube, ich habe die Aufgaben des Versuchsleiters gut erledigt -> Auf Kognitives Vertrauen analisieren Pro person! Hier ist einseitige Signifikanz in der Korrelation von Wie Erfolgreich haben Sie ihrer Meinung nach die vom Versuchleiter ... mit Cog_Trust_Windsorizing vorhanden!

% Self-Avatare

%Jedoch wurde ein größeres Gefühl von Zusammenheit festgestellt, wenn mit einem Menschenähnlichem Avatar interagiert wurde. \citep{kerr1983motivation}
%Dies auch evenntl überprüfen?
\newpage
\subsection{Auswertung subjektiver Daten}
Im folgenden Werden die subjektiven Fragebögen im Bezug auf die beiden Avatarkonditionen ausgewertet.

Die subjektiven Daten wurden auf Extremwerte überprüft. Jedoch lassen sich diese Extremwerte nicht pauschal als Ausreißer klassifizieren. Dies geht aus den jeweiligen Feedback der Nutzer hervor, sowie dem Vergleich anderer angegebener Werte der unterschiedlichen Fragebögen. Der milde Extremwert der \textit{Presence}, der \textit{Selbst wahrgenommenen Co-Presence}, der \textit{Wahrgenommenen Co-Presence des \dq{}anderen\dq{}}, sowie der \textit{Telepresence} werden als plausibel eingestuft.
Der starke Extremwert (ID9) der \textit{Selbst wahrgenommenen Co-Presence} schrieb als Feedback \dq{}\textit{Es war eine spannende Erfahrung am VR-Experiment teilzunehmen und hat Spaß gemacht. Die Personen waren mir nah, obwohl sie eindeutig unreal wirkten}\dq{}, weshalb dieser starke Extremwert als plausibel eingestuft wird und nicht verworfen werden kann.
Der starke Extremwert (ID25) der \textit{Wahrgenommenen Co-Presence des anderen} wird ebenfalls als plausibel eingestuft. Dieser erzielt in der \dq{}Social-Presence\dq{} sowie der \dq{}Telepresence\dq{} ebenfalls überdurchschnittlich hohe Werte. Daher wird auch dieser als Plausibel eingestuft und kann nicht verworfen werden.

\paragraph{Presence}
Die \acf{ipq} ist laut Shapiro-Wilk-Test mit ($p = .150 > \alpha = 0.05$) normalverteit. 
Die \ac{ipqi} beträgt bei Kondition \ac{ik} im Mittel ($\tilde x = 4.1571$) mit einer Standardabweichung von ($\sigma = .70152$).
Die \ac{ipqn} beträgt bei Kondition \ac{nik} im Mittel ($\tilde x = 4.2762$) mit einer Standardabweichung von ($\sigma = .53443$).
Der T-Test für unabhängige Stichproben zeigt \textbf{keine signifikanten Unterschiede} der Mittelwerte  ($t(28) = -.523; p = .606 > \alpha = 0.05$).

	\begin{figure}[H]
		\begin{footnotesize}
		\centering
			\includegraphics[scale=0.3]{Abbildungen/AuswertungDiagramme/BP_Presence.png}
			\caption[Abbildung 1]{Boxplot der Wahrgenommenen Presence}
			\label{BP_Presence}
		\end{footnotesize}
	\end{figure}

\paragraph{Selbst wahrgenommene Co-Präsenz}
Die \acf{scp} ist laut Shapiro-Wilk-Test mit ($p = .001 < \alpha = 0.05$) nicht normalverteit.
Die \ac{scpi} beträgt bei Kondition \ac{ik} im Mittel ($\tilde x = 3.044$) mit einer Standardabweichung von ($\sigma = .4733$).\newline 
Der \ac{scpn} beträgt bei der Kondition \ac{nik} im Mittel ($\tilde x = 2.9000$) mit einer Standardabweichung von ($\sigma = .21639$). Laut Mann-Whitney-U-Test gab es \textbf{keinen signifikanten Unterschied} der Mittelwerte ($ Z = -.595;. p = .552 > \alpha = 0.05$).

	\begin{figure}[H]
		\begin{footnotesize}
		\centering
			\includegraphics[scale=0.3]{Abbildungen/AuswertungDiagramme/BP_SelfCoPresence.png}
			\caption[Abbildung 1]{Boxplot der selbst wahrgenommenen Co-Presence}
			\label{BP_Presence}
		\end{footnotesize}
	\end{figure}

\paragraph{Wahrgenommene Co-Präsenz des anderen}
Die \acf{ocp} ist laut Shapiro-Wilk-Test mit ($p = .001 < \alpha = 0.05$) nicht normalverteit.
Die \ac{ocpi} beträgt bei Kondition \ac{ik} im Mittel ($\tilde x = 2.977$) mit einer Standardabweichung von ($\sigma = .2928$).
Der \ac{ocpn} beträgt beider Kondition \ac{nik} im Mittel ($\tilde x =3.1333$) mit einer Standardabweichung von ($\sigma =.37108$). Laut Mann-Whitney-U-Test gab es \textbf{keinen signifikanten Unterschied} der Mittelwerte ($ Z = -1.092; p = .275 > \alpha = 0.05$).

	\begin{figure}[H]
		\begin{footnotesize}
		\centering
			\includegraphics[scale=0.3]{Abbildungen/AuswertungDiagramme/BP_OtherCoPresence.png}
			\caption[Abbildung 1]{Boxplot der wahrgenommenen Co-Presence der \dq{}anderen\dq{}}
			\label{BP_Presence}
		\end{footnotesize}
	\end{figure}

\paragraph{Telepresence} 
Die \acf{tp} ist laut Shapiro-Wilk-Test mit ($p = .024 < \alpha = 0.05$) nicht normalverteit. 
Die \ac{tpi} beträgt bei Kondition \ac{ik} im Mittel ($\tilde x = 5.2133$) mit einer Standardabweichung von ($\sigma = 1.2659$).
Der \ac{tpn} beträgt bei der Kondition \ac{nik} im Mittel ($\tilde x = 5.3600$) mit einer Standardabweichung von ($\sigma = .7971$). Laut Mann-Whitney-U-Test gab es \textbf{keinen signifikanten Unterschied} der Mittelwerte ($ Z = -.021; p = .983 > \alpha = 0.05$).

	\begin{figure}[H]
		\begin{footnotesize}
		\centering
			\includegraphics[scale=0.3]{Abbildungen/AuswertungDiagramme/BP_Telepresence.png}
			\caption[Abbildung 1]{Boxplot der wahrgenommenen Telepresence}
			\label{BP_Presence}
		\end{footnotesize}
	\end{figure}

\paragraph{Social-Presence} 
Die \acf{sp} ist laut Kolmogoroff-Smirnov-Test mit ($p = .031 < \alpha = 0.05$) nicht normalverteit. 
Die \ac{spi} beträgt bei Kondition \ac{ik} im Mittel ($\tilde x = 4.685$) mit einer Standardabweichung von ($\sigma = 1.883$).
Der \ac{spn} beträgt bei der Kondition \ac{nik} im Mittel ($\tilde x = 4.4952$) mit einer Standardabweichung von ($\sigma = 1.54039$). Laut Mann-Whitney-U-Test gab es \textbf{keinen signifikanten Unterschied} der Mittelwerte ($ Z = -.062; p = .950 > \alpha = 0.05$).

	\begin{figure}[H]
		\begin{footnotesize}
		\centering
			\includegraphics[scale=0.3]{Abbildungen/AuswertungDiagramme/BP_SocialPresence.png}
			\caption[Abbildung 1]{Boxplot der wahrgenommenen Social-Presence}
			\label{BP_Presence}
		\end{footnotesize}
	\end{figure}

\paragraph{NASA-TLX \dq{}Anstrengungsmaß\dq{}} 
Der \acf{tlx} ist laut Kolmogoroff-Smirnov-Test mit ($p = .337 > \alpha = 0.05$) normalverteit. 
Der \ac{tlxi} beträgt bei Kondition \ac{ik} im Mittel ($\tilde x = 8.1667$) mit einer Standardabweichung von ($\sigma = 1.5443$).
Der \ac{tlxn} beträgt bei der Kondition \ac{nik} im Mittel ($\tilde x = 9.033$) mit einer Standardabweichung von ($\sigma = 2.5813$). Der T-Test für unabhängige Stichproben zeigt \textbf{keine Signifikanten Unterschied} der Mittelwerte ($t(28) = -1.116; p = .276 > \alpha = 0.05$).

	\begin{figure}[H]
		\begin{footnotesize}
		\centering
			\includegraphics[scale=0.3]{Abbildungen/AuswertungDiagramme/BP_Anstrengungsmaß.png}
			\caption[Abbildung 1]{Boxplot des Anstrengungsmaßes}
			\label{BP_Presence}
		\end{footnotesize}
	\end{figure}

\paragraph{Team-Kommunikation} 
Die \acf{tc} ist laut Kolmogoroff-Smirnov-Test mit ($p = .018 < \alpha = 0.05$) nicht normalverteit. 
Die \ac{tci} beträgt bei Kondition \ac{ik} im Mittel ($\tilde x = 4.0133$) mit einer Standardabweichung von ($\sigma = .6653$).
Der \ac{tcn} beträgt bei der Kondition \ac{nik} im Mittel ($\tilde x = 4.48$) mit einer Standardabweichung von ($\sigma = .47689$). Laut Mann-Whitney-U-Test gab es einen \textbf{signifikanten Unterschied} der Mittelwerte ($ Z = -2.062; p = .039 < \alpha = 0.05$).

	\begin{figure}[H]
		\begin{footnotesize}
		\centering
			\includegraphics[scale=0.3]{Abbildungen/AuswertungDiagramme/BP_Teamkommunikation.png}
			\caption[Abbildung 1]{Boxplot der wahrgenommenen Teamkommunikation}
			\label{BP_Presence}
		\end{footnotesize}
	\end{figure}

	\begin{figure}[H]
		\begin{footnotesize}
		\centering
			\includegraphics[scale=0.3]{Abbildungen/AuswertungDiagramme/TC_Mittelwerte.png}
			\caption[Abbildung 1]{Mittelwerte und Standardabweichung der kognitiven Vertrauenswerte. Links die Kondition \ac{ik} und Rechts die Kondition \ac{nik}}
			\label{H1_Mittelwerte}
		\end{footnotesize}
	\end{figure}
\newpage

\paragraph{Wahrgenommene Team-Effektivität}
Die \acf{te} ist laut Kolmogoroff-Smirnov-Test mit ($p = .065 < \alpha = 0.05$) normalverteit. 
Die \ac{tei} beträgt bei Kondition \ac{ik} im Mittel ($\tilde x = 4.440$) mit einer Standardabweichung von ($\sigma = 1.0933$).
Die \ac{ten} beträgt bei der Kondition \ac{nik} im Mittel ($\tilde x = 4.8533$) mit einer Standardabweichung von ($\sigma = .81929$). Laut Mann-Whitney-U-Test gab es einen \textbf{keinen signifikanten Unterschied} der Mittelwerte ($ Z = -1.172; p = .251 > \alpha = 0.05$).

	\begin{figure}[H]
		\begin{footnotesize}
		\centering
			\includegraphics[scale=0.3]{Abbildungen/AuswertungDiagramme/BP_Teameffektivität.png}
			\caption[Abbildung 1]{Boxplot der wahrgenommenen Teameffektivität}
			\label{BP_Presence}
		\end{footnotesize}
	\end{figure}

\paragraph{Weitere Analysen}
Es ist eine positive Korrelation starken Effektes (vgl. \cite{cohen2013statistical}) mit dem Spearman-Korrelationskoeffizient ($r = .729$) zwischen \textbf{\ac{ctn} und \ac{ten}} auf Konditionsebene zu erkennen. Die Spearman-Korrelation ist \textbf{signifikant} ($p = .002 < \alpha = 0.05$). Siehe \textbf{\textit{Abbildung \ref{SubSig1}}}\\

\begin{figure}[h]
  \centering
  \subfloat[][]{\includegraphics[width=0.45\linewidth]{Abbildungen/AuswertungDiagramme/ctn_ten_KorrIK.png}}
  \qquad
  \subfloat[][]{\includegraphics[width=0.45\linewidth]{Abbildungen/AuswertungDiagramme/ctn_ten_KorrNIK.png}}
  \caption{Spearman-Korrelationen der kognitiven Vertrauenswerte auf Konditionslevel mit der subjektiven Teameffektivität. Links (a) für die Kondition \ac{ik} und Rechts (b) für die Kondition \ac{nik}}
  \label{SubSig1}
\end{figure}

Weiterhin ist eine positive Korrelation starken Effektes (vgl. \cite{cohen2013statistical}) mit dem Spearman-Korrelationskoeffizient ($r = .532$) zwischen \textbf{\ac{ctn} und \ac{tcn}} auf Konditionsebene zu erkennen. Die Spearman-Korrelation ist \textbf{signifikant} ($p = .041 < \alpha = 0.05$). Siehe \textbf{\textit{Abbildung \ref{SubSig2}}}\\

\begin{figure}[h]
  \centering
  \subfloat[][]{\includegraphics[width=0.45\linewidth]{Abbildungen/AuswertungDiagramme/ctn_tcn_KorrIK.png}}
  \qquad
  \subfloat[][]{\includegraphics[width=0.45\linewidth]{Abbildungen/AuswertungDiagramme/ctn_tcn_KorrNIK.png}}
  \caption{Spearman-Korrelationen der kognitiven Vertrauenswerte mit der wahrgenommenen Teamkommunikation. Links (a) für die Kondition \ac{ik} und Rechts (b) für die Kondition \ac{nik}}
  \label{SubSig2}
\end{figure}

\newpage

\begin{table}[H]
	\centering\footnotesize\setstretch{0.8}
	\caption{Variablen, Mittelwerte, Standartabweichungen und Anzahl der Teilnehmer}
	\label{MittelwerteUndCo}
	\begin{tabularx}{\textwidth}{p{3cm} | p{3.0cm} p{1.5cm} p{1.5cm} p{2cm} p{1cm} p{2cm}} 
		Was wurde gemessen? & Abkürzung & Min/Max & Erreicht Min/Max & Mittelwert & Std. Abweichung & $N$ \\
		\hline \\
		Genereller Hang zum Vertrauen &\ac{gt} & $1/7$ & $3.9/5.7$ & $4.665$ & $.3371$ & $30$ \\ \\
		Kognitives Vertrauen &\ac{ct} & $1/7$ & $2.57/5.14$ & $4.357$ & $.6.252$ & $30$ \\ \\
		Kognitives Vertrauen-IK &\ac{cti} & $1/7$ & $2.57/5.00$ & $4.1238$ & $.6719$ & $15$ \\ \\
		Kognitives Vertrauen-NIK &\ac{ctn} & $1/7$ & $3.57/5.14$ & $4.590$ & $.4915$ & $15$ \\ \\
		Team-Kommunikation &\ac{tcn} & $1/5$ & $3/5$ & $4.2467$ & $.61629$ & $30$ \\ \\
		Team-Kommunikation-IK &\ac{tci} & $1/5$ & $3/5$ & $4.0133$ & $.66533$ & $15$ \\ \\
		Team-Kommunikation-NIK &\ac{tcn} &$1/5$ & $3.4/5$ & $4.48$ & $.47689$ & $15$ \\ \\
		Wahrg.Team-Effektivität &\ac{te} & $1/7$ & $2.2/6.6$ & $4.646$ & $.9722$ & $30$ \\ \\
		Wahrg.Team-Effektivität-IK &\ac{tei} & $1/7$ & $2.2/5.8$ & $4.4$ & $1.093$ & $15$ \\ \\
		Wahrg.Team-Effektivität-NIK &\ac{ten} & $1/7$ & $3.4/6.6$ & $4.853$ & $.8192$ & $15$ \\ \\
		NASA-TLX &\ac{tlx} & $1/21$ &$5.17/14$ & $8.6$ & $2.135$ & $30$ \\ \\
		NASA-TLX-IK &\ac{tlxi} & $1/21$ & $5.67/10.67$ & $8.166$ & $1.544$ & $15$ \\ \\
		NASA-TLX-NIK&\ac{tlxn} & $1/21$ & $5.17/14.00$ & $9.033$ & $2.581$ & $15$ \\ \\
		IPQ & \ac{ipq} & $1/7$ & $2.43/5.43$ & $4.216$ & $.6157$ & $30$ \\ \\
		IPQ-IK & \ac{ipqi} & $1/7$ & $2.43/4.93$ & $4.1571$ & $.7015$ & $30$ \\ \\
		IPQ-NIK & \ac{ipqn} & $1/7$ & $3.36/5.43$ & $4.276$ & $.5344$ & $30$ \\ \\
		Self - Co-Presence &\ac{cp} & $1/5$ & $2.5/4.17$ & $2.9722$ & $.3690$ & $30$ \\ \\
		Self - Co-Presence-IK &\ac{cpi} & $1/5$ & $2.5/4.17$ & $3.044$ & $.4733$ & $15$ \\ \\
		Self - Co-Presence-NIK &\ac{cpn} & $1/5$ & $2.5/3.33$ & $2.900$ & $.2163$ & $15$ \\ \\
		Others - Co-Presence &\ac{cp} & $1/5$ & $2.42/4.25$ & $3.055$ & $.3378$ & $30$ \\ \\
		Others - Co-Presence-IK &\ac{cpi} & $1/5$ & $2.42/3.58$ & $2.977$ & $.2928$ & $15$ \\ \\
		Others - Co-Presence-NIK &\ac{cpn} & $1/5$ & $2.83/4.25$ & $3.133$ & $.3710$ & $15$ \\ \\
		Telepresence &\ac{cp} & $1/7$ & $2.0/7.0$ & $5.286$ & $1.042$ & $30$ \\ \\
		Telepresence-IK &\ac{cpi} & $1/7$ & $2.0/7.0$ & $5.213$ & $1.265$ & $15$ \\ \\
		Telepresence-NIK &\ac{cpn} & $1/7$ & $3.4/6.8$ & $5.360$ & $.7971$ & $15$ \\ \\
		Social Presence &\ac{cp} & $1/10$ & $2.86/7.71$ & $4.590$ & $1.693$ & $30$ \\ \\
		Social Presence - IK &\ac{cpi} & $1/10$ & $2.14/7.43$ & $4.685$ & $1.883$ & $15$ \\ \\
		Social Presence - NIK &\ac{cpn} & $1/10$ & $2.86/7.71$ & $4.495$ & $1.540$ & $15$ \\ \\
	\end{tabularx}
\end{table}
\clearpage

\begin{table}[H]
	\centering\footnotesize\setstretch{0.8}
	\caption{Variablen, Mittelwerte, Standartabweichungen und Anzahl der Teilnehmer}
	\label{MittelwerteUndCo2}
	\begin{tabularx}{\textwidth}{p{3cm} | p{3.0cm} p{1.5cm} p{1.5cm} p{2cm} p{1cm} p{2cm}} 
		Was wurde gemessen? & Abkürzung & Min/Max &Angegeben Min/Max & Mittelwert & Std. Abweichung & $N$ \\
		\hline \\
		Anz. abg. Runden &\ac{tRound} & &$4/13$ & $9$ & $2.5730$ & $10$ \\ \\
		Anz. abg. Runden-IK Team &\ac{tRoundIK} & & $4/13$ & $9$ & $3.273$ & $5$ \\ \\
		Anz. abg. Runden-NIK Team &\ac{tRoundNIK} & & $7/12$ & $9$ & $1.732$ & $5$ \\ \\	
		Anz. abg. Runden Team &\ac{tRound} & & $4/13$ & $9$ & $2.6667$ & $10$ \\ \\
		Anz. abg. Runden Team-IK&\ac{tRoundIK} & & $4/13$ & $9$ & $3.535$ & $5$ \\ \\
		Anz. abg. Runden Team-NIK &\ac{tRoundNIK} & & $7/12$ & $9$ & $1.870$ & $5$ \\ \\	
		Kognitives Vertrauen im Team &\ac{teamCog} & 3/21 & $11.71/15.00$ & $13.068$ & $1.1838$ & $10$ \\ \\
		Kognitives Vertrauen im Team-IK &\ac{teamCogIK} & 3/21& $11.71/13.29$ & $12.36$ & $.839$ & $5$ \\ \\
		Kognitives Vertrauen im Team-NIK &\ac{teamCogNIK} & 3/21& $12.28/15.00$ & $13.76$ & $1.106$ & $5$ \\
		Generelles Vertrauen im Team &\ac{teamGen} & 3/21& $13.35/14.60$ & $13.99.53$ & $.3854$ & $10$ \\ \\
		Generelles Vertrauen im Team-IK &\ac{teamGenIK} & 3/21& $13.35/14.15$ & $13.74$ & $.3361$ & $5$ \\ \\
		Generelles Vertrauen im Team-NIK &\ac{teamGenNIK} & 3/21& $13.35/14.60$ & $14.25$ & $.2423$ & $5$ \\
	\end{tabularx}
\end{table}
\clearpage

%\subsection{Berechnung der Werte für die Auswertung}
%Bspw. Wie wurde CT oder GT etc. berechnet? UND WARUM BENÖTIGE ICH DIESE? WARUM Z.B. BESTANDENE TEAMRUNDEN ? -> WEIL NUR SO TEAMBUILDING ERFOLG GEMESSEN WERDEN KANN
%\newpage

\clearpage
\newpage
\section{Analyse}
\subsection{Analyse der Hypothesen}
Insgesamt zeigen die Ergebnisse Anzeichen dafür, dass unterschiedliche Avatar Konditionen sich auf das gebildete kognitive Vertrauen der einzelnen Personen auswirkt. 
Weiterhin zeigen die Ergebnisse kein Anzeichen dafür, dass unterschiedliche Avatar Konditionen sich auf die Teameffektivität auswirken. Es gibt keine Anzeichen dafür, dass es einen Zusammenhang zwischen dem generellem Vertrauen einer Person und dem gebildeten kognitiven Vertrauen einer Person gibt. Es gibt ebenfalls keine Anzeichen dafür, dass der generelle Hang des Vertrauens im Team sich auf die Teameffektivität auswirkt. Weiterhin gibt es keine Anzeichen dafür, dass sich das gemeinsame Vertrauen im Team sich auf die Teameffektivität eines Teams auswirkt.

Eine Übersicht der Hypothesen und ob diese als signifikant bewertet werden, befindet sich in \textbf{\textit{Tabelle \ref{SignifikanzOverview}}}

\paragraph{Auswirkungen auf das kognitive Vertrauen}
Es konnte ein \textit{signifikanter Unterschied} in den Mittelwerten anhand eines durchgeführten Mann-Whitney-U Test zwischen dem gebildeten kognitiven Vertrauen in das Team der unterschiedlichen Avatarkonditionen festgestellt werden. Die \textbf{Hypothese 1 (H1$_{1}$)}, dass der erzielte kognitive Vertrauenswert sich bei den Konditionen \ac{ik} und \ac{nik}  \textit{signifikant} voneinander unterscheidet, wird angenommen. Es ist ein höheres durchschnittliches kognitives Vertrauen bei der Kondition \ac{nik} ($\bar(x)=4.590$) als bei der Kondition \ac{ik} ($\bar(x)= 4.123$) gebildet worden.

Durch die statistische Auswertung der \textbf{Hypothese 2$_{0}$} konnten \textit{keine signifikanten Zusammenhänge} anhand einer Spearman-Korrelation zwischen den erzielten generellen Vertrauenswerten einer Person und den erzielten kognitiven Vertrauenswerten einer Person, bei unterschiedlichen Avatarkonditionen, festgestellt werden. Weiterhin unterscheidet sich der Zusammenhang zwischen dem \ac{gt} und dem \ac{ct} bei unterschiedlichen Avatarkonditionen \textit{nicht signifikant voneinander}. \newline Die beiden durchgeführten Spearman-Korrelationen weisen positive Korrelationskoeffizienten moderaten bis starken Effektes auf. Weiterhin zeigt eine durchgeführte Fisher-Z-Transformation \textit{keinen signifikanten Unterschied} der Verteilungsformen zwischen den Spearman-Korrelationen auf. Es muss jedoch die Alternativhypothese (H2$_{0}$), dass kein Zusammenhang zwischen dem erzieltem generellen Vertrauenswert einer Person und dem erzielten kognitiven Vertrauenswert bei Unterschiedlichen Avatarkonditionen besteht, angenommen werden.

\paragraph{Auswirkungen auf die Teameffektivität}
Die Ergebnisse der statistischen Auswertung der \textbf{Hypothese 3$_{0}$} zeigen anhand einer durchgeführten Spearman-Korrelation \textit{keine signifikanten Zusammenhänge} der kognitiven Vertrauenswerte von Teams und der realen Teameffektivität bei unterschiedlichen Avatarkonditionen. Es konnte nicht gezeigt werden, dass der Zusammenhang der kognitive Vertrauenswert eines Teams mit der realen Teameffektivität eines Teams \ac{tRound} und der Kondition \ac{ik} stärker, als der von Teams mit der Kondition \ac{nik}, ist.
Die beiden Verteilungsformen der Spearman-Korrelationen zwischen den kognitiven Vertrauenswerten im Team \ac{teamCog} und der Teameffektivität des Teams \ac{tRound} bei unterschiedlichen Avatarkonditionen unterscheiden sich anhand einer Fishers-Z-Transformation \textit{nicht signifikant} voneinander.
Die Alternativhypothese  (H3$_{0}$), dass keine signifikanten Zusammenhänge der kognitiven Vertrauenswerte von Teams und der realen Teameffektivität bei unterschiedlichen Avatarkonditionen bestehen, wird angenommen. \newline

Es wurde auf jedoch Konditionsebene ist zwischen dem kognitiven Vertrauenswert (\ac{ctn}) und der wahrgenommenen Teameffektivität (\ac{ten}) der Kondition \ac{nik} eine \textit{signifikante positive} Korrelation festgestellt. 

Die Mittelwerte der realen Teameffektivität bei unterschiedlichen Avatarkonditionen sind gleich und unterscheiden sich \textit{nicht signifikant} voneinander. Die \textbf{Hypothese 4$_{0}$}, dass die reale Teameffektivität sich bei bei unterschiedlichen Avatarkonditionen signifikant voneinander unterscheidet, wird abgelehnt. Es muss die Alternativhypothese (H4$_{0}$) angenommen werden, die besagt, dass die reale Teameffektivität sich bei unterschiedlichen Avatarkonditionen nicht unterscheidet, angenommen.\newline

Durch die Analyse der \textbf{Hypothese 5$_{0}$} konnten \textit{keine signifikanten} Zusammenhänge der generellen Vertrauenswerte eines Teams und der realen Teameffektivität bei unterschiedlichen Avatarkonditionen festgestellt werden. Durch diese nicht signifikanten Zusammenhänge und dadurch, dass eine durchgeführte Fishers-Z-Transformation keine Unterschiede der Zusammenhänge zeigt, muss die Hypothese 5 (H5$_{1}$) verworfen und die Alternativhypothese (H5$_{0}$) angenommen werden. Diese besagt, dass der generelle Vertrauenswert eines Teams nicht stärker mit der realen Teameffektivität eines Teams und der Kondition \ac{ik} korreliert als von Teams mit der Kondition \ac{nik}.

\paragraph{Subjektive Daten}
Eine Tabelle der einzelnen Werte der Variablen befindet sich in \textbf{Tabelle \ref{MittelwerteUndCo} }
Die Nutzer des Systems zeigen anhand der Minimal und Maximal anzugebenden Werte im jeweiligen Fragebogen, unabhängig der Avatarkonditionen, ein überdurchschnittliches Gefühl der Presence  ($\bar(x)=4.21; \sigma=.615$). Auch die, \textit{der selbst wahrgenommenen Co-Presencen} ($\bar(x)=2.972; \sigma=.3690$), der \textit{wahrgenommenen Co-Presence des anderen} ($\bar(x)=3.055; \sigma=.3378$), der \textit{Telepresence} ($\bar(x)=5.286; \sigma=1.042$) und der \textit{Social-Presence} ($\bar(x)=4.590; \sigma=1.693$) sind überdurchschnittliche Werte erzielt worden.
Es konnten jedoch \textit{keine signifikanten Unterschiede} bei den einzelnen Avatarkonditionen dieser subjektiven Daten festgestellt werden.

Das \textit{wahrgenommene Anstrengungsmaß} ($\bar(x)=8.6; \sigma=2.135$) liegt, anhand der Minimal und Maximal anzugebenden Werte im jeweiligen Fragebogen, unter dem Durchschnitt, während die wahrgenommene Teameffektivität ($\bar(x)=4.646; \sigma=..972$) über dem Durschnitt liegt. Auch hier konnten jedoch \textit{keine signifikanten Unterschiede} bei den einzelnen Avatarkonditionen festgestellt werden.

Anhand eines Mann-Whitney-U Test konnte bei der Teamkommunikation ein signifikanter unterschied der Mittelwerte bei unterschiedlichen Avatarkonditionen festgestellt. werden. Der Mittelwert der Teamkommunikation der Kondition \ac{nik} beträgt ($\bar(x) = 4.013$), während der Mittelwert der Teamkommunikation der Kondition \ac{ik} ($\bar(x) = 4.48$) beträgt. Somit ist die subjektiv empfundene Teamkommunikation bei der Kondition \ac{ik} höher als bei der Kondition \ac{nik}.

\paragraph{Fazit der Analyse}
Anhand der statistischen Analyse, lässt sich feststellen, dass unterschiedliche Avatarkonditionen einen \textit{signifikanten Einfluss} auf das gebildete kognitive Vertrauen eines Teams besitzen, dieses gebildete kognitive Vertrauen jedoch textit{keinen Einfluss} auf die Teameffektivität besitzt. 
Es wurde ein signifikanter unterschied der Mittelwerte der Teamkommunikation auf Konditionsebene festgestellt. Es hat sich ein erhöhtes Presencegefühl (\ac{ipq}, \ac{tp}, \ac{scp}, \ac{ocp}, \ac{sp}) auf Konditionsebene während der Versuchsdurchführung gebildet. Das Anstrengungsmaß (\ac{tlx} lag unter dem Durchschnitt der möglichen Anzugebenden Werte und die wahrgenommene Teameffektivität war überdurchschnittlich hoch.


\begin{table}[H]
	\centering\footnotesize\setstretch{1.2}
	\caption{Signifikanz der Hypothesen}
	\label{SignifikanzOverview}
	\begin{tabularx}{\textwidth}{p{12cm} | p{3.5cm}} 
		Hypothese & Signifikanz  \\
		\hline \\
%		H1$_{0}$ : Ein hoher General-Trust-Score (\ac{gt}) wirkt sich \textbf{nicht positiv} auf den Cognitiven-Trust-Score (\ac{ct}) aus.\\
%\textbf{H1$_{1}$} : Der erzielte kognitive Vertrauenswert (\ac{ct}) unterscheidet sich bei den Konditionen \ac{ik} und \ac{nik} signifikant voneinander.\\ \\
%Alternative formulierung :
\textbf{H1$_{1}$} : Die Mittelwerte der erzielten kognitive Vertrauenswerte (\ac{ct}) unterscheiden sich bei den Konditionen \ac{ik} und \ac{nik} signifikant voneinander.\\
		& nicht signifikant \\
		\hline \\
		
%		H2$_{0}$ : Die Konditionen \ac{ik} oder \ac{nik} haben \textbf{keinen signifikant abweichenden Einfluss} auf den Cognitive-Trust-Score (\ac{ct}). \\
%		\textbf{H2$_{1}$} : Es besteht ein Zusammenhang zwischen dem erzielten generellen Vertrauenswert \ac{gt} einer Person und dem erzielten kognitiven Vertrauenswert \ac{ct} einer Person.\\ \\
%		Alternative formulierung :
				\textbf{H2$_{1}$} : Je höher die erzielten generelle Vertrauenswert \ac{gt} einer Person desto höher ist der erzielte kognitive Vertrauenswert \ac{ct} einer Person.
		& \textbf{signifikant} \\
		
		\hline 	\\	
%
%		\textbf{H3$_{1}$} : Der kognitive Vertrauenswert von Teams \ac{teamCog} korreliert mit der realen Teameffektivität von Teams \ac{tRound} und der Kondition \ac{ik} stärker als von Teams mit der Kondition \ac{nik}. \\ \\
%		H3a$_{0}$ : Ein hoher Cognitiver-Trust-Score hat \textbf{keinen Einfluss} auf die Team-Effektivität (\ac{tRound}).
%Alternative Formulierung :
\textbf{H3$_{1}$} : Der Zusammenhang des kognitive Vertrauenswert von Teams \ac{teamCog} und der realen Teameffektivität von Teams \ac{tRound} mit der Kondition \ac{ik} ist stärker als der von Teams mit der Kondition \ac{nik}. \\
		& nicht signifikant \\		
		
		\hline 	\\	
%		\textbf{H4$_{1}$} : Die reale Teameffektivität \ac{tRound} unterscheidet sich bei den Konditionen \ac{ik} und \ac{nik} signifikant voneinander.\\ \\
%%		H4$_{0}$ : Die Konditionen \ac{ik} oder \ac{nik} haben \textbf{keinen signifikant abweichenden Einfluss} auf die Anzahl der erfolgreich abgeschlossenen Runden \ac{tRound} eines Teams.\\
%Alternative Formulierung :
		\textbf{H4$_{1}$} : Die Mittelwerte der realen Teameffektivität \ac{tRound} unterscheidet sich bei den Konditionen \ac{ik} und \ac{nik} signifikant voneinander.\\
		& nicht signifikant\\

		\hline 	\\	
%		\textbf{H5$_{1}$} : Der generelle Vertrauenswert eines Teams \ac{teamGen} korreliert mit der realen Teameffektivität eines Teams \ac{tRound} und der Kondition \ac{ik} stärker als von Teams mit der Kondition \ac{nik}. \\ \\
%%		H5$_{0}$ : Teams, aufgeteilt nach Avatar Verkörperungen (\ac{ik}, \ac{nik}), mit einem hohem General-Trust-Score (\ac{gt}) \textbf{schließen nicht mehr Runden ab}, (\ac{tRound}) als die mit einem niedrigen General-Trust-Score \ac{gt}\\
%Alternative Formulierung :
			\textbf{H5$_{1}$} : Der Zusammenhang der generelle Vertrauenswert eines Teams \ac{teamGen} und der realen Teameffektivität eines Teams \ac{tRound} mit der Kondition \ac{ik} ist stärker als von Teams mit der Kondition \ac{nik}.
		& nicht signifikant \\
		
		\hline 	\\
		
%		H5a$_{0}$ : Teams mit einem hohem General-Trust-Score (\ac{gt}) erzielen \textbf{schließen nicht mehr Runden ab} (\ac{tRound} als die mit einem niedrigen General-Trust-Score \ac{gt}.\\		
%		& nicht signifikant \\
	\end{tabularx}
\end{table}		


\subsection{Analyse der technischen Anforderungen}

Schweizer \citep[p.270]{schweitzer2010conceptualizing} erläutert die 4 größten Kennzeichen eines \ac{vts} in der Literatur.
\ac{vts} sind :
\begin{itemize}
\item Zustandegekommen aufgrund von Kommunikationstechnologie.Durch technische Hilfsmittel wird Kommuniziert, es werden Entscheidungen getroffen oder Informationen ausgetauscht.
\item Räumlich getrennt. \ac{vts} arbeiten \textit{nicht} am selben Arbeitsplatz
\item Grenzübergreifend. Es gibt Zusammenarbeit der Teammitglieder aus verschiedenen Organisationen oder Organisationseinheiten.
\item Asynchron. \ac{vts} arbeiten zu unterschiedlichen Zeiten/Zeitzonen oder in der selben Zeitzone in unterschiedlichen
\end{itemize}

Die technischen Anforderungen, die im Kapitel \textit{\ref{TechnischeAnforderungen}} aufgestellt werden, konnten hinreichend umgesetzt werden.
\begin{itemize}
\item \textbf{Online-Fähigkeit und Unterstützung mehrerer HMD's}: Durch das verwendete Framework \textit{Normcore} ist es mehreren Nutzern gleichzeitig möglich gewesen, von jedem beliebigen Ort an dem Experiment teilzunehmen. Die einzige Voraussetzung war, dass die teilnehmenden Personen ein \ac{hmd}, einen Internetanschluss sowie einen Computer besitzen. Es ist möglich gewesen, die Anforderungen eines virtuellen Teams laut Schweitzer \citep[p.270]{schweitzer2010conceptualizing} (vgl. Kapitel \textit{AnforderungenLautSchweitzer}) zu erfüllen. 
\item \textbf{Geringe Latenz}: Es wurden zwei End-to-End Latenzmessungen durchgeführt. Die End-to-End Latenzmessung bezeichnet die Zeit, die benötigt wird, bis eine reale Aktion eines Nutzers von einem anderen Nutzers wahrgenommen wird. Die die erste Latenzmessung betraf die Geschwindigkeit der Audioübertragung des Netzwerks. Es wurde eine Durchschnittliche Latenz von \textbf{356ms} festgestellt.
Die zweite Messung betraf die Geschwindigkeit, in der die Avatare ihre Körperpositionen verändern. Es wurde eine durchschnittliche Geschwindigkeit von 294ms festgestellt.
Es wurden subjektiv keine störende Bewegungsverzögerung und Audioübertragung wahrgenommen, weshalb die Anfoderung als erfüllt angesehen wird.
\item \textbf{Steuerung und Verwaltung der Anwendung durch einen Spectator}: Durch einen eigenen Spectator-Client ist es möglich gewesen, die Anwendung von Außen zu steuern und zu überwachen.
\item \textbf{Interaktionsmöglichkeiten}:  Durch das Verwendete Framework \textit{Normcore} und dessen globalen Datenspeicher für unterschiedliche Variablen, ist es möglich gewesen, alle Zustände und Informationen während der laufenden Versuchsdurchführung zwischen dem Client und Spectator zu synchronisieren. Durch Nutzerfeedback wurde ebenfalls die intuitive Steuerung bestätigt.
\item \textbf{Sprachübertragung}: Die Implementierung eines RCP-Basiertem Netzwerkhandlings ermöglichte, eine Sprachübertragung von Client zu Spectator und von Spectator zu Client.
\item \textbf{Avatararten}: Der Spectator hatte bei dem verwendetem System die Möglichkeit zur Laufzeit zwei verschiedene Avatararten für die Clients auszuwählen. Es wurden zwei Avatararten (\ac{ik} und \ac{nik}) implementiert, anhand diesem Zusammenhänge und Unterschiede gemessen wurden.
\end{itemize}
%
%Weiterhin besitzt das System noch \textbf{zusätzliche Anforderungen} :
%
%\begin{itemize}
%\item \textbf{Teamfähigkeit} Das Team muss eine Aufgabe gemeinsam lösen. Jedes Teammitglied muss zu gleichen Teilen involviert sein. Fehlt ein Teammitglied, soll die Aufgabe nicht lösbar sein.
%\item \textbf{Inkrementelle Erhöhung der Schwierigkeit} Um einen Vergleich zwischen der Teameffektivität herzustellen, mussten die Runden inkrementell schwieriger werden. Dies wurde dadurch gelöst, dass die teilnehmenden Personen jede drei Runden ein Symbol mehr gestikulieren  oder erraten mussten.
%\item \textbf{Neutrale Avatare} Die Avatare sollen so neutral wie möglich gestaltet sein. 
%\item \textbf{Vermeidung von Bekanntheit} Um die gegenseitige Bekanntheit  teilnehmenden Personen auszuschließen, wurde jedem Teilnehmer zu beginn des Versuchs ein Zufälliger Name zugeordnet. Vor dem gesamten Versuch konnten sich die teilnehmenden Personen nicht sehen oder hören.
%Können die teilnehmenden Personen sich vor dem Versuch gesehen, würde sich ein Kognitives-Vertrauen oder Kognitives-Misstrauen zu der anderen Person aufbauen. 
%Kennen sich die Personen untereinander schon, wirkt sich dies auf die Vertrauensbildung in der \ac{vr} in Form von schon vorhandenem Affektiven-Vertrauen auf die Ergebnisse aus.
%\item \textbf{Keine verbale Kommunikation} Da die Stimme der teilnehmenden Personen ein Residuum ist, wurde die verbale Kommunikation von Versuchsteilnehmer zu Versuchsteilnehmer nicht gestattet. 
%\end{itemize}


\section{TODO}
H5 die Konditionen entfernen! Da das nicht Sinnvoll ist!

\newpage
\section{Diskussion}

	Was habe ich gemacht?
	Was habe ich gefunden? 
	Was bedeutet das genau?
	
Die Ergebnisse werden interpretiert und Erkenntnisse werden erläutert.

Die Beschränkungen der Forschung und deren Auswirkungen werden dargelegt.

Es wird beschrieben, inwiefern die Erwartungen erfüllt worden sind.

Eventuelle Ursachen und Folgen für die Ergebnisse werden besprochen.

Vorschläge für eventuelle weiterführende Forschungen werden gemacht.

Ergebnisse werden interpretiert.

Es werden keine zu vagen Empfehlungen für weiterführende Forschungen gemacht.
	

\paragraph{Allgemeines}
	Diese Studie trägt zum einen dazu bei, besser zu verstehen, welche Avatarkonditionen in einem \ac{sve} zum Vertrauensaufbau eingesetzt werden sollten, und zum anderen, wie sich das gebildete Vertrauen auf die Effektivität eines Teams auswirkt. Durch die Berücksichtigung des aktuellen Forschungsstandes, wurde das kognitive Vertrauen der einzelnen Personen und des gesamtem Teams, sowie deren Teameffektivität, gemessen. Insgesamt werden 1 der 5 aufgestellten Hypothesen als signifikant angesehen.

Die Stichprobengröße betrug in dem Experiment $N=30$, wobei, um Unterschiede und Zusammenhänge aufgeteilt nach Avatarkonditionen festzustellen, diese Stichprobengröße auf $N=15$ halbiert werden musste. Um Teameffekte zu messen, wurde das Team als Einheit gemessen. Dies führt dazu, dass bei $N=5$ Teams mit der Kondition \ac{ik} und $N=5$ Teams mit der Kondition \ac{nik} Zusammenhänge und Unterschiede gemessen werden konnten.

Hier auf größe der Stichprobe eingehen. 
Hier auf die Hauptfaktoren eingehen, weshalb die Hypothesen nicht signifikant wurden
	-3er Teams zu groß
	-Dadurch, dass die Personen wussten, dass diese mit 3 Lebendigen Personen arbeiten, wurde viel vertrauen gebildet
	-Visualisierung der Avatare
	-Komplexität der Aufgabe - Eventuell zu leicht? Eventuell nicht genügend Vertrauensaufbau vorhanden? Vertrauensaufbau eher im Sinne von "Wenn mein Team Versagt, dann Leite ich das Versagen auf mein gesamtes Team ab" "Wenn mein Team gut ist, dann denke ich, dass mein Team sehr Zuverlässig ist.

\subsection{Diskussion des Experimentes}
\paragraph{Avatarkonditionen}
Es gibt keine unterschiede in der erzeugten Selbstpräsenz sowie der erzeugten Co-Präsenz der eingesetzten Avatare. Ein Grund diesbezüglich könnte das fehlende Selbst-Avatar der einzelnen Nutzer sein. Dies erhöht Laut Biocca ebenfalls die Selbstpräsenz der Avatare \citep[421-427]{construal2014connected}.
Die teilnehmende Person (ID2) mit der Kondition \ac{ik} schrieb dazu : \newline
\dq{}\textit{Ich wusste nicht genau wie meine Arme zu sehen sind, ob ich ein Buchstabe auch mit den Armen formen, statt mit den Händen schreiben konnte.}\dq{}\newline
Diese Verwirrung der fehlenden Selbstpräsenz hat eventuelle Präsenzdefizite zur folge.
Weiterhin hat sich die fehlende Geschlechtsunterscheidung der Avatare merklich gemacht. So schrieb die teilnehmende Person (ID4) mit der Kondition \ac{ik} dazu \newline 
\dq{}\textit{Warum gab es keine Frauenfiguren? Das würde mehr Realität erzeugen}\dq{}.\newline

\paragraph{Schwierigkeit des Experiments}
Es ist möglich, dass das Experiment nicht komplex genug aufgebaut war, um einen Zusammenhang zwischen der Vertrauensbildung und der Teameffektivität zu messen.
Dies ging besonders durch die statistisch unsignifikanten Ergebnisse der Analyse der Hypothese 3, 4 und 5 hervor. 
So schrieb eine teilnehmende Person (ID17) mit der Kondition \ac{nik} dazu :\newline
\dq{}\textit{Zu kurz - zu einfach - zu grau}\dq{} \newline
Ähnliches schrieb die die teilnehmende Person mit der ID21 \newline 
\dq{}\textit{Schwierigkeitsgrad hätte noch stärker sein dürfen}\dq{} 
\newline
Die Runden des Experiments hätten eventuell schneller abgearbeitet werden können, um eine größere Varianz der einzelnen Effektivitätswerte der Teams zu messen. 
Diese Varianz der erzielten Teameffektivität könnte sich noch weiter erhöhen, wenn ein Team aus zwei statt aus drei Personen besteht. Einzelne Runden des Experiments würden schneller abgearbeitet werden und das gebildete kognitive Vertrauen könnte in einem direktem Vergleich gemessen werden.
Durch diese Art des Kollaboration wäre allerdings die komplexe Komponente eines drei Personen Teams verloren gegangen.\newline

Durch diese Einfachheit könnten innerhalb eines Teams schneller Lerneffekte durch Repetitivität entstanden sein, die die Teameffektivität untereinander angeglichen haben. 
Eine teilnehmende Person (ID7) mit der Kondition \ac{ik} schreibt zu den Natürliche Lerneffekten : \newline 
\dq{}\textit{Am Anfang habe ich ein wenig Druck verspürt, die richtigen Zeichen zu geben. Als ich gesehen habe, dass es gut funktioniert, hat es sehr viel Spaß gemacht.}\dq{} \newline

%Der Titel dieser Masterarbeit lässt sich in 2 Fragestellungen unterteilen :

%Ist die Avatarkondition \ac{ik} oder die Avatarkondition \ac{nik} effektiver um Vertrauen in einem \ac{sve} zu bilden?

%Weshalb gibt es keine Auswirkungen?
%Was hat der Versuchsaufbau mit den Ergebnissen zu tun?
%

%Wirkt sich das gebildete Vertrauen der verschiedenen Avatarkonditionen auf die Effektivität eines virtuellen Teams aus?
%Wenn ja : Welche Avatarkondition ist effektiver?
%Wie viel effektiver ist eine Avatarkonditione?
%Wenn nein : Woran liegt es, dass die Teams gleich effektiv sind?

\subsection{Effektivität des Vertrauensaufbau zwischen den Avatarkonditionen}

Die während der Analyse der Hypothese 1 wurde einen höheren Mittelwert der Kondition \ac{nik} als bei der Kondition \ac{ik} festgestellt. Diese Feststellung über den Vertrauensaufbau unter Avataren widerspricht der Untersuchung von Bente und Rüggen (vgl. \textbf{\textit{ Kapitel \ref{AvatarTrust}}}). Die Kondition \ac{ik} sieht menschenähnlich aus, weshalb sich hätte vermuten lassen, dass diese Kondition den teilnehmenden Personen ein größerer kognitiver Vertrauensaufbau stattfindet, als bei der Kondition \ac{nik}. Gründe für dieses Ergebnis könnte sein, dass der Avatar von dem \textit{Uncanny Valley} Effekt betroffen sind. Schon in der Studie von Waldow tendiert der in dieser Studie eingesetzte Avatar zum Unbehagen, was auch durch den Uncanny-Valley Effekt zu Stande kommen kann. Weiterhin kann es sein, dass die Gestukulation, während des Versuchs nicht eindeutig von den teilnehmenden Personen mit der Kondition \ac{ik} ausgewertet werden konnten. Die Inverse-Kinematik kann falsch interpretiert worden sein, oder es sind durch diese unnatürliche Bewegungsmuster der Avatare aufgetreten. Dadurch, dass die Avatare \dq{}anders als normale Menschen \dq{} waren, konnte diesen eventuell nicht all zu viel Vertrauen zugesprochen werden.
Eine teilnehmende Person (ID9) mit der Kondition \ac{ik} schrieb im Feedback Teil des Fragebogens dazu : \newline
\dq{}\textit{Es war eine spannende Erfahrung am VR-Experiment teilzunehmen und hat Spaß gemacht. Die Personen waren mir nah, obwohl sie eindeutig unreal wirkten; Vertrauen war nichtsdestotrotz vorhanden.}\dq{}

Dass kein signifikanter Zusammenhang des kognitiven Vertrauens mit dem generellem Vertrauen bei unterschiedlichen Avatarkonditionen besteht, kann auch als positiver Effekt gesehen werden. 
So lässt sich vermuten, dass es nicht von relevanz ist, wie hoch oder niedrig das generelle Vertrauen einer Person im \ac{sve} ist. Das generelle Vertrauen einer Person, während der Kennenlernphase, besitzt keinen signifikanten Zusammenhang mit dem gebildeten kognitive Vertrauen. Dies ist auch im Zusammenhang mit der signifikant unterschiedlichen Bildung von kognitiven Vertrauen von Avatarkonditionen zu betrachten. Dadurch, dass nur die unterschiedlichen Avatarkonditionen einen signifikanten Einfluss auf die Bildung des kognitiven Vertrauens besitzten, kann davon ausgegangen werden, dass das generelle Vertrauen isoliert betrachtet werden kann und bei einer Kennenlernphase eines \ac{vt} keine größere Rolle spielt.
%Weshalb könnte es keine Auswirkungen gegeben haben?


\subsection{Einfluss des Vertrauens im Team auf die Teameffektivität}
\paragraph{Hypothese 3, Hypothese 4 und Hypothese 5}
Während der Analyse der Hypothese 4 wurde festgestellt, dass es keinen Unterschied auf die Teameffektivität bei unterschiedlichen Avatarkonditionen gibt. Dies spiegelt sich auch in der Teameffektivität der Teams wieder. Bei beiden Teams ist die durchschnittliche Teameffektivität $x=9$.  
Ein Team bestand aus 3 Personen, was die Stichprobengröße verkleinert hat. Somit nahmen 10 Teams mit insgesamt 30 Personen an dem Versuch teilnahmen. Diese Stichprobe ist sehr klein und könnte eine mögliche Ursache dafür sein, dass in Hypothese 3 und 5 keine signifikanten Ergebnisse erzielt wurden.

Auch wenn kein Zusammenhang des kognitiven Vertrauens und der Teameffektivität vorliegt, lässt sich vermuten, dass nichts desto trotz zu einem bestimmten Grad Vertrauen durch eine erhöhte Teameffektivität zustande kommt.
Eine teilnehmende Person (ID20) mit der Kondition \ac{nik} schrieb dazu : \newline
\dq{}\textit{[...]Ohne Sprache ist die Kommunikation mit den anderen schon sehr schwierig. Trotzdem fand ich interessant, dass die Aufgabe überhaupt zu bewältigen war, und tatsächlich nach den ersten Erfolgen auch das Vertrauen zu den anderen entstand, dass sie in der Lage sind, die Aufgabe zu bewältigen, zu kommunizieren und das kommunizierte zu verstehen.}\dq{}
	
\paragraph{Einfluss des Experiments auf die Ergebnisse}

Nochmal revué passieren lassen, ob das Teamwork-Spiel das richtige war. Hätte ich vielleicht etwas ändern können am Aufbau? Eventuell ein Social dillemma szenario? Hätte ich da eventuell validere informationen raus bekommejn?
%Was hat der Versuchsaufbau mit den Ergebnissen zu tun?

Rundenaufbau -> Werden schwerer
Größere Stichproben der einzelnen Teams

Die 3 Teammitglieder WUSSTEN, dass diese mit menschen interagieren. Vielleicht gab es daher keinen unterschied zum kognitiven Vertrauen - Wie wäre es, wenn man denen sagen würde, dass die anderen Teilnehmer Computerfiguren sind und KEINE Menschen? Super interessant!
Dazu schreibt ID8 (\ac{ik}) : "Es war ein sehr interessantes Erlebnis. Ich bin gespannt zu Erfahren, wer die Teilnehmer waren"

Hätten die Runden kürzer sein können um einen positiven Teameffekt zu messen?


\paragraph{Diskussion der Hypothesen}

Es könnte die gleiche Anzahl von abgeschlossenen Runden pro Team sein, weil :
-Das Teambuilding game wie ich es aufgebaut habe war ein schlechtes Messinstrument
-Game hat relativ wenig mit Vertrauen zu tun.

Wie wäre das Spiel verlaufen, wenn nur 2 Avatare an dem Game teilgenommen hätten?
Hätte sich was geändert? Was haben die aktuellen 3 Avatare für eine Auswirkungen und welche folgen hat dies auf die mögliche Teameffektivität und das gebildete Cognitive Vertrauen?

-3er Teams -> So nahe an der Realität wie möglich. Daher aber eventuell auch generell höheres Vertrauensgefühl, da man die Arbeit "von sich abdrängen kann", da eigentlich immer nur die erklärende Person zur Zeit unter druck steht.

\subsection{Einfluss der subjektiven Daten}

Keine Signifikanz von Social presence und co präsenz kann daher kommen, weil die testpersonen die ganze zeit miteinander interagiert haben. daher haben sich beide gruppen stark drin gefühlt oder stark dass der andere da ist
Abh var : tRoundsIK mit unabh general-Trust score scheint Kubisch oder quadratisch

Die subjektiven Daten alle in verbindung mit der Teambuldingmaßnahme setzen. Anstrengungsmaß z.B., weil man die Leute erst kennenlernen muss etc.

Hürde : Fehlinterpretation der Gestikulationen

\subsection{Limitierungen}
Wo waren die Grenzen der Forschung?
Habe ich bestimmte Limitiationen stark zu spüren bekommen?
Sind Fragen offen geblieben?
Was kann ich auf Basis meiner Resultate für zukünftige Forschungen empfehlen?

Diese Arbeit behandelt XY nicht … .
Es würde über den Rahmen dieser Arbeit hinausgehen, … .
Es ist nicht das Ziel dieser Arbeit … .
Die LeserInnen sollten beachten, dass diese Arbeit auf … basiert.
Begrenzungen ergaben sich beim Durchführen der Interviews, da … .
Eine Limitation der Bachelorarbeit ist zum einen … .
Weiterer Forschungsbedarf ergibt sich aus … .

\paragraph{Implikationen für Zukünftige Forschungen}
Die teilnehmenden Personen wussten bevor das Experiment startete, dass diese in einem \ac{sve} mit 3 weiteren realen Personen arbeiten würden. Eine interessante Fragestellung wäre es nun, den teilnehmenden Personen im vorherein nicht zu erzählen, dass die Personen real sind und anschließend das gebildete kognitive Vertrauen und die Effektivität im Team zu messen.

Halte ebenfalls fest, inwiefern deine Resultate neue oder andere Erkenntnisse in Bezug auf dein Thema mit sich bringen. Gehe alle Möglichkeiten durch: Was genau hast du herausgefunden?

\paragraph{Diskussion der eingesetzten Methoden}
\paragraph{Auswirkungen auf die Gegenwart}
\paragraph{Vorschläge für zukünftige Untersuchungen}
		Social Identity und Teambuilding - Ein Avatar ist anders.

\paragraph{IDEEN}
Warum sind die Ergebnisse so wie diese sind?
Haben die Probanten während der Anwendung Stress verspürt?


\paragraph{Was kann besser gemacht werden?}
Durch die Symbole ist das Experiment auch änderübergreifend einsetzbar gewesen.. pr eund post questionnaire haben das jedoch vehindert, da deutsch

	
\section{Fazit}

Welche Avatarkondition sollte man nutzen um frühzeitig Vertrauen zu bilden?
Warum ist eine Avatarkondition effektiver?

Steigern nun unterschiedliche Avatarkonditionen die Teameffektivität?
Steigern unterschiedliche Avatararten das kognitive Kognitive Vertrauen?
Gibt es einen Zusammenhang zwischen generellem Vertrauen und der gebildeten Teameffektivität?
Gibt es einen Zusammenhang zwischen dem kognitiven Vertrauen und der gebildeten Teameffektivität?
Gibt es einen Zusammenhang zwischen dem generellen Vertrauen und dem gebildeten kognitiven Vertrauen der unterschiedlichen Avatarkonditionen?

Insgesamt nochmal auf die EFFEKTIVITÄT der Vertrauensbildung UND der Effektivität an sich eingehen.
Auch auf virtuelle Teams eingehen ...! Wichtig!
\newpage

\bibliographystyle{dcu}
\bibliography{bibfile}

	\newpage
	\appendix	
\section*{Anhang}\markboth{Anhang}{Anhang}\addcontentsline{toc}{section}{Anhang}

\section{Auswertungsergebnisse}
	
	\begin{figure}[H]
	\centering
		\begin{footnotesize}
			\includegraphics[scale=0.6]{Abbildungen/Post_QuestionnaireStatistiks/Normalverteilung_30}
			
			\caption{Kolmogorov-Smirnoff Normalverteilung für Individuelle Stichproben}
			\label{fig:KolSmirInd}
		\end{footnotesize}
	\end{figure}	
	
	\begin{figure}[H]
	\centering
		\begin{footnotesize}
			\includegraphics[scale=0.6]{Abbildungen/Post_QuestionnaireStatistiks/Normalverteilung_15_IK}
			
			\caption{Kolmogorov-Smirnoff Normalverteilung für die IK-Kondition}
			\label{fig:KolSmirIndIK}
		\end{footnotesize}
	\end{figure}	
	
	\begin{figure}[H]
	\centering
		\begin{footnotesize}
			\includegraphics[scale=0.6]{Abbildungen/Post_QuestionnaireStatistiks/Normalverteilung_15_NON_IK}\\
			\caption{Kolmogorov-Smirnoff Normalverteilung für die NON-IK-Kondition}
			\label{fig:KolSmirIndNONIK}
		\end{footnotesize}
	\end{figure}	
	
	
%\begin{figure}[H]
%\centering
%		\begin{footnotesize}
%			\includegraphics[scale=0.5]{Abbildungen/Post_QuestionnaireStatistiks/boxplot_cognitive_trust_winsorisiert}\\
%			\caption{Boxplot kognitives Vertrauen winsorisiert}
%			\label{fig:boxplot_cognitive_trust_winsorisiert}
%		\end{footnotesize}
%	\end{figure}	

\begin{figure}[H]
\centering
		\begin{footnotesize}
			\includegraphics[scale=0.6]{Abbildungen/Post_QuestionnaireStatistiks/h3_regression_ik}
			\includegraphics[scale=0.6]{Abbildungen/Post_QuestionnaireStatistiks/h3_regression_nik}
			\caption{Regressionsergebnisse der Regressionen \ac{cti} und \ac{tei} sowie \ac{ctn} und \ac{ten} }
			\label{fig:h3_regression}
		\end{footnotesize}
	\end{figure}	
	
\begin{figure}[H]
\centering
		\begin{footnotesize}
%			\includegraphics[scale=0.6]{Abbildungen/Post_QuestionnaireStatistiks/h3_both_korrelation}
			\includegraphics[scale=0.6]{Abbildungen/Post_QuestionnaireStatistiks/h3_both_diagramm_korr}
			\includegraphics[scale=0.6]{Abbildungen/Post_QuestionnaireStatistiks/h3_both_regression}
			\includegraphics[scale=0.6]{Abbildungen/Post_QuestionnaireStatistiks/H3_NIK_Korrelation_Spearman}
			\includegraphics[scale=0.6]{Abbildungen/Post_QuestionnaireStatistiks/H3_IK_Korrelation_Spearman}
			\caption{Regressionsergebnisse der Regressionen \ac{ct} und \ac{te} }
			\label{fig:h3_both}
		\end{footnotesize}
	\end{figure}	
	
	
\begin{figure}[H]
\centering
		\begin{footnotesize}
			\includegraphics[scale=0.6]{Abbildungen/Post_QuestionnaireStatistiks/h5_both_korrelation}
			\includegraphics[scale=0.6]{Abbildungen/Post_QuestionnaireStatistiks/h5_both_diagramm_korr}
			\includegraphics[scale=0.6]{Abbildungen/Post_QuestionnaireStatistiks/h5_both_regression}
			\caption{Regressionsergebnisse der Regressionen \ac{gt} und \ac{te} }
			\label{fig:h5_both}
		\end{footnotesize}
	\end{figure}	

\begin{figure}[H]
\centering
		\begin{footnotesize}
			\includegraphics[scale=0.6]{Abbildungen/Post_QuestionnaireStatistiks/h3_regression_ik}
			\includegraphics[scale=0.6]{Abbildungen/Post_QuestionnaireStatistiks/h3_regression_nik}
			\caption{Regressionsergebnisse der Regressionen \ac{gti} und \ac{tei} sowie \ac{gtn} und \ac{ten} }
			\label{fig:h5_regression}
		\end{footnotesize}
	\end{figure}	

\clearpage
\newpage
	\section{Demografieauswertung}
	\begin{figure}[H]
	\centering
		\begin{footnotesize}
			\includegraphics[scale=0.6]{Abbildungen/Pre_QuestionnaireStatistiks/teilnehmerStatistik}\\
			\includegraphics[scale=0.5]{Abbildungen/Demographie/alter}\\
			\caption{Alter der Teilnehmer}
			\label{fig:Teilnehmer}
		\end{footnotesize}
	\end{figure}	
	
	\begin{figure}[H]
	\centering
		\begin{footnotesize}
			\includegraphics[scale=0.6]{Abbildungen/Pre_QuestionnaireStatistiks/teilnehmerGeschlecht}\\
			\includegraphics[scale=0.5]{Abbildungen/Demographie/teilnehmerGeschlecht}\\
			\caption{Biologisches Geschlecht der Teilnehmer}
			\label{fig:biologischesGeschlecht}
		\end{footnotesize}
	\end{figure}	
	
	\begin{figure}[H]
	\centering
		\begin{footnotesize}
			\includegraphics[scale=0.6]{Abbildungen/Pre_QuestionnaireStatistiks/teilnehmerBildungsstand}\\
			\includegraphics[scale=0.5]{Abbildungen/Demographie/teilnehmerBildungsstand}\\
			\caption{Bildungsstand der Teilnehmer}
			\label{fig:teilnehmerBildungsstand}
		\end{footnotesize}
	\end{figure}	
	
	\begin{figure}[H]
	\centering
		\begin{footnotesize}
			\includegraphics[scale=0.6]{Abbildungen/Pre_QuestionnaireStatistiks/teilnehmerVRErfahrung}\\
			\includegraphics[scale=0.5]{Abbildungen/Demographie/teilnehmerVRErfahrung}\\
			\caption{VR-Erfahrung der Teilnehmer}
			\label{fig:teilnehmerVRErfahrung}
		\end{footnotesize}
	\end{figure}	
	
	\begin{figure}[H]
		\begin{footnotesize}
			\includegraphics[scale=0.6]{Abbildungen/Pre_QuestionnaireStatistiks/teilnehmerVorexperimente}\\
			\includegraphics[scale=0.5]{Abbildungen/Demographie/teilnehmerVorexperimente}\\
			\caption{VR-Studien Vorerfahrungen}
			\label{fig:teilnehmerVorexperimente}
		\end{footnotesize}
	\end{figure}		
	
	\begin{figure}[H]
	\centering
		\begin{footnotesize}
			\includegraphics[scale=0.6]{Abbildungen/Pre_QuestionnaireStatistiks/teilnehmerPCAußmaß}\\
			\includegraphics[scale=0.5]{Abbildungen/Demographie/teilnehmerPCAußmaß}\\
			\caption{PC-Nutzung Außmaß sowie zugehörige Perzentile}
			\label{fig:teilnehmerPCAußmaß}
		\end{footnotesize}
	\end{figure}		
	
	\begin{figure}[H]
	\centering
		\begin{footnotesize}
			\includegraphics[scale=0.6]{Abbildungen/Pre_QuestionnaireStatistiks/teilnehmerVideospieleAußmaß}\\
			\includegraphics[scale=0.5]{Abbildungen/Demographie/teilnehmerVideospieleAußmaß}\\
			\caption{Videospieleaußmaß der Teilnehmer}
			\label{fig:teilnehmerVideospieleAußmaß}
		\end{footnotesize}
	\end{figure}		

	
	\section{Pre-Questionnaire}
	\label{Pre-Questionnaire}
	
	\begin{figure}[H]
	\centering
		\begin{footnotesize}
			\includegraphics[scale=0.6]{Abbildungen/Fragebogen/Pre-Questionnaire/PQ1}\\
		\end{footnotesize}
	\end{figure}	
	
	\begin{figure}[H]
	\centering
		\begin{footnotesize}
			\includegraphics[scale=0.6]{Abbildungen/Fragebogen/Pre-Questionnaire/PQ2}\\
		\end{footnotesize}
	\end{figure}	
	
	\begin{figure}[H]
	\centering
		\begin{footnotesize}
			\includegraphics[scale=0.6]{Abbildungen/Fragebogen/Pre-Questionnaire/PQ3}\\
		\end{footnotesize}
	\end{figure}	
	
	\begin{figure}[H]
	\centering
		\begin{footnotesize}
			\includegraphics[scale=0.6]{Abbildungen/Fragebogen/Pre-Questionnaire/PQ4}\\
		\end{footnotesize}
	\end{figure}	
	
	\begin{figure}[H]
	\centering
		\begin{footnotesize}
			\includegraphics[scale=0.6]{Abbildungen/Fragebogen/Pre-Questionnaire/PQ5}\\
		\end{footnotesize}
	\end{figure}	
	
	\begin{figure}[H]
	\centering
		\begin{footnotesize}
			\includegraphics[scale=0.6]{Abbildungen/Fragebogen/Pre-Questionnaire/PQ6}\\
		\end{footnotesize}
	\end{figure}	
	
\section{Post-Questionnaire - Konditionsabfrage}
\label{Post-Questionnaire - Konditionsabfrage}

	\begin{figure}[H]
	\centering
		\begin{footnotesize}
			\includegraphics[scale=0.6]{Abbildungen/Fragebogen/Post-Questionnaire/PQM1}
		\end{footnotesize}
	\end{figure}	

\newpage

\section{Post-Questionnaire - Generelles Vertrauen}
\label{Post-Questionnaire - Generelles Vertrauen}

	\begin{figure}[H]
	\centering
		\begin{footnotesize}
			\includegraphics[scale=0.6]{Abbildungen/Fragebogen/Post-Questionnaire/PQG1}
		\end{footnotesize}
	\end{figure}	
	
	\begin{figure}[H]
	\centering
		\begin{footnotesize}
			\includegraphics[scale=0.6]{Abbildungen/Fragebogen/Post-Questionnaire/PQG2}
		\end{footnotesize}
	\end{figure}	
	
	\begin{figure}[H]
	\centering
		\begin{footnotesize}
			\includegraphics[scale=0.6]{Abbildungen/Fragebogen/Post-Questionnaire/PQG3}
		\end{footnotesize}
	\end{figure}	
	
	\begin{figure}[H]
	\centering
		\begin{footnotesize}
			\includegraphics[scale=0.6]{Abbildungen/Fragebogen/Post-Questionnaire/PQG4}
		\end{footnotesize}
	\end{figure}	

\newpage
\section{Post-Questionnaire - Kognitives Vertrauen}	
\label{Post-Questionnaire - Kognitives Vertrauen}	

	\begin{figure}[H]
	\centering
		\begin{footnotesize}
			\includegraphics[scale=0.6]{Abbildungen/Fragebogen/Post-Questionnaire/PQC1}
		\end{footnotesize}
	\end{figure}	
	
	\begin{figure}[H]
	\centering
		\begin{footnotesize}
			\includegraphics[scale=0.6]{Abbildungen/Fragebogen/Post-Questionnaire/PQC2}
		\end{footnotesize}
	\end{figure}	

\newpage
\section{Post-Questionnaire - Teamkommunikation}	
\label{Post-Questionnaire - Teamkommunikation}	

\begin{figure}[H]
	\centering
		\begin{footnotesize}
			\includegraphics[scale=0.4]{Abbildungen/Fragebogen/Post-Questionnaire/PQKQ1}
		\end{footnotesize}
	\end{figure}	

\newpage
\section{Post-Questionnaire - Team-Effektivität}	
\label{Post-Questionnaire - Team-Effektivität}	

	\begin{figure}[H]
	\centering
		\begin{footnotesize}
			\includegraphics[scale=0.6]{Abbildungen/Fragebogen/Post-Questionnaire/PQTE1}
		\end{footnotesize}
	\end{figure}	

\newpage
\section{Post-Questionnaire - NASA-TLX}
\label{Post-Questionnaire - NASA-TLX}

	\begin{figure}[H]
	\centering
		\begin{footnotesize}
			\includegraphics[scale=0.6]{Abbildungen/Fragebogen/Post-Questionnaire/PQTLX1}
		\end{footnotesize}
	\end{figure}	
	
	\begin{figure}[H]
	\centering
		\begin{footnotesize}
			\includegraphics[scale=0.6]{Abbildungen/Fragebogen/Post-Questionnaire/PQTLX2}
		\end{footnotesize}
	\end{figure}	
	
	\begin{figure}[H]
	\centering
		\begin{footnotesize}
			\includegraphics[scale=0.6]{Abbildungen/Fragebogen/Post-Questionnaire/PQTLX3}
		\end{footnotesize}
	\end{figure}	
	
	\begin{figure}[H]
	\centering
		\begin{footnotesize}
			\includegraphics[scale=0.6]{Abbildungen/Fragebogen/Post-Questionnaire/PQTLX4}
		\end{footnotesize}
	\end{figure}	
	
	\begin{figure}[H]
	\centering
		\begin{footnotesize}
			\includegraphics[scale=0.6]{Abbildungen/Fragebogen/Post-Questionnaire/PQTLX5}
		\end{footnotesize}
	\end{figure}	
	
	\begin{figure}[H]
	\centering
		\begin{footnotesize}
			\includegraphics[scale=0.6]{Abbildungen/Fragebogen/Post-Questionnaire/PQTLX6}
		\end{footnotesize}
	\end{figure}	

\newpage
\section{Post-Questionnaire - IPQ}
\label{Post-Questionnaire - IPQ}

	\begin{figure}[H]
	\centering
		\begin{footnotesize}
			\includegraphics[scale=0.6]{Abbildungen/Fragebogen/Post-Questionnaire/PQP1}
		\end{footnotesize}
	\end{figure}	
	
	\begin{figure}[H]
	\centering
		\begin{footnotesize}
			\includegraphics[scale=0.6]{Abbildungen/Fragebogen/Post-Questionnaire/PQP2}
		\end{footnotesize}
	\end{figure}	
	
	\begin{figure}[H]
	\centering
		\begin{footnotesize}
			\includegraphics[scale=0.6]{Abbildungen/Fragebogen/Post-Questionnaire/PQP3}
		\end{footnotesize}
	\end{figure}	

\newpage
\section{Post-Questionnaire - Co-Presence}			
\label{Post-Questionnaire - Co-Presence}		

	\begin{figure}[H]
	\centering
		\begin{footnotesize}
			\includegraphics[scale=0.6]{Abbildungen/Fragebogen/Post-Questionnaire/PQCP1}
		\end{footnotesize}
	\end{figure}	
	
	\begin{figure}[H]
	\centering
		\begin{footnotesize}
			\includegraphics[scale=0.6]{Abbildungen/Fragebogen/Post-Questionnaire/PQCP2}
		\end{footnotesize}
	\end{figure}	
	
	\begin{figure}[H]
	\centering
		\begin{footnotesize}
			\includegraphics[scale=0.6]{Abbildungen/Fragebogen/Post-Questionnaire/PQCP3}
		\end{footnotesize}
	\end{figure}	
	
	\begin{figure}[H]
	\centering
		\begin{footnotesize}
			\includegraphics[scale=0.6]{Abbildungen/Fragebogen/Post-Questionnaire/PQCP4}
		\end{footnotesize}
	\end{figure}	
	
	\begin{figure}[H]
	\centering
		\begin{footnotesize}
			\includegraphics[scale=0.6]{Abbildungen/Fragebogen/Post-Questionnaire/PQCP5}
		\end{footnotesize}
	\end{figure}	
	
	\begin{figure}[H]
	\centering
		\begin{footnotesize}
			\includegraphics[scale=0.6]{Abbildungen/Fragebogen/Post-Questionnaire/PQCP6}
		\end{footnotesize}
	\end{figure}	

\section{Other}	
	
\begin{figure}[H]
		\begin{footnotesize}
			\includegraphics[scale=1]{Abbildungen/criteriaForVirtualTeams.JPG}
			\caption[Abbildung 1]{Kriterien für \ac{vts} von \citep[p. 27]{schweitzer2010conceptualizing}}
			\label{criteriaForVirtualTeams}
		\end{footnotesize}
	\end{figure}	
	
%	\begin{figure}[H]
%	\centering
%		\begin{footnotesize}
%			\includegraphics[scale=0.6]{Abbildungen/CouchEtAl_1996_TrustScale}\\
%			\caption[Abbildung 1]{Items of the Trust Inventory \citep[305-323]{couch1996assessment}}
%			\label{Trust-Inventory}
%		\end{footnotesize}
%	\end{figure}
	

\end{document}
